% !TeX root = easycrypt.tex

%%%%%%%%%%%%%%%%%%%%%%%%%%%%%%%%%
% DEFS
%%%%%%%%%%%%%%%%%%%%%%%%%%%%%%%%%

\newcommand{\ambientKeywords}{}

\newcommand{\tacname}{Error tacname}
\newcommand{\vtacname}{Error tacname}

\newcommand{\makeAmbient}[3]{
  \expandafter\def\expandafter\ambientKeywords\expandafter{\ambientKeywords,#1}
  \renewcommand{\tacname}{\rawec{#1}}
  \renewcommand{\vtacname}{#1}
  \index{ambient}{#1@\rawec{#1}}
  \subsubsection{#1}
  \Syntax \rawec{#1} #2
  \Description #3
  \renewcommand{\tacname}{Error tacname}
  \renewcommand{\vtacname}{Error tacname}
}

\newcommand{\ambientLambda}{}
\newcommand{\addLambda}[3]{
  \expandafter\def\expandafter\ambientLambda\expandafter{\ambientLambda\makeAmbient{#1}{#2}{#3}}
}
\newcommand{\ambientLogic}{}
\newcommand{\addLogic}[3]{
  \expandafter\def\expandafter\ambientLogic\expandafter{\ambientLogic\makeAmbient{#1}{#2}{#3}}
}
\newcommand{\ambientAuto}{}
\newcommand{\addAuto}[3]{
  \expandafter\def\expandafter\ambientAuto\expandafter{\ambientAuto\makeAmbient{#1}{#2}{#3}}
}
\newcommand{\ambientOther}{}
\newcommand{\addOther}[3]{
  \expandafter\def\expandafter\ambientOther\expandafter{\ambientOther\makeAmbient{#1}{#2}{#3}}
}


\newcommand{\example}[6]%proof,context,goal
{
\vspace*{3ex}
\begin{tabular}{ccc}
\parbox{100pt}{#1} & {\expandafter\rawec\expandafter{#3 #4.}} & \parbox{100pt}{#5} \\
\cline{0-0} \cline{3-3} {\ec{#2}} & ~ & {\ec{#6}} \\
\end{tabular}\\
}

\newcommand{\env}[2]{\ec{#1 : #2}\\}

\newcommand{\tacarg}[2]{($#1$:\var{#2})}

\newcommand{\refdef}[1]{\emph{#1}(\ref{#1})}

\newcommand{\ambientDesc}{
  \subsection{Generalities}
    \subsubsection{judgements}\label{judgements} TODO
    \subsubsection{variables}\label{variables} TODO
    \subsubsection{context}\label{context} TODO
    \subsubsection{scope}\label{scope} TODO
    \subsubsection{environment}\label{environment} TODO
    \subsubsection{convertible}\label{convertible} This notion apply to two terms, they are convertible if they have the same normal form
      for lambda calculus. You can see it as making a \emph{TODO:simplify} and testing equality. It is automatically computed by easycrypt when needed.
      It will be denoted $\equiv$.
  \subsection{Lambda calculus}
    The tactics of this section correspond to lambda calculus reduction. It will find a redex, reduce it
    and do it again until there is no more redex. You can chain tactics by separating them with space and it
    will act as a new tactic that search for the redex of the two tactics and reduce them and do it again.
    \ambientLambda
  \subsection{Logic}
    \ambientLogic
  \subsection{Automatic}
    \ambientAuto
  \subsection{Others}
    \ambientOther
}


%%%%%%%%%%%%%%%%%%%%%%%%%%%%%%%%%
% LAMBDA
%%%%%%%%%%%%%%%%%%%%%%%%%%%%%%%%%

%beta
\addLambda{beta}{}{Apply all possible beta-reduction}

%iota
\addLambda{iota}{}{Reduce let tuple and if}

%zeta
\addLambda{zeta}{}{Reduce let}

%logic
\addLambda{logic}{}{Do logical simplification}

%delta
\addLambda{delta}{\tacarg{name}{ident}*}{Replace all occurence of all the specify $name$ by their values if no $name$ is  specify it replaces all possible names.}

%simplify
\addLambda{simplify}{\tacarg{name}{ident}|\var{delta}?}{Compute the normal form by applying the previous tactics in the given order and it applys delta if delta is specify and just delta names if namess is specify}

%congr
\addLambda{congr}{}{If the current goal is a function equality of arity $n$, it replaces this goal by $n$ new goals corresponfin to the argument equality.\\
You must be aware that this tactics will transform your goal in a stronger one.}

%pose
\addLambda{pose}{TODO}{TODO}


%%%%%%%%%%%%%%%%%%%%%%%%%%%%%%%%%
% LOGIC
%%%%%%%%%%%%%%%%%%%%%%%%%%%%%%%%%

%Tactics working only on the goal

%change
\addLogic{change}{\tacarg{f}{formula}}{This tactics permits to change the current goal to f.
\begin{displaymath}
  \infrule{G \equiv F ~ ~ ~ F}{G}
\end{displaymath}}

%split
\addLogic{split}{}
{\tacname{} tactic permits to do several things :
\begin{itemize}
  \item end a proof when you just have a term \refdef{convertible} to true.
  \item end a proof when you just have an equality between two terms that are \refdef{convertible}.
  \item replace a iff by the direct and indirect implication.
  \item replace an and statement by two goals : the left side and the right side.
  \item replace a n-tuple equality by n equality of the component.
\end{itemize}
\begin{displaymath}
  \infrule{a \equiv true}{a}
  ~~~~~~
  \infrule{a \equiv b}{a = b}
  ~~~~~~
  \infrule{\Gamma \vdash A \Rightarrow B ~ ~ ~ \Gamma \vdash B \Rightarrow A}{\Gamma \vdash A \Leftrightarrow B}
  ~~~~~~
  \infrule{\Gamma \vdash A ~ ~ ~ \Gamma \vdash B}{\Gamma \vdash A \land B}
  ~~~~~~
  \infrule{\Gamma \vdash a_1 = b_1  ~ ~ ~ ...  ~ ~ ~ \Gamma \vdash a_n = b_n}{\Gamma \vdash (a_1, ..., a_n) = (b_1, ..., b_n)}
  ~~~~~~
\end{displaymath}
}

%left
\addLogic{left}{}{This tactic permits to prove a or statement by proving only the left side. You must be aware that this tactics will transform your goal in a stronger one.
\begin{displaymath}
  \infrule{\Gamma \vdash A}{\Gamma \vdash A \lor B}
\end{displaymath}}

%right
\addLogic{right}{}{This tactic permits to prove a or statement by proving only the right side. You must be aware that this tactics will transform your goal in a stronger one.
\begin{displaymath}
  \infrule{\Gamma \vdash B}{\Gamma \vdash A \lor B}
\end{displaymath}}

%generalize
\addLogic{generalize}{\tacarg{C}{formula}}{
You must be aware that this tactics will transform your goal in a stronger one.
\begin{displaymath}
  \infrule{\Gamma \vdash C ~ ~ ~ \Gamma \vdash \forall x, G[C <- x]}{\Gamma \vdash G}
\end{displaymath}}

%case
\addLogic{case}{\tacarg{F}{formula}}{This tactic creates two subgoals, one where \emph{f} is true and an other where \emph{f} is false.
\begin{displaymath}
  \infrule{\Gamma \vdash f => G ~ ~ ~ \Gamma \vdash \neg f => G}{\Gamma \vdash G}
\end{displaymath}}



%Tactics working with the env

%assumption
\addLogic{assumption}{\tacarg{h}{ident}?}{It close the current goal by checking you have it in your context up to \refdef{conversion}. If you don't specify
\emph{h} it test all the \refdef{judgement} present in the context, else it just tests \emph{h}
\begin{displaymath}
  \infrule{H \equiv G}{\Gamma,h:H \vdash G}
\end{displaymath}}

%intros
\addLogic
{intros}
{\tacarg{x}{\_|ident}}
{This tactics permits to remove of your goal : a forall, the left side af an application or a let assignement by pushing it into your \refdef{context}.
Easycrypt checks that x is not already present in the \refdef{environment}.
\begin{displaymath}
  \infrule{\Gamma,x = a \vdash G(x)}{\Gamma \vdash let x = a in G(x)}
  ~~~~~~
  \infrule{\Gamma,x \vdash G(x)}{\Gamma \vdash \forall x, G(x)}
  ~~~~~~
  \infrule{\Gamma,H \vdash G}{\Gamma \vdash H => G}
  ~~~~~~
\end{displaymath}

\example
{}{forall (x y:int), x = 3 => x = 3}
{\vtacname}{a b hyp1}
{
\env{a}{int}
\env{b}{int}
\env{h1}{a=3}
}
{b = 3}
}

%cut
\addLogic{cut}{\tacarg{h}{ident} : \tacarg{C}{formula}}{It will ask you to first prove the lemma and then push it into your \refdef{context}.
\begin{displaymath}
  \infrule{\Gamma \vdash C ~ \Gamma,C \vdash G}{\Gamma \vdash G}
\end{displaymath}}

%elim
\addLogic{elim}{\tacarg{h}{ident}}{This tactics take as argument the name of a \refdef{judgment} from the \refdef{context} or the \refdef{scope}.
\begin{displaymath}
  \infrule{\Gamma, h:A \land B \vdash A \Rightarrow B \Rightarrow G}{\Gamma, h:A \land B \vdash G}
  ~~~~~~
  \infrule{\Gamma, h:\exists x, A(x) \vdash \forall x, A(x) \rightarrow G}{\Gamma, h:\exists x, A \vdash G}
\end{displaymath}\\
\begin{displaymath}
  \infrule{\Gamma, h:(a_1, ..., a_n) = (b_1, ..., b_n) \vdash a_1 = b_1 \Rightarrow ... \Rightarrow a_n = b_n \Rightarrow G}{\Gamma, h:(a_1, ..., a_n) = (b_1, ..., b_n) \vdash G}
\end{displaymath}}

%clear
\addLogic{clear}{\tacarg{h}{ident}?}{It removes the judgement $h$ from the context
\begin{displaymath}
  \infrule{\Gamma \vdash G}{\Gamma,h:H \vdash G}
\end{displaymath}}



%%%%%%%%%%%%%%%%%%%%%%%%%%%%%%%%%
% AUTO
%%%%%%%%%%%%%%%%%%%%%%%%%%%%%%%%%

%smt
\addLogic{smt}{TODO}{This tactic send your current goal to provers.\\
\warningbox{Some judgements can't be converted see \ref{smtProgram} TODO} }

%apply
\addLogic{apply}{TODO}{Apply a lemma in order to prove the current goal TODO}

%rewrite
\addAuto{rewrite}{TODO}{Rewrite an expression in your goal according to a lemma TODO}

%elimT
\addAuto{elimT}{TODO}{Apply an induction princple to your current goal TODO}

%subst
\addAuto{subst}{\tacarg{x}{ident}?}{Find a value for var \emph{x} in the hypothesis, subsitute it everywhere and remove it. If \emph{x} is not specify
it does it for all variable. TODO}

%progress
\addAuto{progress}{TODO}{Split all you hypothesis, make all subsitution possible and then split your goal and do it again on all subgoals. It
permits to easly break a big judgements in smaller one TODO}

%trivial
\addAuto{trivial}{TODO}{TODO}



%%%%%%%%%%%%%%%%%%%%%%%%%%%%%%%%%
% OTHER
%%%%%%%%%%%%%%%%%%%%%%%%%%%%%%%%%

%idtac
\addOther
{idtac}
{\tacarg{x}{string}}
{TODO\\
\example
{env}{formula(env)}
{\vtacname}{}
{env}{formula(env)}
}

%same
\addOther{same}{}{TODO}

%modpath
\addOther{modpath}{}{TODO}

%using
\addOther{using}{}{TODO}

%compute
\addOther{compute}{}{TODO}

%field
\addOther{field}{}{TODO}

%fieldsimplify
\addOther{fieldsimplify}{}{TODO}
