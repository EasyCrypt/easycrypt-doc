% !TeX root = easycrypt.tex
\chapter{Writing Proofs}

\section{First-Order Tactics}

\subsection{The \texttt{case} tactic}
%

\subsection{The \texttt{unfold} tactic}
%

\subsection{The \texttt{simplify} tactic}
%

\subsection{The \texttt{trivial} tactic}
%

\section{Probabilistic Hoare logic}



\section{Bounded probabilistic Hoare logic}

\subsection{The \texttt{rnd} tactic}
%
\Syntax 
\verb+rnd+ (\textit{formula} $|$ \_ ) (\textit{formula} $|$ \_ )

\Description
the first optional parameter $p$ is a computable predicate (i.e., \verb+'a cPred+)
(i.e., \verb+'a -> bool+ ). Assume $d$ of type \verb+A Distr.distr+:
\begin{displaymath}
\begin{array}{c}
  \infrule{
    \bdHoareSeq{c}{\pre}{\mu\, d\, p \leq f \land 
      (\forall v\in \mathsf{support}(d).~ \post\subst{x}{v} \Rightarrow p\, v)}{1} 
  }{
    \bdHoareSle{c;\Rand{x}{d}}{\pre}{\post}{f}
  }\left[\verb+rnd+\ p\right]
\\[4ex]
\end{array}
\end{displaymath}
If $p$ is not given then the tool attempts to build it from $\post$
(not implemented yet).

For judgments bounded by $(=,\geq)$ the tactic additionally accepts as second parameter a real
number $g$ representing a bound:
\begin{displaymath}
  \infrule{
    \bdHoareSge{c}{\pre}{\mu\, d\, p \geq g \land 
      (\forall v\in \mathsf{support}(d).~ \post\subst{x}{v} \Rightarrow p\, v)}{\frac{f}{g}} 
  }{
    \bdHoareSge{c;\Rand{x}{d}}{\pre}{\post}{f}
  }\left[\verb+rnd+\ p\ g\right]
\\[4ex]
\end{displaymath}
If $g$ is not given then $g=f$ in the rule.

\subsection{The \texttt{app} tactic}
%
\Syntax 
\verb+app+ \verb+[>>|<<]+ \textit{num} \textit{formula} [\textit{formula}]

\Description
Direction \verb+<<+ by default.
The first formula is the intermediate predicate.
Assume $s_2$ is at program position $n$:
\begin{displaymath}
  \infrule{
    \hoareS{s1}{P}{R} \qquad \bdHoareSle{s2}{R}{Q}{f}
  }{
    \bdHoareSle{s1;s2}{P}{Q}{f}
  }\left[\verb+app+\ n\ R\right]
\end{displaymath}
%
Which, if preferred, can be rewritten to:
\begin{displaymath}
  \infrule{
    \hoareS{s1}{P}{\lambda m. \Prm{s_2}{m}{Q}\leq f} \qquad 
  }{
    \bdHoareSle{s1;s2}{P}{Q}{f}
  }\left[\verb+app+\ n\ R\right]
\end{displaymath}


The following rule is a generalization but not implemented:
\begin{displaymath}
  \infrule{
    \begin{array}{c}
      \bdHoareSle{s1}{P}{R}{f_1} \qquad \bdHoareSle{s2}{R}{Q}{f_2}
      \\
      \bdHoareSle{s1}{P}{R}{g_1} \qquad \bdHoareSle{s2}{R}{Q}{g_2}
      \\
      f_1 f_2 + g_1 g_2 \leq f 
    \end{array}
  }{
    \bdHoareSle{s1;s2}{P}{Q}{f}
  }
\end{displaymath}


The second parameter (optional) is a real number representing a bound (only supported for $=$ and $\geq$).
\begin{displaymath}
\begin{array}{c}
  \infrule{
    \bdHoareSge{s1}{P}{R}{f/g} \qquad \bdHoareSge{s2}{R}{Q}{g}
  }{
    \bdHoareSge{s1;s2}{P}{Q}{f}
  }\left[\verb+app+\ n\ R\ g\right]
\\[4ex]
  \infrule{
    \bdHoareSge{s1}{P}{R}{g} \qquad \bdHoareSge{s2}{R}{Q}{f/g}
  }{
    \bdHoareSge{s1;s2}{P}{Q}{f}
  }\left[\verb+app>>+\ n\ R\ g\right]
\end{array}
\end{displaymath}
Similar rules hold for $=$. If the parameter $g$ is not given then
$g=f$ in the rule. 

\subsection{The \texttt{if} tactic}
%
(similar to probabilistic Hoare \verb+if+ tactic)
\begin{displaymath}
\begin{array}{c}
  \infrule{
    \bdHoareSle{c_1}{\pre \land b}{\post}{f}\qquad
    \bdHoareSle{c_2}{\pre \land \neg b}{\post}{f}
  }{
    \bdHoareSle{\Cond{b}{c_1}{c_2}}{\pre}{\post}{f}
  }\left[\verb+if+ \right] 
\\[4ex]
\end{array}
\end{displaymath}
Similar rules hold for $=,\geq$.

\subsection{The \texttt{wp} tactic}
%
\Syntax \verb+wp+

\Description
Assuming \verb+wp+ complete (weakest) and $c$ is straight-line code,
$\bdHoareSeq{c}{\mathsf{wp}(c,\post)}{\post}{1}$, plus sequential
composition rule, we have:
\begin{displaymath}
  \infrule{
    \bdHoareSle{c_1}{\pre }{\mathsf{wp}(c_2,\post)}{f}
  }{
    \bdHoareSle{c_1;c_2}{\pre}{\post}{f}
  }\left[\verb+wp+ \right] 
\end{displaymath}
Similar rules hold for $=,\geq$.


\subsection{The \texttt{skip} tactic}
%

\subsection{The \texttt{condt,condf} tactic}
%


%***** NOT IMPLEMENTED 

\subsection{The \texttt{while} tactic}
\Syntax \verb+while+ \textit{formula} \textit{formula} 
[\textit{formula} \textit{formula}]
%

\Description
%
The first formula is the loop invariant.
%
The second one is a variant expression. 
%
The third one is a real expression bound $g$ and the fourth one an
integer expression $n$.
%
If $g$ is not given then it is interpreted as $g=1$, and the fourth
formula is ignored, otherwise required. $M$ stands for the variables
that may be modified by $c$.

\begin{displaymath}
  \infrule{
    \begin{array}{c}
    \bdHoareSge{c'}{\pre }{\chi \land 
      \forall M.~ (\chi \land 0 \leq e \Rightarrow \neg b)  \land
      \chi \land \neg b \Rightarrow \post}{f} 
    \\[.5ex]
    \forall k.~ \bdHoareSeq{c}{\chi \land b \land e = k}{\chi \land e
      < k}{1}
  \end{array}
}{
    \bdHoareSge{c';\While{b}{c}}{\pre}{\post}{f}
  }\left[\verb+while+\ \chi\ e \right] 
\end{displaymath}
Similarly for (=).

For an arbitrary bound $g$ the following rule generalizes the one
above for lower bounded judgments:
\begin{displaymath}
  \infrule{
    \begin{array}{c}
    \bdHoareSge{c'}{\pre }{\chi \land e \leq n \land 
      \forall M.~ (\chi \land 0 \leq e \Rightarrow \neg b) 
      \land (\chi \land \neg b \Rightarrow \post)}{\frac{f}{g^n}} 
    \\[.5ex]
    \bdHoareSge{c}{\chi \land b}{\chi}{g}
    \\[.5ex]
    \forall k.~ \bdHoareSeq{c}{\chi \land b \land e = k}{e<k}{1}
  \end{array}
}{
    \bdHoareSge{c';\While{b}{c}}{\pre}{\post}{f}
  }\left[\verb+while+\ \chi\ e\ g\ n \right] 
\end{displaymath}

and the folowing one for exact judgments (=):
\begin{displaymath}
  \infrule{
    \begin{array}{c}
    \bdHoareSge{c'}{\pre }{\chi \land e = n \land 
      \forall M.~ (\chi\Rightarrow (0\leq e \Leftrightarrow \neg b)) 
        \land (\chi \land \neg b \Rightarrow \post)}
      {\frac{f}{g^n}}
    \\[.5ex]
    \bdHoareSge{c}{\chi \land b}{\chi}{g}
    \\[.5ex]
    \forall k.~ \bdHoareSeq{c}{\chi \land b \land e = k}{e=k-1}{1}
  \end{array}
}{
    \bdHoareSge{c';\While{b}{c}}{\pre}{\post}{f}
  }\left[\verb+while+\ \chi\ e\ g\ n \right] 
\end{displaymath}


There is no appropriate rule for $(\leq)$.


\subsection{The \texttt{call} tactic}
%


\subsection{The \texttt{inline} tactic}
%

\subsection{The \texttt{swap} tactic}
%


\section{Relational probabilistic Hoare logic}


\section{Combination of Tactics}


\section{Automated Tactics}


%%% Local Variables: 
%%% mode: latex
%%% TeX-master: "easycrypt"
%%% End: 
