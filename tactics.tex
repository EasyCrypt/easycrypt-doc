% !TeX root = easycrypt.tex

%% TODO (Francois): For index, rather than \texttt, use \rawec and make a class of keywords for tactics and tacticals

\chapter{Writing Proofs\label{chap:tactics}}

\EasyCrypt comes with a proof engine that allows to state and prove properties
about programs written in the various languages described in
Chapter~\ref{chap:theories}.
%
Proofs are built interactively by applying \define[tactic]{tactics}, that transform a
current \define[goal]{proof goal} into a (possibly empty) set of
\define[subgoal]{subgoals} such that the conjunction of the subgoals implies the
original goal.
%
This process is repeated, starting from the theorem statement, up to the point
where all the subgoals correspond to general axioms or premises of the theorem.

In this chapter, we describe this proof engine in general before listing and
describing the existing tactics for various fragments of \EasyCrypt's
underlying logic.

\section{The proof engine}

The proof engine deals with \define[judgment]{judgments} or goals of the form
$\Env; \Gamma \vdash \phi$ where $\Env$ is the (global) \define{environment},
$\Gamma$ is the \define{context} (a set of local facts) and $\phi$ is the
formula we want to prove. Here is an example of such a judgment:

\begin{center}
$\Int; x, y , z: \tint, x \le y \vdash x + z \le y + z$.
\end{center}

It states that in the environment ($\Env$) solely composed of the
theory $\Int$, having three local variables $x, y, z$ of type $\tint$ along
with the fact $x \le y$ (the context $\Gamma$), we are interested
in proving $x + z \le y + z$.

\medskip

On top on this, a set of \define[deduction rule]{deduction rules} is given. They
describe how one can derive a judgment $\Env; \Gamma \vdash \phi$ given
that a set of prerequisites (or \define[premise]{premises}) are fulfilled. The
general form of such a rule is given as follow:

\begin{displaymath}
 \infrule{A_1 \cdots A_n}{\Env; \Gamma \vdash \phi}
\end{displaymath}

It has to be read as: \emph{given that $A_1 \cdots A_n$ are derivable, then
so is $\Env, \Gamma \vdash \phi$}. We give three examples of such deduction
rules:

\begin{displaymath}
 \infrule
         {\Env; \Gamma \vdash \phi_1 \quad
          \Env; \Gamma \vdash \phi_1 \Rightarrow \phi_2}
         {\Env; \Gamma \vdash \phi_2}
         {\rname{MP}}
 \quad\quad
 \infrule
         {\Env; \Gamma, \phi_1 \vdash \phi_2}
         {\Env; \Gamma \vdash \phi_1 \Rightarrow \phi_2}
         {\rname{$\Rightarrow$-I}}
 \quad\quad
 \infrule{ }{\Env; \Gamma, \phi, \Delta \vdash \phi}{\rname{Ax}}
\end{displaymath}

The first, the \emph{modus ponens}, states that one can derive
$\Env; \Gamma \vdash \phi_2$ given that $\Env; \Gamma \vdash \phi_1
\Rightarrow \phi_2$ and $\Env; \Gamma \vdash \phi_1$ are derivable.
%
The next provides a way for deriving $\phi_1 \Rightarrow \phi_2$ from
a derivation of $\phi_2$, but with a context augmented by $\phi_1$.
%
The last states that $\Env; \Gamma, \phi, \Delta \vdash \phi$ is derivable as-is.

Combining these deduction rules, it is possible to build a tree rooted by
a judgment $\Env; \Gamma \vdash \phi$ and with leaves composed of deduction
rules with no premises (as the third one in the previous example). Such a
tree forms a \define{proof} of $\Env; \Gamma \vdash \phi$.
%
For instance, Figure~\ref{fig:LJproof} gives a proof of
%
\[\Env; b_1, b_2 : \tbool \vdash (b_1 \Rightarrow b_2) \Rightarrow b_1 \Rightarrow b_2\]

\begin{figure}
  \begin{displaymath}
    \infrule
      {\infrule{ }{\Env; b_1, b_2 : \tbool, b_1 \Rightarrow b_2, b_1 \vdash b_1 \Rightarrow b_2} \quad
       \infrule{ }{\Env; b_1, b_2 : \tbool, b_1 \Rightarrow b_2, b_1 \vdash b_1}}
      {\infrule
        {\Env; b_1, b_2 : \tbool, b_1 \Rightarrow b_2, b_1 \vdash b_2}
        {\infrule
           {\Env; b_1, b_2 : \tbool, b_1 \Rightarrow b_2 \vdash b_1 \Rightarrow b_2}
           {\Env; b_1, b_2 : \tbool \vdash (b_1 \Rightarrow b_2) \Rightarrow b_1 \Rightarrow b_2}}}
  \end{displaymath}

  \caption{\label{fig:LJproof} Proof tree of
    $\Env; b_1, b_2 : \tbool \vdash
        (b_1 \Rightarrow b_2) \Rightarrow b_1 \Rightarrow b_2$}
\end{figure}

The \EasyCrypt proof engine helps the user build such proofs. At each step
of the proof building, the system presents to the user the set of goals
that have to be proved. The user can then \emph{apply} a tactic to one of
them, each tactic corresponding to a deduction rule. If the conclusion
of the rule corresponding to the applied tactic matches the goal to which
it is applied, the proof engine replaces it with the set of the
premises of the applied rule - the subgoals. This application may generate
no, one or several subgoals depending on the rule. This process is repeated
iteratively, up to the point where no goals remain.

\section{Tacticals}

% !TeX root = easycrypt.tex

\newcommand{\addTclTacticPlain}[2]{\addTacticPlain{#1}{#2}}
\newcommand{\addTclTacticIdx}[1]{\addTacticIdx{tactical}{#1}}
\newcommand{\addTclTactic}[2]{\addTactic{tactical}{#1}{#2}}

%seq
\addTclTacticPlain{Tactics sequencing}{\ec{t1; t2}}
Execute \ec{t1} and then \ec{t2} on all the subgoals generated by
\ec{t1}.

%try
\addTclTacticPlain{Failure recovering}{\ec{try t}}
\addTclTacticIdx{try}
Execute the tactic \ec{t} or do noting if this last failed.

%repetition
\addTclTacticPlain{Tactics repetition}{\ec{do t}}
\addTclTacticIdx{do}

%selection
\addTclTacticPlain{Goal selection}{\ec{t1; first t2}}
\addTclTacticIdx{first}
\addTclTacticIdx{last}
Apply the tactic \ec{t1} and then \ec{2} on the first subgoal
generated by \ec{1}. It is an error if no subgoals have been
generator by \ec{1}.

\paragraph{Variants}\strut\\

\noindent\begin{tabularx}{\textwidth}{@{}ll@{}}
 {\ec{t1; first n t2}} & apply {\ec{t2}} on the first {\ec{n}} subgoals
   generated by {\ec{t1}}\\
 {\ec{t1; last t2}} & apply {\ec{t2}} on the last subgoal
   generated by {\ec{t1}}\\
 {\ec{t1; last n t2}} & apply {\ec{t2}} on the last {\ec{n}} subgoals
   generated by {\ec{t1}}
\end{tabularx}



\section{Ambient Logic}
% !TeX root = easycrypt.tex

%%%%%%%%%%%%%%%%%%%%%%%%%%%%%%%%%
% DEFS
%%%%%%%%%%%%%%%%%%%%%%%%%%%%%%%%%

\newcommand{\ambientKeywords}{}

\newcommand{\tacname}{Error tacname}
\newcommand{\vtacname}{Error tacname}

\newcommand{\addTactic}[2]{
  \expandafter\def\expandafter\ambientKeywords\expandafter{\ambientKeywords,#1}
  \renewcommand{\tacname}{\rawec{#1}}
  \renewcommand{\vtacname}{#1}
  \index{ambient}{#1@\rawec{#1}}
  \subsubsection{#1}
  \Syntax \ec{#1} #2
  \Description%
}


\newcommand{\example}[6]%proof,context,goal
{
\vspace*{3ex}
\begin{tabular}{ccc}
\parbox{100pt}{#1} & {\expandafter\rawec\expandafter{#3 #4.}} & \parbox{100pt}{#5} \\
\cline{0-0} \cline{3-3} {\ec{#2}} & ~ & {\ec{#6}} \\
\end{tabular}\\
}

\newcommand{\env}[2]{\ec{#1 : #2}\\}

\newcommand{\vararg}[1]{\ec{#1}}
\newcommand{\cstarg}[1]{\ec{#1}}
\newcommand{\typarg}[1]{\textit{#1}}

\newcommand{\tacarg}[2]{(\vararg{#1}:\typarg{#2})}

\newcommand{\refdef}[1]{\emph{#1}(\ref{#1})}


%%%%%%%%%%%%%%%%%%%%%%%%%%%%%%%%%
% END DEFS
%%%%%%%%%%%%%%%%%%%%%%%%%%%%%%%%%

\subsection{Generalities}

\EasyCrypt ambient logic is based on non-dependent higher-order logic.

\subsection{Convertibility}\label{convertible}

\EasyCrypt ambient logic enjoys a mechanism that identifies all formulas
that are equal up to a given amount of computations.

\begin{center}
\begin{tabular}{l@{$\quad$}l@{$\quad$}ll}
{\rawec{(lambda (x : t), phi1)\ phi2}} & $\rightarrow_\beta$ &
  \multicolumn{2}{@{}l}{{\rawec{phi2} \{\rawec{x} $\leftarrow$ \rawec{phi1}\}}}\\
{\rawec{if (true) \{ phi1 \} else \{ phi2 \}}} & $\rightarrow_\iota$ &
  {\rawec{phi1}}\\
{\rawec{if (false) \{ phi1 \} else \{ phi2 \}}} & $\rightarrow_\iota$ &
  \multicolumn{2}{@{}l}{{\rawec{phi2}}}\\
{\rawec{let (x1, ..., xn) = (phi1, ..., phin) in phi}} & $\rightarrow_\iota$ &
  \multicolumn{2}{@{}l}{{\rawec{phi} \{ \rawec{x1, ..., xn} $\leftarrow$ \rawec{phi1, ..., phin} \}}}\\
{\rawec{let x = phi1 in phi2}} & $\rightarrow_\zeta$ &
  \multicolumn{2}{@{}l}{{\rawec{phi2} \{ \rawec{x} $\leftarrow$ \rawec{phi1} \}}}\\
{\rawec{o}} & $\rightarrow_\delta^{\Env,\Gamma}$ &
  {\rawec{e}} & if {\rawec{op o := e}} $\in \Env$\\
{\rawec{x}} & $\rightarrow_\delta^{\Env,\Gamma}$ &
  {\rawec{phi}} & if {\rawec{x := phi}} $\in \Gamma$\\
\end{tabular}
\end{center}


%%%%%%%%%%%%%%%%%%%%%%%%%%%%%%%%%
% LAMBDA
%%%%%%%%%%%%%%%%%%%%%%%%%%%%%%%%%

\subsection{Lambda}

%change
\addTactic{change}{\tacarg{f}{formula}}
Change the current goal to the $\leftrightarrow^*$-equivalent one \ec{f}
\begin{displaymath}
  \infrule{\phi_1 \leftrightarrow^*_{\Env;\Gamma} \phi_2 \quad
           \Env; \Gamma \vdash \phi_1}
          {\Env; \Gamma \vdash \phi_2}
\end{displaymath}

%beta
\addTactic{beta}{}
Change the goal with its $\beta$-head normal-form.

%iota
\addTactic{iota}{}
Change the goal with its $\iota$-head normal-form.

%zeta
\addTactic{zeta}{}
Change the goal with its $\zeta$-head normal-form.

%logic
\addTactic{logic}{}
Change the goal with its $\Lambda$-head normal-form.

%delta
\addTactic{delta}{\tacarg{names}{ident*}}
Do one step of parallel, strong $\delta$-reduction, restricted to
 the symbols designed by \ec{names}. If \ec{names} if empty, no restriction
 on the $\delta$-reduction is applied.

%simplify
\addTactic{simplify}{\tacarg{names}{ident*} | \var{delta}$\!\!$?}
Change the goal with its $\beta\iota\zeta\Lambda$-head normal-form, followed
 by one step of parallel, strong $\delta$-reduction if \ec{delta} is given.
 The $\delta$-reduction can be restricted to a set of defined symbols by
 replacing \ec{delta} by the non-empty sequence of targeted symbols.

%congr
\addTactic{congr}{}
This tactic applies to a goal of the form \ec{f t1 ... tn = f u1 ... un}
 replacing it by  the subgoals \ec{ti = ui} for all \ec{i}. Note that subgoals
 solvable by \ec{reflexivity} are automatically closed.

%generalize
\addTactic{generalize}{\tacarg{p}{pattern}}
Search for the first subterm of the goal matching \ec{p} and leading
to the full instantiation of the pattern. Then, do a logical
generalization of all the occurrences of \ec{p}, after instantiation,
in the goal.
\begin{displaymath}
  \infrule{\Env; \Gamma \vdash p \quad
           \Env; \Gamma \vdash \forall x, \phi(x)}
          {\Env; \Gamma \vdash \phi(p)}
\end{displaymath}

%pose
\addTactic{pose}{\tacarg{x}{ident} \rawec{:=} \tacarg{p}{pattern}}
Search for the first subterm of the goal matching \ec{p} and leading
to the full instantiation of the pattern. Then, introduce, after
instantiation, the local definition \rawec{x := p} and abstract
all the occurrences of \ec{p} in the goal by \ec{x}
\begin{displaymath}
  \infrule{\Env; \Gamma \vdash p \quad
           \Env; \Gamma, x := p \vdash \phi(x)}
          {\Env; \Gamma \vdash \phi(p)}
\end{displaymath}

%%%%%%%%%%%%%%%%%%%%%%%%%%%%%%%%%
% LOGIC
%%%%%%%%%%%%%%%%%%%%%%%%%%%%%%%%%

\subsection{Logic}

%split
\addTactic{split}{}
\tacname{} breaks a goal that is intrinsically conjunctive into multiple subgoals.
 For instance, it
 \begin{itemize}
  \item closes any goal that is \refdef{convertible} to \ec{true} or provable
        by \ec{reflexivity},

  \begin{displaymath}
  \infrule{\Env; \Gamma \vdash a \equiv true}{a}
  ~~~~~~
  \infrule{\Env; \Gamma \vdash a \equiv b}{a = b}
  \end{displaymath}
       
  \item replaces a logical equivalence by the direct and indirect implication,

  \begin{displaymath}
  \infrule{\Env; \Gamma \vdash \phi_1 \Rightarrow \phi_2 \quad
           \Env; \Gamma \vdash \phi_2 \Rightarrow \phi_1}
          {\Gamma \vdash \phi_1 \Leftrightarrow \phi_2}
  \end{displaymath}
  
  \item replaces a goal of the form \rawec{f1 /\\ f2} or \rawec{f1 \&\& f2} by the two
        subgoals for \ec{f1} and \ec{f2},

  \begin{displaymath}
  \infrule{\Env; \Gamma \vdash \phi_1 \quad
           \Env; \Gamma \vdash \phi_2}
          {\Env; \Gamma \vdash \phi_1 \land \phi_2}
  ~~~~~~
  \infrule{\Env; \Gamma \vdash \phi_1 \quad
           \Env; \Gamma \vdash \phi_2}
          {\Env; \Gamma \vdash \phi_1 \&\& \phi_2}
  \end{displaymath}
        
  \item replaces an equality between two $n$-tuples by the $n$ equalities of
        of the paired components.

  \begin{displaymath}
  \infrule{\Env; \Gamma \vdash a_1 = b_1  \quad \cdots \quad
           \Env; \Gamma \vdash a_n = b_n}
          {\Gamma \vdash (a_1, ..., a_n) = (b_1, ..., b_n)}
  \end{displaymath}
\end{itemize}

%left
\addTactic{left}{}
Reduce a disjunctive goal to its left part
\begin{displaymath}
  \infrule{\Env; \Gamma \vdash \phi_1}{\Env; \Gamma \vdash \phi_1 \lor \phi_2}
\end{displaymath}

%right
\addTactic{right}{}
Reduce a disjunctive goal to its right part
\begin{displaymath}
  \infrule{\Env; \Gamma \vdash \phi_2}{\Env; \Gamma \vdash \phi_1 \lor \phi_2}
\end{displaymath}

%case
\addTactic{case}{\tacarg{f}{formula}}
Do an excluded-middle case analysis on \ec{f}
\begin{displaymath}
  \infrule{\Env; \Gamma \vdash b \Rightarrow \phi(true) \quad
           \Env; \Gamma \vdash \neg b \Rightarrow \phi(false)}
          {\Env; \Gamma \vdash \phi(b)}
\end{displaymath}

%assumption
\addTactic{assumption}{}
Search in the context an hypothesis \refdef{convertible} to the goal and close it.
 If no such hypothesis exists, the tactic fails
\begin{displaymath}
  \infrule{(h : \phi) \in \Gamma}{\Env; \Gamma \vdash \phi}
\end{displaymath}

%intros
\addTactic{intros}{\tacarg{x}{\_|ident}}
This tactics permits to remove of your goal : a forall, the left side af an application or a let assignement by pushing it into your \refdef{context}.
Easycrypt checks that \vararg{x} is not already present in the \refdef{environment}.
\begin{displaymath}
  \infrule{\Gamma,x = a \vdash G(x)}{\Gamma \vdash let x = a in G(x)}
  ~~~~~~
  \infrule{\Gamma,x \vdash G(x)}{\Gamma \vdash \forall x, G(x)}
  ~~~~~~
  \infrule{\Gamma,H \vdash G}{\Gamma \vdash H => G}
  ~~~~~~
\end{displaymath}

\example
{}{forall (x y:int), x = 3 => x = 3}
{\vtacname}{a b hyp1}
{
\env{a}{int}
\env{b}{int}
\env{h1}{a=3}
}
{b = 3}


%cut
\addTactic{cut}{\tacarg{ip}{intro-pattern} : \tacarg{C}{formula}}
Logical cut. Generates two subgoals: on for $C$ (the cut formula),
 and one for $C \Rightarrow G$ where $G$ is the initial goal. Moreover,
 the intro-pattern \ec{ip} is applied to the second subgoal.
\begin{displaymath}
  \infrule{\Env; \Gamma \vdash C \quad
           \Env; \Gamma, \vdash C \Rightarrow G}
          {\Env; \Gamma \vdash G}
\end{displaymath}

%elim
\addTactic{elim}{\tacarg{h}{ident}}
This tactics take as argument the name of a \refdef{judgment} from the \refdef{context} or the \refdef{scope}.
\begin{displaymath}
  \infrule{\Gamma, h:A \land B \vdash A \Rightarrow B \Rightarrow G}{\Gamma, h:A \land B \vdash G}
  ~~~~~~
  \infrule{\Gamma, h:\exists x, A(x) \vdash \forall x, A(x) \rightarrow G}{\Gamma, h:\exists x, A \vdash G}
\end{displaymath}\\
\begin{displaymath}
  \infrule{\Gamma, h:(a_1, ..., a_n) = (b_1, ..., b_n) \vdash a_1 = b_1 \Rightarrow ... \Rightarrow a_n = b_n \Rightarrow G}{\Gamma, h:(a_1, ..., a_n) = (b_1, ..., b_n) \vdash G}
\end{displaymath}


%%%%%%%%%%%%%%%%%%%%%%%%%%%%%%%%%
% AUTO
%%%%%%%%%%%%%%%%%%%%%%%%%%%%%%%%%

\subsection{Automatic}

%smt
\addTactic{smt}{[\ec{nolocal}]}
Try to solve the goal using SMT solvers. The goal is sent along with all the
 lemmas proved so far plus the local hypotheses, unless the \ec{nolocal} is
 given.
 
 \noindent
 \warningbox{Not all lemmas can be sent translated in such a way that they can
  be sent to the SMT provers. For instance, any formulas involving pRHL
  constructions are ignored.}

%apply
\addTactic{apply}{\tacarg{p}{proof-term}}
Modus Ponens. If \ec{p} is a proof-term for the pattern (formula) for
  \begin{center}
    \ec{forall (x1 : t1) ... (xn : tn), A1 -> ... -> An -> B}
  \end{center}
  \noindent then \tacname{} tries to match B with the current G. If the
  match succeeds and leads to the full instantiation of the pattern,
  then the goal is replaced, after instantiation, with the $n$ subgoals
  \ec{A1, ..., An}

%rewrite
\addTactic{rewrite}{rw1 ... rw${}_n$ where the rw${}_i$ are of the form \ec{//},
\ec{/=}, \ec{//=}, a proof-term or a pattern prefixed by \ec{/}
(slash). The two last forms can be prefixed by a direction indicator (the sign
\ec{-}), followed by an occurrence selector (\ec{\{i1 ... in\}}),
followed by repetition marker (\ec{!}, \ec{?}, \ec{i!} or \ec{i?}). All
these prefixes are optional.}
Depending on the form of \ec{rw}, \tacname{} \ec{rw} does the following:
  \begin{itemize}
   \item Call \rawec{trivial} if \ec{rw} is \ec{//},
   \item Call \rawec{simplify} if \ec{rw} is \ec{/=},
   \item Call \rawec{simplify; trivial} as \ec{rw} is \ec{//=},
   \item If \ec{rw} is a proof-term for the pattern (formula)
     \begin{center}
      \ec{forall (x1 : t1) ... (xn : tn), A1 -> ... -> An -> f1 = f2}
     \end{center}
     \noindent then \tacname{} searches for the first subterm of the goal
     matching \ec{f1} and resulting in the full instantiation of the pattern.
     It then replaces, after instantiation of the pattern, all the occurrences
     of \ec{f1} by \ec{f2} in the goal, and creates $n$ new subgoals for the
     \ec{Ai}'s. If no subterms of the goal match \ec{f1} or if the pattern
     cannot be fully instanciated by matching, the tactic fails.
     The tactic works the same if the pattern ends by \ec{f1 <-> f2}. If the
     direction indicator \ec{-} is given, \tacname{} works in the reverse
     direction, searching for a match of \ec{f2} and then replacing all
     occurrences of \ec{f2} by \ec{f1}.
   \item If \ec{rw} is a \ec{/}-prefixed pattern of the form \ec{(o p1 ... pn)},
     with \ec{o} a defined symbol, then \tacname{} searches for the first subterm
     of the goal matching \ec{(o p1 ... pn)} and resulting in the full instantiation
     of the pattern. It then replaces, after instantiation of the pattern, all
     the occurrences of \ec{(o p1 ... pn)} by the $\beta\delta_{\rm o}$ head-normal form
     of \ec{(o p1 ... pn)}. If no subterms of the goal match \ec{(o p1 ... pn)} or
     if the pattern cannot be fully instanciated by matching, the tactic fails. If the
     direction indicator \ec{-} is given, \tacname{} works in the reverse
     direction, searching for a match of the $\beta\delta_{\rm o}$ head-normal
     of \ec{(o p1 ... pn)} and then replacing all occurrences of this head-normal
     form with \ec{(o p1 ... pn)}.
  \end{itemize}
  
  \smallskip
  
  The occurrence selector \ec{\{i1 ... in\}} allows to restrict which occurrences
  of the matching pattern are replaced in the goal. If given, only the
  \ec{i1}-th, ..., \ec{in}-th ones are replaced (considering that the goal is
  traversed in DFS mode). Note that this selection applies after the matching has
  been done.
  
  \medskip
  
  Repetition markers allow the repetition of the same rewriting. For instance,
  \tacname{} \ec{!rw} leads to \ec{do!} \tacname{} \ec{rw}. See \ec{do} for
  more information.
  
  \medskip

  Last, \tacname{} \ec{rw1 ... rwn} is equivalent to
  \tacname{} \ec{rw1}; ...; \tacname{} \ec{rwn}

%elimT
\addTactic{elim}{$\!\!$/\tacarg{h}{ident} \tacarg{f}{pattern}}
Apply the induction principle \vararg{h} on \vararg{x}

%subst
\addTactic{subst}{\tacarg{x}{ident}?}
Search for the first equation of the form \ec{x = f} or \ec{f = x} in the context
 and replace all the occurrences of \ec{x} by \ec{f} everywhere in the context and the
 goal before clearing it. If no idents are given, repeatedly apply the tactic to
 all identifiers for which such an equation exists.

%progress
\addTactic{progress}{}
Split all you hypothesis, make all subsitution possible and then split your goal and do it again on all subgoals. It
permits to easly break a big judgements in smaller one.

%trivial
\addTactic{trivial}{}
If what remains in yout goal is very simple and not need an smt you can try using this tactic. TODO



%%%%%%%%%%%%%%%%%%%%%%%%%%%%%%%%%
% OTHER
%%%%%%%%%%%%%%%%%%%%%%%%%%%%%%%%%

\subsection{Other}

%idtac
\addTactic{idtac}{\tacarg{x}{string}?}
The identity tactic, leaving the goal unchanged and printing the string argument, if any.



\section{Common Hoare tactics}
The tactics mentioned so far permit to build proofs of purely logical statements
ranging over operators. The tactics introduced in the following sections allow
the user to write and prove judgments on module functions. Many of them
appear in some form or another in the various fragments of \EasyCrypt's logic,
with varying interpretations. This section discusses those program tactics whose
interpretation is similar in all logical fragments.

\subsection{Trivial judgements: the \rawec{exfalso} tactic}

\subsubsection*{The \rawec{exfalso} tactic}
\index{commonhl}{\rawec{exfalso}}

The \rawec{exfalso} tactic discharges a Hoare, Probabilistic Hoare or
Probabilistic relational Hoare judgement if the precondition can be
trivially proved false. The \rawec{trivial} tactic incorporates the
functionality of the \rawec{exfalso} tactic.


\subsubsection*{The \rawec{trivial} tactic}
\index{commonhl}{\rawec{trivial}}

Discharges trivial judgements by applying the \rawec{exfalso} and
\rawec{pr_bounded} tactics, removing Hoare judgements with
postconditions equivalent to true, modulo trivial reasoning
in the ambient logic.


\subsection{Probability expressions: the \rawec{pr_false, pr_or} tactics}
\index{commonhl}{\rawec{pr_false,pr_or}}

\begin{displaymath}
\infrule{
  \false \Rightarrow \post
}{
  \Prm{c}{m}{\post} = 0
}
\end{displaymath}

\begin{displaymath}
\infrule{
\Prm{c}{m}{\pre} \land
  \Prm{c}{m}{\post} \land \Prm{c}{m}{\pre \wedge \post} = \delta
}{
  \Prm{c}{m}{\pre \vee \post} = \delta
}
\end{displaymath}



\subsection{Program transformations}
%

\subsubsection*{The \rawec{inline} tactic}
\index{commonhl}{\rawec{inline}}
%

\subsubsection*{The \rawec{swap} tactic}
\index{commonhl}{\rawec{swap}}
%
\Syntax \rawec{swap} [\textit{side}] \textit{swap\_pos}

\textbf{where:} 
\begin{tabular}[t]{l}
  \textit{swap\_pos} ::= 
  \textit{n} \textit{n} \textit{n} $\mid$ \textit{n} \textit{z} $\mid$ [\textit{n}:\textit{n}] \textit{z}
  \\
  $n$ a natural number
  \\
  $z$ an integer number
\end{tabular}
  

The tactic [\rawec{swap} $p_1$ $p_2$ $p_3$] swaps the code between
positions $p_1$ and $p_2$ with the code between positions $p_2$ and
$p_3$. That is, assuming that $c_1$ and $c_2$ are syntactically
independent, that $c_1$ is between positions $p_1$ and $p_2$ and that
$c_2$ is between positions $p_2$ and $p_3$, the tactic implements the
following rule:
\begin{displaymath}
\infrule{
  \Hoare{c;c_2;c_1;c_3}{\pre}{\post}
}{
  \Hoare{c;c_1;c_2;c_3}{\pre}{\post}
} [\mathec{swap}\ p_1\ p_2\ p_3]
\end{displaymath}

If $k$ is positive (negative) then [\rawec{swap} $k$] moves the first
(last) instruction $k$ positions forwards (backwards). Similarly,
[\rawec{swap} $i$ $k$] moves the $i^{th}$ instruction forwards or
backwards, and [\rawec{swap} $[i_1:i_2]$ $k$] moves the instructions
between positions $i_1$ and $i_2$.

\subsubsection*{The \rawec{fun} tactic}
\index{commonhl}{\rawec{fun}}
\NotDocumented
\subsubsection*{The \rawec{unroll} tactic}
\index{commonhl}{\rawec{unroll}}
\NotDocumented
\subsubsection*{The \rawec{splitwhile} tactic}
\index{commonhl}{\rawec{splitwhile}}
\NotDocumented
\subsubsection*{The \rawec{fusion,fission} tactic}
\index{commonhl}{\rawec{fusion,fission}}
\NotDocumented
\subsubsection*{The \rawec{condt,condf} tactic}
\index{commonhl}{\rawec{condt,condf}}
\NotDocumented
\subsubsection*{The \rawec{kill} tactic}
\index{commonhl}{\rawec{kill}}
\NotDocumented


%%% Local Variables: 
%%% mode: latex
%%% TeX-master: "easycrypt"
%%% End: 


\section{Hoare Logic}
% --------------------------------------------------------------------
\begin{tactic}{hoare}
  \begin{tsyntax}[empty]{hoare}
  \fix{Missing description of hoare}.
  \end{tsyntax}
\end{tactic}


\section{Probabilistic Hoare Logic}


\subsection{Reasoning on the program structure}
\subsubsection*{Empty statements: the \rawec{skip} tactic}
\index{phl}{\rawec{skip}}

\Syntax \rawec{skip}

\Description Reduces logical program judgements with empty statements
to a higher-order logical goal. The generated subgoals can then be
processed using the ambient logic. The behaviour of the \rawec{skip}
tactic can be briefly described by the following rules:
%
\begin{displaymath}
\infrule{
  \pre \Rightarrow \post \qquad \bound = 1
}{
  \HoareLe{\Skip}{\pre}{\post}{\bound}
}
\end{displaymath}
%


\subsubsection*{Random samplings: the \rawec{rnd} tactic}
\index{phl}{\rawec{rnd}}

\Syntax 
\rawec{rnd} [\textit{formula} | \textit{formula} \textit{formula}
\textit{formula} \textit{formula} \textit{formula} \textit{formula} ] 

\Description

The invocation of this tactic expects a goal of any of the following
forms:
\begin{itemize}
\item $\HoareLe{\pre}{c;\Rand{x}{d}}{\post}{\bound}$
\item $\HoareEq{\pre}{c;\Rand{x}{d}}{\post}{\bound}$
\item $\HoareGe{\pre}{c;\Rand{x}{d}}{\post}{\bound}$
\end{itemize}

The following rule, corresponding to the invocation of
$
\left[\mbox{\rawec{rnd}}\ \varphi\ \bound_1\ \bound_2\ \bound_3\ \bound_4\ p\right],
$
describes the most general variant implemented by
the \rawec{rnd} tactic for probabilistic Hoare judgements:
%
\begin{displaymath}
\infrule{
  \begin{array}{c}
    \bound_1 \bound_2 + \bound_3 \bound_4 \leq \bound \\
    \HoareLe{c}{\pre}{\varphi}{\bound_1} \\
    \varphi \Rightarrow \mu\, d\, p \leq \bound_2 \land (\forall v,~\insupp{v}{d}\Rightarrow
    \post\subst{x}{v} \Rightarrow p\, v) \\
    \HoareLe{c}{\pre}{\neg \varphi}{\bound_3} \\
    \neg\varphi \Rightarrow \mu\, d\, p \leq \bound_4 \land (\forall v,~\insupp{v}{d}\Rightarrow
    \post\subst{x}{v} \Rightarrow p\, v) \\
  \end{array}
}{
  \HoareLe{c;\Rand{x}{d}}{\pre}{\post}{\bound}
}
\end{displaymath}
%
where $\insupp{z}{d}$ stands for the predicate $\ECinsupp{z}{d}$.
%
Similar rules hold by substituting every occurrence of the
comparison operator $\leq$ by the operator $=$, or by the operator $\geq$.

If a single parameter is given, then the \rawec{rnd} tactic implements
the following simpler rules:
\begin{displaymath}
  \infrule{
    \Hoare{c}{\pre}{\mu\, d\, p \leq f \land 
      (\forall v,~\insupp{v}{d}\Rightarrow \post\subst{x}{v} \Rightarrow p\, v)}
  }{
    \HoareLe{c;\Rand{x}{d}}{\pre}{\post}{f}
  }\left[\mathec{rnd}\ p\right]
\end{displaymath}
%
\begin{displaymath}
  \infrule{
    \HoareEq{c}{\pre}{\mu\, d\, p \geq \delta \land 
      (\forall v,~\insupp{v}{d} \Rightarrow p\, v \Rightarrow \post\subst{x}{v} )}{1} 
  }{
    \HoareGe{c;\Rand{x}{d}}{\pre}{\post}{\delta}
  }\left[\mathec{rnd}\ p\right]
\end{displaymath}
%
\begin{displaymath}
  \infrule{
    \HoareEq{c}{\pre}{\mu\, d\, p = \delta \land 
      \forall v,~ (\insupp{v}{d} \Rightarrow (p\, v \Leftrightarrow \post\subst{x}{v} ))}{1} 
  }{
    \HoareEq{c;\Rand{x}{d}}{\pre}{\post}{\delta}
  }\left[\mathec{rnd}\ p\right]
\end{displaymath}

If the \rawec{rnd} tactic is invoked with no parameters, and the
assigned variable does not occur in the postcondition, then the
following rule variants are applied:
%
\begin{displaymath}
  \infrule{
    \HoareLe{c}{\pre}{\post}{\bound}
  }{
    \HoareLe{c;\Rand{x}{d}}{\pre}{\post}{\bound}
  }\left[\mathec{rnd}\ p\right]
\end{displaymath}
%
\begin{displaymath}
  \infrule{
    \HoareEq{c}{\pre}{\post}{\bound} \qquad
    \mu\, d\, \mathec{cpTrue} = 1
  }{
    \HoareEq{c;\Rand{x}{d}}{\pre}{\post}{\delta}
  }\left[\mathec{rnd}\ p\right]
\end{displaymath}
%
\begin{displaymath}
  \infrule{
    \HoareGe{c}{\pre}{\post}{\bound} \qquad
    \mu\, d\, {cpTrue} = 1
  }{
    \HoareGe{c;\Rand{x}{d}}{\pre}{\post}{\bound}
  }\left[\mathec{rnd}\ p\right]
\end{displaymath}


If the judgement bound expression contains variables that are modified
by the judgement statement, then the same rules are applied after a
renaming of the bound expression enabled by the following equivalence: 
\begin{displaymath}
  \HoareLe{c;\Rand{x}{d}}{\pre}{\post}{\bound}
  \Longleftrightarrow
  \forall \bound', \HoareLe{c;\Rand{x}{d}}{\pre \land \bound=\bound'}{\post}{\bound'}
\end{displaymath}



\subsubsection*{Sequential composition: the \rawec{seq} tactic}
\index{phl}{\rawec{seq}}

\Syntax 
\rawec{seq}  \textit{codepos} \textit{formula}
[\textit{formula} | \textit{formula} \textit{formula}
\textit{formula} \textit{formula}]

\Description

\begin{displaymath}
  \infrule{
    \begin{array}{c}
      \Hoare{c_1}{\pre}{\varphi}
      \\
      \HoareLe{c_1}{\pre}{\chi}{\bound_1} \qquad 
          \HoareLe{c_2}{\varphi \land \chi }{\post}{\bound_2}
      \\
      \HoareLe{c_1}{\pre}{\neg \chi}{\bound_3} \qquad 
             \HoareLe{c_2}{\varphi \land \neg\chi}{\post}{\bound_4}
      \\
      \bound_1 \bound_2 + \bound_3 \bound_4 \leq \bound
    \end{array}
  }{
    \HoareLe{c_1;c_2}{\pre}{\post}{\bound}
  }\left[\mathec{seq}\ \varphi\ \chi\ f_1\ f_2\ f_3\ f_4 \right]
\end{displaymath}

(idem with the comparison operators ($=$) and ($\geq$)).



\subsubsection*{Conditional statements: the \rawec{condt,condf} tactic}
\index{phl}{\rawec{condt,condf}}
%
\NotDocumented


\subsubsection*{Conditional statements: the \rawec{if} tactic}
\index{phl}{\rawec{if}}
Applies the following rule for conditional statements. It expects a
conditional statement at the first program position.
\begin{displaymath}
\begin{array}{c}
  \infrule{
    \HoareLe{c_1;c}{\pre \land b}{\post}{f}\qquad
    \HoareLe{c_2;c}{\pre \land \neg b}{\post}{f}
  }{
    \HoareLe{\Cond{b}{c_1}{c_2};c}{\pre}{\post}{f}
  }\left[\mathec{if} \right] 
\\[4ex]
\end{array}
\end{displaymath}
Similar rules hold for the other comparison operators $=,\geq$.

\subsubsection*{Deterministic straight-line code: the \rawec{wp} tactic}
\index{phl}{\rawec{wp}}


\Syntax \rawec{wp} [\textit{codepos}]

\Description The \rawec{wp} tactic computes the weakest-precondition of
deterministic, loop and procedure-call free program fragments
(i.e. deterministic assignments and conditionals).   
The computation of the weakest precondition over a
conditional statement is only possible if its branches do not
contain random samplings, while-loops nor function calls.

The optional code position parameter \textit{pos} restricts the range
of instructions that may be affected by the tactic invocation. 
%
If the optional code position parameter is not provided, The tactic
processes instructions bottom-up until a random sampling, a loop or a
function call is reached.

\begin{displaymath}
  \infrule{
    \HoareLe{c_1}{\pre }{\mathsf{wp}(c_2,\post)}{f}
  }{
    \HoareLe{c_1;c_2}{\pre}{\post}{f}
  }\left[\mathec{wp} \right] 
\end{displaymath}
Similar rules hold for $=,\geq$.

\subsubsection*{Function call statements: the \rawec{call} tactic}
\index{phl}{\rawec{call}}

\Syntax \mathec{call} \textit{formula} \textit{formula} [\textit{formula}]

\Description

Let $p$ stand for the formal parameters of function $f$, $\result_f$
the result variable of function $f$, and $\vec{m}$ the set of
variables modifiable by $f$.
\begin{displaymath}
  \infrule{
    \begin{array}{c}
      \Hoare{c}{\pre}{\pre_f\subst{\vec{p}}{\vec{y}} \land
        \forall v.~ \forall \vec{z}.~ 
        \post_f\subst{\result_f}{v}\subst{\vec{m}}{\vec{z}}
        \Rightarrow \post\subst{x}{v}\subst{\vec{m}}{\vec{z}}
      }
      \\[.5ex]
      \HoareLe{f}{\pre_f}{\post_f}{\delta}
    \end{array}
  }{
    \HoareLe{c;\Call{x}{f}{\vec{y}}}{\pre}{\post}{\delta}
  } \left[\mathec{call}~ \pre_f~ \post_f \right]
\end{displaymath}

\begin{displaymath}
  \infrule{
    \begin{array}{c}
      \HoareEq{c}{\pre}{\pre_f\subst{\vec{p}}{\vec{y}} \land
        \forall v.~ \forall \vec{z}.~ 
        \post_f\subst{\result_f}{v}\subst{\vec{m}}{\vec{z}}
        \Rightarrow \post\subst{x}{v}\subst{\vec{m}}{\vec{z}}}{1}
    \\[.5ex]
    \HoareEq{f}{\pre_f}{\post_f}{\delta}
  \end{array}
  }{
    \HoareEq{c;\Call{x}{f}{\vec{y}}}{\pre}{\post}{\delta}
  } \left[\mathec{call}~ \pre_f~ \post_f \right]
\end{displaymath}

\begin{displaymath}
  \infrule{
    \begin{array}{c}
      \HoareEq{c}{\pre}{\pre_f\subst{\vec{p}}{\vec{y}} \land
        \forall v.~ \forall \vec{z}.~ 
        \post_f\subst{\result_f}{v}\subst{\vec{m}}{\vec{z}}
        \Rightarrow \post\subst{x}{v}\subst{\vec{m}}{\vec{z}}}
      {1}
    \\[.5ex]
    \HoareGe{f}{\pre_f}{\post_f}{\delta}
  \end{array}
  }{
    \HoareGe{c;\Call{x}{f}{\vec{y}}}{\pre}{\post}{\delta}
  } \left[\mathec{call}~ \pre_f~ \post_f \right]
\end{displaymath}


\subsubsection*{While loop statements: the \rawec{while} tactic}
\index{phl}{\rawec{while}}

\Syntax \rawec{while} \textit{formula} \textit{formula} 
%

\Description
%
The argument is the loop invariant, and the second one is a variant
expression used to prove termination. 
%
$M$ stands for the variables that may be modified by $c$.

\begin{displaymath}
  \infrule{
    \begin{array}{c}
    \HoareLe{c'}{\pre }{\chi \land 
      \forall M.~ (\chi \land 0 \leq e \Rightarrow \neg b)  \land
      \chi \land \neg b \Rightarrow \post}{f} 
    \\[.5ex]
    \forall k.~ \HoareEq{c}{\chi \land b \land e = k}{\chi \land e
      < k}{1}
  \end{array}
}{
    \HoareLe{c';\While{b}{c}}{\pre}{\post}{f}
  }\left[\mathec{while}\ \chi\ e \right] 
\end{displaymath}
Similarly for the comparison operators $=$ and $\leq$.


\subsection{Abstract adversaries: the \rawec{fun} tactic}
\index{phl}{\rawec{fun}}


\begin{displaymath}
\infrule{
    \pre \Rightarrow \chi  \qquad \chi \Leftrightarrow\post \qquad
    \HoareEq{O_i}{\chi}{\chi}{1}
}{
  \HoareEq{A}{\pre}{\post}{1}
} [\mathec{fun}~\chi]
\end{displaymath}


\subsection{Trivial judgements: the \rawec{pr_bounded} tactic}
\index{phl}{\rawec{pr_bounded}}

The \rawec{pr_bounded} tactic is only applicable to probabilistic
Hoare judgements. It discharges proof goals by trivial probability axioms:
\begin{displaymath}
\begin{array}{cc}
\infrule{
}{
  \HoareLe{c}{\pre}{\post}{1}
}
&
\infrule{
}{
  \HoareGe{c}{\pre}{\post}{0}
}
\end{array}
\end{displaymath}


\subsection{Strengthening goals}

\subsubsection*{The \rawec{conseq} tactic}
\index{phl}{\rawec{conseq}}


\begin{displaymath}
\infrule{
  \HoareLe{c}{\pre'}{\post'}{\delta} \qquad \pre\Rightarrow\pre' \qquad  \post\Rightarrow\post'
}{
  \HoareLe{c}{\pre}{\post}{\delta}
}\left[\mathec{conseq}~ (\_ : \pre'~ \mathec{==>}~ \post') \right]
\end{displaymath}

\begin{displaymath}
\infrule{
  \HoareEq{c}{\pre'}{\post'}{\delta} \qquad \pre\Rightarrow\pre' \qquad  \post\Leftrightarrow\post'
}{
  \HoareEq{c}{\pre}{\post}{\delta}
}\left[\mathec{conseq}~ (\_ : \pre'~ \mathec{==>}~ \post') \right]
\end{displaymath}

\begin{displaymath}
\infrule{
  \HoareGe{c}{\pre'}{\post'}{\delta} \qquad \pre\Rightarrow\pre' \qquad  \post'\Rightarrow\post
}{
  \HoareGe{c}{\pre}{\post}{\delta}
}\left[\mathec{conseq}~ (\_ : \pre'~ \mathec{==>}~ \post') \right]
\end{displaymath}


One can additionally strengthen the bounding condition with the syntax:

\rawec{conseq} (\_ : \_ \rawec{==>} \_ : [\textit{cmp}]
\textit{formula} ) 

that implements strengthening rules like the following:

\begin{displaymath}
\infrule{
  \HoareLe{c}{\pre}{\post}{\delta'} \qquad \delta'\leq\delta 
}{
  \HoareLe{c}{\pre}{\post}{\delta}
}\left[\mathec{conseq}~ (\_ : \_ ~\mathec{==>}~ \_) : \delta' \right]
\end{displaymath}

\begin{displaymath}
\infrule{
  \HoareEq{c}{\pre}{\post}{\delta'} \qquad \delta\leq\delta' 
}{
  \HoareGe{c}{\pre}{\post}{\delta}
}\left[\mathec{conseq}~ (\_ : \_ ~\mathec{==>}~ \_) : = \delta' \right]
\end{displaymath}



\subsubsection*{The \rawec{bd\_eq}}
\index{phl}{\rawec{bd_eq}}
%
\Syntax \rawec{bd_eq}
\begin{displaymath}
\begin{array}{cc}
\infrule{
  \HoareEq{c}{\pre}{\post}{f}
}{
  \HoareLe{c}{\pre}{\post}{f}
}
&
\infrule{
  \HoareEq{c}{\pre}{\post}{f}
}{
  \HoareGe{c}{\pre}{\post}{f}
}
\end{array}
\end{displaymath}


\subsection{Possibilistic or probabilistic: the \rawec{hoare,hoare\_bd} tactic}
\index{phl}{\rawec{hoare,hoare_db}}

\Syntax \mathec{hoare}, \mathec{hoare_bd}
allows to switch between possibilistic and probabilistic logics
according to these rules:
\begin{displaymath}
\infrule{
  \Hoare{c}{\pre}{\neg \post} \quad f = 0
}{
  \HoareEq{c}{\pre}{\post}{f}
}
\end{displaymath}

\subsection{Probability expressions: \rawec{deno} tactics}

\index{phl}{\rawec{deno,hoare_deno}}

\begin{displaymath}
\infrule{
    \pre 
    \qquad 
    \chi\Rightarrow\post 
    \qquad 
    \HoareLe{f}{\pre}{\post}{\delta}
}{
  \Prm{c}{m}{\chi} \leq \delta
}\left[\mathec{hoare_deno}\ \pre\ \post\right]
\end{displaymath}

\begin{displaymath}
\infrule{
    \pre 
    \qquad 
    \post\Leftrightarrow \chi 
    \qquad 
    \HoareEq{f}{\pre}{\post}{\delta}
}{
  \Prm{c}{m}{\chi} = \delta
}\left[\mathec{hoare_deno}\ \pre\ \post\right]
\end{displaymath}

\begin{displaymath}
\infrule{
    \pre 
    \qquad 
    \post\Rightarrow\chi
    \qquad 
    \HoareGe{f}{\pre}{\post}{\delta}
}{
  \delta \leq \Prm{c}{m}{\chi}
}\left[\mathec{hoare_deno}\ \pre\ \post\right]
\end{displaymath}



\subsection{Failure events: the \rawec{fel} tactic}
\index{phl}{\rawec{fel}}
%
The following rule describes the application of the tactic
$\mathec{fel}\ k\ q\ c\ \delta\ F\ P$.  Assume $f$ is defined and
$c_1,c_2$ stands for the splitting of its body at position $n$. Let
$\left\{O_i\right\}_{i=0}^k$ stand for all oracles accessed by any
adversary called at $c_2$. Assume that variables in $F$ can at most be
modified by $\left\{O_i\right\}_{i=0}^k$.
 
\begin{displaymath}
\infrule{
  \begin{array}{c}
    \left\{
    \begin{array}{l}
      \HoareLe{O_i}{\neg F}{F}{h(c)} \\
      \forall c_0,\ \Hoare{O_i}{P\land c=c_0}{c_0 < c} \\
      \forall c_0,\ \forall f_0,\ \Hoare{O_i}{\neg P\land F=f_0 \land c=c_0}{F=f_0 \land c=c_0} \\
    \end{array}\right\}_{i=0}^k\\[5ex]
    \forall m', (\varphi \Rightarrow F \land c\leq q) 
    \qquad 
    \sum_{i=0}^{q-1} h(i) \leq \epsilon 
    \qquad
    \Hoare{c_1}{\true}{\neg F \land c=0}
  \end{array}
}{
  \Prm{f}{m}{\varphi} \leq \epsilon  
} \left[\mathec{fel}\ n\ q\ h\ c\ F\ P\right]
\end{displaymath}


%%% Local Variables: 
%%% mode: latex
%%% TeX-master: "easycrypt"
%%% End: 


\section{Probabilistic Relational Hoare Logic}


\subsection{Reasoning on the program structure}
\subsubsection*{Empty statements: the \rawec{skip} tactic}
\index{prhl}{\rawec{skip}}

\Syntax \rawec{skip}

\Description Reduces logical program judgements with empty statements
to a higher-order logical goal. The generated subgoals can then be
processed using the ambient logic. The behaviour of the \rawec{skip}
tactic can be briefly described by the following rule:
%
\begin{displaymath}
\infrule{
  \pre \Rightarrow \post
}{
  \Equiv{\Skip}{\Skip}{\pre}{\post}
}
\end{displaymath}
%


\subsubsection*{Random samplings: the \rawec{rnd} tactic}
\index{prhl}{\rawec{rnd}}

\Syntax \rawec{rnd}[\textit{side}] [\textit{form} | \textit{form} \textit{form}]

\Description

The logical rule implemented by the \rawec{rnd} tactic depends on the
the optional parameter \textit{side}. If a left/right side flag is
provided then the one-sided logical rule for random sampling is
applied. If missing, then the two-sided rule for random assignment is
considered.
%

\paragraph*{Two-sided application.} 
When no side flag is provided, then the \rawec{rnd} tactic takes as
parameter a representation of a bijective function.

When two formulae are provided as the bijection parameter,
they are verified to be bijective functions and inverse of each
other. If only one function is provided then this function is verified
to be an involution, and lastly if no argument is given then the
identity function is considered.

The description of the rule below assumes that a bijective function
$f$ and its inverse is provided and generates according verification
conditions. Furthermore, it requires the following type constraints
for some types \rawec{'a} and \rawec{'b}: 
\begin{itemize}
\item $d_1$ \rawec{: 'a Distr.distr}, 
\item $d_2$ \rawec{: 'b Distr.distr}
\item $f$ \rawec{: 'a} $\to$ \rawec{'b},
\item $g$ \rawec{: 'b} $\to$ \rawec{'a}, 
\end{itemize}

\begin{displaymath}
\infrule{
  \Equiv{c_1}{c_2}{\pre} 
  { % \begin{array}{l}
      \forall z\,v,\insupp{z}{d_1}\Rightarrow
      \insupp{v}{d_2}\Rightarrow \mathsf{bij}\,f\,g
       \land
      \post\subst{x}{z}\subst{y}{f\,z}
    % \end{array}
  }
}{
  \Equiv{c_1;\Rand{x}{d_1}}{c_2;\Rand{y}{d_2}}{\pre}{\post}
}\left[\mathec{rnd}\ f\ g\right]
\end{displaymath}
where
$\mathsf{bij}\,f\,g$ stands for
\begin{displaymath}
  \mu\,d_1\,\charfun_{\{z\}}=\mu\,d_2\,\charfun_{\{f\,z\}} 
  \land \insupp{(g\,v)} {d_1} \land
  g\,(f\,z)=z \land f\,(g\,v)=v
\end{displaymath}
    

\paragraph*{One-sided application.} 
If the \emph{side} optional argument is provided, then the \rawec{rnd}
tactic implements the following rule:
%
\begin{displaymath}
\infrule{
  \Equiv{c}{c'}{\pre}{\mathsf{weight}\ d=1 \land  \forall z,\insupp{z}{d} \Rightarrow \post\subst{x}{z}}
}{
  \Equiv{c;\Rand{x}{d}}{c'}{\pre}{\post}
}\left[\mathec{rnd}\{1\}\right]
\end{displaymath}
where $\mathsf{weight}\ d$ stands for the expression
$\mathec{Distr.weight}\ d$ defined as $\mathec{mu}\ d\ \mathec{cpTrue}$.
%
A similar rule holds for an invocation to $\left[\mathec{rnd}\{2\}\right]$.



\subsubsection*{Sequential composition: the \rawec{seq} tactic}
\index{prhl}{\rawec{seq}}


\Syntax
\rawec{seq} \textit{codepos} \textit{form}

\Description
Applies the RHL rule for sequential composition:
$$
\infrule{\Equiv{c_1}{c_2}{\post}{\post'} \quad
         \Equiv{c_1'}{c_2'}{\post'}{\post''}}
        {\Equiv{c_1;c_1'}{c_2;c_2'}{\post}{\post''}}[\textrm{R-Seq}]
$$
The application of tactic $\mathec{app}\ m\ n\ p$ defines $c_1$ as the first
$m$ instructions of the program on the left-hand side and $c_2$ as
the first $n$ instructions of the program on the right-hand side
and $\post'$ as $p$.







\subsubsection*{Conditional statements: the \rawec{condt,condf} tactic}
\index{prhl}{\rawec{condt,condf}}
%
\NotDocumented


\subsubsection*{Conditional statements: the \rawec{if} tactic}
\index{prhl}{\rawec{if}}

\Syntax \rawec{if} [\textit{side}]

\Description Applies the pRHL rule for conditional.
If the \textit{side} argument is given then the corresponding
one-side rule is used, else the two side rule is used.
The \rawec{if} tactic expects a conditional as first instruction. 
\begin{center}
\begin{tabular}{c|c}
Syntax & Rule \\
\hline\\
\mathec{if\{1\}} &
$
\infrule{\Equiv{c_1;c}{c'}{\pre \land e\sidel}{\post}
        \quad \Equiv{c_2;c}{c'}{\pre \land \neg e\sidel}{\post}}
        {\Equiv{\Cond{e}{c_1}{c_2};c}{c'}{\pre}{\post}}
$\\
\\\hline\\
\mathec{if\{2\}} &
$
\infrule{\Equiv{c'}{c_1;c}{\pre \land e\sider}{\post}
        \quad \Equiv{c'}{c_2;c}{\pre \land \neg e\sider}{\post}}
        {\Equiv{c'}{\Cond{e}{c_1}{c_2};c}{\pre}{\post}}
$\\
\\\hline\\
\mathec{if} &
$
\infrule{
 \begin{array}{c}
   \vdash \pre \Rightarrow e\sidel = e'\sider \\
   \Equiv{c_1;c}{c'_1;c'}{\pre \land e\sidel \land e'\sider}{\post}\\
   \Equiv{c_2;c}{c'_2;c'}{\pre \land \neg e\sidel \land \neg e'\sider}{\post}
 \end{array}
}{\Equiv{\Cond{e}{c_1}{c_2};c}
        {\Cond{e'}{c'_1}{c'_2};c'}
        {\pre}{\post}}
$\\
\end{tabular}
\end{center}





\subsubsection*{Deterministic straight-line code: the \rawec{wp} tactic}
\index{prhl}{\rawec{wp}}


\Syntax \rawec{wp} [\textit{codepos}]

\Description The \rawec{wp} tactic computes the weakest-precondition of
deterministic, loop and procedure-call free program fragments
(i.e. deterministic assignments and conditionals).   
The computation of the weakest precondition over a
conditional statement is only possible if its branches do not
contain random samplings, while-loops nor function calls.

The optional code position parameter \textit{pos} restricts the range
of instructions that may be affected by the tactic invocation. 
%
If the optional code position parameter is not provided, The tactic
processes instructions bottom-up until a random sampling, a loop or a
function call is reached.


\subsubsection*{Function call statements: the \rawec{call} tactic}
\index{prhl}{\rawec{call}}



\subsubsection*{While loop statements: the \rawec{while} tactic}
\index{prhl}{\rawec{while}}

\Syntax  \rawec{while} [\textit{side}] \textit{form} [\textit{form}]

\Description This tactic applies the pRHL verification rules for
loops:
\begin{itemize}
\item the optional argument \textit{side} can be either \rawec{{1}} or
  \rawec{{2}} to indicate the application of one-sided versions of the
  rule. If missing, the two-sided rule for loops is considered.
\item the first \textit{form} argument is mandatory and is used as
  loop invariant. It can refer to variables in both the left and right
  programs.
\item the optional parameter \textit{form} is required (and accepted
  only) in the one-sided application of the rule. This parameter
  corresponds to the decreasing variant expression used to prove loop
  termination.
\end{itemize}



\paragraph{Two-sided version.}
%
\Syntax \rawec{while} \textit{form} 
%
\Description Applies the two-sided RHL rule for while loops, using the
\textit{form} parameter as loop invariant. This tactic requires that
the last instruction of both left and right statements are while loops.
In the rule, $M$ refers to the variables that may be modified by the
loop bodies.

\begin{displaymath}
\infrule{ 
  \begin{array}{c}
    \Equiv{c_2}{c'_2}{I \land e\sidel \land e'\sider}{I \land  e\sidel = e'\sider}\\
    \Equiv{c_1}{c'_1}{\pre}{ I \land e\sidel = e'\sider \land 
      \forall M, (I \land \neg e\sidel \land \neg e'\sider \Rightarrow \post)}
  \end{array}
}{
  \Equiv{c_1;\While{e}{c_2}}{c'_1;\While{e'}{c'_1}}{\pre}{\post}
}
\end{displaymath}

\paragraph{One-sided version.}

\Syntax \rawec{while} \textit{side} \textit{form} \textit{form} 

\Description Applies the one-sided pRHL rule for while loops, using
the first parameter \textit{form} as loop invariant and the second
parameter \textit{form} as a decreasing \textit{variant}
expression. The variant is used to verify the loop termination. The
one-sided rule are described below. Only the left (i.e., flagged with
\rawec{\{1\}}) variant is shown; the right (i.e., flagged with
\rawec{\{2\}}) variant is symmetric. The expressions $\forall
X,~\varphi$ and $\exists X,~\varphi$ denote, respectively, universal
and existential quantification over the set of variables $X$ modified
in the loop body $c$. 

\begin{displaymath}
\infrule{
  \begin{array}{c}
    \vdash I \land v \leq b \Rightarrow \neg e  \\
    \Equiv{c}{\Skip}{b=B \land v=C \land e \land I }{b=B \land v<C \land I} \\
    \Equiv{c_1}{c_2}{\pre}{I \land \forall X, (I \land \neg e
      \Rightarrow \post)}
  \end{array}
}{
  \Equiv{c_1;\While{e}{c}}{c_2}{\pre}{\post}
}
\end{displaymath}


\subsection{Abstract adversaries: the \rawec{fun} tactic}
\index{prhl}{\rawec{fun}}

\paragraph*{ }
The formula given as parameter represents the general oracle
invariant. 
%
The tactic implements the following rule:
\begin{displaymath}
\infrule{
  \begin{array}{c}
    \pre \Rightarrow \chi \land \glob_A = \glob_B \land \vec{p}_A=\vec{p}_B
    \\[.5ex]
    \chi\land\glob_A=\glob_B\land\result_A=\result_B\Rightarrow\post
    \\ 
    \Equiv{O_i}{O_i'}{\chi\land
      \vec{p}_{O_i}=\vec{p}_{O'_i}}{\chi\land \result_{o_i}=\result_{o'_i}}
  \end{array}
}{
  \Equiv{A}{B}{\pre}{\post}
} [\mathec{fun}~\chi]
\end{displaymath}
%
where $\vec{p}_f$ represent the formal parameters of a function
(abstract adversary or oracle) $f$, $\result_f$ represents the result of
a function (abstract adversary or oracle) $f$, $\left\{O_i\right\}_{i=0}^k$ and
$\left\{O'_i\right\}_{i=0}^k$ are the oracles of the abstract adversaries $A$ and
$B$, $\glob_A$ and $\glob_B$ represent the global state of the abstract
adversaries $A$ and $B$.

\subsubsection{Verifying invariants upto bad}
\NotDocumented

\subsection{Strengthening goals}
\index{prhl}{\rawec{conseq}}

\subsubsection*{The \rawec{conseq} tactic}

\subsubsection*{The \rawec{conseq} tactic}
\index{prhl}{\rawec{conseq}}

\Syntax \rawec{conseq} (\_ : \textit{formula} \rawec{==>} \textit{formula} )

\begin{displaymath}
\infrule{
  \Equiv{c_1}{c_2}{\pre'}{\post'} \qquad \pre\Rightarrow\pre' \qquad  \post'\Rightarrow\post
}{
  \Equiv{c_1}{c_2}{\pre}{\post}
}\left[\mathec{conseq}~ ( \_ : \pre'~\mathec{==>}~ \post' \right]
\end{displaymath}


\subsubsection*{The \rawec{bypr} tactic}
\index{prhl}{\rawec{bypr}}
%

\begin{displaymath}
\infrule{
  \forall m_1,\ \forall m_2,\ 
  \Prm{f_1}{m_1}{\varphi_1} = \Prm{f_2}{m_2}{\varphi_2}
}{
  \Equiv{f_1}{f_2}{\pre}{\post}
}
\end{displaymath}









\subsection{Probability expressions: \rawec{deno} tactics}

\index{prhl}{\rawec{deno}}

\begin{displaymath}
\infrule{
  \Equiv{c_1}{c_2}{\pre}{\post} 
  \qquad
  \pre
  \qquad
  \post \Rightarrow \chi_1 \Rightarrow \chi_2
}{
  \Prm{c_1}{m_1}{\chi_1} \leq \Prm{c_2}{m_2}{\chi_2}
}\left[\mathec{deno}\ \pre\ \post\right]
\end{displaymath}

\begin{displaymath}
\infrule{
  \Equiv{c_1}{c_2}{\pre}{\post} 
  \qquad
  \pre
  \qquad
  \post \Rightarrow (\chi_1 \Leftrightarrow \chi_2)
}{
  \Prm{c_1}{m_1}{\chi_1} = \Prm{c_2}{m_2}{\chi_2}
}\left[\mathec{deno}\ \pre\ \post\right]
\end{displaymath}



%%% Local Variables: 
%%% mode: latex
%%% TeX-master: "easycrypt"
%%% End: 



%%% Local Variables: 
%%% mode: latex
%%% TeX-master: "easycrypt"
%%% End: 
