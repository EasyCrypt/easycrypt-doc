% --------------------------------------------------------------------
\usepackage{amsmath}

\newcommand{\rel}[1]{\mathrel{#1}}

% --------------------------------------------------------------------
\newenvironment{tightcenter}{%
  \setlength\topsep{0pt}
  \setlength\parskip{0pt}
  \begin{center}}
{\end{center}}

% --------------------------------------------------------------------
% Acronyms, names, ...

\usepackage{xspace}

\def\EasyCrypt{\textsc{EasyCrypt}\xspace}
\def\prhl{\textsc{pRHL}\xspace}
\def\phl{\textsc{pHL}\xspace}
\def\hl{\textsc{HL}\xspace}
\newcommand{\WhyThree}{\textsf{Why3}\xspace}

% --------------------------------------------------------------------
% Inference rules
\usepackage{mathpartir}

\newenvironment{cmathpar}
{\begin{tightcenter}\begin{mathpar}}
{\end{mathpar}\end{tightcenter}}

% --------------------------------------------------------------------
% EasyCrypt listings

\usepackage{textcomp}

\usepackage[final]{listings}

\newcommand{\ensuretext}[1]{\ensuremath{\text{#1}}}

\lstdefinestyle{mylistings}{
  upquote=true,
  escapechar=\#,
  mathescape=false,
  columns=fullflexible,
  keepspaces=true,
  captionpos=b,
  frame=tb,
  xleftmargin=.1\textwidth,
  xrightmargin=.1\textwidth,
  rangebeginprefix={(**\ begin\ },
  rangeendprefix={(**\ end\ },
  rangesuffix={\ *)},
  includerangemarker=false,
  basicstyle=\small\sffamily,
  identifierstyle={}
}

\lstdefinelanguage{mylistings}{
  style=mylistings,
}

\lstset{language=mylistings}

\documentclass[a4paper,notitlepage,oneside]{book}

\usepackage[utf8]{inputenc}
\usepackage[T1]{fontenc}
\usepackage{lmodern}

\usepackage{multicol}
\usepackage[authoryear,longnamesfirst,round]{natbib}
\usepackage{multind}
\usepackage{xspace}
\usepackage{mdframed}
\usepackage[procnames]{listings}
\usepackage{amssymb,dsfont,stmaryrd}
\usepackage{infer}
\usepackage[pagebackref,colorlinks=true,linkcolor=black,linktoc=all,citecolor=blue]{hyperref}
\usepackage[usenames,dvipsnames]{xcolor}
\usepackage{tabularx}

\DeclareMathVersion{sans}
\SetSymbolFont{operators}{sans}{OT1}{cmbr}{m}{n}
\SetSymbolFont{letters}{sans}{OML}{cmbrm}{m}{it}
\SetSymbolFont{symbols}{sans}{OMS}{cmbrs}{m}{n}
\SetMathAlphabet{\mathit}{sans}{OT1}{cmbr}{m}{sl}
\SetMathAlphabet{\mathbf}{sans}{OT1}{cmbr}{bx}{n}
\SetMathAlphabet{\mathtt}{sans}{OT1}{cmtl}{m}{n}
\SetSymbolFont{largesymbols}{sans}{OMX}{iwona}{m}{n}

\makeindex{easycrypt}
\makeindex{ambient}
\makeindex{hoare}
\makeindex{phl}
\makeindex{prhl}
\makeindex{commonhl}

% !TeX root = easycrypt.tex
%% Misc
\newcommand{\DONE}{}% {{\color{red}DONE}}
\newcommand{\Example}{\paragraph*{Example}}
\newcommand{\Syntax}{\paragraph*{Syntax}}
\newcommand{\Description}{\paragraph*{Description}}
\setcounter{secnumdepth}{3}
\renewcommand{\thesubsubsection}{\arabic{chapter}.\arabic{section}.\arabic{subsection}.\arabic{subsubsection}}
\newbox\minicodebox
\newenvironment{minicode}[1]{%
\minipage[t]{#1\linewidth} %
\centering %
\verbatim %
}{%
\endverbatim %
\endminipage% 
}


\newcounter{alarmcounter}
\setcounter{alarmcounter}{1}
\newcommand{\alarm}[1]
            {\begingroup
              \def\thefootnote{{\normalsize\color{red}(\arabic{footnote})}}
              \footnote{\textsf{\textbf{{\color{red}\sc \bf ALARM:}
                  #1}}}\endgroup}


\newcommand{\infr}[2]{
{\renewcommand{\arraystretch}{1.1}
\begin{array}{c}
{#1}\\
\hline
{#2}
\end{array}}}

% \providecommand{\eqref}[1]{\textup{(\ref{#1})}}
% \providecommand{\eqdef}{\raisebox{-.2ex}[.2ex]{$\stackrel{\textrm{\tiny def}}{~=~}$}}
\newcommand{\result}{\mathsf{res}}

%% Names

% \newcommand{\CertiCrypt}{\textsf{CertiCrypt}\xspace}
% \newcommand{\CertiPriv}{\textsf{CertiPriv}\xspace}
% \newcommand{\ProVerif}{\textsf{ProVerif}\xspace}
% \newcommand{\CryptoVerif}{\textsf{CryptoVerif}\xspace}
% \newcommand{\Coq}{\textsf{Coq}\xspace}
% \newcommand{\CIL}{\textsf{CIL}\xspace}
% \newcommand{\pWHILE}{\textsf{p}\textsc{While}\xspace}
% \newcommand{\Why}{\textsf{Why}\xspace}
\newcommand{\WhyThree}{\textsf{Why3}\xspace}

%% Security properties and schemes

\newcommand{\INDCPA}{\textsf{IND-CPA}\xspace}
\newcommand{\INDCCAone}{\textsf{IND-CCA1}\xspace}
\newcommand{\INDCCA}{\textsf{IND-CCA}\xspace}
\newcommand{\EFCMA}{\textsf{EF-CMA}\xspace}
\newcommand{\LCDH}{\ensuremath{\mathsf{LCDH}}\xspace}
\newcommand{\CDH}{\textsf{CDH}\xspace}
\newcommand{\DDH}{\textsf{DDH}\xspace}
\newcommand{\ElGamal}{\textsf{ElGamal}\xspace}
\newcommand{\HElGamal}{\textsf{HElGamal}\xspace}
\newcommand{\RSA}{\textsf{RSA}\xspace}
\newcommand{\OAEP}{\textsf{OAEP}\xspace}
\newcommand{\FDH}{\textsf{FDH}\xspace}
\newcommand{\HMAC}{\textsf{HMAC}\xspace}
\newcommand{\CS}{\textsf{CS}\xspace}
\newcommand{\Skein}{\textsf{Skein}\xspace}
\newcommand{\TCR}{\textsf{TCR}\xspace}

\newcommand{\KG}{\mathcal{KG}}
\newcommand{\Enc}{\mathcal{E}}
\newcommand{\Dec}{\mathcal{D}}

%% Complexity and termination

\newcommand{\lossless}{\mathsf{lossless}}

%% Sets

\newcommand{\zeroone}{[0,1]}
\newcommand{\bit}{\{0,1\}}
\newcommand{\bitstring}[1]{\ensuremath{\bit^{#1}}}
\newcommand{\bool}{\mathbb{B}}
\newcommand{\nat}{\mathbb{N}}
\newcommand{\real}{\mathbb{R}}
\newcommand{\option}[1]{#1_\bot}

%% Mathematics

\renewcommand{\Pr}[2]{\mathrm{Pr}\left[#1 : #2\right]}
\newcommand{\Prm}[3]{\mathrm{Pr}\left[#1,#3 : #2\right]}
\newcommand{\labs}{\left\lvert}
\newcommand{\rabs}{\right\rvert}
\newcommand{\charfun}{\mathds{1}}

%% Distribution monad

\newcommand{\supp}{\mathsf{support}}
\newcommand{\range}[2]{\mathsf{range}~{#1}~{#2}}

%% Semantics 

% \newcommand{\sem}[1]{\llbracket #1 \rrbracket}
\newcommand{\subst}[2]{\left[{}^{#2}/{}_{#1}\right]}
\newcommand{\fv}{\mathsf{fv}}
\newcommand{\modifies}{\mathsf{mod}}
\newcommand{\glob}{\mathsf{glob}}
\newcommand{\Env}{\mathcal{E}}

%% Well-formed adversaries


\newcommand{\EC}{\textsc{EasyCrypt}\xspace}
\newcommand{\ECversion}{0.$\beta$\xspace}

\newcommand{\pRHL}{\mathsf{pRHL}\xspace}
\newcommand{\rname}[1]{[\textsc{#1}]}

%% Program Judgements 
\newcommand{\Hoare}[3]{\left[{#1}:{#2}\Longrightarrow{#3}\right]}
\newcommand{\Equiv}[4]{\left[{#1}\sim{#2}:{#3}\Longrightarrow{#4}\right]}
\newcommand{\bdHoareS}[5]{{\Hoare{#1}{#2}{#3}}\,{#4}\,{#5}}
\newcommand{\HoareLe}[4]{\bdHoareS{#1}{#2}{#3}{\leq}{#4}}
\newcommand{\HoareEq}[4]{\bdHoareS{#1}{#2}{#3}{=}{#4}}
\newcommand{\HoareGe}[4]{\bdHoareS{#1}{#2}{#3}{\geq}{#4}}


% \newcommand{\Equiv}[4]{\models {#1} \sim {#2} : {#3} \Longrightarrow {#4}}
% \newcommand{\AEquiv}[6]{\models {#2} \sim_{#5,#6} {#3} : {#1} \Longrightarrow {#4}}
% \newcommand{\JAEquiv}[6]{{#2} \sim_{#5,#6} {#3} : {#1} \Longrightarrow {#4}}
% \newcommand{\EquivMem}[2]{\models {#1} \equiv {#2}}
% \newcommand{\EqObs}[4]{\models {#1} \simeq^{#3}_{#4} {#2}}
% \newcommand{\AEqObs}[5]{\models {#1} \simeq^{#3}_{{#4}} {#2} \preceq {#5}} 
% \newcommand{\ACEqObs}[7]
%         {\AEqObs{\left[ #1 \right]_{#6}}{\left[ #2 \right]_{#7}}{#3}{#4}{#5}}
% \newcommand{\Triple}[3]{\sem{#2} {#3} \preceq {#1}}
% \newcommand{\DTriple}[3]{\sem{#2} {#3} \succeq {#1}}
% \newcommand{\dequiv}[3]{{#1} \simeq_{#3} {#2}}
% \newcommand{\fequiv}[3]{{#1} =_{#3} {#2}}
% \newcommand{\Pre}{\Psi}
% \newcommand{\Post}{\Phi}
% \newcommand{\Inv}{\Phi}
\newcommand{\side}[1]{\langle #1 \rangle}
\newcommand{\sidel}{\side{1}}
\newcommand{\sider}{\side{2}}
% \newcommand{\eqobsin}{\mathsf{eqobs\_in}}
% \newcommand{\eqobsout}{\mathsf{eqobs\_out}}
\newcommand{\pre}{\Psi}
\newcommand{\post}{\Phi}

%% Variables

% \newcommand{\LH}{\gl{L}_H}
% \newcommand{\LD}{\gl{L}_\Dec}
% \newcommand{\cdef}{\gl{\gamma_\mathsf{def}}}
\newcommand{\var}[1]{\ensuremath{\mathit{#1}} \xspace}

%% Constants and operators
% TODO: Update!
\newcommand{\true}{\mathsf{true}}
\newcommand{\false}{\mathsf{false}}
\newcommand{\nil}{\mathsf{nil}}
\newcommand{\hd}{\mathsf{hd}}
\newcommand{\tl}{\mathsf{tl}}
\newcommand{\app}{\mathbin{+\mkern-7mu+}}
\newcommand{\concat}{\parallel}
\newcommand{\xor}{\oplus}
\newcommand{\msb}[2]{[#1]^{#2}}
\newcommand{\lsb}[2]{[#1]_{#2}}
\newcommand{\dom}{\mathsf{dom}}
\newcommand{\ran}{\mathsf{ran}}
\newcommand{\fst}{\mathsf{fst}}
\newcommand{\snd}{\mathsf{snd}}
\newcommand{\some}[1]{#1}
\newcommand{\none}{\bot}

\newcommand{\Int}{\mathsf{Int}}
\newcommand{\tint}{\mathsf{int}}

\newcommand{\tbool}{\mathsf{bool}}

%% Language
% TODO: Update!
\newcommand{\Skip}{\mathsf{skip}}
\newcommand{\Seq}[2]{#1;\ #2}
\newcommand{\Ass}[2]{#1 \leftarrow #2}
\newcommand{\Rand}[2]{#1 \stackrel{\raisebox{-.25ex}[.25ex]%
{\tiny $\mathdollar$}}{\raisebox{-.2ex}[.2ex]{$\leftarrow$}} #2}
\newcommand{\Randi}[2]{\Rand{#1}{[0..#2]}}
\newcommand{\Randb}[1]{\Rand{#1}{\bit}}
\newcommand{\Randbs}[2]{\Rand{#1}{\bitstring{#2}}}
\newcommand{\Cond}[3]{\mathsf{if}\ #1\ \mathsf{then}\ #2\ \mathsf{else}\ #3}
\newcommand{\Condt}[2]{\mathsf{if}\ #1\ \mathsf{then}\ #2}
\newcommand{\Else}{\mathsf{else}\ }
\newcommand{\Elsif}{\mathsf{elsif}\ }
\newcommand{\nWhile}[3]{\mathsf{while}_{#1}\ #2\ \mathsf{do}\ #3}
\newcommand{\While}[2]{\mathsf{while}\ #1\ \mathsf{do}\ #2}
\newcommand{\Call}[3]{#1 \leftarrow #2\mathsf{(}#3\mathsf{)}}
\newcommand{\Return}{\mathsf{return}}
% \newcommand{\Assert}[1]{\mathsf{assert}~#1}

%% Language definition
\lstnewenvironment{easycrypt}[2][]%
  {\lstset{language=easycrypt,caption=#2,#1}}%
  {}

\newcommand{\rawec}[2][]{\lstinline[language=easycrypt,#1]{#2}}
\newcommand{\ec}[2][]{\lstinline[language=easycrypt,style=easycrypt-pretty,#1]{#2}}

\newcommand{\ecimport}[4][]{\lstinputlisting[language=easycrypt,linerange=#4,caption=#2,#1]{#3}}

\def\createEasycrypt#1\relax{
\lstdefinelanguage{easycrypt}{
  style=easycrypt-default,
%  procnamekeys={op,pred,fun},
%  procnamestyle={\sffamily\itshape},
  keywordsprefix={'},
  morekeywords=[1]{unit,bool,int,real,bitstring,array,list,matrix,word},
  morekeywords=[2]{type,op,axiom,lemma,module,pred},
  morekeywords=[3]{var,fun},
  morekeywords=[4]{while,if},
  morekeywords=[5]{theory,end},
  morekeywords=[6]{forall,exists,lambda},
  morekeywords=[7]{#1},
%  moredirectives={prover,print}, % Incomplete
  morecomment=[n][\itshape]{(*}{*)},
  morecomment=[n][\bfseries]{(**}{*)}
}
}

% !TeX root = easycrypt.tex

%%%%%%%%%%%%%%%%%%%%%%%%%%%%%%%%%
% DEFS
%%%%%%%%%%%%%%%%%%%%%%%%%%%%%%%%%

\newcommand{\ambientKeywords}{}

\newcommand{\tacname}{Error tacname}
\newcommand{\vtacname}{Error tacname}

\newcommand{\addTactic}[2]{
  \expandafter\def\expandafter\ambientKeywords\expandafter{\ambientKeywords,#1}
  \renewcommand{\tacname}{\rawec{#1}}
  \renewcommand{\vtacname}{#1}
  \index{ambient}{#1@\rawec{#1}}
  \subsubsection{#1}
  \Syntax \ec{#1} #2
  \Description%
}


\newcommand{\example}[6]%proof,context,goal
{
\vspace*{3ex}
\begin{tabular}{ccc}
\parbox{100pt}{#1} & {\expandafter\rawec\expandafter{#3 #4.}} & \parbox{100pt}{#5} \\
\cline{0-0} \cline{3-3} {\ec{#2}} & ~ & {\ec{#6}} \\
\end{tabular}\\
}

\newcommand{\env}[2]{\ec{#1 : #2}\\}

\newcommand{\vararg}[1]{\ec{#1}}
\newcommand{\cstarg}[1]{\ec{#1}}
\newcommand{\typarg}[1]{\textit{#1}}

\newcommand{\tacarg}[2]{(\vararg{#1}:\typarg{#2})}

\newcommand{\refdef}[1]{\emph{#1}(\ref{#1})}


%%%%%%%%%%%%%%%%%%%%%%%%%%%%%%%%%
% END DEFS
%%%%%%%%%%%%%%%%%%%%%%%%%%%%%%%%%

\subsection{Generalities}

\EasyCrypt ambient logic is based on non-dependent higher-order logic.

\subsection{Convertibility}\label{convertible}

\EasyCrypt ambient logic enjoys a mechanism that identifies all formulas
that are equal up to a given amount of computations.

\begin{center}
\begin{tabular}{l@{$\quad$}l@{$\quad$}ll}
{\rawec{(lambda (x : t), phi1)\ phi2}} & $\rightarrow_\beta$ &
  \multicolumn{2}{@{}l}{{\rawec{phi2} \{\rawec{x} $\leftarrow$ \rawec{phi1}\}}}\\
{\rawec{if (true) \{ phi1 \} else \{ phi2 \}}} & $\rightarrow_\iota$ &
  {\rawec{phi1}}\\
{\rawec{if (false) \{ phi1 \} else \{ phi2 \}}} & $\rightarrow_\iota$ &
  \multicolumn{2}{@{}l}{{\rawec{phi2}}}\\
{\rawec{let (x1, ..., xn) = (phi1, ..., phin) in phi}} & $\rightarrow_\iota$ &
  \multicolumn{2}{@{}l}{{\rawec{phi} \{ \rawec{x1, ..., xn} $\leftarrow$ \rawec{phi1, ..., phin} \}}}\\
{\rawec{let x = phi1 in phi2}} & $\rightarrow_\zeta$ &
  \multicolumn{2}{@{}l}{{\rawec{phi2} \{ \rawec{x} $\leftarrow$ \rawec{phi1} \}}}\\
{\rawec{o}} & $\rightarrow_\delta^{\Env,\Gamma}$ &
  {\rawec{e}} & if {\rawec{op o := e}} $\in \Env$\\
{\rawec{x}} & $\rightarrow_\delta^{\Env,\Gamma}$ &
  {\rawec{phi}} & if {\rawec{x := phi}} $\in \Gamma$\\
\end{tabular}
\end{center}


%%%%%%%%%%%%%%%%%%%%%%%%%%%%%%%%%
% LAMBDA
%%%%%%%%%%%%%%%%%%%%%%%%%%%%%%%%%

\subsection{Lambda}

%change
\addTactic{change}{\tacarg{f}{formula}}
Change the current goal to the $\leftrightarrow^*$-equivalent one \ec{f}
\begin{displaymath}
  \infrule{\phi_1 \leftrightarrow^*_{\Env;\Gamma} \phi_2 \quad
           \Env; \Gamma \vdash \phi_1}
          {\Env; \Gamma \vdash \phi_2}
\end{displaymath}

%beta
\addTactic{beta}{}
Change the goal with its $\beta$-head normal-form.

%iota
\addTactic{iota}{}
Change the goal with its $\iota$-head normal-form.

%zeta
\addTactic{zeta}{}
Change the goal with its $\zeta$-head normal-form.

%logic
\addTactic{logic}{}
Change the goal with its $\Lambda$-head normal-form.

%delta
\addTactic{delta}{\tacarg{names}{ident*}}
Do one step of parallel, strong $\delta$-reduction, restricted to
 the symbols designed by \ec{names}. If \ec{names} if empty, no restriction
 on the $\delta$-reduction is applied.

%simplify
\addTactic{simplify}{\tacarg{names}{ident*} | \var{delta}$\!\!$?}
Change the goal with its $\beta\iota\zeta\Lambda$-head normal-form, followed
 by one step of parallel, strong $\delta$-reduction if \ec{delta} is given.
 The $\delta$-reduction can be restricted to a set of defined symbols by
 replacing \ec{delta} by the non-empty sequence of targeted symbols.

%congr
\addTactic{congr}{}
This tactic applies to a goal of the form \ec{f t1 ... tn = f u1 ... un}
 replacing it by  the subgoals \ec{ti = ui} for all \ec{i}. Note that subgoals
 solvable by \ec{reflexivity} are automatically closed.

%generalize
\addTactic{generalize}{\tacarg{p}{pattern}}
Search for the first subterm of the goal matching \ec{p} and leading
to the full instantiation of the pattern. Then, do a logical
generalization of all the occurrences of \ec{p}, after instantiation,
in the goal.
\begin{displaymath}
  \infrule{\Env; \Gamma \vdash p \quad
           \Env; \Gamma \vdash \forall x, \phi(x)}
          {\Env; \Gamma \vdash \phi(p)}
\end{displaymath}

%pose
\addTactic{pose}{\tacarg{x}{ident} \rawec{:=} \tacarg{p}{pattern}}
Search for the first subterm of the goal matching \ec{p} and leading
to the full instantiation of the pattern. Then, introduce, after
instantiation, the local definition \rawec{x := p} and abstract
all the occurrences of \ec{p} in the goal by \ec{x}
\begin{displaymath}
  \infrule{\Env; \Gamma \vdash p \quad
           \Env; \Gamma, x := p \vdash \phi(x)}
          {\Env; \Gamma \vdash \phi(p)}
\end{displaymath}

%%%%%%%%%%%%%%%%%%%%%%%%%%%%%%%%%
% LOGIC
%%%%%%%%%%%%%%%%%%%%%%%%%%%%%%%%%

\subsection{Logic}

%split
\addTactic{split}{}
\tacname{} breaks a goal that is intrinsically conjunctive into multiple subgoals.
 For instance, it
 \begin{itemize}
  \item closes any goal that is \refdef{convertible} to \ec{true} or provable
        by \ec{reflexivity},

  \begin{displaymath}
  \infrule{\Env; \Gamma \vdash a \equiv true}{a}
  ~~~~~~
  \infrule{\Env; \Gamma \vdash a \equiv b}{a = b}
  \end{displaymath}
       
  \item replaces a logical equivalence by the direct and indirect implication,

  \begin{displaymath}
  \infrule{\Env; \Gamma \vdash \phi_1 \Rightarrow \phi_2 \quad
           \Env; \Gamma \vdash \phi_2 \Rightarrow \phi_1}
          {\Gamma \vdash \phi_1 \Leftrightarrow \phi_2}
  \end{displaymath}
  
  \item replaces a goal of the form \rawec{f1 /\\ f2} or \rawec{f1 \&\& f2} by the two
        subgoals for \ec{f1} and \ec{f2},

  \begin{displaymath}
  \infrule{\Env; \Gamma \vdash \phi_1 \quad
           \Env; \Gamma \vdash \phi_2}
          {\Env; \Gamma \vdash \phi_1 \land \phi_2}
  ~~~~~~
  \infrule{\Env; \Gamma \vdash \phi_1 \quad
           \Env; \Gamma \vdash \phi_2}
          {\Env; \Gamma \vdash \phi_1 \&\& \phi_2}
  \end{displaymath}
        
  \item replaces an equality between two $n$-tuples by the $n$ equalities of
        of the paired components.

  \begin{displaymath}
  \infrule{\Env; \Gamma \vdash a_1 = b_1  \quad \cdots \quad
           \Env; \Gamma \vdash a_n = b_n}
          {\Gamma \vdash (a_1, ..., a_n) = (b_1, ..., b_n)}
  \end{displaymath}
\end{itemize}

%left
\addTactic{left}{}
Reduce a disjunctive goal to its left part
\begin{displaymath}
  \infrule{\Env; \Gamma \vdash \phi_1}{\Env; \Gamma \vdash \phi_1 \lor \phi_2}
\end{displaymath}

%right
\addTactic{right}{}
Reduce a disjunctive goal to its right part
\begin{displaymath}
  \infrule{\Env; \Gamma \vdash \phi_2}{\Env; \Gamma \vdash \phi_1 \lor \phi_2}
\end{displaymath}

%case
\addTactic{case}{\tacarg{f}{formula}}
Do an excluded-middle case analysis on \ec{f}
\begin{displaymath}
  \infrule{\Env; \Gamma \vdash b \Rightarrow \phi(true) \quad
           \Env; \Gamma \vdash \neg b \Rightarrow \phi(false)}
          {\Env; \Gamma \vdash \phi(b)}
\end{displaymath}

%assumption
\addTactic{assumption}{}
Search in the context an hypothesis \refdef{convertible} to the goal and close it.
 If no such hypothesis exists, the tactic fails
\begin{displaymath}
  \infrule{(h : \phi) \in \Gamma}{\Env; \Gamma \vdash \phi}
\end{displaymath}

%intros
\addTactic{intros}{\tacarg{x}{\_|ident}}
This tactics permits to remove of your goal : a forall, the left side af an application or a let assignement by pushing it into your \refdef{context}.
Easycrypt checks that \vararg{x} is not already present in the \refdef{environment}.
\begin{displaymath}
  \infrule{\Gamma,x = a \vdash G(x)}{\Gamma \vdash let x = a in G(x)}
  ~~~~~~
  \infrule{\Gamma,x \vdash G(x)}{\Gamma \vdash \forall x, G(x)}
  ~~~~~~
  \infrule{\Gamma,H \vdash G}{\Gamma \vdash H => G}
  ~~~~~~
\end{displaymath}

\example
{}{forall (x y:int), x = 3 => x = 3}
{\vtacname}{a b hyp1}
{
\env{a}{int}
\env{b}{int}
\env{h1}{a=3}
}
{b = 3}


%cut
\addTactic{cut}{\tacarg{ip}{intro-pattern} : \tacarg{C}{formula}}
Logical cut. Generates two subgoals: on for $C$ (the cut formula),
 and one for $C \Rightarrow G$ where $G$ is the initial goal. Moreover,
 the intro-pattern \ec{ip} is applied to the second subgoal.
\begin{displaymath}
  \infrule{\Env; \Gamma \vdash C \quad
           \Env; \Gamma, \vdash C \Rightarrow G}
          {\Env; \Gamma \vdash G}
\end{displaymath}

%elim
\addTactic{elim}{\tacarg{h}{ident}}
This tactics take as argument the name of a \refdef{judgment} from the \refdef{context} or the \refdef{scope}.
\begin{displaymath}
  \infrule{\Gamma, h:A \land B \vdash A \Rightarrow B \Rightarrow G}{\Gamma, h:A \land B \vdash G}
  ~~~~~~
  \infrule{\Gamma, h:\exists x, A(x) \vdash \forall x, A(x) \rightarrow G}{\Gamma, h:\exists x, A \vdash G}
\end{displaymath}\\
\begin{displaymath}
  \infrule{\Gamma, h:(a_1, ..., a_n) = (b_1, ..., b_n) \vdash a_1 = b_1 \Rightarrow ... \Rightarrow a_n = b_n \Rightarrow G}{\Gamma, h:(a_1, ..., a_n) = (b_1, ..., b_n) \vdash G}
\end{displaymath}


%%%%%%%%%%%%%%%%%%%%%%%%%%%%%%%%%
% AUTO
%%%%%%%%%%%%%%%%%%%%%%%%%%%%%%%%%

\subsection{Automatic}

%smt
\addTactic{smt}{[\ec{nolocal}]}
Try to solve the goal using SMT solvers. The goal is sent along with all the
 lemmas proved so far plus the local hypotheses, unless the \ec{nolocal} is
 given.
 
 \noindent
 \warningbox{Not all lemmas can be sent translated in such a way that they can
  be sent to the SMT provers. For instance, any formulas involving pRHL
  constructions are ignored.}

%apply
\addTactic{apply}{\tacarg{p}{proof-term}}
Modus Ponens. If \ec{p} is a proof-term for the pattern (formula) for
  \begin{center}
    \ec{forall (x1 : t1) ... (xn : tn), A1 -> ... -> An -> B}
  \end{center}
  \noindent then \tacname{} tries to match B with the current G. If the
  match succeeds and leads to the full instantiation of the pattern,
  then the goal is replaced, after instantiation, with the $n$ subgoals
  \ec{A1, ..., An}

%rewrite
\addTactic{rewrite}{rw1 ... rw${}_n$ where the rw${}_i$ are of the form \ec{//},
\ec{/=}, \ec{//=}, a proof-term or a pattern prefixed by \ec{/}
(slash). The two last forms can be prefixed by a direction indicator (the sign
\ec{-}), followed by an occurrence selector (\ec{\{i1 ... in\}}),
followed by repetition marker (\ec{!}, \ec{?}, \ec{i!} or \ec{i?}). All
these prefixes are optional.}
Depending on the form of \ec{rw}, \tacname{} \ec{rw} does the following:
  \begin{itemize}
   \item Call \rawec{trivial} if \ec{rw} is \ec{//},
   \item Call \rawec{simplify} if \ec{rw} is \ec{/=},
   \item Call \rawec{simplify; trivial} as \ec{rw} is \ec{//=},
   \item If \ec{rw} is a proof-term for the pattern (formula)
     \begin{center}
      \ec{forall (x1 : t1) ... (xn : tn), A1 -> ... -> An -> f1 = f2}
     \end{center}
     \noindent then \tacname{} searches for the first subterm of the goal
     matching \ec{f1} and resulting in the full instantiation of the pattern.
     It then replaces, after instantiation of the pattern, all the occurrences
     of \ec{f1} by \ec{f2} in the goal, and creates $n$ new subgoals for the
     \ec{Ai}'s. If no subterms of the goal match \ec{f1} or if the pattern
     cannot be fully instanciated by matching, the tactic fails.
     The tactic works the same if the pattern ends by \ec{f1 <-> f2}. If the
     direction indicator \ec{-} is given, \tacname{} works in the reverse
     direction, searching for a match of \ec{f2} and then replacing all
     occurrences of \ec{f2} by \ec{f1}.
   \item If \ec{rw} is a \ec{/}-prefixed pattern of the form \ec{(o p1 ... pn)},
     with \ec{o} a defined symbol, then \tacname{} searches for the first subterm
     of the goal matching \ec{(o p1 ... pn)} and resulting in the full instantiation
     of the pattern. It then replaces, after instantiation of the pattern, all
     the occurrences of \ec{(o p1 ... pn)} by the $\beta\delta_{\rm o}$ head-normal form
     of \ec{(o p1 ... pn)}. If no subterms of the goal match \ec{(o p1 ... pn)} or
     if the pattern cannot be fully instanciated by matching, the tactic fails. If the
     direction indicator \ec{-} is given, \tacname{} works in the reverse
     direction, searching for a match of the $\beta\delta_{\rm o}$ head-normal
     of \ec{(o p1 ... pn)} and then replacing all occurrences of this head-normal
     form with \ec{(o p1 ... pn)}.
  \end{itemize}
  
  \smallskip
  
  The occurrence selector \ec{\{i1 ... in\}} allows to restrict which occurrences
  of the matching pattern are replaced in the goal. If given, only the
  \ec{i1}-th, ..., \ec{in}-th ones are replaced (considering that the goal is
  traversed in DFS mode). Note that this selection applies after the matching has
  been done.
  
  \medskip
  
  Repetition markers allow the repetition of the same rewriting. For instance,
  \tacname{} \ec{!rw} leads to \ec{do!} \tacname{} \ec{rw}. See \ec{do} for
  more information.
  
  \medskip

  Last, \tacname{} \ec{rw1 ... rwn} is equivalent to
  \tacname{} \ec{rw1}; ...; \tacname{} \ec{rwn}

%elimT
\addTactic{elim}{$\!\!$/\tacarg{h}{ident} \tacarg{f}{pattern}}
Apply the induction principle \vararg{h} on \vararg{x}

%subst
\addTactic{subst}{\tacarg{x}{ident}?}
Search for the first equation of the form \ec{x = f} or \ec{f = x} in the context
 and replace all the occurrences of \ec{x} by \ec{f} everywhere in the context and the
 goal before clearing it. If no idents are given, repeatedly apply the tactic to
 all identifiers for which such an equation exists.

%progress
\addTactic{progress}{}
Split all you hypothesis, make all subsitution possible and then split your goal and do it again on all subgoals. It
permits to easly break a big judgements in smaller one.

%trivial
\addTactic{trivial}{}
If what remains in yout goal is very simple and not need an smt you can try using this tactic. TODO



%%%%%%%%%%%%%%%%%%%%%%%%%%%%%%%%%
% OTHER
%%%%%%%%%%%%%%%%%%%%%%%%%%%%%%%%%

\subsection{Other}

%idtac
\addTactic{idtac}{\tacarg{x}{string}?}
The identity tactic, leaving the goal unchanged and printing the string argument, if any.


\expandafter\createEasycrypt \ambientKeywords \relax

\lstdefinestyle{easycrypt-default}{
  columns=fullflexible,
  captionpos=b,
  frame=tb,
  xleftmargin=.1\textwidth,
  xrightmargin=.1\textwidth,
  rangebeginprefix={(**\ begin\ },
  rangeendprefix={(**\ end\ },
  rangesuffix={\ *)},
  includerangemarker=false,
  basicstyle=\small\sffamily,
  identifierstyle={},
  keywordstyle=[1]{\itshape\color{OliveGreen}},
  keywordstyle=[2]{\bfseries\color{Blue}},
  keywordstyle=[3]{\bfseries},
  keywordstyle=[4]{\bfseries},
  keywordstyle=[5]{\bfseries\color{OliveGreen}},
  keywordstyle=[6]{\itshape\color{Blue}},
  keywordstyle=[7]{\itshape\color{Red}},
  literate={phi}{{$\!\phi\,$}}1
           {phi1}{{$\!\phi_1$}}1
           {phi2}{{$\!\phi_2$}}1
           {phi3}{{$\!\phi_3$}}1
           {phin}{{$\!\phi_n$}}1
}

\lstdefinestyle{easycrypt-pretty}{
    basicstyle=\small\sffamily,
    literate={:=}{{$\mathrel{\gets}$}}1
              {<=}{{$\mathrel{\leq}$}}1
              {>=}{{$\mathrel{\geq}$}}1
              {<>}{{$\mathrel{\neq}$}}1
              {=\$}{{$\stackrel{\$}{\gets}$}}1
              {forall}{{$\forall$}}1
              {exists}{{$\exists$}}1
              {->}{{$\rightarrow\;$}}1
              {<-}{{$\leftarrow\;$}}1
              {=>}{{$\Rightarrow\;$}}1
              {==>}{{$\Rrightarrow\;$}}1
              {\/\\}{{$\wedge$}}1
              {\\\/}{{$\vee$}}1
              {.\[}{{[}}1
              {'a}{{$\alpha\,$}}1
              {'b}{{$\beta\,$}}1
              {'c}{{$\gamma\,$}}1
              {'t}{{$\tau\,$}}1
              {'x}{{$\chi\,$}}1
              {lambda}{{$\lambda\,$}}1
}

%% Typesetting
\newcommand{\titledbox}[4]{{\color{#1}\fbox{\begin{minipage}{#2}{\textbf{#3:} \color{black}#4}\end{minipage}}}}
\newcommand{\warningbox}[1]{\titledbox{red}{.9\textwidth}{Warning}{#1}}

%%% Local Variables: 
%%% mode: latex
%%% TeX-master: "easycrypt"
%%% End: 



\lstset{numberbychapter=false} %% Set to true if chapter numbering is desired


\title{\EasyCrypt Manual}
\date{Version \ECversion{} --- \today}
\author{The \EasyCrypt Team}

\begin{document}

\maketitle
\thispagestyle{empty}

\tableofcontents

\part{User Manual}
% Getting Started: Installation and Basic Usage
% !TeX root = easycrypt.tex

\chapter{Getting Started}
\section{Installation}

\section{Basic Example (Tutorial)\label{sec:tutorial}}
In this section we present a formalization of the proof of security
against chosen plaintext attacks of a public-key encryption scheme
introduced by \citet{br93}.

This scheme is based on one-way trapdoor permutations and uses a
random oracle. The proof of security follows by reduction: given an
adversary against the chosen plaintext attack experiment that has
access to the random oracle, there exists an inverter (making use of
the adversary) that succeds in inverting the underlying trapdoor
permutation with at least as much probability.

While the proof of this scheme is not involved, it allows us to
introduce several features of \EC both on the modelling and on
the proving side.

On the modelling side, we will describe how to define abstract operators
and axioms; how to clone theories; how to define module types, modules
and how to model adversaries as abstract procedures.

On the proving side, we will explain how to perform reasoning up to
failure; optimistic sampling and simple reasoning about probabilites
of events in games.

\subsection{Setting}
We begin by introducing some basic concepts of cryptography and how we
model them in \EC.

\paragraph{Bitstrings}
As part of its standard library, \EC provides a theory of fixed-length
bitstrings coined \ec{Word}. For the purpose of this example, we will
need bitstrings of three different lengths, \ec{k} for messages,
\ec{l} for randomness and \ec{n} for ciphers; such that \ec{k + l =
  n}. The following piece of code declares three integer constants,
states the desired relationship between them as an axiom, clones the
theories (see Section~\ref{sec:cloning} for details on cloning) with
appropriate lengths and defines synonyms for the types.

\ecimport{ }{../examples/br93_tutorial.ec}{bitstrings}

The \ec{import} keyword allows us to ommit the qualified names when
they can be inferred from the context. 

We will be interested in sampling from uniform distributions of types
\ec{randomness} and \ec{plaintext}. These distributions are defined in
the standard library, but we define shortcuts to them as follows

\ecimport{ }{../examples/br93_tutorial.ec}{distributions}


\paragraph{Random oracles}
A hash function in the random oracle model is a consistent random
function: new queries trigger samplings that are stored in a map; old
queries trigger a map lookup. 

Adversaries are normally given restricted access to the oracle, in the
sense that the amount of calls allowed is bounded. Moreover, we keep
track of the queries triggered by the adversary. 

The following piece of code defines two signatures for random oracles:
the one that will be used in the encryption of messages and the one
that will be provided to adversaries. Moreover, we define an
implementation between types \ec{randomness} and \ec{plaintext} as
defined above.

\ecimport{ }{../examples/br93_tutorial.ec}{random_oracles}


The module \ec{RO} is assigned two module types: \ec{Oracle} and
\ec{ARO} and implements both the normal oracle and the adversary
interface. It consists of a global variable \ec{m} that models the map
of the random oracle; a global variable \ec{s} that models the queries
performed by the adversary, a function \ec{init} that initializes the
map to \ec{empty} and the set to \ec{empty}, a function \ec{o} that
models the actual oracle and a function that models the adversary
version. The former takes an argument \ec{x} of type \ec{randomness},
samples a value \ec{y} of type \ec{plaintext}, updates the value of
the map in case it's necesary and finally returns the lookup of \ec{x}
in the map \ec{m}. The latter checks that the adversary has not
exceeded the amount of calls as prescribed by \ec{qO}; and if this is
the case, records the query in \ec{s}, calls \ec{o} and retunrs its
result; otherwise, it returns a bitstring consisting of zeros.


\paragraph{One-way trapdoor permutations}
Trapdoor permutations are functions that are ``easy'' to compute, but
hard to invert without knowing a secret key. Formally, a family
of trapdoor permutations on a type \ec{T} is a triple of algorithms
$(\KG,f,f^{-1})$ such that for any pair of keys $(pk,sk)$ output by
$\KG$, $f_{pk}$ and $f^{-1}_{sk}$ are permutations on
\ec{T} and inverse of each other. 

In these case, we are interested in trapdoor permutations on type
\ec{randomness}. In \EC we model them as follows

\ecimport{ }{../examples/br93_tutorial.ec}{one_way}

First, we declare two types \ec{pkey} and \ec{skey} that represent
public and secret keys respectively. We then declare a distribution
\ec{keypairs} that will be used by the $\KG$ algorithm to produce fresh
keys. We then declare an abstract predicate that will model when two
keys are matching and specify that all the pairs of keys that can be
sampled from \ec{keypairs} are matching. Afterwards, we declare \ec{f}
and \ec{finv} with the appropriate type and we state the axioms saying
that one is inverse of eachother for valid key pairs.
 
The module \ec{OW} is defined by declaring an interface corresponding
to the inverter, which comprises a function that given a challenge
value and a public key returns a guess for the pre-image of the
challenge. The OW experiment is then defined by sampling a value
\ec{x} and a pair of keys \ec{pk} and \ec{sk} and providing the
inverter with \ec{f pk x}. The inverter returns a value \ec{x'} and
the experiment succeds if \ec{x = x'}.

\paragraph{Public-key encryption}
A public-key encryption scheme consists of a triple of algorithms
$(\KG,\Enc,\Dec)$:

\begin{description}
\item[Key Generation] 
  The key generation algorithm $\KG$ produces a pair of keys $(pk,sk)$;
  $pk$ is a \emph{public-key} used for encryption, $sk$ is
  a \emph{secret-key} used for decryption;

\item[Encryption] 
  Given a public-key $pk$ and a message $m$, $\Enc_{pk}(m)$ outputs a
  ciphertext $c$;

\item[Decryption] 
  Given a secret-key $sk$ and a ciphertext $c$, $\Dec_{sk}(c)$ outputs
  either message $m$ or a distinguished value $\bot$ denoting failure.
\end{description}
%
We require that for pairs of keys $(pk,sk)$ generated by $\KG$,
$\Dec_{sk}(\Enc_{pk}(m))= m$ holds for any message $m$. 

In \EC, we model public-key encryption schemes in the following way

\ecimport{ }{../examples/br93_tutorial.ec}{scheme}

We declare the interface of an encryption scheme, that consists of a
key generation algorithm \ec{kg}, an encryption function \ec{enc} and
a decryption function \ec{dec}. Note that we are actually modelling
the particular class of encryption schemes that use at most one random
oracle.

\paragraph{The BR93 scheme}
The scheme we consider defines encryption as 
$\Enc_{pk}(m) = f_{pk}(r) \concat m \xor H(r)$, where $r$ is a
randomly sampled value and $H(r)$ denotes the result of a hash oracle
call with value $r$.

 In easycrypt we model the encryption scheme as an instance of our
 \ec{Scheme} signature as follows

\ecimport[morecomment={[is]{proof.}{qed.}}]{ }{../examples/br93_tutorial.ec}{br93}

First, we define a concatenation operation with the appropriate
type. In doing so, we make use of the array library, in particular,
the concat operation \ec{||}. Note that we have to insert the
appropriate cast functions that allows us to move from boolean arrays
to bitstrings of fixed size. We then define some useful lemmas (we
ommit the proofs for conciseness).

Afterwards, we define the module \ec{BR}, that makes use of a random
oracle \ec{R} and has type \ec{Scheme(R)}. This module defines the
required operations in the \ec{Scheme} signature. \ec{kg} samples a
pair of keys and returns them. \ec{enc} takes a public key \ec{pk} and
a plaintext \ec{m} and encrypts by samplying a random value \ec{r},
querying the oracle with this value and applying \ec{f} to \ec{pk} and
\ec{r} and appending it to the xor of result of the hash call and the
message. Decryption is defined by projecting the first part of \ec{c}
and applying \ec{finv} with \ec{sk} to it to recover the
randomness. Afterwards, we query the oracle with the randomness and
xor the result with the second part of \ec{c}.

\paragraph{The \textsf{IND-CPA} experiment}
The CPA experiment can be described in \EC as follows:

\ecimport{ }{../examples/br93_tutorial.ec}{cpa}

We start by declaring a module type for the adversary. This interface
establishes that an adversary has to provide two functions. The
function \ec{a1} corresponds to the first round of the CPA game in
which the adverasay is given a public key \ec{pk} and produces two
plaintext of its choice. The function \ec{a2} corresponds to the
second phase, in which the adversary is presented with a challenge
ciphertext \ec{c} and he has to determine (by producing a boolean)
which plaintext was encrypted.

The module \ec{CPA} is parametrized by an adversary \ec{A_} and a
scheme \ec{S}. Note however, that in order to use these modules we
have to instantiate them with a random oracle. However, we will not
use the same oracle for both: for the encryption we use the bare
random oracle \ec{RO}, but for the adversary we first define \ec{AO},
a wrapped version of the oracle that provides the increased
functionality mentioned previously, and then we instantiate \ec{A_}
with it.

The function \ec{main} defines the CPA experiment. We initialize the
adversary oracle, and we call the key generation algorithm and obtain
\ec{pk} and \ec{sk}. Then we provide \ec{a1} with the public key and get
two plaintext \ec{m0} and \ec{m1}. We proceed by sampling a random bit
and we encrypt either \ec{m0} or \ec{m1} according to its value and we
call \ec{a2} with this ciphertext as argument and obtain a boolean
\ec{b'}. The experiment returns true if the adversary manages to
establish the value of \ec{b}, {\em i.e.} it returns \ec{b=b'}.

\paragraph{Security property}
We are interested in proving that for any adversary \ec{A} against CPA,
that manages to win with probability $P$, there exists a inverter
I that manages to win the one-way experiment with probability $Q$ such
that $| P - \frac{1}{2} |\leq Q$. The intuition is that an adversary
that constantly returns \ec{true} wins with probability
$\frac{1}{2}$. What we are measuring is essentially how much better
can an adversary do, and we are bounding it in terms of the
probability of suceeding in the one-way experiment.

This property is formalized in \EC as follows

\ecimport{ }{../examples/br93_tutorial.ec}{conclusion}

The lemma \ec{Conclusion} is parametrized by an adversary \ec{A}. The
annotations \ec{\{Rand, RO\}} are memory access restrctions:
essentially, the adversary cannot access the memories of these modules
directly---only through oracle calls. The next parameter of the lemma
is a memory \ec{m}. Next, we have one hypothesis that states that if
the wrapped version of the random oracle terminates with probability
$1$, so does the adversary. The \ec{\%r} next to constants establishes
that they should be interpreted as real values. The existential
quantification binds a module \ec{I} of type \ec{Inverter}. The
expression \ec{Pr[CPA(BR,A).main() @ &m : res]} denotes the
probability of the event \ec{res} in the distribution obtained after
executing the \ec{main} function of module \ec{CPA}, instantiated with
the module \ec{BR} and the parametrized adversary $A$ in initial
memory \ec{m}. Similarly, the expression \ec{Pr[OW(BR_OW(A)).main() @
  &m : res]} denotes the probability of the event \ec{res} in the
distribution obtained after executing the \ec{main} function of module
\ec{OW}, instantiated with the existentially quantified module \ec{I},
in initial memory \ec{m}. These two expressions are of type real and
the \ec{<=} real less or equal operator.

The rest of the tutorial is devoted to introducing and explaining the
reasoning principles of \EC required to establih this property.
%TODO: put forward pointers to françois' and guillaume's section

\subsection{Overview of the proof}
The proof proceeds by defining a sequence of transformation starting
on the initial game and proving a relationship between probabilities
of events of interests in each of these games. Each of these steps is
justified by reasoning in terms of probabilistic relational hoare
logic (pRHL for short).

The proof of the BR93 encryption scheme hinges on the following
sequence of games:
\begin{enumerate}
\item The first transformation we apply is to replace the call to the
  hash oracle in the encryption by a fresh randomly sampled value. The
  obtained encryption expression is $\Enc^2_{pk}(m) = f_{pk}(r) \concat
  m \xor r'$. We can show that the results of the encryption are
  observationally equivalent and that the resulting maps of the random
  oracle differ only in the query for value \ec{r}, provided that
  query done by the original encryption is fresh, therefore, if the
  adversary did not query \ec{r} to the random oracle. Hence, we bound
  the probability of \ec{b=b'} in the intitial game, in terms of the
  probability of winning in the second game, plus the probability of
  the event \ec{mem M.r ARO.s}, which means that the adversary was
  able to query \ec{r} to the random oracle.

\item Next, we modify the encryption scheme once again and replace the
  occurrence of $m\xor r'$ by $r'$ and we obtain $\Enc^3_{pk}(m) =
  f_{pk}(r) \concat r'$. This step is justified by means of optimistic
  sampling: $\Rand{r}{\bitstring{k};\Ass{v}{r\xor x}}$ and
  $\Rand{v}{\bitstring{k}}$ yield the same distribution of values of
  $v$.

\item  Note that at this point, the encryption algorithm does not depend on
  its input $m$ anymore. In particular, in the \ec{CPA} game, this
  means that the challenge \ec{c} that the adversary receives in the
  second round, does not depend on the bit \ec{b} anymore. We exploit
  this fact and at this point we modify de \ec{CPA} game by pushing
  the sampling of \ec{b} to the bottom of \ec{main}. At this point,
  the probability of \ec{b = b'} is exactly $\frac{1}{2}$.

\item We show that if the adversary manages to query \ec{r} to the
  random oracle, he can invert the one way function, since the only
  occurrence of \ec{r} in the cipher is as argument to the one way
  function. In doing so, we construct an inverter that receives a
  challenge \ec{y} and a public key, runs the first phase of the
  adversary, samples a value \ec{r'} and calls the second phase of the
  adversary with argument \ec{y || r'}. Afterwards, we go through
  the sets of queries performed by the adversary and look for $x$
  s.t. \ec{f pk x = y}. If we succeed, we return x.
\end{enumerate}

\subsection{Easycrypt proof}
In this section, we describe thoroughly the proof, as it is done in
easycrypt. We present intermediate games, and formally define the
relationships between them.

\paragraph{Reasoning up to failure: replacing \ec{H(r)} by fresh \ec{h}}
As mentioned above, the first step of the proof is to remove the hash
call in the encryption scheme and replace it by a randomly sampled
value. We define \ec{BR2} then, as follows

\ecimport{ }{../examples/br93_tutorial.ec}{br2}

Now, the question that arises is: when will an adversary be able to
distiguish \ec{h = H(r)} and \ec{h = $uniform_plain}? %$
If the query in the original scheme is fresh, it means that it will be
a randomly sampled value and in this case, the two values above are
indistinguishable. We formalize this reasoning step in pRHL as follows

\ecimport{ }{../examples/br93_tutorial.ec}{eq1enc}

The precondition states that when we run both encryptions with equal
arguments \ec{pk, m} and in inital states that coincide in the value
of \ec{RO.m}; if the value \ec{Rnd.r} has not been queried to
\ec{RO.m} on the right execution, the results coincide, the values of
\ec{Rnd.r} coincide and the mappings \ec{RO.m} are equal except in
\ec{Rnd.r}.

The proof proceeds by making use of a sequence of tactics. Among them
we have
\begin{itemize}
\item \ec{fun}: when called in a pRHL goal with two concrete
  procedures (such as in this case), it expands their definition. 
\item \ec{inline}: it expects a function identifier and inlines the
  calls in the current goal. In this case, it is used to inline the
  definition of RO.o in the current goal.
\item \ec{wp}: implements relational weakest precondition.
\item \ec{rnd}: implements the pRHL rule for random samplings. It
  takes a bijection (in the form of two functions $f$ and $g$) and ensures that
  the result of the random sampling on the left side is equal to $f$
  of the random sampling of the right side. When we ommit paramenters,
  the bijection is the identity.
\item \ec{skip}: allows to reduce the validity of the current
  judgement to the validity of a logical entailment between pre and
  post condition when the programs are ``empty''.
\item \ec{smt}: calls smt solvers in order to decide on the validity
  of a formula.
\end{itemize} 
For a detailed explanation of the tactics see Chapter (writting proofs).

At the point where the encryption scheme is called
within the CPA game, we know that all the values queried to the random
oracle correspond to adversary queries. Hence, the call is fresh if
the adversary has not queried this value. On the other hand, after the
encryption is called, we know that the oracle maps are equal, except
in the value of \ec{Rand.r}. Therefore, if the adversary in the second
round queries the hash oracle with \ec{Rand.r}, he manages to
distinguish between the two. To sum up, the behaviour is equivalent
except if the adversary manages to query \ec{Rand.r} either in the
first phase or in the second one. We formalize this reasoning in
easycrypt as follows

\ecimport{ }{../examples/br93_tutorial.ec}{eq1}

Upon starting the proof, after \ec{intros A Hll1 Hll2;fun}, we have
the following goal

%formatting needs to be fixed
\begin{easycrypt}[columns=fullflexible]{ }{\label{ec:proof1}}
Current goal

Type variables: <none>

A : Adv{RO, Rnd}
Hll1: forall (R <: ARO), islossless R.o_a => islossless A(R).a1
Hll2: forall (R <: ARO), islossless R.o_a => islossless A(R).a2
--------------------------------------
&1 (left ) : CPA(BR, A).main
&2 (right) : CPA(BR2, A).main

pre = (glob A){1} = (glob A){2}

RO.init()                                       (1)  RO.init()                       
(pk, sk) := CPA(BR, A).SO.kg()      (2)  (pk, sk) := CPA(BR2, A).SO.kg() 
(m0, m1) := CPA(BR, A).A.a1(pk)  (3)  (m0, m1) := CPA(BR2, A).A.a1(pk)
b =$ {0,1}                                    (4)  b =$ {0,1}                      
c := CPA(BR, A).SO.enc(pk,          (5)  c := CPA(BR2, A).SO.enc(pk,     
                       b ? m0 : m1)                                             b ? m0 : m1)                  
b' := CPA(BR, A).A.a2(c)               (6)  b' := CPA(BR2, A).A.a2(c)       

post = !(mem Rnd.r{2} RO.s{2}) => (b{1} = b'{1}) = (b{2} = b'{2})

\end{easycrypt}

The strategy we apply is the following:
\begin{itemize}
\item Before the second adversary call at the line 6, we know that the
  postcondition of \ec{enc} holds. Then, we use precondition

\ec{(!mem Rnd.r RO.s){2} => (={RO.s,c} /\\  eq_except RO.m{1} RO.m{2} Rnd.r{2}
                  /\\ (glob A){1} = (glob A){2})}

and postcondition \ec{((!mem Rnd.r RO.s){2} => ={res})}

In order to prove this specification, we use \ec{fun} with two
arguments:
\begin{itemize}
  \item The first one is a failure event $F$, in this case \ec{mem Rnd.r
      RO.s}.
    \item The second one, is an invariant $I$, in this case 
      \ec{(={RO.s} /\\ eq_except RO.m{1} RO.m{2} Rnd.r{2})}.
\end{itemize}
The idea is that the invariant will hold, as long as the failure event
doesn't. This tactic generates the following proof obligations:
\begin{itemize}
\item a logical entailment of the form $P \Rightarrow \textsf{ if } F
  \textsf{ then true else } I$, where $P$ is the precondition
  (discharged by \ec{smt};

\item a logical entailment of the form $(\lnot F \Rightarrow
  I)\Rightarrow Q$, where $Q$ is the postcondition (discharged by
  \ec{smt});

\item losslessness of the adversary, assuming losslessness of the
  oracles he has access to (in our case this is an assumption);

\item a pRHL judgement for each of the oracles the adversary has
  access to, with precondition $!F \land I \land =\{\mathsf{args}\}$
  and postcondition $!F \Rightarrow (I \land=\{\mathsf{res}\})$;

\item losslessness of the oracles the adversary has access to,
  assuming $F$ holds in the initial state;

\item preservation of $F$ for all the oracles the adversary has access
  to (a bounded probabilistic hoare logic judgement with precondition
  $F$ and postcondition $F$ and probability 1).
\end{itemize}
For a more detailes account of reasoning up to failure, we refer the
reader to Section (upto bad section).


\item For the call to the encryption scheme in line 5, we use the
  specification we established previously. This is done using
  \ec{call} with the same spec, and then applying the lemma
  \ec{eq_enc1}.

\item The random samplings of b in line 5 are handled using the
  \ec{rnd} tactic. This ensures that the values are considered equal. 

\item For the first adversary call in line 4, we provide a
  specification. For the precondition we use
 \ec{(={RO.m,RO.s,pk} /\\ (glob A){1} = (glob A){2} /\\ dom RO.m{2} =
   RO.s{2})}, and for the postcondition, the same formula, except that
 we require equality on \ec{res} rather than in the argument
 \ec{pk}. We prove this specification using \ec{fun} with invariant
 (I) defined as \ec{(={RO.m,RO.s} /\\ dom RO.m{2} = RO.s{2})}. This
 tactic generates the following proof obligations:
\begin{itemize}
\item a logical entaliment of the form $P \Rightarrow I \land
  (\mathsf{glob}\;A)\{1\} = (\mathsf{glob}\;A)\{2\} \land
  =\{\mathsf{args}\}$, where $P$ is the precondition,
  $\mathsf{glob}\;A$ represents the global variables of adversary
  \ec{A} (all the state variables in all the modules, except those
  modules to which access is explcitly restricted in the annotations)
  and $\mathsf{args}$ represents the arguments of the adversary;
\item a logical entailment of the form $I \land
  (\mathsf{glob}\;A)\{1\} = (\mathsf{glob}\;A) \land =\{\mathsf{res}\}
  \Rightarrow Q$, where $\mathsf{res}$ is the result of the adversary
  and $Q$ is the postcondition;
\item a pRHL judgement for each of the oracles the adversary has
  access to, with precondition $I \land =\{\mathsf{args}\}$ and
  postcondition $I \land =\{\mathsf{res}\}$.
 \end{itemize} 
For a more detailed account of the rule for adversary calls, we refer
the reader to Section(call for adversaries).
\item It remains to prove that the code before the call in line 3
  establishes the precondition. This is done by inlining the code and
  using \ec{wp} and \ec{rnd} when necessary. 

\item Once we have two empty programs, we use \ec{skip} to reduce the
  validity of the judgement to the validity of a logic formula. Then
  we apply \ec{progress}, which implements a simplification strategy
  for formulas, and we discharge the resulting goals using the
  \ec{smt} tactic, which interfaces with smt solvers and concludes the
  proof. 
\end{itemize}
From the pRHL judgement we have just proved, we can conclude the
following property

\ecimport{ }{../examples/br93_tutorial.ec}{prob1}

The proof proceeds by transitivity:
\begin{itemize}
\item First, we prove 

\ec{Pr[CPA(BR,A).main() @ &m : res] <=
Pr[CPA(BR2,A).main() @ &m : res \\/ mem Rnd.r RO.s]},
using our pRHL proof and the tactic \ec{equiv_deno}.

\item Next, we prove

\ec{Pr[CPA(BR2,A).main() @ &m : res \\/ mem Rnd.r RO.s] <=} 

\ec{Pr[CPA(BR2,A).main() @ &m : res]} +

\ec{Pr[CPA(BR2,A).main() @ &m : mem Rnd.r RO.s]}

using the tactic \ec{pr_or}, that implements the following reasoning
principle:
\ec{Pr[ f() @ &m : a \\/ b]} =  \ec{Pr[ f() @ &m : a]} + \ec{Pr[ f() @
  &m : b]} - \ec{Pr[ f() @ &m : a /\\ b]}.
 
\end{itemize}

\paragraph{Optmistic sampling: replacing \ec{m \^\^ r'} by \ec{r'}}
The second step of the proof consists on modifying again the
computation of the ciphertext. In this case, instead of computing the
$\xor$ of the message $m$ with the randomly sampled value $r'$, we
directly place $r'$ in that position. This transformation is defined
as follows
\ecimport{ }{../examples/br93_tutorial.ec}{br3}

We can prove that the results of the two versions of the encryption
are indistinguishable resorting to optimistic sampling. The following
\EC script shows the desired property

\ecimport{ }{../examples/br93_tutorial.ec}{eq2enc}

The specification states that when the two versions of the encryption
function are called with equal arguments and in two states where the
maps associated to the random oracle are equal, the functions yield
the same result and produce states where the maps of the oracle are
equal and the randomness is equal.  

The key reasoning step is embodied by the tactic \ec{rnd (lambda v, v
  \^\^ m) (lambda v, v\^\^ m)}. In its most general form, the \ec{rnd}
tactic takes a bijection, represented as a pair of functions inverse
from eachother. When the arguments are ommitted, the identity function
is used as bijection. This tactic implments the pRHL rule for random
assignments shown in (Reference to Reasoning about random samplings)
and in this case, it allows us to conclude that ${h\{2\} = m\{1\} \xor
  h\{1\}}$.

Using this specification for \ec{enc}, we can conclude this step as
follows
\ecimport{ }{../examples/br93_tutorial.ec}{eq2}

Note that here we have the same code on both sides, operating on the
same state. The only difference is in the code for encryption, that we
already proved yields equal results. Hence, all the invariants we are
use consist of equalities on variables on both sides. 

Once more, using our pRHL logic judgements, we conclude a property on
the probabilities of th events we are interested in as follows

\ecimport{ }{../examples/br93_tutorial.ec}{prob2}





\paragraph{Code movement: sampling \ec{b} at the end}
At this point, the encryption function does not depend on the input
message anymore. We take advantage of this fact and modify the CPA
game to sample the bit \ec{b} at the bottom of the game; hence making
explicit that the adversary cannot gain any information regarding its
value by inspecting the cipher. The \ec{CPA2} game is defined as
follows
\ecimport{ }{../examples/br93_tutorial.ec}{cpa2}

Again, we are interested in showing the equivalence of the events of
interest. In \EC we carry out this step in the following way
\ecimport{ }{../examples/br93_tutorial.ec}{eq3}

The key step in this case, is \ec{swap{2} -2}. This tactic moves the
last instruction of the second game \ec{b = $\{0,1\}} %$
up by two instructions. After this, we have the exact same code on
both sides, so we conclude the proof using a serie of invariants that
consists of equalities on variables on both states; as done in the
previous step. 

At this point we can prove the following result on probabilities
\ecimport{ }{../examples/br93_tutorial.ec}{prob3}

The proof consists of three steps:
\begin{itemize}
\item \ec{Pr[CPA(BR3,A).main() @ &m : res] = Pr[CPA2(BR3,A).main() @
    &m : res]} 
\item \ec{Pr[CPA(BR3,A).main() @ &m : mem Rnd.r RO.s] = Pr[CPA2(BR3,A).main() @
    &m : mem Rnd.r RO.s]}
\item \ec{Pr[CPA2(BR3,A).main() @  &m : res] = 1\%r/2\%r}
\end{itemize}
The first two are proved, as usual, relying on the previously proved
pRHL judgement and in \ec{equiv_deno}. The last one relies on bounded
probabilistic Hoare logic. This logic allows us to reason regarding
bounds of events in our probabilistic language. In particular, in this
case we show that the probability of \ec{b = b'} in a game where \ec{b}
is sampled at the end is $\frac{1}{2}$. This proof is relatively
straightforward and consists of two parts:
\begin{itemize}
\item Showing that all the code before the sampling of \ec{b} is
  lossless, {\em i.e.} terminates with probability 1;
\item showing that a program that consists uniquely of the sampling of
  a boolean \ec{b} yields a distribution where the event \ec{b = true}
  and \ec{b = false} have probability $\frac{1}{2}$.
\end{itemize}

For the first part, we do the same kind of reasoning as when we did
reasoning up to failure. Basically, we have to show that the program
terminates unconditionally. In doing so, we show that all function
calls are lossless: concrete calls are handled by inlining and
adversaries are handeled using our losslessness hypotheses.

For the second part, the key step is the use of the \ec{rnd (1\%r /
  2\%r) (lambda b, b = b')}. This tactic takes a bound and an event
(depending on the value sampled), consumes the random sampling and
produces the following goal


\ec{mu \{0,1\} (lambda (b0 : bool), b0 = CPA2(BR3, A).main.b') = 1\%r / 2\%r},

where \ec{mu} is an axiomatized that measures the probability of an
event in a distribution. In our case, we conclude by relying on the
definition of the \ec{\{0,1\}} distribution.


\paragraph{Constructing the inverter}

%%% Local Variables: 
%%% mode: latex
%%% TeX-master: "easycrypt"
%%% End: 

% Theories: Types and Operators, Modules, and Cloning
% !TeX root = easycrypt.tex

\chapter{Theories (Fran\c{c}ois + Gilles)\label{chap:theories}}

Definitions and lemmas can be grouped in theories, that can be imported to
provide new language functionalities as required for a particular proof.
Currently, \EC supports user-defined types and operators
(Section~\ref{sec:types}) and user-defined distributions
(Section~\ref{sec:distributions}). In addition, theories can declare modules and
functors (Section~\ref{sec:modules}), used to represent schemes, oracles and
experiments, as well as abstract types for such modules, which can be used to
modularize proofs or represent abstract adversaries. Finally, theories can be
cloned and refined (Section~\ref{sec:cloning}), which allows the user to
consider small variants of a theory without having to create a new one from
scratch, or to consider concrete implementation details only when necessary for
the proof.

% TODO: adapt depending on what I actually end up doing
We illustrate each of the basic concepts with commented excerpts from the
standard library distributed with \EC. The standard library is documented
further in Chapter~\ref{chap:libraries}.

\section{The \EC Expression Language}

At the core of any \EC specification is a collection of types and functional
operators on those types, that create and extend a rich language of expressions
for use in cryptographic games. Informally, \EC types can be seen as
\emph{non-empty} sets, and operators are \emph{mathematical} functions between
them.
%
The reader familiar with polymorphic lambda-calculus~\cite{} may skip forward to
Section~\ref{sec:ec-specifics}.

\subsection{A quick overview}
Following this analogy and standard functional programming notation, we write
denote with \ec{f : 'a -> 'b}  the fact that \ec{f} is a function from
type \ec{'a} to type \ec{'b}. Given function \ec{f} above, and an
\ec{x : 'a}, we denote with \ec{f x} the value (of type \ec{'b})
resulting from applying \ec{f} to \ec{x}.

Functions expecting several arguments take their arguments separated by spaces.
For example, the logarithm base 2 of an integer \ec{n} would be denoted
\ec{log 2 n}, where the type of \ec{log} is
\ec{log: int -> int -> int}. For now, the reader unfamiliar with functional
languages should just read this type as the type of functions that take two
integers and return an integer.



\section{Types and Operators\label{sec:types}}

At the core of any \EC specification is a collection of types and functional
operators on those types. Formally, \EC types are \emph{non-empty} sets of
values, and operators are \emph{mathematical} functions between them.

\warningbox{Currently, the easiest way to define types and operators is to
declare them abstractly and specify them using axioms. It is \emph{very}
important for the consistency of the logical context to remember that types are
\emph{always} assumed to be non-empty, and that operators are functionals
(total, deterministic and side-effect free) when writing such definitions.}

The \EC type system supports polymorphic, higher-order types: the type of lists
can be declared independently of the type of their contents, and operators can
take and return other operators.

\subsection{Built-Ins}
The language is equipped with a few built-in types\footnote{This is only to
bridge our language with \WhyThree's logic and avoid the introduction of
indirection layers in the proof obligations sent to the SMT solvers: all these
built-in types could in theory be defined in \EC itself.}:
\begin{itemize}
\item the \rawec{unit} type,
\item the \rawec{bool} type of booleans,
\item the \rawec{int} type of arithmetic integers (in $\mathbb{Z}$),
\item the \rawec{real} type of real numbers (in $\mathbb{R}$).
\end{itemize}

Some symbols are defined, declared or imported from \WhyThree in the
corresponding libraries, and are discussed further in
Chapter~\ref{chap:libraries}.

\subsubsection*{The \rawec{unit} type}
The \rawec{unit} type contains a unique element \rawec{tt}. It is \emph{not}
equal to the empty tuple type in \EC.

\subsubsection*{The \rawec{bool} type: termination, operators and predicates}
In \EC, booleans are used not only to represent binary program variables, but
also to type logical formulas. The \rawec{bool} type is therefore equipped with
symbols for universal and existential quantification, implication, logical and
short-circuiting conjunction and disjunction and xor. However, since the program
logic assumes that expressions are terminating, we need to restrict the class of
boolean formulas that can be used when defining operators and writing programs.
To this effect, we consider two distinct classes of functional symbols:
\emph{predicates}, that may not be computable in general, and cannot be used to
write programs (although they can be used to specify them), and
\emph{operators}, that are computable and can be used for defining predicates
and other operators, and to write programs.

Universal and existential quantification (\rawec{forall} and \rawec{exists}),
implication (\rawec{=>}) and equivalence (\rawec{<=>}) are internally defined as
predicate symbols, and therefore cannot be used to define operators. All other
boolean function symbols are defined as operators and can be used in any
context. They include logical conjunction and disjunction (\rawec{/\\} and
\rawec{\\\/}), short-circuiting conjunction and disjunction (\rawec{\&\&} and
\rawec{\|\|}), and xor (\rawec{+} or \rawec{\^\^}).

\subsubsection*{The \rawec{int} type}
The \rawec{int} type represents mathematical integers. It is possible to
directly write constants between $0$ and $2^{62} - 1$ (non-negative OCaml
integers). The unary negation operator is denoted \rawec{-!}, and all arithmetic
and comparison operators use standard notations (\rawec{+}, \rawec{-},
\rawec{*}, \rawec{/}, \rawec{\%}, \rawec{^}, \rawec{=}, \rawec{<}, \rawec{<=},
\rawec{>}, \rawec{>=} \ldots). It may be necessary to import the integer library
(\rawec{require import Int.}) to have access to these operators. Since this type
is identified with the \WhyThree type of integers, all \WhyThree lemmas on it
can be used by the SMT solvers, but may not be accessible by name for use in
interactive proofs.

\subsubsection*{The \rawec{real} type}
The \rawec{real} type represents real numbers. Integer variables and constants
can be cast to reals by suffixing them with a \rawec{\%r}. Basic operations and
comparisons are defined, with standard notations, on real numbers. Here again,
it may be necessary to import the library (using \rawec{require import Real.})
to put the real operators in scope. This type is also identified with the
\WhyThree type of reals, but SMT solvers often have very limited support for
real numbers and will often fail to prove seemingly trivial results.
%
We will be working on enriching the library of lemmas available in this theory
to allow more complex interactive proofs.

\warningbox{Note that \rawec{(1/2)\%r} and \rawec{1\%r/2\%r} are very different
expressions. The former evaluates to $0$ (the division is in $\mathbb{Z}$, and
the result ($0$) is interpreted in $\mathbb{R}$) whereas the latter evaluates to
$0.5$ (the operands are interpreted and the operator is applied in
$\mathbb{R}$). Eventually, we will introduce notations for scoping that should
allow less cluttered real expressions.}

\subsection{Types}
From these base types, many more can be built. \EC allows the user to declare
 (polymorphic) abstract types, and combine existing types into tuple types or
function types.

Types are declared using the \rawec{type} keyword followed by an identifier and,
if desired, an \rawec{=} sign and a concrete definition. Listing~\ref{lst:types}
declares a polymorphic abstract \rawec{multiset} type, and (rather daftly)
defines naturals as multisets of units.

\begin{easycrypt}[label={lst:types}]{Declaring and applying polymorphic types}
type 'a multiset.
type nat = unit multiset.
\end{easycrypt}

\subsubsection*{Tuple types}
Given two types $\tau$ and $\tau'$, one can easily define the type of pairs of
elements of $\tau$ and $\tau'$ as $\left(\tau * \tau'\right)$. More generally,
the product can be iterated to produce tuple types of arbitrary arity.

\warningbox{Types $\left(\tau * \tau' * \tau''\right)$,
$\left(\left(\tau *\tau'\right) * \tau''\right)$ and
$\left(\tau * \left(\tau' * \tau''\right)\right)$ are distinct.}

\subsubsection*{Function types}

\subsubsection*{A note on polymorphism}


\subsection{Operators}

\subsection{Predicates, Axioms and Ambient Lemmas\label{sec:ec-specifics}}

% TODO: redistribute the following
%
%The following \EC code (Listing~\ref{lst:bitstrings}) declares an abstract type of bitstrings, equipped
%with a length operator (whose result is never negative) and an infix addition
%operator that enjoys interesting properties both with respect to length and by
%itself.
%
%\begin{easycrypt}[label={lst:bitstrings}]{[A type for bitstrings]A core type for bitstrings}
%type bitstring.
%
%op length: bitstring -> int.
%axiom length_pos: forall bs, 0 <= length bs.
%
%op (+): bitstring -> bitstring -> bitstring.
%axiom length_xor: forall b0 b1,
%  length b0 = length b1 => length (b0 + b1) = length b0.
%axiom xor_associative: forall b0 b1 b2,
%  (b0 + b1) + b2 = b0 + (b1 + b2).
%axiom xor_commutative: forall b0 b1, b0 + b1 = b1 + b0.
%\end{easycrypt}
%
%\subsubsection{Types}
%\EC expressions are equipped with polymorphic, higher-order types, that can be declared abstractly (as in Listing~\ref{lst:bitstrings}), or given concrete definitions, depending on the needs of the proof.
%
%
%\warningbox{One could expect the type annotations in the \rawec{extensionality}
%axiom to be optional. However, they are currently required so that the type
%variable can be instantiated.}
%
%\subsubsection{Operators}
%The language of operators and predicates is functional in style, and
%comma-separated lists of parameters are in fact currified, in this context, to
%yield functional symbols that can be partially applied. (For example, the infix
%extensional equality predicate \ec{(==)} defined in Listing~\ref{lst:arrays} has
%type \ec{'x array -> 'x array -> bool}.)
%
%\paragraph{Mixfix operators}
%Special syntax is used to introduce two hard-coded mixfix operators (generally
%considered to correspond to set and get operations on various types). The
%``\rawec[literate={\_}{{$\cdot$}}1]{_.[_]}'' operator (named \emph{get}
%throughout this manual) is hard-coded as a binary operator. The
%``\rawec[literate={_}{{$\cdot$}}1]{_.[_<-_]}'' operator (not yet encountered,
%and named \emph{set} in the rest of this manual) is hard-coded as a ternary
%operator. In addition, a constant operator ``\rawec{[]}'' can be defined, and
%will often be referred to as the \emph{empty} constant, depending on its type
%and the current scope.
%
%\paragraph{Infix operators}
%Infix operators are declared between parentheses, and can be used either in
%infix syntax using the symbol itself (for example, \rawec{x == y}), or in prefix
%syntax using their parenthesized form (for example, \rawec{(==) x y}).
%
%\subsection{Higher-Order Operators}
%Operators can be higher-order. For example, the code sample shown in
%Listing~\ref{lst:init_arrays} illustrates how one can axiomatically specify an
%operator which creates a fresh array and initializes its elements with
%index-dependent values. We illustrate its usage by defining an operator
%\ec{init} that, given an integer $n$, produces the array containing integers $0$
%through $n$, in ascending order.
%
%\begin{easycrypt}[label={lst:init_arrays}]{[Index-dependent array initializer]An operator initializing an array with index-dependent values}
%op init: (int -> 'x) -> int -> 'x array.
%
%axiom init_length: forall (f:int -> 'x) l,
%  0 <= l => length (init f l) = l.
%
%axiom init_get: forall (f:int -> 'x) l i,
%  0 <= l => 0 <= i => i < l =>
%  (init f l).[i] = f i.
%
%op first_ints n = init (lambda i, i) n.
%\end{easycrypt}
%
%The \rawec{lambda} notation works as expected. The argument types can be
%specified where necessary. Equality on lambda terms is extensional.

\section{Describing Distributions\label{sec:distributions}}

%%%% From Gilles:
\EC features a polymorphic type \rawec{'a distr} of \emph{discrete
sub-distributions} over a base type \rawec{'a}. The primary operation over
sub-distributions is the function \rawec{op mu: 'a distr -> ('a -> bool) ->
real.} which measures the probability of an event. The function is assumed to
satisfy the basic axioms of probability sub-distributions:
\begin{itemize}
\item probabilities lie in the unit interval;
\item the probability of the union of events is equal to the sum of
  their probabilities, minus the sum of their intersection;
\item probabilities are monotonic with respect to event inclusion;
\item two sub-distributions over the same base type are equal if and only if
  they are equal on each of the elements in the base type.
\end{itemize}

Sub-distributions are introduced axiomatically, and no well-formedness checks
are performed. For example, the uniform distribution on booleans is defined, in
the standard library, as displayed in Listing~\ref{lst:dbool}, where
\rawec{caract P x} is \rawec{1\%r} if \rawec{P} holds on \rawec{x} and
\rawec{0\%r} otherwise.

\begin{easycrypt}[label={lst:dbool}]{[Uniform Boolean distribution]Defining the uniform distribution on booleans}
op dbool: bool distr.

axiom mu_def: forall (p:bool -> bool), 
  mu dbool p =
    (1%r/2%r) * caract p true +
    (1%r/2%r) * caract p false.
\end{easycrypt}

The standard library on distributions, discussed further in
Chapter~\ref{chap:libraries}, introduces several auxiliary quantities on
distributions that are useful in probability computations and for defining more
complex distributions that are difficult, or impossible, to describe purely
using \rawec{mu}. The definitions of these quantities is shown in
Listing~\ref{lst:mu_aux}, where \rawec{cPtrue} is the constantly true predicate.
Note the use of partial application to the equality operator when defining
\rawec{mu_x}.

\begin{easycrypt}[label={lst:mu_aux}]{Auxiliary operators on distributions}
op mu_x(d:'a distr, x:'a) = mu d ((=) x).
op mu_weight(d:'a distr) = mu d cPtrue.
op in_supp(x:'a, d:'a distr) = 0%r < mu_x d x.
\end{easycrypt}

With these auxiliary operators, it is in fact fairly easy to check that the simple axiom from Listing~\ref{lst:dbool} does indeed define a distribution, by proving the following lemma, which is discharged automatically by the SMT solvers.

\begin{easycrypt}[]{}
lemma lossless : weight dbool = 1%r.
\end{easycrypt}

\subsection{Old Stuff}
\EC offers support for user-defined distributions. Distributions are defined by
formally defining their density function, represented by the built-in operator
\rawec{op mu: 'a distr -> ('a -> bool) -> real.} For example, the uniform
distribution on booleans is defined, in the standard library, as displayed in
Listing~\ref{lst:dbool}, where \rawec{caract P x} is \rawec{1\%r} if \rawec{P}
holds on \rawec{x} and \rawec{0\%r} otherwise.

It may not always be this easy to define distributions. More examples of
user-provided definitions, and some distribution transformers that make it
possible to define new distributions from existing ones, can be found in the
standard library (Chapter~\ref{chap:libraries}).

\warningbox{Distributions are defined axiomatically, and no well-formedness checks are performed.}

\section{Modules and Functors\label{sec:modules}}

So far, we have only considered features whose goal is to extend the language of
expressions and the semantic domain of values. Specifications of schemes,
oracles, assumptions and game-based security properties all use modules (and
module signatures) and functors, which we now discuss, using a simple modular
definition of a random oracle from bitstrings to bitstrings as an example.

We first formally define the functionalities a random oracle is expected to
provide, as a \emph{module type}, or \emph{signature}
(Listing~\ref{lst:modulesig}). Any module implementation \rawec{M} that provides
\emph{at least} the functions from a module type \rawec{Mt} is said to be of
type \rawec{Mt} (denoted \rawec{M :> Mt}). Module types cannot specify state
directly. Instead, if a global variable of the module should be exposed to the
outside, getter and setter functions can be added to the module signature.

\begin{easycrypt}[label={lst:modulesig}]{[A signature for random oracles]A signature for random oracles from bitstrings to bitstrings}
module type RO = {
  fun init():unit
  fun h(x:bitstring):bitstring }.
\end{easycrypt}

Such a module signature can then be given various realizations. For example, Listing~\ref{lst:modules} shows two possible realizations of a random oracle, both of which assume a positive integer constant \rawec{qH}, used to bound the number of calls to the oracle, and two positive integer constants \rawec{inLen} and \rawec{outLen} representing the input and output lengths of the random oracles.

%% The following is probably the single most disgusting thing I've ever typeset in Latex... and I've done ugly things.
\begin{minipage}{\textwidth}
\hrule
\begin{multicols}{2}
\begin{easycrypt}[frame=none,xleftmargin=0pt,xrightmargin=0pt]{}
module RO_L: RO = {
  var cG: int
  var mG: (bitstring,bitstring) map

  fun init() = {
    mG = empty;
    cG = 0;
  }

  fun h(x:bitstring) = {
    var r:bitstring;
    var res:bitstring = empty;
    if (length x = inLen && cG < qG)
    {
      cG = cG + 1;
      r = $dbitstring(outLen);
      if (!mem(x,dom mG)) mG[x] = r;
      res = mG[x];
    }
    return res;
  }
}.
\end{easycrypt}
\columnbreak
\begin{easycrypt}[frame=none,xleftmargin=0pt,xrightmargin=0pt]{}
module RO_E: RO = {
  var tape: bitstring list
  var mG: (bitstring,bitstring) map

  fun init() = {
    var r:bitstring;
    tape = [];
    mG = empty;
    while (length tape < qG)
    {
      r = $dbitstring(outLen);
      tape = r :: tape;
    }
  }

  fun h(x:bitstring) = {
    var r:bitstring;
    var res:bitstring = empty;
    if (length x = inLen &&
       length tape <> 0)
    {
      r = hd tape;
      if (!mem(x,dom mG))
        mG[x] = r;
      res = mG[x];
      tape = tl tape;
    }
    return res;
  }
}.
\end{easycrypt}
\end{multicols}
\hrule
\begin{easycrypt}[frame=non,xleftmargin=0pt,xrightmargin=0pt,label={lst:modules}]{[Two random oracles]Two possible implementations of module type \rawec{RO}}
\end{easycrypt}
\end{minipage}

\subsection{Functors and Oracle Annotations}
Modules can in fact be specified as functors, parameterized by other modules
whose functions can be accessed as oracles. For example, the signature shown in
Listing~\ref{lst:functorsig} describes a class of modules that are given access
to a random oracle but whose unique function can in fact only call the \rawec{h}
function provided by the parameter.

\begin{easycrypt}[label={lst:functorsig}]{[A functor signature]A functor signature}
module type RO_adv(O:RO) = {
  fun a(): bool { O.h }
}.
\end{easycrypt}

\section{Lemmas and Judgements}

\warningbox{Note that we use different implication symbols. $\Rightarrow$ is implication in the ambient logic (corresponding to the \rawec{=>} predicate), whereas $\Longrightarrow$ is used to express contracts on functions.\footnote{From Francois: it may be worth considering the use of another arrow symbol (for example $\Rrightarrow$) for contracts.}}

\subsection{Ambient Lemmas}

\subsection{(Possibilistic) Hoare Judgements}

Interpretation:
\begin{displaymath}
\Hoare{c}{\pre}{\post}
\quad\doteq\quad
\forall m.~ \pre\,m \Rightarrow \range Q ([\![ c ]\!]\,m)
\end{displaymath}
%
where $\range \varphi \mu \doteq \forall f.~ (\forall x.~ \varphi\,x
\Rightarrow f\,x=0) \Rightarrow \mu\,f=0$




\subsection{Probabilistic Hoare Judgements}

Interpretation:
\begin{displaymath}
\HoareLe{c}{\pre}{\post}{\delta} 
\quad\doteq\quad
\forall m.~ \pre\,m \Rightarrow [\![ c ]\!]\,m\,\charfun_\post \leq
\delta\,m
\end{displaymath}

Property?:
\begin{displaymath}
\Hoare{c}{\pre}{\post}
\Leftrightarrow
\HoareEq{c}{\pre}{\neg\post}{0}
\end{displaymath}

Property?:
\begin{displaymath}
\Hoare{c}{\pre}{\post} \land \HoareEq{c}{\pre}{\true}{1}
\Leftrightarrow
\HoareEq{c}{\pre}{\post}{1}
\end{displaymath}


\subsection{Relational Hoare Judgements}

\section{Requiring, Cloning and Realizing Theories\label{sec:cloning}}




%%% Local Variables: 
%%% mode: latex
%%% TeX-master: "easycrypt"
%%% End: 


% Tactics: First-Order and pRHL Tactics
\chapter{Tactics}
\label{chap:tactics}

Proofs in \EasyCrypt are carried out using \emph{tactics}, logical
rules embodying general reasoning principles, which transform the
current lemma (or \emph{goal}) into zero or more
\emph{subgoals}---sufficient conditions for the lemma/goal to
hold. Simple ambient logic goals may be automatically proved using SMT
solvers.

In this chapter, we introduce \EasyCrypt's proof engine, before
describing the tactics for \EasyCrypt's four logics: ambient, \prhl,
\phl and \hl.

\section{Proof Engine}

\EasyCrypt's proof engine works with goal lists, where a \emph{goal}
has three parts:
\begin{itemize}
\item A set of type variables.

\item An \emph{ordered} set of \emph{assumptions}, consisting of
  identifiers with their types, memories, module names with their
  module types and restrictions, and named formulas. An identifier's
  type may involve the type variables, the formulas may involve the
  type variables, identifiers, memories and module names.

\item A \emph{conclusion}, consisting of a single formula, with
  the same constraints as the assumption formulas.
\end{itemize}
Informally, to prove a goal, one must show the conclusion to be true,
given the truth of the formula assumptions, for all valid instantiations
of the assumption identifiers, memories and module names.

For example,
\ecinput{examps/parts/tactics-examp0-2.1.ec}{}{}{}{}
is a goal.
And, in the context of the declarations
\ecinput{examps/tactics-examp2.ec}{}{3-11}{}{}
this is a goal:
\ecinput{examps/parts/tactics-examp2-2.0.ec}{}{}{}{}
The conclusion of this goal is just a nonlinear rendering of the formula
\begin{easycrypt}{}{}
phoare [G(X).g : G.x = n ==> G.x = n] = 1%r.
\end{easycrypt}
\EasyCrypt's pretty printer renders \prhl, \phl and \hl judgements
in such a nonlinear style when the judgements appear as
(as opposed to in) the conclusions of goals.

Internally, \EasyCrypt's proof engine also works with \prhl, \phl and
\hl judgments involving lists of statements rather than procedure
names, which we'll call \emph{statement judgements}, below. For
example, given this declaration
\ecinput{examps/tactics-examp1.ec}{}{3-9}{}{}
this is an \phl statement judgement:
\ecinput{examps/parts/tactics-examp1-3.0.ec}{}{}{}{}
The pre- and post-conditions of a statement judgement may refer to the
parameters and local variables of the procedure that's the
\emph{context} of the conclusion---\ec{M.f} in the preceding
example. They may also refer to the memories \ec{&$1$} and \ec{&$2$}
in the case of \prhl statement judgements.
When a statement judgement appears anywhere other than as the conclusion of
a goal, the pretty printer renders it in abbreviated linear syntax.
E.g., the preceding goal is rendered as
\begin{easycrypt}{}{}
hoare[if (x %% 3 = 1) {...} : x %% 3 = n ==> x %% 3 = n %% 2 + 1]
\end{easycrypt}
Statement judgements can't be directly input by the user.

When the proof of a lemma is begun, the proof engine starts out with
a single goal, consisting of the lemma's statement. E.g.,
the lemma
\ecinput{examps/tactics-examp0.ec}{}{1-3}{}{}
gives rise to the goal
\ecinput{examps/parts/tactics-examp0-1.0.ec}{}{}{}{}
For parameterized lemmas, the goal includes the lemma's parameters
as assumptions. E.g.,
\begin{easycrypt}{}{}
lemma PairEq (x x' : 'a) (y y' : 'b) :
  x = x' => y = y' => (x, y) = (x', y').
\end{easycrypt}
gives rise to
\ecinput{examps/parts/tactics-examp0-2.0.ec}{}{}{}{}

\EasyCrypt's tactics, when applicable, reduce the first goal to zero
or more subgoals.  E.g., if the first goal is
\ecinput{examps/parts/tactics-examp1-3.0.ec}{}{}{}{}
then applying the \ec{if} tactic (handle a conditional) reduces
(replaces) this goal with the two goals
\ecinput{examps/parts/tactics-examp1-3.1.ec}{}{}{}{} and
\ecinput{examps/parts/tactics-examp1-3.2.ec}{}{}{}{}
(leaving the remaining goals, if any, unchanged).
If the first goal is
\ecinput{examps/parts/tactics-examp1-6.0.ec}{}{}{}{}
then applying the \ec{smt} tactic (try to solve the goal using
SMT provers) solves the goal, i.e., replaces it with no subgoals.
Applying a tactic may fail; in this case an error message is issued
and the list of goals is left unchanged.

A lemma's proof may be saved, using the step \ec{qed}, when the list
of goals becomes empty. And this must be done before anything else may
be done.

\section{Ambient logic}
\label{sec:ambientlogic}

In this section, we describe the proof terms, tactics and tacticals of
\EasyCrypt's ambient logic.

\subsection{Proof Terms}
\label{subsec:proofterms}

Formulas introduce identifier and formula assumptions using universal
quantifiers and implications. For example, the formula
\begin{easycrypt}{}{}
forall (x y : bool), x = y => forall (z : bool), y = z => x = z.
\end{easycrypt}
introduces the assumptions
\begin{easycrypt}{}{}
x     : bool
y     : bool
eq_xy : x = y
z     : bool
eq_yz : y = z
\end{easycrypt}
(where the names of the two formulas were chosen to be meaningful),
and has \ec{x = z} as its conclusion. We refer to the first assumption
of a formula as the formula's \emph{top assumption}. E.g., the top
assumption of the preceding formula is \ec{x : bool}.

\EasyCrypt has \emph{proof terms}, which partially describe how
to prove a formula.  Their syntax is described in Figure~\ref{fig:proofterms},
where $X$ ranges over lemma (or formula assumption) names.
\begin{figure}
  \begin{center}
  \begin{tabular}{rcl>{\bf}l}
    $p$ & ::=
      & {\ec{_}} & proof hole \\
     && {\ec{($X$, $\;q_1$, $\;\ldots$, $\;q_n$)}} & lemma application \\[.2cm]
    $q$ & ::=
      & {$e$} & expression \\
      && {$p$} & proof term \\
  \end{tabular}
  \end{center}
  \caption{\label{fig:proofterms} \EasyCrypt's Proof Terms}
\end{figure}
A proof term for a lemma (or formula assumption) $X$ has components
corresponding to the assumptions introduced by $X$.  A component
corresponding to a variable consists of an expression of the
variable's type. The proof term is explaining how the instantiation of
the lemma's conclusion with these expressions may be proved from the
instantiation of the formula components.  A formula component consists
of a proof term explaining how the instantiation of the formula
component may be proved.  Proof holes will get turned into subgoals
when a proof term is applied, e.g., by the \ec{apply} tactic.

Consider, e.g., the following declarations and axioms
\begin{easycrypt}{}{}
pred P : int.
pred Q : int.
pred R : int.
axiom P (x : int) : P x.
axiom Q (x : int) : P x => Q x.
axiom R (x : int) : P(x + 1) => Q x => R x.
\end{easycrypt}
Then, given that \ec{x : int} is an assumption,
\begin{easycrypt}{}{}
(R x (P(x + 1)) (Q x (P x)))
\end{easycrypt}
is a proof term proving the conclusion \ec{R x}. And
\begin{easycrypt}{}{}
(R x _ (Q x _))
\end{easycrypt}
is a proof term that, when applied to a goal with conclusion \ec{R x},
will generated subgoals with conclusions \ec{P(x + 1)} and \ec{P x},
corresponding to the first and second occurrences of \ec{_}.  When
using a proof term at the top-level, e.g., via \ec{apply}, one may
abbreviate some or all of its expressions to \ec{_}, letting
\EasyCrypt infer the expressions from the conclusion of the goal at
hand. Going even further, one may abbreviate a single-level proof term
with lemma $X$ to just $X$, e.g., writing \ec{R} for \ec{(R _ _ _)}.
Applying \ec{R} to \ec{R x} will generate two subgoals: \ec{P(x + 1)}
and \ec{Q x}. Below, we'll consider such abbreviates to \emph{be} proof
terms.

\subsection{Occurrence Selectors and Rewriting Directions}
\label{subsec:occsels}

Some ambient logic tactics use \emph{occurrence selectors} to restrict
their operation to certain occurrences of a term or formula in a
goal's conclusion or formula assumption. The syntax is \ec{\{$i_1$,
  $\;\ldots$, $\;i_n$\}}, specifying that only occurrences $i_1$
throught $i_n$ of the term/formula in a depth-first, left-to-right
traversal of the goal's conclusion or formula assumption should be
operated on. Specifying \ec{\{- $i_1$, $\;\ldots$, $\;i_n$\}}
restricts attention to all occurrences \emph{not} in the following
list. They may also be empty, meaning that all applicable occurrences
should be operated on.

Some ambient logic tactics uses \emph{rewriting directions}, $\mathit{dir}$,
which may either be empty (meaning rewriting from left to right), or \ec{-},
meaning rewriting from right to left.

\subsection{Introduction Tactical  and Generalization}
\label{subsec:intropatterns}

One moves the assumptions of a goal's conclusions into the goal's
ordered set of assumptions using the introduction tactical.
This tactical uses introduction patterns, $\iota$, which are defined
in Figure~\ref{fig:intropat}.
\begin{figure}
  \begin{center}
  \begin{tabular}{rcl>{\bf}l}
    $\iota$ & ::=
      & {$b$} & name \\
      && {\ec{$b\,$!}} & rename \\
      && {\ec{_}} & no name \\
      && {\ec{+}} & auto revert \\
      && {\ec{?}} & find name \\
      && {$\mathit{occ}\;$\ec{->}} & rewrite using assumption \\
      && {$\mathit{occ}\;$\ec{<-}} & rewrite in reverse using assumption \\
      && {\ec{->>}} & substitute using assumption \\
      && {\ec{<<-}} & substitute in reverse using assumption \\
      && {\ec{/$p$}} & replace assumption by application of proof term to it \\
      && {\ec{\{$b_1\cdots b_n$\}}} & clear introduced assumptions \\
      && {\ec{/=}} & simplify \\
      && {\ec{//}} & trivial \\
      && {\ec{//=}} & simplify then trivial \\
      && {\ec{$\mathit{dir}\;\mathit{occ}\;$@/$\mathit{op}$}} & unfold definition of operator \\
      && {\ec{[$\iota_{11}\cdots\iota_{1{m_1}}$ | $\;\cdots$ | $\;\iota_{r1}\cdots\iota_{r{m_r}}$]}} & case pattern \\[.2cm]
    $\mathit{b}$ & ::=
      & {$x$} & identifier \\
      && {$M$} & module name \\
      && {\ec{&$m$}} & memory name \\
  \end{tabular}
  \end{center}
  \caption{\label{fig:intropat} \EasyCrypt's Proof Terms}
\end{figure}
In this definition, $\mathit{occ}$ ranges over occurrence selectors,
and $\mathit{dir}$ ranges over directions---see
see Subsection~\ref{subsec:occsels}).

\begin{tactic}[$t$ =>$\;\iota_1 \cdots \iota_n$]{introduction}
  \begin{tsyntax}[empty]{t=> ip1 ... ipn}
    Runs the tactic $t$, matching the resulting goals, $G_1,\ldots,G_l$,
    with the introduction patterns $\iota_1,\ldots,\iota_n$:
    \begin{itemize}
    \item  If $l=0$, the tactical produces no subgoals.

    \item Otherwise, if $n=0$, i.e., there are no remaining
      introduction patterns, the subgoals of the tactical are the
      $G_i$.

    \item Otherwise, if $\iota_1$ is not a case pattern, each subgoal
      $G_i$ is matched against $\iota_1,\ldots,\iota_n$ by the
      procedure described below, with the resulting subgoals being
      collected into a list of goals (maintaining order viz a viz the
      indices $i$) as the tactical's result.

    \item Otherwise, $\iota_1$ is a case pattern
          \ec{[$\iota_{11}\cdots\iota_{1{m_1}}$ | $\;\cdots$ | $\;\iota_{r1}\cdots\iota_{r{m_r}}$]}.

     \item If $t$ is not equivalent to \rtactic{idtac}, the tactic fails
       unless $r = l$, in which case each $G_i$ is matched against 
        \begin{displaymath}
          \iota_{i1}\cdots\iota_{i{m_i}}\iota_2\cdots \iota_n
        \end{displaymath}
       by the procedure described below, with the resulting subgoals
       being collected into the tactical's result.

     \item Otherwise, $t$ is equivalent to \rtactic{idtac} (and so
       $l=1$). In this case $G_1$ is matched against
       $\iota_1,\ldots,\iota_n$ by the procedure described below, with
       the resulting subgoals being collected into a list of goals as
       the tactical's result.
    \end{itemize}

    \paragraph{Matching a single goal against a list of patterns:}

    The introduction patterns are processed from left-to-right, as
    follows:
    \begin{itemize}
    \item ($b$)\quad The top assumption (universally quantified
      identifier, module name or memory; or left side of implication)
      is consumed, and introduced with this name. Fails if the top
      assumption has neither of these forms.

    \item (\ec{$b\,$!})\quad Same as the preceding case, except that
      $b$ is used as the base of the introduced name, extending the
      base to avoid naming conflicts.

    \item (\ec{_})\quad Same as the preceding case, except the
      assumption is introduced with an anonymous name (which can't be
      uttered by the user).

    \item (\ec{+})\quad Same as the preceding case, except that after
      a branch of the procedure completes, yielding a goal, the
      assumption will be reverted, i.e., un-introduced (using a
      universal quantifier or implication as appropriate).

    \item (\ec{?})\quad Same as the preceding case, except \EasyCrypt
      chooses the name by which the assumption is introduced (using
      universally quantified names as assumption bases).

    \item ($\mathit{occ}\;$\ec{->})\quad Consume the top assumption,
      which must be an equality, and use it as a left-to-right rewriting
      rule in the remainder of the goal's conclusion, restricting rewriting
      to the specified occurrences of the equality's left side.

    \item ($\mathit{occ}\;$\ec{<-})\quad The same as the preceding case,
      except the rewriting is from right-to-left.

    \item (\ec{->>})\quad The same as \ec{->}, except the consumed
      equality assuption is used to perform a left-to-right substitution
      in the entire goal, i.e., in its assumptions, as well as its
      conclusion.

    \item (\ec{<<-})\quad The same as the preceding case, except
      the substitution is from right-to-left.

    \item (\ec{/$p$})\quad Replace the top assumption by the result
    of applying the proof term $p$ to it using forward reasoning.

    \item (\ec{\{$b_1\cdots b_n$\}})\quad Doesn't affect the goal's
      conclusion, but clears the specified assumptions, i.e., removes
      them. Fails if one or more of the assumptions can't be cleared,
      because a remaining assumption depends upon it.

    \item (\ec{/=})\quad Apply \rtactic{simplify} to goal's conclusion.

    \item (\ec{//})\quad Apply \rtactic{trivial} to goal's conclusion;
      this may solve the goal, i.e., so that the procedure's current
      branch yields no goals.

    \item (\ec{/=})\quad Apply \rtactic{simplify} and then \rtactic{trivial}
      to goal's conclusion; this may solve the goal, so that the
      procedure's current branch yields no goals.

    \item ({\ec{$\mathit{dir}\;\mathit{occ}\;$@/$\mathit{op}$}})\quad
      Unfold (fold, if the direction is \ec{-}) the definition of
      operator $\mathit{op}$ at the specified occurrences of the
      goal's conclusion.

    \item (\ec{[$\iota_{11}\cdots\iota_{1{m_1}}$ | $\;\cdots$ | $\;\iota_{r1}\cdots\iota_{r{m_r}}$]})\quad
      \begin{itemize}
      \item If $r=0$, then the top assumption of the goal is destructed
        using the \rtactic{case} tactic, the resulting goals are
        matched against $\iota_2,\ldots,\iota_n$, and their subgoals
        are assembled into a list of goals.

      \item Otherwise $r>0$. The goal's top assumption is destructed
        using the \rtactic{case} tactic, yielding subgoals
        $H_1,\ldots H_p$.  If $p\neq r$, the procedure fails. Otherwise
        each subgoal $H_i$ is matched against
        \begin{displaymath}
          \iota_{i1}\cdots\iota_{i{m_i}}\iota_2\cdots \iota_n
        \end{displaymath}
        with the resulting goals being collected into a list as
        the procedure's result.
      \end{itemize}
    \end{itemize}

    The following examples use the tactic \rtactic{move}, which is
    equivalent to \rtactic{idtac}.
    In its simplest form, the introduction tactical simply gives names
    to assumptions.  For example, if the current goal is
    \ecinput{examps/parts/tactics/introduction/1-1.0.ec}{}{}{}{}
    then running
    \ecinput{examps/parts/tactics/introduction/1-1.ec}{}{}{}{}
    produces
    \ecinput{examps/parts/tactics/introduction/1-1.1.ec}{}{}{}{}
    Alternatively, we can use the introduction pattern \ec{?}
    to let \EasyCrypt choose the assumption names, using
    \ec{H} as a base for formula assumptions and starting
    from the identifier names given in universal quantifiers:
    \ecinput{examps/parts/tactics/introduction/2-1.ec}{}{}{}{}
    produces
    \ecinput{examps/parts/tactics/introduction/2-1.1.ec}{}{}{}{}
    Or we can use the \ec{!} pattern suffix to specify our
    own base assumption names: running
    \ecinput{examps/parts/tactics/introduction/3-1.ec}{}{}{}{}
    produces
    \ecinput{examps/parts/tactics/introduction/3-1.1.ec}{}{}{}{}

    To see how the \ec{->} rewriting pattern works, suppose
    the current goal is
    \ecinput{examps/parts/tactics/introduction/4-1.0.ec}{}{}{}{}
    Then running
    \ecinput{examps/parts/tactics/introduction/4-1.ec}{}{}{}{}
    produces
    \ecinput{examps/parts/tactics/introduction/4-1.1.ec}{}{}{}{}
    Alternatively, one can introduce the assumption \ec{x = y},
    and then use the \ec{->>} substitution pattern:
    if the current goal is
    \ecinput{examps/parts/tactics/introduction/8-1.0.ec}{}{}{}{}
    then running
    \ecinput{examps/parts/tactics/introduction/8-1.ec}{}{}{}{}
    produces
    \ecinput{examps/parts/tactics/introduction/8-1.1.ec}{}{}{}{}

    To see how a view may be applied to a not-yet-introduced formula
    assumption, suppose the current goal is
    \ecinput{examps/parts/tactics/introduction/5-1.0.ec}{}{}{}{}
    Then running
    \ecinput{examps/parts/tactics/introduction/5-1.ec}{}{}{}{}
    produces
    \ecinput{examps/parts/tactics/introduction/5-1.1.ec}{}{}{}{}
    And then running
    \ecinput{examps/parts/tactics/introduction/5-2.ec}{}{}{}{}
    on this goal produces
    \ecinput{examps/parts/tactics/introduction/5-2.1.ec}{}{}{}{}

    Finally, let's see examples of how a disjunction assumption
    may be destructed, either using the \ec{case} tactic followed
    by a case introduction pattern, or by making the
    case introduction pattern do the destruction.
    For the first case, if the current goal is
    \ecinput{examps/parts/tactics/introduction/6-1.0.ec}{}{}{}{}
    then running
    \ecinput{examps/parts/tactics/introduction/6-1.ec}{}{}{}{}
    produces the two goals
    \ecinput{examps/parts/tactics/introduction/6-1.1.ec}{}{}{}{}
    and
    \ecinput{examps/parts/tactics/introduction/6-1.2.ec}{}{}{}{}
    And for the second case, if the current goal is
    \ecinput{examps/parts/tactics/introduction/7-1.0.ec}{}{}{}{}
    then running
    \ecinput{examps/parts/tactics/introduction/7-1.ec}{}{}{}{}
    produces the two goals
    \ecinput{examps/parts/tactics/introduction/7-1.1.ec}{}{}{}{}
    and
    \ecinput{examps/parts/tactics/introduction/7-1.2.ec}{}{}{}{}
    Note how we used the clear pattern to discard the assumption
    \ec{X}.
  \end{tsyntax}
\end{tactic}

Generalization moves assumptions back into a goal's conclusion:

\begin{tactic}[$t$: $\;\pi_1 \cdots \pi_n$]{generalization}
  \begin{tsyntax}[empty]{$t$: $\;\pi_1 \cdots \pi_n$}
    Generalize the patterns $\pi_1, \cdots, \pi_n$, starting from
    $\pi_n$ and going back, and then run tactic $t$.

    \begin{itemize}
    \item If a pattern $\pi$ is an assumption, it's moved back into
      the conclusion, using universal quantification or an
      implication, as appropriate. If one assumption depends on
      another, one can't generalize the later without also
      generalizing the former.

      For example, if the current goal is
      \ecinput{examps/parts/tactics/generalize/2-1.0.ec}{}{}{}{} then
      running \ecinput{examps/parts/tactics/generalize/2-1.ec}{}{}{}{}
      produces
      \ecinput{examps/parts/tactics/generalize/2-1.1.ec}{}{}{}{} In
      this example, one can't generalize \ec{x} without also
      generalizing \ec{eq_xy}.

    \item A pattern $\pi$ may also be a subformula or subterm of the
      goal, or \ec{_}, which stands for the whole goal, possibly
      prefixed by an occurrence selector. This replaces the formula or
      subterm with a universally quantified identifier of the
      approprate type.

      For example, if the current goal is
      \ecinput{examps/parts/tactics/generalize/1-1.0.ec}{}{}{}{} then
      running \ecinput{examps/parts/tactics/generalize/1-1.ec}{}{}{}{}
      produces
      \ecinput{examps/parts/tactics/generalize/1-1.1.ec}{}{}{}{}
      Alternatively, running
      \ecinput{examps/parts/tactics/generalize/3-1.ec}{}{}{}{}
      produces
      \ecinput{examps/parts/tactics/generalize/3-1.1.ec}{}{}{}{}
    \end{itemize}
  \end{tsyntax}
\end{tactic}

\subsection{Ambient Logic Tactics}

% --------------------------------------------------------------------
\begin{tactic}{idtac}
\end{tactic}

% --------------------------------------------------------------------
\begin{tactic}[move | move: $\;\pi_1 \cdots \pi_n$]{move}
  \begin{tsyntax}{move}
     Does nothing, equivalent to \rtactic{idtac}. This form is mainly
     used in conjonction with an introduction pattern (see
     Section~\ref{s:intro-pattern}), e.g. \ls!move=> $\iota_1 \cdots \iota_n$!.
  \end{tsyntax}

  \begin{tsyntax}{move: $\;\pi_1 \cdots \pi_n$}
    Generalize the patterns $\pi_1, \cdots, \pi_n$, starting from
    $\pi_n$ and going back.
    %See Section~\ref{s:gen-pattern} for more
    %information on the generalization mechanism.
  \end{tsyntax}
\end{tactic}

% --------------------------------------------------------------------
\begin{tactic}{clear}
\end{tactic}

% --------------------------------------------------------------------
\begin{tactic}{done}
  \begin{tsyntax}[empty]{done}
  \fix{Missing description of done}.
  \end{tsyntax}
\end{tactic}

% --------------------------------------------------------------------
\begin{tactic}[apply (p : proof-term)]{apply}
  \begin{tsyntax}[empty]{apply (p : proof-term)}
   If \tct{p} is a proof-term for the pattern (formula)
  \begin{center}
    \tct{forall (x1 : t1) ... (xn : tn), A1 -> ... -> An -> B}
  \end{center}
  \noindent then \tct{apply} tries to match B with the current G. If the
  match succeeds and leads to the full instantiation of the pattern,
  then the goal is replaced, after instantiation, with the $n$ subgoals
  \tct{A1, ..., An}.
  \end{tsyntax}
\end{tactic}

% --------------------------------------------------------------------
\begin{tactic}{exact (p : proofterm)}
  \begin{tsyntax}[empty]{exact}
  Equivalent to \ec{by apply (p : proofterm)}, i.e. apply the given
  proof-term and the try to close the goals with \ec{trivial} - failing
  if not all goals can be closed.
  \end{tsyntax}
\end{tactic}

% --------------------------------------------------------------------
\begin{tactic}{assumption}
  \begin{tsyntax}[empty]{assumption}
    Search in the context for a hypothesis that is convertible to the
    goal's conclusion, solving the goal if one is found. Fail if none
    can be found.

    For example, if the current goal is
    \ecinput{../examps/parts/tactics/assumption/1-1.0.ec}{}{}{}{} then
    running
    \ecinput{../examps/parts/tactics/assumption/1-1.ec}{}{}{}{} solves
    the goal.
  \end{tsyntax}
\end{tactic}

% --------------------------------------------------------------------
\begin{tactic}[pose x := $\;\pi$]{pose}
  \begin{tsyntax}[empty]{pose}
  Search for the first subterm \ec{p} of the goal matching $\pi$ and
  leading to the full instantiation of the pattern. Then introduce,
  after instantiation, the local definition \ec{x := p} and abstract
  all occurrences of \ec{p} in the goal as \ec{x}. An occurence
  selector can be used (see \rtactic{rewrite}).
  \end{tsyntax}
\end{tactic}

\begin{tactic}[have $\;\iota$: $\;\phi$]{have}
  \begin{tsyntax}[empty]{have}
    Logical cut. Generate two subgoals: one for the cut formula
    $\phi$, and one for \ec{$\phi$ => $\;\psi$} where $\psi$ is the
    current goal's conclusion. Moreover, the introduction pattern
    \ec{$\iota$} is applied to the second subgoal.

  For example, if the current goal is
  \ecinput{../examps/parts/tactics/have/1-1.0.ec}{}{}{}{} then
  running \ecinput{../examps/parts/tactics/have/1-1.ec}{}{}{}{}
  produces the goals
  \ecinput{../examps/parts/tactics/have/1-1.1.ec}{}{}{}{}
  and
  \ecinput{../examps/parts/tactics/have/1-1.2.ec}{}{}{}{}
  \end{tsyntax}

  \begin{tsyntax}{have $\;\iota$: $\;\phi$ by $\;\tau$}
  Attempts to use tactic $\tau$ to close the first subgoal (corresponding to
  the cut formula $\phi$), and fails if impossible.
  \end{tsyntax}
\end{tactic}


% --------------------------------------------------------------------
\begin{tactic}[cut $\;\iota$: $\;\phi$]{cut}
  \begin{tsyntax}[empty]{cut}
  Logical cut. Generate two subgoals: one for the cut formula $\phi$,
  and one for $\phi \Rightarrow G$ where $G$ is the current goal. Moreover,
  the intro-pattern \tct{$\iota$} is applied to the second subgoal.
  \end{tsyntax}
\end{tactic}


% --------------------------------------------------------------------
\begin{tactic}[rewrite $\;\pi_1 \cdots \pi_n$ | rewrite $\;\pi_1 \cdots \pi_n$ in $\;H$]{rewrite}
  \begin{tsyntax}{rewrite $\;\pi_1 \cdots \pi_n$}
  Rewrite the rewrite-pattern $\pi_1 \cdots \pi_n$ from left to right,
  where the $\pi_i$ can be of the following form:
  \begin{itemize}
  \item one of \ec{//}, \ec{/=}, \ec{//=},
  \item a proof-term $p$, or
  \item a pattern prefixed by \ec{/} (slash).
  \end{itemize}
  The two last forms can be prefixed by a direction indicator (the
  sign \ec{-}, see Subsection~\ref{subsec:occsels}), followed by an
  occurrence selector (see Subsection~\ref{subsec:occsels}), followed
  (for proof-terms only) by a repetition marker (\ec{!}, \ec{?},
  \ec{n!} or \ec{n?}). All these prefixes are optional.

  \smallskip
  Depending on the form of $\pi$, \ec{rewrite $\;\pi$} does the following:
    \begin{itemize}
    \item For \ec{//}, \ec{/=}, and \ec{//=}, see
      Subsection~\ref{subsec:introgen}.

    \item If $\pi$ is a proof-term with conclusion $f_1=f_2$, then
      \ec{rewrite} searches for the first subterm of the goal's
      conclusion matching $f_1$ and resulting in the full
      instantiation of the pattern.  It then replaces, after
      instantiation of the pattern, all the occurrences of $f_1$ by
      $f_2$ in the goal's conclusion, and creates new subgoals for the
      instantiations of the assuptions of $p$.  If no subterms of the
      goal's conclusion match $f_1$ or if the pattern cannot be fully
      instantiated by matching, the tactic fails.  The tactic works
      the same if the pattern ends by \ec{$f_1\!$ <=> $\;f_2$}. If the
      direction indicator \ec{-} is given, \ec{rewrite} works in the
      reverse direction, searching for a match of $f_2$ and then
      replacing all occurrences of $f_2$ by $f_1$.

    \item If $\pi$ is a \ec{/}-prefixed pattern of the form
      $o\,p_1\,\cdots\,p_n$, with $o$ a defined symbol, then
      \ec{rewrite} searches for the first subterm of the goal's
      conclusion matching $o\,p_1\,\cdots\,p_n$ and resulting in the
      full instantiation of the pattern. It then replaces, after
      instantiation of the pattern, all the occurrences of
      $o\,p_1\,\cdots\,p_n$ by the $\beta\delta$ head-normal form of
      $o\,p_1\,\cdots\,p_n$, where the $\delta$-reduction is
      restricted to subterms headed by the symbol $o$. If no subterms
      of the goal's conclusion match $o\,p_1\,\cdots\,p_n$ or if the
      pattern cannot be fully instantiated by matching, the tactic
      fails. If the direction indicator \ec{-} is given, \ec{rewrite}
      works in the reverse direction, searching for a match of the
      $\beta\delta_o$ head-normal of $o\,p_1\,\cdots\,p_n$ and then
      replacing all occurrences of this head-normal form with
      $o\,p_1\,\cdots\,p_n$.
    \end{itemize}
    
    \smallskip
    The occurrence selector restricts which occurrences of the
    matching pattern are replaced in the goal's conclusion---see
    Subsection~\ref{subsec:occsels}.

    Repetition markers allow the repetition of the same rewriting. For
    instance, \ec{rewrite $\;\pi$} leads to \ec{do! rewrite
      $\;\pi$}. See the tactical \ec{do} for more information.
    
    Lastly, \ec{rewrite $\;\pi_1 \cdots \pi_n$} is equivalent to
    \ec{rewrite} $\;\pi_1$; ...; \ec{rewrite} $\;\pi_n$.

    \smallskip
    For example, if the current goal is
    \ecinput{examps/parts/tactics/rewrite/1-1.0.ec}{}{}{}{} then
    running \ecinput{examps/parts/tactics/rewrite/1-1.ec}{}{}{}{}
    produces \ecinput{examps/parts/tactics/rewrite/1-1.1.ec}{}{}{}{}
    from which
    running \ecinput{examps/parts/tactics/rewrite/1-2.ec}{}{}{}{}
    produces \ecinput{examps/parts/tactics/rewrite/1-2.1.ec}{}{}{}{}
    from which
    running \ecinput{examps/parts/tactics/rewrite/1-3.ec}{}{}{}{}
    produces \ecinput{examps/parts/tactics/rewrite/1-3.1.ec}{}{}{}{}
  \end{tsyntax}

  \begin{tsyntax}{rewrite $\;\pi_1 \cdots \pi_n$ in $\;H$} Like the
    preceding case, except rewriting is done in the hypothesis
    $H$ instead of in the goal's conclusion.  Rewriting using a proof
    term is only allowed when the proof term was defined globally
    or before the assumption $H$.

    For example, if the current goal is
    \ecinput{examps/parts/tactics/rewrite/2-1.0.ec}{}{}{}{} then
    running \ecinput{examps/parts/tactics/rewrite/2-1.ec}{}{}{}{}
    produces \ecinput{examps/parts/tactics/rewrite/2-1.1.ec}{}{}{}{}
  \end{tsyntax}
\end{tactic}

% --------------------------------------------------------------------
\begin{tactic}[subst | subst x]{subst}
  \begin{tsyntax}[empty]{subst}
  Search for the first equation of the form \ec{x = f} or \ec{f = x} in the context
  and replace all the occurrences of \ec{x} by \ec{f} everywhere in the context and the
  goal before clearing it. If no identifier is given, repeatedly apply the tactic to
  all identifiers for which such an equation exists.
  \end{tsyntax}
\end{tactic}


% --------------------------------------------------------------------
\begin{tactic}{split}
  \begin{tsyntax}[empty]{split}
  Break an intrinsically conjunctive goal into its component subgoals.
  For instance, it can:
  \begin{itemize}
  \item close any goal that is convertible to \ec{true} or provable by
    \ec{reflexivity},
    \item replace a logical equivalence by the direct and indirect implications,
    \item replace a goal of the form \ec{$\phi_1\!$ /\\ $\;\,\phi_2$} by the two
      subgoals for $\phi_1$ and $\phi_2$. The same applies for a goal of
      the form \ec{$\phi_1\!$ && $\;\,\phi_2$},
    \item replace an equality between $n$-tuples by $n$ equalities
          on their components.
  \end{itemize}

  For example, if the current goal is
  \ecinput{../examps/parts/tactics/split/1-1.0.ec}{}{}{}{} then
  running \ecinput{../examps/parts/tactics/split/1-1.ec}{}{}{}{}
  produces the goals
  \ecinput{../examps/parts/tactics/split/1-1.1.ec}{}{}{}{}
  and
  \ecinput{../examps/parts/tactics/split/1-1.2.ec}{}{}{}{}
  And if the current goal is
  \ecinput{../examps/parts/tactics/split/2-1.0.ec}{}{}{}{} then
  running \ecinput{../examps/parts/tactics/split/2-1.ec}{}{}{}{}
  produces the goals
  \ecinput{../examps/parts/tactics/split/2-1.1.ec}{}{}{}{}
  and
  \ecinput{../examps/parts/tactics/split/2-1.2.ec}{}{}{}{}
  Repeating the last example with \ec{&&} rather than \ec{/\\},
  if the current goal is
  \ecinput{../examps/parts/tactics/split/2a-1.0.ec}{}{}{}{} then
  running \ecinput{../examps/parts/tactics/split/2a-1.ec}{}{}{}{}
  produces the goals
  \ecinput{../examps/parts/tactics/split/2a-1.1.ec}{}{}{}{}
  and
  \ecinput{../examps/parts/tactics/split/2a-1.2.ec}{}{}{}{}
  This illustrates the difference between \ec{/\\} and \ec{&&}.
  And if the current goal is
  \ecinput{../examps/parts/tactics/split/3-1.0.ec}{}{}{}{} then
  running \ecinput{../examps/parts/tactics/split/3-1.ec}{}{}{}{}
  produces the goals
  \ecinput{../examps/parts/tactics/split/3-1.1.ec}{}{}{}{}
  and
  \ecinput{../examps/parts/tactics/split/3-1.2.ec}{}{}{}{}
  \end{tsyntax}
\end{tactic}

% --------------------------------------------------------------------
\begin{tactic}{left}
  \begin{tsyntax}[empty]{left}
    Reduce a goal whose conclusion is a disjunction to one whose
    conclusion is its left member.

    For example, if the current goal is
    \ecinput{examps/parts/tactics/left/1-1.0.ec} then
    running \ecinput{examps/parts/tactics/left/1-1.ec}
    produces the goal
    \ecinput{examps/parts/tactics/left/1-1.1.ec}
  \end{tsyntax}
\end{tactic}

% --------------------------------------------------------------------
\begin{tactic}{right}
  \begin{tsyntax}[empty]{right}
  Reduce a disjunctive goal to its right member.

  For example, if the current goal is
  \ecinput{../examps/parts/tactics/right/1-1.0.ec}{}{}{}{} then
  running \ecinput{../examps/parts/tactics/right/1-1.ec}{}{}{}{}
  produces the goal
  \ecinput{../examps/parts/tactics/right/1-1.1.ec}{}{}{}{}
  \end{tsyntax}
\end{tactic}


% --------------------------------------------------------------------
\begin{tactic}[case $\;\phi$ | case]{case}
  \begin{tsyntax}{case $\;\phi$}
    Do an excluded-middle case analysis on $\phi$, substituting $\phi$
    in the goal's conclusion.

    For example, if the current goal is
    \ecinput{examps/parts/tactics/case/1-1.0.ec}{}{}{}{} then
    running \ecinput{examps/parts/tactics/case/1-1.ec}{}{}{}{}
    produces the goals
    \ecinput{examps/parts/tactics/case/1-1.1.ec}{}{}{}{}
    and
    \ecinput{examps/parts/tactics/case/1-1.2.ec}{}{}{}{}
  \end{tsyntax}

  \begin{tsyntax}{case}
    Destruct the top assumption of the goal's conclusion, generating
    subgoals that are dependent upon the kind of assumption
    destructed. \emph{This form of the tactic can be followed by
    the generalization tactical---see Subsection~\ref{subsec:introgen}.}

    \begin{itemize}
    \item (\textbf{conjunction})
    For example, if the current goal is
    \ecinput{examps/parts/tactics/case/2-1.0.ec}{}{}{}{} then
    running \ecinput{examps/parts/tactics/case/2-1.ec}{}{}{}{}
    produces the goal
    \ecinput{examps/parts/tactics/case/2-1.1.ec}{}{}{}{}
    \ec{&&} works identically.

    \item (\textbf{disjunction})
    For example, if the current goal is
    \ecinput{examps/parts/tactics/case/3-1.0.ec}{}{}{}{} then
    running \ecinput{examps/parts/tactics/case/3-1.ec}{}{}{}{}
    produces the goals
    \ecinput{examps/parts/tactics/case/3-1.1.ec}{}{}{}{}
    and
    \ecinput{examps/parts/tactics/case/3-1.2.ec}{}{}{}{}
    \ec{||} works identically.

    \item (\textbf{existential})
    For example, if the current goal is
    \ecinput{examps/parts/tactics/case/4-1.0.ec}{}{}{}{} then
    running \ecinput{examps/parts/tactics/case/4-1.ec}{}{}{}{}
    produces the goal
    \ecinput{examps/parts/tactics/case/4-1.1.ec}{}{}{}{}

    \item (\textbf{unit}) Substitutes \ec{tt} for the assumption in
      the remainder of the conclusion.

    \item (\textbf{bool})
    For example, if the current goal is
    \ecinput{examps/parts/tactics/case/5-1.0.ec}{}{}{}{} then
    running \ecinput{examps/parts/tactics/case/5-1.ec}{}{}{}{}
    produces the goals
    \ecinput{examps/parts/tactics/case/5-1.1.ec}{}{}{}{}
    and
    \ecinput{examps/parts/tactics/case/5-1.2.ec}{}{}{}{}

    \item (\textbf{product type})
    For example, if the current goal is
    \ecinput{examps/parts/tactics/case/6-1.0.ec}{}{}{}{} then
    running \ecinput{examps/parts/tactics/case/6-1.ec}{}{}{}{}
    produces the goal
    \ecinput{examps/parts/tactics/case/6-1.1.ec}{}{}{}{}

    \item (\textbf{inductive datatype})
    Consider the inductive datatype declaration:
\begin{easycrypt}{}{}
type tree = [Leaf | Node of bool & tree & tree].
\end{easycrypt}
    Then, if the current goal is
    \ecinput{examps/parts/tactics/case/7-1.0.ec}{}{}{}{} then
    running \ecinput{examps/parts/tactics/case/7-1.ec}{}{}{}{}
    produces the goals
    \ecinput{examps/parts/tactics/case/7-1.1.ec}{}{}{}{}
    and
    \ecinput{examps/parts/tactics/case/7-1.2.ec}{}{}{}{}
    \end{itemize}
  \end{tsyntax}
\end{tactic}

% --------------------------------------------------------------------
\begin{tactic}[elim | elim /$L$]{elim}
  \begin{tsyntax}{elim}
    Eliminates the top assumption of the goal's conclusion, generating
    subgoals that are dependent upon the kind of assumption
    eliminated. \emph{This tactic can be followed by
    the generalization tactical---see Subsection~\ref{subsec:introgen}.}

    \ec{elim} mostly works identically to \rtactic{case}, the exception
    being inductive datatype and the integers (for which a built-in
    induction principle is applied---see the other form).

    Consider the inductive datatype declaration:
\begin{easycrypt}{}{}
type tree = [Leaf | Node of bool & tree & tree].
\end{easycrypt}
    Then, if the current goal is
    \ecinput{examps/parts/tactics/elim/1-1.0.ec}
    running \ecinput{examps/parts/tactics/elim/1-1.ec}
    produces the goals
    \ecinput{examps/parts/tactics/elim/1-1.1.ec}
    and
    \ecinput{examps/parts/tactics/elim/1-1.2.ec}
  \end{tsyntax}

  \begin{tsyntax}{elim /$L$}
    Eliminates the top assumption of the goal's conclusion using the
    supplied induction principle lemma. \emph{This tactic can be
      followed by the generalization tactical---see
      Subsection~\ref{subsec:introgen}.}
    For example, consider the declarations
\begin{easycrypt}{}{}
type tree = [Leaf | Node of bool & tree & tree].
op rev (tr : tree) : tree =
  with tr = Leaf => Leaf
  with tr = Node b tr1 tr2 => Node b (rev tr1) (rev tr2).
\end{easycrypt}
and suppose we've already proved
\begin{easycrypt}{}{}
lemma IndPrin :
  forall (p : tree -> bool) (tr : tree),
  p Leaf =>
  (forall (b : bool) (tr1 tr2 : tree),
   p tr1 => p tr2 => p(Node b tr1 tr2)) =>
  p tr.
\end{easycrypt}
    Then, if the current goal is
    \ecinput{examps/parts/tactics/elim/2-1.0.ec}
    running \ecinput{examps/parts/tactics/elim/2-1.ec}
    produces the goals
    \ecinput{examps/parts/tactics/elim/2-1.1.ec}
    and
    \ecinput{examps/parts/tactics/elim/2-1.2.ec}
  \end{tsyntax}

\smallskip
When we consider the \ec{Int} theory in Chapter~\ref{chap:library},
we'll discuss the induction principle on the integers.
\end{tactic}


% --------------------------------------------------------------------
\begin{tactic}{simplify}
  \begin{tsyntax}[empty]{simplify}
  \fix{Missing description of simplify}.
  \end{tsyntax}
\end{tactic}

% --------------------------------------------------------------------
\begin{tactic}[change $\;\phi$]{change}
  \begin{tsyntax}[empty]{change $\;\phi$}
  Change the current goal for $\phi$ --- $\phi$ must be \emph{convertible}
  to the current goal.

  For example, if the current goal is
  \ecinput{../examps/parts/tactics/change/1-1.0.ec}{}{}{}{} then
  running \ecinput{../examps/parts/tactics/change/1-1.ec}{}{}{}{}
  produces the goal
  \ecinput{../examps/parts/tactics/change/1-1.1.ec}{}{}{}{}
  \end{tsyntax}
\end{tactic}

% --------------------------------------------------------------------
\begin{tactic}[progress | progress $\;\tau$]{progress}
  \begin{tsyntax}[empty]{progress}
  Break the goal into multiple \emph{simpler} ones by repeatedly applying
  \ec{split}, \ec{subst} and \ec{move=>}. The tactic $\tau$ given to
  \ec{progress} is tentatively applied after each step.

  For example, if the current goal is
  \ecinput{examps/parts/tactics/progress/1-1.0.ec}{}{}{}{} then
  running \ecinput{examps/parts/tactics/progress/1-1.ec}{}{}{}{}
  solves the goal.
  \end{tsyntax}

  \fix{Describe \ec{progress} options.}
\end{tactic}


% --------------------------------------------------------------------
\begin{tactic}{reflexivity}
  \begin{tsyntax}[empty]{reflexivity}
  Solve goals of the form \ec{b = b} (up to computation).

  For example, if the current goal is
  \ecinput{../examps/parts/tactics/reflexivity/1-1.0.ec}{}{}{}{} then
  running \ecinput{../examps/parts/tactics/reflexivity/1-1.ec}{}{}{}{}
  solves the goal.
  \end{tsyntax}
\end{tactic}

% --------------------------------------------------------------------
\begin{tactic}{congr}
  \begin{tsyntax}[empty]{congr}
  Replace a goal of the form \ec{f t$_1$ ... t$_n$ = f u$_1$ ... u$_n$}
  with the subgoals \ec{t$_i$ = u$_i$} for all \ec{$i$}. Subgoals solvable
  by \ec{reflexivity} are automatically closed.
  \end{tsyntax}
\end{tactic}

% --------------------------------------------------------------------
\begin{tactic}{algebra}
\end{tactic}


% --------------------------------------------------------------------
\begin{tactic}{trivial}
  \begin{tsyntax}[empty]{trivial}
  Try to solve the goal by using a mixture of low-level tactics.
  This tactic is called by the intro-pattern \ec{//}.
  \end{tsyntax}
\end{tactic}

% --------------------------------------------------------------------
\begin{tactic}{smt}
  \begin{tsyntax}[empty]{smt}
  \fix{Missing description of smt}.
  \end{tsyntax}
\end{tactic}


% --------------------------------------------------------------------
\begin{tactic}{admit}
  \begin{tsyntax}[empty]{admit}
  Close the current goal by admitting it.

  For example, if the current goal is
  \ecinput{examps/parts/tactics/admit/1-1.0.ec} then
  running \ecinput{examps/parts/tactics/admit/1-1.ec}
  solves the goal.
  \end{tsyntax}
\end{tactic}


\section{Tacticals}
\label{sec:tacticals}

Tactics can be combined together, composed and modified by
\emph{tacticals}. Tacticals do not correspond to any deduction rule
but make the proof process smoother, and sometimes permit the reuse of
proofs with similar patterns, but where the fine minutiae might
differ.

\begin{tactic}[t1; t2]{sequence}
  \begin{tsyntax}[empty]{t1; t2}
  Execute \ec{t1} and then \ec{t2} on all the subgoals generated by \ec{t1}.
  \end{tsyntax}
\end{tactic}

\begin{tactic}[try t]{failure recovery}\label{tactic-try}
  \begin{tsyntax}[empty]{try t}
  Execute the tactic \ec{t} if it succeeds; do nothing if it fails.

  \paragraph{Remark.}
  By default, \EasyCrypt proofs are run in \ec{strict} mode. In this
  mode, \ec{smt} failures cannot be caught using \ec{try}. This allows
  \EasyCrypt to always build the proof tree correctly, even in weak
  check mode, where \ec{smt} calls are assumed to succeed. Inside a
  strict proof, weak check mode can be turned on and off at will,
  allowing for the fast replay of proof sections during
  development. In any event, we recommend \emph{never} using \ec{try
    smt}: a little thought is much more cost-effective than failing
  \ec{smt} calls.
  \end{tsyntax}
\end{tactic}

\begin{tactic}[do! t]{tactic repetition}
  \begin{tsyntax}[empty]{do! t}
  Apply \ec{t} to the current goal, then repeatedly apply it to all subgoals,
  stopping only when it fails. An error is produced it \ec{t} does not apply to
  the current goal.
  \end{tsyntax}

  \paragraph{Variants.}\strut

  \begin{tabularx}{\textwidth}{@{}ll@{}}
  {\ec{do ?t}} & apply {\ec{t}} 0 or more times, until it fails\\
  {\ec{do n !t}} & apply {\ec{t}} with depth exactly {\ec{n}}\\
  {\ec{do n ?t}} & apply {\ec{t}} with depth at most {\ec{n}}
  \end{tabularx}
\end{tactic}

\begin{tactic}[t1; first t2]{goal selection}
  \begin{tsyntax}[empty]{t1; first t2}
  Apply the tactic \ec{t1}, then apply \ec{t2} on the first subgoal
  generated by \ec{t1}. An error is produced if no subgoals have been
  generated by \ec{t1}.

  \paragraph{Variants.}\strut

  \noindent\begin{tabularx}{\textwidth}{@{}ll@{}}
  {\ec{t1; first n t2}} & apply {\ec{t2}} on the first {\ec{n}} subgoals
    generated by {\ec{t1}}\\
  {\ec{t1; last t2}} & apply {\ec{t2}} on the last subgoal
    generated by {\ec{t1}}\\
  {\ec{t1; last n t2}} & apply {\ec{t2}} on the last {\ec{n}} subgoals
    generated by {\ec{t1}}\\
  {\ec{t; first n last}} & \parbox{200pt}{reorder the subgoals generated by {\ec{t}}, moving
    the first n to the end of the list}
  \end{tabularx}
  \end{tsyntax}
\end{tactic}

\begin{tactic}[by t]{closing goals}
  \begin{tsyntax}[empty]{by t}
  Apply the tactic \ec{t} and try to close all the generated subgoals using
  \rtactic{trivial}. Fail if not all subgoals can be closed.
  \end{tsyntax}
\end{tactic}

\section{Program Logics}
\label{sec:programlogics}

In this section, we describe the tactics of \EasyCrypt's three program
logics: \prhl, \phl and \hl.  There are five rough classes of program
logic tactics:
\begin{enumerate}
\item those that actually reason about the program in Hoare logic
  style;

\item those that correspond to semantics-preserving program
  transformations or compiler optimizations;

\item those that operate at the level of specifications,
  strenghtening, combining or splitting goals without modifying the
  program;

\item tactics that automate the application of other tactics;

\item advanced tactics for handling eager/lazy sampling and bounding
  the probability of failure.
\end{enumerate}
We discuss these five classes in turn.

\subsection{Tactics for Reasoning about Programs}
\label{subsec:reasoningprograms}

Unless specified, the following program logic tactics operate on a
program's last statement. Although we describe these tactics as if
they operated on single statements, their practical implementation
automatically and implicitly applies tactic \rtactic{seq} to deal with
context when necessary.

For simple proofs, it is often enough to simply apply the program
tactic corresponding to the last statement in the program and let
\ec{smt} deal with the residual program-free formula, once the program
has been consumed.

Most of the program reasoning tactics discussed in this subsection
have two modes when used on \prhl proof obligations. Their default
mode is to operate on both programs at once. When a side is specified
(using \ec{<tactic>\{1\}} or \ec{<tactic>\{2\}}), a one-sided variant
is used. Apart from the \rtactic{if} tactic, the one-sided variant is
in fact a combination of the \phl tactic and \rtactic{conseq}.

\medskip

% --------------------------------------------------------------------
\begin{tactic}{skip}
\end{tactic}

% --------------------------------------------------------------------
\begin{tactic}{seq}
  \begin{tsyntax}{seq $\;n_1$ $\;n_2$ : $\;R$}
    \textbf{\prhl sequence rule.} If $n_1$ and $n_2$ are natural
    numbers and the goal's conclusion is a \prhl statement judgement
    with precondition $P$, postcondition $Q$ and such that the lengths
    of the first and second programs are at least $n_1$ and $n_2$,
    respectively, then reduce the goal to two subgoals:
    \begin{itemize}
    \item A first goal whose conclusion has precondition $P$,
      postcondition $R$, first program consisting of the first $n_1$
      statements of the original goal's first program, and second
      program consisting of the first $n_2$ statements of the original
      goal's second program.

    \item A second goal whose conclusion has precondition $R$,
      postcondition $Q$, first program consisting of all but the first
      $n_1$ statements of the original goal's first program, and
      second program consisting of all but the first $n_2$ statements
      of the original goal's second program.
    \end{itemize}

  \bigskip
  For example, if the current goal is
  \ecinput{../examps/parts/tactics/seq/1-1.0.ec}{}{}{}{} then
  running \ecinput{../examps/parts/tactics/seq/1-1.ec}{}{}{}{}
  produces the goals
  \ecinput{../examps/parts/tactics/seq/1-1.1.ec}{}{}{}{}
  and
  \ecinput{../examps/parts/tactics/seq/1-1.2.ec}{}{}{}{}
  \end{tsyntax}

  \begin{tsyntax}{seq $\;n$ : $\;R$}
  \textbf{\hl sequence rule.} If $n$ is a natural
    number and the goal's conclusion is an \hl statement judgement
    with precondition $P$, postcondition $Q$ and such that the length
    of the program is at least $n$, then reduce the goal to two subgoals:
    \begin{itemize}
    \item A first goal whose conclusion has precondition $P$,
      postcondition $R$, and program consisting of the first $n$
      statements of the original goal's program.

    \item A second goal whose conclusion has precondition $R$,
      postcondition $Q$, and program consisting of all but the first
      $n$ statements of the original goal's program.
    \end{itemize}

  \bigskip
  For example, if the current goal is
  \ecinput{../examps/parts/tactics/seq/2-1.0.ec}{}{}{}{} then
  running \ecinput{../examps/parts/tactics/seq/2-1.ec}{}{}{}{}
  produces the goals
  \ecinput{../examps/parts/tactics/seq/2-1.1.ec}{}{}{}{}
  and
  \ecinput{../examps/parts/tactics/seq/2-1.2.ec}{}{}{}{}
  \end{tsyntax}

%  \begin{tsyntax}{seq $\ n$: R $\ \delta_1\ \delta_2\ \delta_3\ \delta_4$ I}
%  Non-relational probabilistic sequence rule. Argument \ec{I} is
%  optional (and defaults to $\mathsf{true}$). When one of
%  $(\delta_1,\delta_2)$ (resp. $(\delta_3,\delta_4)$) is 0, the other
%  can be replaced with a wildcard \ec{_}, and the corresponding goal
%  is not generated, as it is not relevant to the proof. When none of
%  the $\delta$s are given, the following default values are used:
%  $\delta_1 = 1$, $\delta_2 = \delta$, $\delta_3 = 0$.
%
%  \paragraph{Examples:}\strut
%  
%  \begin{cmathpar}
%  \texample[\phl{}]
%    {\ec{seq $\ \left|c\right|$: R $\ \delta_1$ $\ \delta_2$ $\ \delta_3$ $\ \delta_4$ I}}
%    {\HL{P}{c}{I} \\
%     \pHL{P}{c}{R}{\diamond}{\delta_1}  \\
%     \pHL{R \wedge I}{c'}{Q}{\diamond}{\delta_2} \\
%     \pHL{P}{c}{\neg R}{\diamond}{\delta_3} \\
%     \pHL{\neg R \wedge I}{c'}{Q}{\diamond}{\delta_4} \\
%     \delta_1 \delta_2 + \delta_3 \delta_4 \diamond \delta}
%    {\pHL{P}{c;c'}{Q}{\diamond}{\delta}}
%  \end{cmathpar}
%
%  \begin{cmathpar}
%  \texample[\phl{}]
%    {\ec{seq $\ \left|c\right|$: R $\ \delta_1$ $\ \delta_2\ \_\ 0$}}
%    {\HL{P}{c}{\mathsf{true}} \\
%     \pHL{P}{c}{R}{\diamond}{\delta_1} \\\\
%     \pHL{R \wedge I}{c'}{Q}{\diamond}{\delta_2} \\
%     \pHL{\neg R \wedge I}{c'}{Q}{\diamond}{0} \\
%     \delta_1 \delta_2 \diamond \delta}
%    {\pHL{P}{c;c'}{Q}{\diamond}{\delta}}
%  \end{cmathpar}
%
%  \textbf{Note:} Since most tactics implicitly apply the \rtactic{seq}
%  rule, most \phl tactics take optional final arguments corresponding
%  to the $\delta$s and \ec{I}.
%  \end{tsyntax}
\end{tactic}

% --------------------------------------------------------------------
\begin{tactic}{sp}
\end{tactic}

% --------------------------------------------------------------------
\begin{tactic}{wp}
  \begin{tsyntax}{wp}
    If the goal's conclusion is a \prhl, \phl or \hl statement
    judgement, consume the longest suffix(es) of the conclusion's
    program(s) consisting entirely of statements built-up from
    ordinary assignments (not random assignments or procedure call
    assignments) and \ec{if} statements, replacing the conclusion's
    postcondition by the weakest precondition $R$ such that the
    statement judgement consisting of $R$, the consumed suffix(es)
    and the conclusion's original postcondition holds.

    \bigskip For example, if the current goal is
    \ecinput{examps/parts/tactics/wp/1-1.0.ec} then
    running \ecinput{examps/parts/tactics/wp/1-1.ec}
    produces the goal
    \ecinput{examps/parts/tactics/wp/1-1.1.ec}
  \end{tsyntax}

  \begin{tsyntax}{wp $\;n_1$ $\;n_2$}
    In \prhl, let \ec{wp} consume \emph{exactly} $n_1$ statements of
    the first program and $n_2$ statements of the second
    program. Fails if this isn't possible.
  \end{tsyntax}

  \begin{tsyntax}{wp $\;n$}
    In \phl and \hl, let \ec{wp} consume \emph{exactly} $n$ statements
    of the program. Fails if this isn't possible.
  \end{tsyntax}
\end{tactic}

% --------------------------------------------------------------------
\begin{tactic}{rnd}
  When describing the variants of this tactic, we ony consider random
  assignments whose left-hand sides consist of single identifiers.
  The generalization to multiple assignment, when distributions over
  tuple types are sampled, is straightforward.

  \bigskip
  \begin{tsyntax}{rnd | rnd $\;f$ | rnd $\;f$ $\;g$} If the conclusion
    is a \prhl statement judgement whose programs end with random
    assignments \ec{$x_1\!$ <$\$$ $\;d_1$} and \ec{$x_2\!$ <$\$$
      $\;d_2$}, and $f$ and $g$ are functions between the types of
    $x_1$ and $x_2$, then consume those random assignments, replacing
    the conclusion's postcondition by the probabilistic weakest
    precondition of the random assignments wrt.\ $f$ and $g$.  The new
    postcondition checks that:
    \begin{itemize}
    \item $f$ and $g$ are an isomorphism between the distributions
      $d_1$ and $d_2$;

    \item for all elements $u$ in the support of $d_1$, the result
      of substituting $u$ and $f\,u$ for \ec{$x_1$\{1\}} and
      \ec{$x_2$\{2\}} in the conclusion's original postcondition
      holds.
    \end{itemize}
    When $g$ is $f$, it can be omitted. When $f$ is the identity, it
    can be omitted.

    \bigskip For example, if the current goal is
    \ecinput{examps/parts/tactics/rnd/2-1.0.ec}{}{}{}{} then
    running \ecinput{examps/parts/tactics/rnd/2-1.ec}{}{}{}{}
    produces the goal
    \ecinput{examps/parts/tactics/rnd/2-1.1.ec}{}{}{}{}
    Note that if one uses the other isomorphism between \ec{\{0,1\}} and
    \ec{[2..3]} the generated subgoal will be false.
  \end{tsyntax}

  \begin{tsyntax}{rnd\{1\} | rnd\{2\}}
    If the conclusion is a \prhl statement judgement whose designated
    program (1 or 2) ends with a random assignment 
    \ec{$x\!$ <$\$$ $\;d$}, then consume that random assignment,
    replacing the conclusion's postcondition with a check that:
    \begin{itemize}
    \item the weight of $d$ is $1$ (so the random assignment can't fail);

    \item for all elements $u$ in the support of $d$, the result
      of substituting $u$ for \ec{$x$\{$i$\}}---where $i$ is the
      selected side---in the conclusion's original
      postcondition holds.
    \end{itemize}

    \bigskip For example, if the current goal is
    \ecinput{examps/parts/tactics/rnd/3-1.0.ec}{}{}{}{} then
    running \ecinput{examps/parts/tactics/rnd/3-1.ec}{}{}{}{}
    produces the (false!) goal
    \ecinput{examps/parts/tactics/rnd/3-1.1.ec}{}{}{}{}
  \end{tsyntax}

  \begin{tsyntax}{rnd}
    If the conclusion is an \hl statement judgement whose program ends
    with a random assignment, then consume that random assignment,
    replacing the conclusion's postcondition by the possibilistic
    weakest precondition of the random assignment.

    \bigskip For example, if the current goal is
    \ecinput{examps/parts/tactics/rnd/1-1.0.ec}{}{}{}{} then
    running \ecinput{examps/parts/tactics/rnd/1-1.ec}{}{}{}{}
    produces the goal
    \ecinput{examps/parts/tactics/rnd/1-1.1.ec}{}{}{}{}
  \end{tsyntax}

  \begin{tsyntax}{rnd | rnd $\;E$}
    In \phl, compute the probabilistic weakest precondition of a
    random sampling with respect to event $E$. When $E$ is not
    specified, it is inferred from the current postcondition.
  \end{tsyntax}
\end{tactic}

% --------------------------------------------------------------------
\begin{tactic}{if}
  \begin{tsyntax}[empty]{if}
  \fix{Missing description of if}.
  \end{tsyntax}
\end{tactic}

% --------------------------------------------------------------------
\begin{tactic}{while}
  \begin{tsyntax}{while $\;I$}
    Here $I$ is an \emph{invariant} (formula), which may reference
    variables of the two programs, interpreted in their memories.  If
    the goal's conclusion is a \prhl statement judgement whose
    programs both end with \ec{while} statements, reduce the goal to
    two subgoals whose conclusions are \prhl statement judgements:
    \begin{itemize}
    \item One whose first and second programs are the bodies of the
      first and second while statements, whose precondition is the
      conjunction of $I$ and the while statements' boolean expressions (the
      first of which is interpreted in memory \ec{&1}, and the second
      of which is interpreted in \ec{&2}) and whose postcondition is
      the conjunction of $I$ and the assertion that the while statements'
      boolean expressions (interpreted in the appropriate memories)
      are equivalent.

    \item One whose precondition is the original goal's precondition,
      whose first and second programs are all the results of removing
      the while statements from the two programs, and whose postcondition is
      the conjunction of:
      \begin{itemize}
      \item the conjunction of $I$ and the assertion that the while statements'
        boolean expressions are equivalent; and

      \item the assertion that, for all values of the variables
        \emph{modified} by the while statements, if the while statements'
        boolean expressions don't hold, but $I$ holds, then the
        original goal's postcondition holds (in $I$, the while statements'
        boolean expressions, and the postcondition, variables modified
        by the while statements are replaced by universally quantified
        identifiers; otherwise, the boolean expressions are
        interpreted in the program's respective memories, and the
        memory references of $I$ and the postcondition are
        maintained).
      \end{itemize}
    \end{itemize}

  \medskip
  For example, if the current goal is
  \ecinput{examps/parts/tactics/while/1-1.0.ec}{}{}{}{} then
  running \ecinput{examps/parts/tactics/while/1-1.ec}{}{}{}{}
  produces the goals
  \ecinput{examps/parts/tactics/while/1-1.1.ec}{}{}{}{}
  and
  \ecinput{examps/parts/tactics/while/1-1.2.ec}{}{}{}{}
  \end{tsyntax}

  \begin{tsyntax}{while\{1\} $\;I$ $\;v$ | while\{2\} $\;I$ $\;v$}
    Here $I$ is an \emph{invariant} (formula) and $v$ is a
    \emph{termination variant} integer expression, both of which may reference
    variables of the two programs, interpreted in their memories.  If
    the goal's conclusion is a \prhl statement judgement whose
    designated program (1 or 2) ends with a \ec{while} statement,
    reduce the goal to two subgoals;
    \begin{itemize}
    \item One whose conclusion is a \phl statement judgement, saying that
      running the body of the while statement in a memory in which
      $I$ holds and the while statement's boolean expression is true
      is guaranteed to result in termination in a memory in which
      $I$ holds and in which the value of the variant expression $v$
      has decreased by at least $1$. (More precisely, the \phl statement
      judgment is universally quantified by the memory of the non-designated
      program and the initial value of $v$. References to the variables
      of the nondesignated program in $I$ and $v$ are interpreted in this
      memory; reference to the variables of the designed program have
      their memory references removed.)

    \item One whose conclusion is a \prhl statement judgement whose
      precondition is the original goal's precondition, whose
      designated program is the result of removing the while statement
      from the original designated program, whose other program is
      unchanged, and whose postcondition is the conjunction of $I$ and
      the assertion that, for all values of the variables modified by
      the while statement, that the conjunction of the following
      formulas holds:
      \begin{itemize}
      \item the assertion that, if $I$ holds, but the variant
        expression $v$ is not positive, then the while statement's
        boolean expression is false;

      \item the assertion that, if the while statement's boolean expression
        doesn't hold, but $I$ holds, then the original goal's postcondition
        holds.
      \end{itemize}
    \end{itemize}

  \bigskip
  For example, if the current goal is
  \ecinput{examps/parts/tactics/while/3-1.0.ec}{}{}{}{} then
  running \ecinput{examps/parts/tactics/while/3-1.ec}{}{}{}{}
  produces the goals
  \ecinput{examps/parts/tactics/while/3-1.1.ec}{}{}{}{}
  and
  \ecinput{examps/parts/tactics/while/3-1.2.ec}{}{}{}{}
  \end{tsyntax}

  \begin{tsyntax}{while $\;I$}
    Here $I$ is an \emph{invariant} (formula), which may reference
    variables of the program.  If the goal's conclusion is an \hl
    statement judgement ending with a \ec{while} statement, reduce the
    goal to two subgoals whose conclusions are \hl statement
    judgements:
    \begin{itemize}
    \item One whose program is the body of the while statement, whose
      precondition is the conjunction of $I$ and the while statement's
      boolean expression, and whose postcondition is $I$.

    \item One whose precondition is the original goal's precondition,
      whose program is the result of removing the while statement from
      the original program, and whose postcondition is the conjunction
      of:
      \begin{itemize}
      \item $I$; and

      \item the assertion that, for all values of the variables
        \emph{modified} by the while statement, if the while statement's boolean
        expression doesn't hold, but $I$ holds, then the original
        goal's postcondition holds (in $I$, the while statement's boolean
        expression, and the postcondition, variables modified by the
        while statement are replaced by universally quantified
        identifiers).
      \end{itemize}
    \end{itemize}

  \bigskip
  For example, if the current goal is
  \ecinput{examps/parts/tactics/while/2-1.0.ec}{}{}{}{} then
  running \ecinput{examps/parts/tactics/while/2-1.ec}{}{}{}{}
  produces the goals
  \ecinput{examps/parts/tactics/while/2-1.1.ec}{}{}{}{}
  and
  \ecinput{examps/parts/tactics/while/2-1.2.ec}{}{}{}{}
  \end{tsyntax}

  \begin{tsyntax}{while $\;I$ $\;v$}
  \phl version...

%  Where \ec{v} is an integer-valued expression. In \phl, performs a
%  weakest precondition computation over a loop, using \ec{I} as
%  invariant and \ec{v} as a decreasing variant to prove
%  termination. In addition to the two invariant-related subgoals (see
%  above), two subgoals regarding the variant are generated; one
%  requiring that the variant be less than 0 exactly when the loop
%  condition is false, and the other requiring that the variant
%  decreases strictly.
  \end{tsyntax}
\end{tactic}

% --------------------------------------------------------------------
\begin{tactic}{call}
  \begin{tsyntax}{call (_ : $\;P$ ==> $\;Q$)}
    If the goal's conclusion is a \prhl or \hl statement judgement
    whose program(s) end(s) with (a) procedure call(s) or (a) procedure
    call assignment(s), then generate two subgoals:
  \begin{itemize}
  \item One whose conclusion is a judgement of the same kind whose
    precondition is $P$, whose procedure(s) are/is the procedure(s)
    being called, and whose postcondition is $Q$.

  \item One whose conclusion is a statement judgement of the same
    kind whose precondition is the original goal's precondition,
    whose program(s) are/is the result of removing the procedure
    call(s) from the program(s), and whose postcondition is the
    conjunction of
    \begin{itemize}
    \item the result of replacing the procedure's/procedures'
      parameter(s) by their actual argument(s) in $P$; and

    \item the assertion that, for all values of the global variable(s)
      modified by the procedure(s) and the result(s) of the procedure
      call(s), if $Q$ holds (where these quantified identifiers have
      been substituted for the modified variables and procedure
      results), then the original goal's postcondition holds (where
      the modified global variables and occurrences of the variable(s)
      (if any) to which the result(s) of the call(s) of the
      procedure(s) are/is assigned have been replaced by the
      appropriate quantified identifiers).
    \end{itemize}
  \end{itemize}

  \medskip
  For example, if the current goal is
  \ecinput{../examps/parts/tactics/call/1-1.0.ec}{}{}{}{}
  and the procedures \ec{M.f} and \ec{N.f} have a single parameter,
  \ec{y}, then running
  \ecinput{../examps/parts/tactics/call/1-1.ec}{}{}{}{} produces the
  goals \ecinput{../examps/parts/tactics/call/1-1.1.ec}{}{}{}{} and
  \ecinput{../examps/parts/tactics/call/1-1.2.ec}{}{}{}{}

  \bigskip
  Alternatively, a proof term whose conclusion is a \prhl or
  \hl judgement involving the procedure(s) called at the
  end(s) of the program(s) may be supplied as the argument to
  \ec{call}, in which case only the second subgoal need be
  generated.

  \medskip
  For example, in the start-goal of the preceding example,
  if the lemma \ec{M_N_f} is
  \ecinput{examps/tactics/call/1.ec}{}{31-33}{}
  then running
  \ecinput{../examps/parts/tactics/call/1-2.ec}{}{}{}{} produces the
  goal \ecinput{../examps/parts/tactics/call/1-2.1.ec}{}{}{}{}
  \end{tsyntax}

  \begin{tsyntax}{call\{1\} (_ : $\;P$ ==> $\;Q$) | call\{2\} (_ : $\;P$ ==> $\;Q$)}
    If the goal's conclusion is a \prhl statement judgement whose
    designated program ends with a procedure call, then generate two
    subgoals:
  \begin{itemize}
  \item One whose conclusion is a \phl judgement whose precondition is
    $P$, whose procedure is the procedure being called, whose
    postcondition is $Q$, and whose bound part specifies equality with
    probability 1.
    (Consequently, $P$ and $Q$ may not mention \ec{&1} and
    \ec{&2}.)

  \item One whose conclusion is a \prhl statement judgement whose
    precondition is the original goal's precondition, whose programs
    are the result of removing the procedure call from the designated
    program, and leaving the other program unchanged, and whose
    postcondition is the conjunction of
    \begin{itemize}
    \item the result of replacing the procedure's
      parameter(s) by their actual argument(s) in $P$; and

    \item the assertion that, for all values of the global variable(s)
      modified by the procedure and the result of the procedure call,
      if $Q$ holds (where these quantified identifiers have been
      substituted for the modified variables and procedure result),
      then the original goal's postcondition holds (where the modified
      global variables and occurrences the variable (if any) to which
      the result of the procedure call is assigned have been replaced
      by the appropriate quantified identifiers).
    \end{itemize}
  \end{itemize}

  For example, if the current goal is
  \ecinput{../examps/parts/tactics/call/2-1.0.ec}{}{}{}{}
  then running
  \ecinput{../examps/parts/tactics/call/2-1.ec}{}{}{}{} produces the
  goals \ecinput{../examps/parts/tactics/call/2-1.1.ec}{}{}{}{} and
  \ecinput{../examps/parts/tactics/call/2-1.2.ec}{}{}{}{}

  \bigskip Alternatively, a proof term whose conclusion is a \phl
  judgement specifying equality with probability 1 and involving the
  procedure called at the end of the designated program may be
  supplied as the argument to \ec{call}, in which case only the second
  subgoal need be generated.

  \medskip
  For example, in the start-goal of the preceding example,
  if the lemma \ec{M_f} is
  \ecinput{examps/tactics/call/2.ec}{}{23-24}{}
  then running
  \ecinput{../examps/parts/tactics/call/2-2.ec}{}{}{}{} produces the
  goal \ecinput{../examps/parts/tactics/call/2-2.1.ec}{}{}{}{}
  \end{tsyntax}

  \begin{tsyntax}{call (_ : $\;I$)}
    If the conclusion is a \prhl or \hl statement judgement whose
    program(s) end(s) with (a) call(s) of (a) \emph{concrete}
    procedure(s), then use the specification argument to \ec{call}
    generated from the \emph{invariant} $I$, and automatically apply
    \ec{proc} to its first subgoal.  In the \prhl case, its
    precondition will assume equality of the procedures' parameters,
    and its postcondition will assert equality of the results of the
    procedure calls.

    \medskip
    For example, if the current goal is
    \ecinput{../examps/parts/tactics/call/1-3.0.ec}{}{}{}{}
    and modules \ec{M} and \ec{N} contain
    \ecinput{examps/tactics/call/1.ec}{}{4-8}{} and
    \ecinput{examps/tactics/call/1.ec}{}{18-22}{}
    respectively, then
    running \ecinput{../examps/parts/tactics/call/1-3.ec}{}{}{}{}
    produces the goals
    \ecinput{../examps/parts/tactics/call/1-3.1.ec}{}{}{}{} and
    \ecinput{../examps/parts/tactics/call/1-3.2.ec}{}{}{}{}
  \end{tsyntax}

  \begin{tsyntax}{call (_ : $\;I$)}
    If the conclusion is a \prhl or \hl statement judgement whose
    program(s) end(s) with (a) call(s) of the same \emph{abstract}
    procedure, then use the specification argument to \ec{call}
    generated from the \emph{invariant} $I$, and automatically apply
    \ec{proc $\;I$} to its first subgoal, pruning the first two
    subgoals the application generates, because their conclusions
    consist of ambient logic formulas that are true by construction,
    and pruning the next goal (showing the losslessness of the abstract
    procedure given the losslessness of the abstract oracles it uses), if
    trivial suffices to solve it.
    In the \prhl case, its precondition will assume equality of the
    procedure's parameters and of the global variables of the module
    containing the procedure, and its postcondition will assume
    equality of the results of the procedure calls and of the global
    variables of the containing module.

    \medskip
    For example, given the declarations
    \ecinput{examps/tactics/call/3.ec}{}{3-24}{}
    if the current goal is
    \ecinput{../examps/parts/tactics/call/3-1.0.ec}{}{}{}{}
    then running
    \ecinput{../examps/parts/tactics/call/3-1.ec}{}{}{}{}
    produces the goals
    \ecinput{../examps/parts/tactics/call/3-1.1.ec}{}{}{}{} and
    \ecinput{../examps/parts/tactics/call/3-1.2.ec}{}{}{}{}
  \end{tsyntax}

  \begin{tsyntax}{call (_ : $\;B$, $\;I$)}
    If the conclusion is a \prhl statement judgement whose programs
    end with calls of the same \emph{abstract} procedure, then use the
    specification argument to \ec{call} generated from the \emph{bad
      event} $B$ and \emph{invariant} $I$, and automatically apply
    \ec{proc $\;B$ $\;I$} to its first subgoal, pruning the first two
    subgoals the application generates, because their conclusions
    consist of ambient logic formulas that are true by construction,
    and pruning the next goal (showing the losslessness of the abstract
    procedure given the losslessness of the abstract oracles it uses), if
    trivial suffices to solve it.
    The specification's precondition will assume equality of the
    procedure's parameters and of the global variables of the module
    containing the procedure as well as $I$, and its postcondition
    will assert $I$ and the equality of the results of the procedure
    calls and of the global variables of the containing module---but
    only when $B$ does not hold.

    \medskip
    For example, given the declarations
    \ecinput{examps/tactics/call/4.ec}{}{3-51}{}
    if the current goal is
    \ecinput{../examps/parts/tactics/call/4-1.0.ec}{}{}{}{}
    then running
    \ecinput{../examps/parts/tactics/call/4-1.ec}{}{}{}{}
    produces the goals
    \ecinput{../examps/parts/tactics/call/4-1.1.ec}{}{}{}{} and
    \ecinput{../examps/parts/tactics/call/4-1.2.ec}{}{}{}{} and
    \ecinput{../examps/parts/tactics/call/4-1.3.ec}{}{}{}{} and
    \ecinput{../examps/parts/tactics/call/4-1.4.ec}{}{}{}{}
  \end{tsyntax}

  \begin{tsyntax}{call (_ : $\;B$, $\;I$, $\;J$)}
    If the conclusion is a \prhl statement judgement whose programs
    end with calls of the same \emph{abstract} procedure, then use the
    specification argument to \ec{call} generated from the \emph{bad
      event} $B$ and \emph{invariants} $I$ and $J$, and automatically
    apply \ec{proc $\;B$ $\;I$ $\;J$} to its first subgoal, pruning the
    first two subgoals the application generates, because their
    conclusions consist of ambient logic formulas that are true by
    construction.  The specification's precondition will assume
    equality of the procedure's parameters and of the global variables
    of the module containing the procedure as well as $I$, and its
    postcondition will assert
    \begin{itemize}
    \item $I$ and the equality of the results of the procedure calls
      and of the global variables of the containing module---if $B$
      does not hold; and

    \item $J$---if $B$ does hold.
    \end{itemize}

    \medskip
    For example, given the declarations of the preceding example
    if the current goal is
    \ecinput{../examps/parts/tactics/call/4-2.0.ec}{}{}{}{}
    then running
    \ecinput{../examps/parts/tactics/call/4-2.ec}{}{}{}{}
    produces the goals
    \ecinput{../examps/parts/tactics/call/4-2.1.ec}{}{}{}{} and
    \ecinput{../examps/parts/tactics/call/4-2.2.ec}{}{}{}{} and
    \ecinput{../examps/parts/tactics/call/4-2.3.ec}{}{}{}{} and
    \ecinput{../examps/parts/tactics/call/4-2.4.ec}{}{}{}{}
  \end{tsyntax}
\end{tactic}

% --------------------------------------------------------------------
\begin{tactic}{proc}
  \begin{tsyntax}{proc}
    Turn a goal whose conclusion is a \prhl (resp., \phl) judgement
    involving concrete procedures (resp., a concrete procedure)
    into one whose conclusion is a \prhl (resp., \phl) statement
    judgement by replacing the \emph{concrete} procedures (resp.,
    procedure) by their (resp., its) bodies (resp., body).

  \bigskip
  For example, if the current goal is
  \ecinput{../examps/parts/tactics/proc/1-1.0.ec}{}{}{}{} then
  running \ecinput{../examps/parts/tactics/proc/1-1.ec}{}{}{}{}
  produces the goal
  \ecinput{../examps/parts/tactics/proc/1-1.1.ec}{}{}{}{}
  \end{tsyntax}

  \begin{tsyntax}{proc $\;I$}
    Reduce a goal whose conclusion is a \prhl (resp., \phl) judgement
    involving an abstract procedure to \prhl (resp., \phl) judgements
    on the oracles the procedure may query.

  \bigskip
  For example, given the declarations
  \ecinput{examps/tactics/proc/2.ec}{}{3-20}{}
  if the current goal is
  \ecinput{../examps/parts/tactics/proc/2-1.0.ec}{}{}{}{} then
  running \ecinput{../examps/parts/tactics/proc/2-1.ec}{}{}{}{}
  produces the goals
  \ecinput{../examps/parts/tactics/proc/2-1.1.ec}{}{}{}{}
  and
  \ecinput{../examps/parts/tactics/proc/2-1.2.ec}{}{}{}{}
  and
  \ecinput{../examps/parts/tactics/proc/2-1.3.ec}{}{}{}{}
  and
  \ecinput{../examps/parts/tactics/proc/2-1.4.ec}{}{}{}{}
  The tactic would fail without the module restriction \ec{T\{Or\}} on
  \ec{M}, as then \ec{M} could directly manipulate \ec{Or.x}.
  \end{tsyntax}

  \begin{tsyntax}{proc B I}
  Derive a specification for an \emph{abstract} procedure from an
  ``upto-failure'' invariant on the oracles it may query. The failure
  event \ec{B} is evaluated in the right memory. The left oracles
  must be lossless once the bad event occurs. The right oracles must
  guarantee the stability of the failure event with probability 1.
  \end{tsyntax}

  \begin{tsyntax}{proc B I I'}
  Similar to \ec{proc B I}, with an additional invariant once the bad
  event occurs. This is particularly useful when additional facts
  about the state need to be known to prove the losslessness and
  stability conditions.
  \end{tsyntax}

  \begin{tsyntax}{proc*}
  Derive a specification on procedures from a specification on a
  program whose code consists in a call to that procedure. This tactic
  is particularly useful in combination with \rtactic{inline} when
  faced with a \prhl judgment where one of the procedures is concrete
  and the other is abstract.
  \end{tsyntax}
\end{tactic}


\subsection{Tactics for Transforming Programs}
\label{subsec:transformingprograms}

% --------------------------------------------------------------------
\begin{tactic}{swap}
All versions of the tactic work for \prhl (an optional side can be given),
\phl and \hl statement judgements. We'll describe their operation
in terms of a single program (list of statements).

\medskip
\begin{tsyntax}{swap $\;n$ $\;m$ $\;l$}
  Fails unless $1\leq n < m \leq l$ and the program has at least $l$
  statements. Swaps the statement block from positions $n$ through
  $m-1$ with the statement block from $m$ through $l$, failing if these
  blocks of statements aren't independent.
\end{tsyntax}

\begin{tsyntax}{swap [$n$..$m$] $\;k$}
  Fails unless $1\leq n \leq m$ and the program has at
  least $m$ statements.
  \begin{itemize}
  \item If $k$ is non-negative, move the statement block from $n$
    through $m$ forward $k$ positions, failing if the program doesn't
    have at least $m + k$ statements or if the swapped statements
    blocks aren't independent.

  \item If $k$ is negative, move the statement block from $n$ through
    $m$ backward $-k$ positions, failing if $n + k < 1$ or if the
    swapped statement blocks aren't independent.
  \end{itemize}
\end{tsyntax}

\begin{tsyntax}{swap $\;n$\ $k$}
  Equivalent to \ec{swap [$n$..$n$] $\;k$}.
\end{tsyntax}

\begin{tsyntax}{swap $\;k$}
  If $k$ is non-negative, equivalent to \ec{swap 1 $\;k$}.
  If $k$ is negative, equivalent to \ec{swap $\;n$ $\;k$},
  where $n$ is the length of the program.
\end{tsyntax}

\medskip For example, suppose the current goal is
  \ecinput{examps/parts/tactics/swap/1-1.0.ec}{}{}{}{}
  Then running
  \ecinput{examps/parts/tactics/swap/1-1.ec}{}{}{}{}
  produces goal
  \ecinput{examps/parts/tactics/swap/1-1.1.ec}{}{}{}{}
  From which running
  \ecinput{examps/parts/tactics/swap/1-2.ec}{}{}{}{}
  produces goal
  \ecinput{examps/parts/tactics/swap/1-2.1.ec}{}{}{}{}
  From which running
  \ecinput{examps/parts/tactics/swap/1-3.ec}{}{}{}{}
  produces goal
  \ecinput{examps/parts/tactics/swap/1-3.1.ec}{}{}{}{}
  From which running
  \ecinput{examps/parts/tactics/swap/1-4.ec}{}{}{}{}
  produces goal
  \ecinput{examps/parts/tactics/swap/1-4.1.ec}{}{}{}{}
  From which running
  \ecinput{examps/parts/tactics/swap/1-5.ec}{}{}{}{}
  produces goal
  \ecinput{examps/parts/tactics/swap/1-5.1.ec}{}{}{}{}
\end{tactic}

% --------------------------------------------------------------------
\begin{tactic}{inline}
  \begin{tsyntax}{inline $\;M_1$.$p_1$ $\;\cdots$ $\;M_n$.$p_n$}
    Inline the selected \emph{concrete} procedures in both programs,
    with \prhl, and in the program, with \hl and \phl, until no more
    inlining of these procedures is possible.

    To inline a procedure call, the procedure's parameters are
    assigned the values of their arguments (fresh parameter
    identifiers are used, as necessary, to avoid naming
    conflicts). This is followed by the body of the procedure. Finally,
    the procedure's return value is assigned to the identifiers (if
    any) to which the procedure call's result is assigned.
  \end{tsyntax}

  \begin{tsyntax}{inline\{1\} $\;M_1$.$p_1$ $\;\cdots$ $\;M_n$.$p_n$ | inline\{2\} $\;M_1$.$p_1$ $\;\cdots$ $\;M_n$.$p_n$}
    Do the inlining in just the first or second program, in the \prhl case.
  \end{tsyntax}

  \begin{tsyntax}{inline* | inline\{1\}* | inline\{2\}*}
    Inline all concrete procedures, continuing until no more inlining
    is possible.
  \end{tsyntax}

  \begin{tsyntax}{inline $\;\mathit{occs}$ $\;M$.$p$ | inline\{1\} $\;\mathit{occs}$ $\;M$.$p$ | inline\{2\} $\;\mathit{occs}$ $\;M$.$p$}
    Inline just the specified occurrences of $M$.$p$, where
    $\mathit{occs}$ is a parenthesized nonempty sequence of positive
    numbers \ec{($n_1$ $\;\cdots$ $\;n_l$)}. E.g., \ec{(1 3)} means the
    first and third occurrences of the procedure.  In the \prhl case,
    a side \ec{\{1\}} or \ec{\{2\}} must be specified.
  \end{tsyntax}

  \bigskip
  For example, given the declarations
  \ecinput{examps/tactics/inline/1.ec}{}{3-18}{}
  if the current goal is
  \ecinput{examps/parts/tactics/inline/1-1.0.ec}{}{}{}{} then
  running \ecinput{examps/parts/tactics/inline/1-1.ec}{}{}{}{}
  produces the goal
  \ecinput{examps/parts/tactics/inline/1-1.1.ec}{}{}{}{}
  From which running
  \ecinput{examps/parts/tactics/inline/1-2.ec}{}{}{}{}
  produces the goal
  \ecinput{examps/parts/tactics/inline/1-2.1.ec}{}{}{}{}
  From which running
  \ecinput{examps/parts/tactics/inline/1-3.ec}{}{}{}{}
  produces the goal
  \ecinput{examps/parts/tactics/inline/1-3.1.ec}{}{}{}{}
  And, if the current goal is
  \ecinput{examps/parts/tactics/inline/2-1.0.ec}{}{}{}{} then
  running \ecinput{examps/parts/tactics/inline/2-1.ec}{}{}{}{}
  produces the goal
  \ecinput{examps/parts/tactics/inline/2-1.1.ec}{}{}{}{}
\end{tactic}


% --------------------------------------------------------------------
\begin{tactic}{rcondf}
\end{tactic}

% --------------------------------------------------------------------
\begin{tactic}{rcondt}
  \begin{tsyntax}{rcondt $\;n$}
    If the goal's conclusion is an \hl statement judgement whose $n$th
    statement is an \ec{if} statement, reduce the goal to two
    subgoals.
    \begin{itemize}
    \item One whose concludion is an \hl statement judgement whose
      precondition is the original goal's precondition, program is the
      first $n-1$ statements of the original goal's program, and
      postcondition is the boolean expression of the \ec{if}
      statement.
   
    \item One whose conclusion is an \hl statement judgement that's
      the same as that of the original goal except that the \ec{if}
      statement has been replaced by its \ec{then} part.
    \end{itemize}

    \medskip For example, if the current goal is
    \ecinput{examps/parts/tactics/rcondt/1-1.0.ec}{}{}{}{} then
    running \ecinput{examps/parts/tactics/rcondt/1-1.ec}{}{}{}{}
    produces the goals
    \ecinput{examps/parts/tactics/rcondt/1-1.1.ec}{}{}{}{} and
    \ecinput{examps/parts/tactics/rcondt/1-1.2.ec}{}{}{}{}
  \end{tsyntax}

  \begin{tsyntax}{rcondt\{1\} $\;n$ | rcondt\{2\} $\;n$}
    If the goal's conclusion is a \prhl statement judgement where the
    $n$th statement of the designated program is an \ec{if} statement,
    reduce the goal to two subgoals.
    \begin{itemize}
    \item One whose conclusion is an \hl statement judgement whose
      precondition is the original goal's precondition, program is the
      first $n-1$ statements of the original goal's designated
      program, and postcondition is the boolean expression of the
      \ec{if} statement. Actually, the \hl statement judgement is
      universally quantified by a memory of the non-designated
      program, and references in the precondition to variables of the
      non-designated program are interpreted in that memory.
   
    \item One whose conclusion is a \prhl statement judgement that's
      the same as that of the original goal except that the \ec{if}
      statement has been replaced by its \ec{then} part.
    \end{itemize}

  \medskip
  For example, if the current goal is
  \ecinput{examps/parts/tactics/rcondt/1-2.0.ec}{}{}{}{} then
  running \ecinput{examps/parts/tactics/rcondt/1-2.ec}{}{}{}{}
  produces the goals
  \ecinput{examps/parts/tactics/rcondt/1-2.1.ec}{}{}{}{}
  and
  \ecinput{examps/parts/tactics/rcondt/1-2.2.ec}{}{}{}{}
  \end{tsyntax}
\end{tactic}


% --------------------------------------------------------------------
\begin{tactic}{splitwhile}
  \begin{tsyntax}[empty]{splitwhile}
  \fix{Missing description of splitwhile}.
  \end{tsyntax}
\end{tactic}

% --------------------------------------------------------------------
\begin{tactic}{unroll}
  \begin{tsyntax}{unroll $\;n$}
    If the goal's conclusion is an \hl statement judgement whose $n$th
    statement is a \ec{while} statement, then insert before that
    statement an \ec{if} statement whose boolean expression is the
    \ec{while} statement's boolean expression, whose \ec{then} part is
    the \ec{while} statements's body, and whose \ec{else} part is
    empty.

    \medskip For example, if the current goal is
    \ecinput{../examps/parts/tactics/unroll/1-1.0.ec}{}{}{}{} then
    running \ecinput{../examps/parts/tactics/unroll/1-1.ec}{}{}{}{}
    produces the goal
    \ecinput{../examps/parts/tactics/unroll/1-1.1.ec}{}{}{}{}
  \end{tsyntax}

  \begin{tsyntax}{unroll\{1\} $\;n$ | unroll\{2\} $\;n$}
    If the goal's conclusion is an \prhl statement judgement where the
    $n$th statement of the designated program is a \ec{while}
    statement, then insert before that statement an \ec{if} statement
    whose boolean expression is the \ec{while} statement's boolean
    expression, whose \ec{then} part is the \ec{while} statements's
    body, and whose \ec{else} part is empty.

    \medskip For example, if the current goal is
    \ecinput{../examps/parts/tactics/unroll/1-2.0.ec}{}{}{}{} then
    running \ecinput{../examps/parts/tactics/unroll/1-2.ec}{}{}{}{}
    produces the goal
    \ecinput{../examps/parts/tactics/unroll/1-2.1.ec}{}{}{}{}
    from which running
    running \ecinput{../examps/parts/tactics/unroll/1-3.ec}{}{}{}{}
    produces the goal
    \ecinput{../examps/parts/tactics/unroll/1-3.1.ec}{}{}{}{}
  \end{tsyntax}
\end{tactic}

% --------------------------------------------------------------------
\begin{tactic}{fission}
\end{tactic}

% --------------------------------------------------------------------
\begin{tactic}{fusion}
  \begin{tsyntax}{fusion $\;c$!$l$ @ $\;m$, $\;n$}
    \hl statement judgement version.  Fails unless $0\leq l$ and
    $0\leq m$ and $0\leq n$ and the $c$th statement of the program
    is a \ec{while} statement, and there are at least $l$ statements right
    before the \ec{while} statement, at its level,
    and the part of the program beginning from the $l$ statements
    before the while loop may be uniquely matched against
\begin{easycrypt}{}{}
#$s_1$# while (#$e$#) { #$s_2$# #$s_4$# }
#$s_1$# while (#$e$#) { #$s_3$# #$s_4$# }
\end{easycrypt}
    where:
    \begin{itemize}
    \item $s_1$ has length $l$;

    \item $s_2$ has length $m$;

    \item $s_3$ has length $n$;

    \item $e$ doesn't reference the variables written by $s_2$ and $s_3$;

    \item $s_1$ and $s_4$ don't read or write the variables written by
      $s_2$ and $s_3$;

    \item $s_2$ and $s_3$ don't write the variables written by $s_1$
      and $s_4$;

    \item $s_2$ and $s_3$ don't read or write the variables written by
      the other.
    \end{itemize}
    The tactic replaces
\begin{easycrypt}{}{}
#$s_1$# while (#$e$#) { #$s_2$# #$s_4$# }
#$s_1$# while (#$e$#) { #$s_3$# #$s_4$# }
\end{easycrypt}
by
\begin{easycrypt}{}{}
#$s_1$# while (#$e$#) { #$s_2$# #$s_3$# #$s_4$# }
\end{easycrypt}

    \medskip For example, if the current goal is
    \ecinput{examps/parts/tactics/fusion/1-1.0.ec}{}{}{}{} then
    running \ecinput{examps/parts/tactics/fusion/1-1.ec}{}{}{}{}
    produces the goal
    \ecinput{examps/parts/tactics/fusion/1-1.1.ec}{}{}{}{}
  \end{tsyntax}

  \begin{tsyntax}{fusion $\;c$ @ $\;m$, $\;n$}
    Equivalent to \ec{fusion $\;c$!1 @ $\;m$, $\;n$}.
  \end{tsyntax}

  \begin{tsyntax}{fusion\{1\} $\;\cdots$ | fusion\{2\} $\;\cdots$}
    The \prhl versions of the above variants, working on the
    designated program.
  \end{tsyntax}
\end{tactic}


% --------------------------------------------------------------------
\begin{tactic}{alias}
  \begin{tsyntax}{alias $\;c$ with $\;x$}
    If the goal's conclusion is an \hl statement judgement whose
    program's $c$th statement is an assignment statement, and $x$ is
    an identifier, then replace the assignment statement by the
    following two statements:
    \begin{itemize}
    \item an assignment statement of the same kind as the original
      assignment statement (ordinary, random, procedure call) whose
      left-hand side is $x$, and whose right-hand side is the
      right-hand side of the original assignment statement;

    \item an ordinary assignment statement whose left-hand side is
      the left-hand side of the original assignment statement,
      and whose right-hand side is $x$.
    \end{itemize}
    If $x$ is a local variable of the program, a fresh name is
    generated by adding digits to the end of $x$.
  \end{tsyntax}

  \begin{tsyntax}{alias $\;c$}
    Equivalent to \ec{alias $\;c$ with x}.
  \end{tsyntax}

  \begin{tsyntax}{alias $\;c$ $\;x$ = $\;e$}
    If the program has an $c$th statement, and the expression $e$ is
    well-typed in the context of the program, insert before the $c$th
    statement an ordinary assignment statement whose left-hand side is
    $x$ and whose right-hand side is $e$.
    If $x$ is a local variable of the program, a fresh name is
    generated by adding digits to the end of $x$.
  \end{tsyntax}

  \begin{tsyntax}{alias\{1\} $\;\cdots$ | alias\{2\} $\;\cdots$}
    The \prhl versions of the preceding forms, where the aliasing
    is done in the designated program.
  \end{tsyntax}

  \bigskip For example, if the current goal is
  \ecinput{../examps/parts/tactics/alias/1-1.0.ec}{}{}{}{} then running
  \ecinput{../examps/parts/tactics/alias/1-1.ec}{}{}{}{} produces the
  goal \ecinput{../examps/parts/tactics/alias/1-1.1.ec}{}{}{}{}
  from which running
  \ecinput{../examps/parts/tactics/alias/1-2.ec}{}{}{}{} produces the
  goal \ecinput{../examps/parts/tactics/alias/1-2.1.ec}{}{}{}{}
  from which running
  \ecinput{../examps/parts/tactics/alias/1-3.ec}{}{}{}{} produces the
  goal \ecinput{../examps/parts/tactics/alias/1-3.1.ec}{}{}{}{}
\end{tactic}

% --------------------------------------------------------------------
\begin{tactic}{cfold}
  \begin{tsyntax}{cfold $\;n$ ! $\;m$}
    Fails unless $n\geq 1$ and $m\geq 0$.  If the goal's conclusion is
    an \hl statement judgement in which statement $n$ of the
    judgement's program is an ordinary assignment statement in which
    constant values are assigned to local identifiers, and the
    following statement block of length $m$ does not write any of
    those identifiers, then replace all occurrences of the assigned
    identifiers in that statement block by the constants assigned to
    them, and move the assignment statement to after the modified
    statement block.

    \medskip For example, if the current goal is
    \ecinput{../examps/parts/tactics/cfold/1-1.0.ec}{}{}{}{} then
    running \ecinput{../examps/parts/tactics/cfold/1-1.ec}{}{}{}{}
    produces the goal
    \ecinput{../examps/parts/tactics/cfold/1-1.1.ec}{}{}{}{}
    from which
    running \ecinput{../examps/parts/tactics/cfold/1-2.ec}{}{}{}{}
    produces the goal
    \ecinput{../examps/parts/tactics/cfold/1-2.1.ec}{}{}{}{}
    from which
    running \ecinput{../examps/parts/tactics/cfold/1-3.ec}{}{}{}{}
    produces the goal
    \ecinput{../examps/parts/tactics/cfold/1-3.1.ec}{}{}{}{}
    from which
    running \ecinput{../examps/parts/tactics/cfold/1-4.ec}{}{}{}{}
    produces the goal
    \ecinput{../examps/parts/tactics/cfold/1-4.1.ec}{}{}{}{}
  \end{tsyntax}

  \begin{tsyntax}{cfold\{1\} $\;n$ ! $\;m$ | cfold\{2\} $\;n$ ! $\;m$}
    Like the \hl version, but operating on the designed program of
    a \prhl judgement's conclusion.
  \end{tsyntax}

  \begin{tsyntax}{cfold\{1\} $\;n$ | cfold\{2\} $\;n$}
    Like the general cases, but where $m$ is set so as to be the
    number of statements after the assignment statement.
  \end{tsyntax}
\end{tactic}

% --------------------------------------------------------------------
\begin{tactic}{kill}
  \begin{tsyntax}{kill $\;n$ ! $\;m$}
    Fails unless $n\geq 1$ and $m\geq 0$.  If the goal's conclusion is
    an \hl statement judgement in which the statement block from
    position $n$ to $n + m - 1$ is well-defined (when $m = 0$, this
    block is empty), and the variables written by this statement
    block aren't used in the judgement's postcondition or read by the
    rest of the program, then reduce the goal to two subgoals.
    \begin{itemize}
    \item One whose conclusion is a \phl statement judgement whose pre-
       and postconditions are \ec{true}, whose program is the
       statement block, and whose bound part is \ec{= 1\%r}.

    \item One that's identical to the original goal except that the
      statement block has been removed.
    \end{itemize}

    \medskip For example, if the current goal is
    \ecinput{../examps/parts/tactics/kill/1-1.0.ec}{}{}{}{} then
    running \ecinput{../examps/parts/tactics/kill/1-1.ec}{}{}{}{}
    produces the goals
    \ecinput{../examps/parts/tactics/kill/1-1.1.ec}{}{}{}{}
    and
    \ecinput{../examps/parts/tactics/kill/1-1.2.ec}{}{}{}{}
  \end{tsyntax}

  \begin{tsyntax}{kill\{1\} $\;n$ ! $\;m$ | kill\{2\} $\;n$ ! $\;m$}
    Like the \hl case but for \prhl judgements, where the statement
    block to be killed is in the designated program.

    \medskip For example, if the current goal is
    \ecinput{../examps/parts/tactics/kill/1-2.0.ec}{}{}{}{} then
    running \ecinput{../examps/parts/tactics/kill/1-2.ec}{}{}{}{}
    produces the goals
    \ecinput{../examps/parts/tactics/kill/1-2.1.ec}{}{}{}{}
    and
    \ecinput{../examps/parts/tactics/kill/1-2.2.ec}{}{}{}{}
  \end{tsyntax}

  \begin{tsyntax}{kill $\;n$ | kill\{1\} $\;n$ | kill\{2\} $\;n$}
    Like the general cases, but with $m = 1$.
  \end{tsyntax}

  \begin{tsyntax}{kill $\;n$ ! * | kill\{1\} $\;n$ ! * | kill\{2\} $\;n$ ! *}
    Like the general cases, but with $m$ set so that the statement
    block to be killed is the rest of the (designated) program.
  \end{tsyntax}
\end{tactic}

% --------------------------------------------------------------------
\begin{tactic}{modpath}
\end{tactic}


\subsection{Tactics for Reasoning about Specifications}

% --------------------------------------------------------------------
\begin{tactic}{symmetry}
  \begin{tsyntax}{symmetry}
  In \prhl, swaps the two programs, transforming the pre and
  postconditions by swapping the memories they refer to.

  \textbf{Examples:} In the following, $\symrel{\cdot}$ inverses its
  argument relation. (That is, for any relation $R$ and any
  $(m_1,m_2)\in{R}$, we have $(m_2,m_1)\in\symrel{R}$.)
  $$
  \inferrule*[left=(\prhl)]%%
    {\pRHL{\symrel{P}}{c_2}{c_1}{\symrel{Q}}}%%
    {\pRHL{P}{c_1}{c_2}{Q}}%%
    \quad\raisebox{.7em}{\tct{symmetry}}
  $$
  \end{tsyntax}
\end{tactic}

% --------------------------------------------------------------------
\begin{tactic}{transitivity}
  \begin{tsyntax}{transitivity c ($P_1$ ==> $\ Q_1$) ($P_2$ ==> $\ Q_2$)}
  In \prhl, applies the transitivity of program equivalence using the
  specified program and specifications. When the goal is a judgment on
  procedures, \ec{c} should be a procedure. When the goal is a
  judgment on statements, \ec{c} should be a statement, and the
  tactic then takes a side argument, used to decide the procedure
  context under which local variables from \ec{c} are evaluated.

  \textbf{Examples:}
  \begin{mathpar}
  \inferrule%%
    {\forall \mem{m_1}\ \mem{m_2}.\, \Rel{P}{\mem{m_1}}{\mem{m_2}} \Rightarrow
        \exists \mem{m}.\, \Rel{P_1}{\mem{m_1}}{\mem{m}}
                           \wedge \Rel{P_2}{\mem{m}}{\mem{m_2}} \\%
     \forall \mem{m_1}\ \mem{m}\ \mem{m_2}.\,
        \Rel{Q_1}{\mem{m_1}}{\mem{m}} \Rightarrow
        \Rel{Q_2}{\mem{m}}{\mem{m_2}} \Rightarrow
        \Rel{Q}{\mem{m_1}}{\mem{m_2}} \\%
     \pRHL{P_1}{f_1}{f}{Q_1} \\%
     \pRHL{P_2}{f}{f_2}{Q_2}}%%
    {\pRHL{P}{f_1}{f_2}{Q}}%%
    \quad\mbox{(\prhl)\quad\parbox{200pt}{\ec{transitivity f ($P_1$ ==> $\ Q_1$) ($P_2$ ==> $\ Q_2$)}}} \\
  \inferrule%%
    {\forall \mem{m_1}\ \mem{m_2}.\, \Rel{P}{\mem{m_1}}{\mem{m_2}} \Rightarrow
        \exists \mem{m}.\, \Rel{P_1}{\mem{m_1}}{\mem{m}}
                           \wedge \Rel{P_2}{\mem{m}}{\mem{m_2}} \\%
     \forall \mem{m_1}\ \mem{m}\ \mem{m_2}.\,
        \Rel{Q_1}{\mem{m_1}}{\mem{m}} \Rightarrow
        \Rel{Q_2}{\mem{m}}{\mem{m_2}} \Rightarrow
        \Rel{Q}{\mem{m_1}}{\mem{m_2}} \\%
     \pRHL{P_1}{s_1}{s}{Q_1} \\%
     \pRHL{P_2}{s}{s_2}{Q_2}}%%
    {\pRHL{P}{s_1}{s_2}{Q}}%%
    \quad\mbox{(\prhl)\quad\parbox{200pt}{\ec{transitivity$\{$1$\}$ $\ \{$ s $\ \}$ ($P_1$ ==> $\ Q_1$) ($P_2$ ==> $\ Q_2$)}}} \\
  \end{mathpar}

  \textbf{Note:} In practice, the existential quantification over
  memory $\mem{m}$ in the first generated subgoal is replaced with an
  existential quantification over the program variables appearing in $P$,
  $P_1$, ot $P_2$.
  \end{tsyntax}
\end{tactic}

% --------------------------------------------------------------------
\begin{tactic}{conseq}
  \begin{tsyntax}{conseq <specification>}
  Rule of consequence. Proves a specification by weakening of a
  stronger result. Any one of the specification places can be filled
  with a wildcard \tct{_} to keep the value it contains in the current
  goal and trivially discharge the corresponding subgoal.

  \textbf{Examples:} In the following, $\leq^\uparrow$ (resp. $=^\uparrow$,
  $\geq^\uparrow$) is $\Leftarrow$ (resp. $\Leftrightarrow$ and
  $\Rightarrow$).
  \begin{mathpar}
  \inferrule*[left=(\prhl),rightskip=10em]%%
    {P' \Rightarrow P \\%
     Q \Rightarrow Q' \\%
     \pRHL{P}{c}{c'}{Q}}%%
    {\pRHL{P'}{c}{c'}{Q'}}%%
    \quad\raisebox{.7em}{\tct{conseq (_: P ==> Q)}} \\
  \inferrule*[left=(\prhl),rightskip=10em]%%
    {Q \Rightarrow Q' \\%
     \pRHL{P'}{c}{c'}{Q}}%%
    {\pRHL{P'}{c}{c'}{Q'}}%%
    \quad\raisebox{.7em}{\tct{conseq (_: _ ==> Q)}} \\
  \inferrule*[left=(\phl),rightskip=10em]%%
    {P' \Rightarrow \delta \mathrel{\diamond} \delta' \\%
     P' \Rightarrow P \\%
     Q \mathrel{\diamond^\uparrow} Q' \\%
     \pHL{P}{c}{Q}{\diamond}{\delta}}%%
    {\pHL{P'}{c}{Q'}{\diamond}{\delta'}}%%
    \quad\raisebox{.7em}{\tct{conseq (_: P ==> Q: $\delta$)}} \\
  \inferrule*[left=(\hl),rightskip=10em]%%
    {P' \Rightarrow P \\%
     Q \Rightarrow Q' \\%
     \HL{P}{c}{Q}}%%
    {\HL{P'}{c}{Q'}}%%
    \quad\raisebox{.7em}{\tct{conseq (_: P ==> Q)}} \\
  \end{mathpar}
  \end{tsyntax}

  \begin{tsyntax}{conseq <lemma>}
  Only works on procedures. Same as \tct{conseq <specification>}, but
  the specification to use is inferred from the lemma provided. Raises
  an error if the lemma does not refer to the expected procedure(s).
  \end{tsyntax}

  \begin{tsyntax}{conseq* <specification>}
  Same as \tct{conseq <specification>}, but the subgoal corresponding
  to the postcondition is refined by a ``may modify'' analysis.
  \end{tsyntax}

  \fix{Missing descriptions of combining variants of conseq}.
\end{tactic}

% --------------------------------------------------------------------
\begin{tactic}[case]{case-pl}
  \begin{tsyntax}{case $\;e$}
    If the goal's conclusion is a \prhl, \hl or \phl \emph{statement}
    judgement and $e$ is well-typed in the goal's context, split the
    goal into two goals:
    \begin{itemize}
    \item a first goal in which $e$ is added as a conjunct to the
      conclusion's precondition; and

    \item a second goal in which $!e$ is added as a conjunct to the
      conclusion's precondition.
    \end{itemize}

    \medskip For example, if the current goal is
    \ecinput{examps/parts/tactics/case_pl/1-1.0.ec} then
    running \ecinput{examps/parts/tactics/case_pl/1-1.ec}
    produces the goals
    \ecinput{examps/parts/tactics/case_pl/1-1.1.ec}
    and
    \ecinput{examps/parts/tactics/case_pl/1-1.2.ec}
    And if the current goal is
    \ecinput{examps/parts/tactics/case_pl/2-1.0.ec} then
    running \ecinput{examps/parts/tactics/case_pl/2-1.ec}
    produces the goals
    \ecinput{examps/parts/tactics/case_pl/2-1.1.ec}
    and
    \ecinput{examps/parts/tactics/case_pl/2-1.2.ec}
  \end{tsyntax}
\end{tactic}

% --------------------------------------------------------------------
\begin{tactic}{phoare split}
  \fxfatal{Update waiting for overhaul of \phl.}

  \begin{tsyntax}{phoare split $\ \delta_{A}$ $\ \delta_{B}$ $\ \delta_{AB}$}
  Splits a \phl judgment whose postcondition is a conjunction or
  disjunction into three \phl judgments following the definition of
  the probability of a disjunction of events.

  \paragraph{Examples:}\strut

  \begin{cmathpar}
  \texample[\phl{}]
    {\ec{phoare split $\ \delta_{A}$ $\ \delta_{B}$ $\ \delta_{AB}$}}
    {\delta_{A} + \delta_{B} - \delta_{AB} \diamond \delta \\
     \pHL{P}{c}{A}{\diamond}{\delta_{A}} \\
     \pHL{P}{c}{B}{\diamond}{\delta_{B}} \\
     \pHL{P}{c}{A \wedge B}{\invrel{\diamond}}{\delta_{AB}}}
    {\pHL{P}{c}{A \vee B}{\diamond}{\delta}}

  \texample[\phl{}]
    {\ec{phoare split $\ \delta_{A}$ $\ \delta_{B}$ $\ \delta_{AB}$}}
    {\delta_{A} + \delta_{B} - \delta_{AB} \diamond \delta \\
     \pHL{P}{c}{A}{\diamond}{\delta_{A}} \\
     \pHL{P}{c}{B}{\diamond}{\delta_{B}} \\
     \pHL{P}{c}{A \vee B}{\invrel{\diamond}}{\delta_{AB}}}
    {\pHL{P}{c}{A \wedge B}{\diamond}{\delta}}
  \end{cmathpar}
  \end{tsyntax}

  \begin{tsyntax}{phoare split $\ {!}$ $\ \delta_{\top}$ $\ \delta_{!}$}
  Splits a \phl judgment into two judgments whose postcondition are
  true and the negation of the original postcondition, respectively.

  \paragraph{Examples:}\strut

  \begin{cmathpar}
  \texample[\phl{}]
    {\ec{phoare split ! $\ \delta_{\top}$ $\ \delta_{!}$}}
    {\delta_{\top} - \delta_{!} \diamond \delta \\
     \pHL{P}{c}{\mathsf{true}}{\diamond}{\delta_{\top}} \\
     \pHL{P}{c}{!Q}{\invrel{\diamond}}{\delta_{!}}}
    {\pHL{P}{c}{Q}{\diamond}{\delta}}
  \end{cmathpar}
  \end{tsyntax}

  \begin{tsyntax}{phoare split $\ \delta_{A}$ $\ \delta_{!A}$: A}
  Splits a \phl judgment following an event $A$.

  \paragraph{Examples:}\strut

  \begin{cmathpar}
  \texample[\phl{}]
    {\ec{phoare split $\ \delta_{A}$ $\ \delta_{!A}$: A}}
    {\delta_{A} + \delta_{!A} \diamond \delta \\
     \pHL{P}{c}{Q \wedge A}{\diamond}{\delta_{A}} \\
     \pHL{P}{c}{Q \wedge \neg A}{\diamond}{\delta_{!A}}}
    {\pHL{P}{c}{Q}{\diamond}{\delta}}
  \end{cmathpar}
  \end{tsyntax}  
\end{tactic}

% --------------------------------------------------------------------
\begin{tactic}{byequiv}
  \begin{tsyntax}{byequiv (_ : $\;P$ ==> $\;Q$)}
    If the goal's conclusion has the form
    \begin{center}
      \ec{Pr[$M_1$.$p_1$($a_{1,1}$, $\ldots$, $a_{1,n_1}$) @ &$m_1$ : $\;E_1$] =
          Pr[$M_2$.$p_2$($a_{2,1}$, $\ldots$, $a_{2,n_2}$) @ &$m_2$ : $\;E_2$]},
    \end{center}
    reduce the goal to three subgoals:
    \begin{itemize}
    \item One with conclusion
          \ec{equiv[$M_1$.$p_1$ ~ $\;M_2$.$p_2$ : $\;P$ ==> $\;Q$]};

    \item One whose conclusion says that $P$ holds, where
      references to memories \ec{&1} and \ec{&2} have been replaced
      by \ec{&$m_1$} and \ec{&$m_2$}, respectively, and references
      to the formal parameters of \ec{$M_1$.$p_1$} and
      \ec{$M_2$.$p_2$} have been replaced by their arguments;

    \item One whose conclusion says that $Q$ implies that
      \ec{$E_1$\{1\} <=> $\;E_2$\{2\}}.
    \end{itemize}

    The argument to \ec{byequiv} may be replaced by a proof term for
    \ec{equiv[$M_1$.$p_1$ ~ $\;M_2$.$p_2$ : $\;P$ ==> $\;Q$]}, in which
    case the first subgoal isn't generated.
    Furthermore, either or both of $P$ and $Q$ may be replaced by
    \ec{_}, asking that the pre- or postcondition be inferred.
    Supplying no argument to \ec{equiv} is the same as replacing
    both $P$ and $Q$ by \ec{_}. By default, inference of $Q$ attempts
    to infer a conjuction of equalities implying   
    \ec{$E_1$\{1\} <=> $\;E_2$\{2\}}. Passing the \ec{[-eq]} option to
    \ec{conseq} takes $Q$ to \emph{be} \ec{$E_1$\{1\} <=> $\;E_2$\{2\}}.

    \medskip
    \emph{The other variants of the tactic behave similarly with
    regards to the use of proof terms and specification inference.}

    \medskip For example, consider the module
    \ecinput{examps/tactics/byequiv/1.ec}{}{3-10}{}
    If the current goal is
    \ecinput{examps/parts/tactics/byequiv/1-1.0.ec}{}{}{}{} then
    running \ecinput{examps/parts/tactics/byequiv/1-1.ec}{}{}{}{}
    produces the goals
    \ecinput{examps/parts/tactics/byequiv/1-1.1.ec}{}{}{}{}
    and
    \ecinput{examps/parts/tactics/byequiv/1-1.2.ec}{}{}{}{}
    and
    \ecinput{examps/parts/tactics/byequiv/1-1.3.ec}{}{}{}{}
    Given the lemma
    \ecinput{examps/tactics/byequiv/1.ec}{}{24-24}{}
    if the current goal is
    \ecinput{examps/parts/tactics/byequiv/1-2.0.ec}{}{}{}{} then
    running \ecinput{examps/parts/tactics/byequiv/1-2.ec}{}{}{}{}
    produces the goals
    \ecinput{examps/parts/tactics/byequiv/1-2.1.ec}{}{}{}{}
    and
    \ecinput{examps/parts/tactics/byequiv/1-2.2.ec}{}{}{}{}
    And, if the current goal is
    \ecinput{examps/parts/tactics/byequiv/1-3.0.ec}{}{}{}{} then
    running \ecinput{examps/parts/tactics/byequiv/1-3.ec}{}{}{}{}
    produces the goals
    \ecinput{examps/parts/tactics/byequiv/1-3.1.ec}{}{}{}{}
    and
    \ecinput{examps/parts/tactics/byequiv/1-3.2.ec}{}{}{}{}
    and
    \ecinput{examps/parts/tactics/byequiv/1-3.3.ec}{}{}{}{}
  \end{tsyntax}

  \begin{tsyntax}{byequiv (_ : $\;P$ ==> $\;Q$)}
    If the goal's conclusion has the form
    \begin{center}
      \ec{Pr[$M_1$.$p_1$($a_{1,1}$, $\ldots$, $a_{1,n_1}$) @ &$m_1$ : $\;E_1$] <=
          Pr[$M_2$.$p_2$($a_{2,1}$, $\ldots$, $a_{2,n_2}$) @ &$m_2$ : $\;E_2$]},
    \end{center}
    then \ec{conseq} behaves the same as in the first variant
    except that the conclusion of the third subgoal says that $Q$
    implies \ec{$E_1$\{1\} => $\;E_2$\{2\}}.

    \medskip For example, if the current goal is
    \ecinput{examps/parts/tactics/byequiv/2-1.0.ec}{}{}{}{} then
    running \ecinput{examps/parts/tactics/byequiv/2-1.ec}{}{}{}{}
    produces the goals
    \ecinput{examps/parts/tactics/byequiv/2-1.1.ec}{}{}{}{}
    and
    \ecinput{examps/parts/tactics/byequiv/2-1.2.ec}{}{}{}{}
    and
    \ecinput{examps/parts/tactics/byequiv/2-1.3.ec}{}{}{}{}
    And, if the current goal is
    \ecinput{examps/parts/tactics/byequiv/2-2.0.ec}{}{}{}{} then
    running \ecinput{examps/parts/tactics/byequiv/2-2.ec}{}{}{}{}
    produces the goals
    \ecinput{examps/parts/tactics/byequiv/2-2.1.ec}{}{}{}{}
    and
    \ecinput{examps/parts/tactics/byequiv/2-2.2.ec}{}{}{}{}
    and
    \ecinput{examps/parts/tactics/byequiv/2-2.3.ec}{}{}{}{}
  \end{tsyntax}

  \begin{tsyntax}{byequiv (_ : $\;P$ ==> $\;Q$)}
    If the goal's conclusion has the form
    \begin{center}
      \ec{Pr[$M_1$.$p_1$($a_{1,1}$, $\ldots$, $a_{1,n_1}$) @ &$m_1$ : $\;E_1$] <=} \\
      \ec{Pr[$M_2$.$p_2$($a_{2,1}$, $\ldots$, $a_{2,n_2}$) @ &$m_2$ : $\;E_2$] +
          Pr[$M_2$.$p_2$($a_{2,1}$, $\ldots$, $a_{2,n_2}$) @ &$m_2$ : $\;B_2$]},
    \end{center}
    then \ec{conseq} behaves the same as in the first variant
    except that the conclusion of the third subgoal says that $Q$
    implies \ec{!$B_2$\{2\} => $\;E_1$\{1\} => $\;E_2$\{2\}}.

    \medskip For example, if the current goal is
    \ecinput{examps/parts/tactics/byequiv/3-1.0.ec}{}{}{}{} then
    running \ecinput{examps/parts/tactics/byequiv/3-1.ec}{}{}{}{}
    produces the goals
    \ecinput{examps/parts/tactics/byequiv/3-1.1.ec}{}{}{}{}
    and
    \ecinput{examps/parts/tactics/byequiv/3-1.2.ec}{}{}{}{}
    \fixme{Why is the second subgoal pruned? (Compare with first and
    second variants, where the corresponding subgoal isn't pruned.)}
    And, if the current goal is
    \ecinput{examps/parts/tactics/byequiv/3-2.0.ec}{}{}{}{} then
    running \ecinput{examps/parts/tactics/byequiv/3-2.ec}{}{}{}{}
    produces the goals
    \ecinput{examps/parts/tactics/byequiv/3-2.1.ec}{}{}{}{}
    and
    \ecinput{examps/parts/tactics/byequiv/3-2.2.ec}{}{}{}{}
    \fixme{Why is the second subgoal pruned?}
  \end{tsyntax}

  \begin{tsyntax}{byequiv (_ : $\;P$ ==> $\;Q$) : $\;B_1$}
    If the goal's conclusion has the form
    \begin{center}
      \ec{`| Pr[$M_1$.$p_1$($a_{1,1}$, $\ldots$, $a_{1,n_1}$) @ &$m_1$ : $\;E_1$] -
          Pr[$M_2$.$p_2$($a_{2,1}$, $\ldots$, $a_{2,n_2}$) @ &$m_2$ : $\;E_2$] | <=} \\
      \ec{Pr[$M_2$.$p_2$($a_{2,1}$, $\ldots$, $a_{2,n_2}$) @ &$m_2$ : $\;B_2$]},
    \end{center}
    then \ec{conseq} behaves the same as in the first variant
    except that the conclusion of the third subgoal says that $Q$
    implies
    \begin{center}
      \ec{($B_1$\{1\} <=> $\;B_2$\{2\}) /\\ ! $\;B_2$\{2\} => ($\;E_1$\{1\} <=> $\;E_2$\{2\})}
    \end{center}

    \medskip For example, if the current goal is
    \ecinput{examps/parts/tactics/byequiv/4-1.0.ec}{}{}{}{} then
    running \ecinput{examps/parts/tactics/byequiv/4-1.ec}{}{}{}{}
    produces the goals
    \ecinput{examps/parts/tactics/byequiv/4-1.1.ec}{}{}{}{}
    and
    \ecinput{examps/parts/tactics/byequiv/4-1.2.ec}{}{}{}{}
    and
    \ecinput{examps/parts/tactics/byequiv/4-1.3.ec}{}{}{}{}
    Given the lemma
    \ecinput{examps/tactics/byequiv/4.ec}{}{37-38}{}
    if the current goal is
    \ecinput{examps/parts/tactics/byequiv/4-2.0.ec}{}{}{}{} then
    running \ecinput{examps/parts/tactics/byequiv/4-2.ec}{}{}{}{}
    produces the goals
    \ecinput{examps/parts/tactics/byequiv/4-2.1.ec}{}{}{}{}
    and
    \ecinput{examps/parts/tactics/byequiv/4-2.2.ec}{}{}{}{}
  \end{tsyntax}
\end{tactic}

%%  \begin{tsyntax}{byequiv [option]? <spec>}
%%  Derives a probability relation from a \prhl judgement on the
%%  procedures involved. \ec{<spec>} can include wildcards when the
%%  tactic should infer the pre or postcondition. In addition,
%%  \ec{<spec>} can be extended with a failure event to infer precise
%%  applications of the Fundamental Lemma.
%%
%%  \textbf{Options:} By default, (\ec{eq} option) specification
%%  inference attempts to infer a conjunction of equalities sufficient
%%  to imply the desired relation. Passing the \ec{-eq} option
%%  overrides this behaviour, instead using the trivial relation on
%%  events.
%%
%%  \paragraph{Examples:}\strut
%%
%%  \begin{cmathpar}
%%    \texample
%%      {\ec{byequiv (_: P ==> Q)}}
%%      {\pRHL{P}{f_1}{f_2}{Q} \\\\
%%       {m_1[\Arg\mapsto\vec{a}_1]} \rel{P} {m_2[\Arg\mapsto\vec{a}_2]} \\
%%       Q \Rightarrow E_1\{1\}  \Leftrightarrow E_2\{2\}}
%%      {\PR{f_1}{\vec{a}_1}{\mem{m_1}}{E_1} = \PR{f_2}{\vec{a}_2}{m_2}{E_2}}
%%  \end{cmathpar}
%%
%%  \begin{cmathpar}
%%    \texample
%%      {\ec{byequiv (_: P ==> Q)}}
%%      {\pRHL{P}{f_1}{f_2}{Q} \\\\
%%       {m_1[\Arg\mapsto\vec{a}_1]} \rel{P} {m_2[\Arg\mapsto\vec{a}_2]} \\
%%       Q \Rightarrow E_1\{1\}  \Rightarrow E_2\{2\}}
%%      {\PR{f_1}{\vec{a}_1}{\mem{m_1}}{E_1} \leq \PR{f_2}{\vec{a}_2}{m_2}{E_2}}
%%  \end{cmathpar}
%%
%%  \begin{cmathpar}
%%    \texample
%%      {\ec{byequiv (_: P ==> Q)}}
%%      {\pRHL{P}{f_1}{f_2}{Q} \\\\
%%       {m_1[\Arg\mapsto\vec{a}_1]} \rel{P} {m_2[\Arg\mapsto\vec{a}_2]} \\
%%       Q \Rightarrow E_2\{2\}  \Rightarrow E_1\{1\}}
%%      {\PR{f_1}{\vec{a}_1}{\mem{m_1}}{E_1} \geq \PR{f_2}{\vec{a}_2}{m_2}{E_2}}
%%  \end{cmathpar}
%%
%%  \begin{cmathpar}
%%    \texample
%%      {\ec{byequiv (_: P ==> Q)}}
%%      {\pRHL{P}{f_1}{f_2}{Q} \\
%%       {m_1[\Arg\mapsto\vec{a}_1]} \rel{P} {m_2[\Arg\mapsto\vec{a}_2]} \\
%%       Q \Rightarrow \neg B_2\{2\} \Rightarrow E_1\{1\}  \Rightarrow E_2\{2\}}
%%      {\PR{f_1}{\vec{a}_1}{\mem{m_1}}{E_1}
%%       \leq \PR{f_2}{\vec{a}_2}{m_2}{E_2}
%%          + \PR{f_2}{\vec{a}_2}{m_2}{B_2}}
%%  \end{cmathpar}
%%
%%  \begin{cmathpar}
%%    \texample
%%      {\ec{byequiv (_: P ==> Q) : B$_1$}}
%%      {\pRHL{P}{f_1}{f_2}{Q} \\
%%       {m_1[\Arg\mapsto\vec{a}_1]} \rel{P} {m_2[\Arg\mapsto\vec{a}_2]} \\
%%       Q \Rightarrow
%%         (B_1\{1\} \Leftrightarrow B_2\{2\})
%%         \wedge (\neg B_2\{2\} \Rightarrow E_1\{1\} \Leftrightarrow E_2\{2\})}
%%      {| \PR{f_1}{\vec{a}_1}{\mem{m_1}}{E_1} - \PR{f_2}{\vec{a}_2}{m_2}{E_2} |
%%       \leq \PR{f_2}{\vec{a}_2}{m_2}{B_2}}
%%  \end{cmathpar}
%%
%%  \begin{cmathpar}
%%    \texample
%%      {\ec{byequiv [-eq] (_: P ==> _)}}
%%      {\pRHL{P}{f_1}{f_2}{E_1\{1\} \Leftrightarrow E_2\{2\}} \\
%%       {m_1[\Arg\mapsto\vec{a}_1]} \rel{P} {m_2[\Arg\mapsto\vec{a}_2]}}
%%      {\PR{f_1}{\vec{a}_1}{\mem{m_1}}{E_1} = \PR{f_2}{\vec{a}_2}{m_2}{E_2}}
%%  \end{cmathpar}
%% \end{tsyntax}
%%
%%  \begin{tsyntax}{byequiv <lemma>}
%%  Same as \ec{byequiv <spec>}, but the specification to use
%%  is inferred from the lemma provided. Raises an error if the lemma
%%  does not refer to the expected procedures. Inference options have no
%%  effect in this setting.
%%  \end{tsyntax}

% --------------------------------------------------------------------
\begin{tactic}{byphoare}
\end{tactic}

% --------------------------------------------------------------------
\begin{tactic}{hoare}
  \begin{tsyntax}[empty]{hoare}
  \fix{Missing description of hoare}.
  \end{tsyntax}
\end{tactic}

% --------------------------------------------------------------------
\begin{tactic}{bypr}
  \begin{tsyntax}{bypr $\;e_1$ $\;e_2$}
    If the goal's conclusion has the form
    \begin{center}
      \ec{equiv[$M$.$p$ ~ $\;N$.$q$ : $\;P$ ==> $\;Q$]},
    \end{center}
    and the $e_i$ are expressions of the same type possibily
    involving memories \ec{&1} and \ec{&2} for \ec{$M$.$p$} and
    \ec{$N$.$q$}, respectively, then reduce the goal to two subgoals:
    \begin{itemize}
    \item One whose conclusion says that for all memories \ec{&1}
      and \ec{&2} for \ec{$M$.$p$} and \ec{$N$.$q$}, if $e_1 = e_2$,
      then $Q$ holds; and

    \item One whose conclusion says that, for all memories \ec{&1} and
      \ec{&2} for \ec{$M$.$p$} and \ec{$N$.$q$} and values $a$ of the
      common type of the $e_i$, if $P$ holds, then the probability of
      running \ec{$M$.$p$} in memory \ec{&1} and with arguments
      consisting of the values of its formal parameters in \ec{&1} and
      terminating in a memory in which the value of $e_1$ (replacing
      references to \ec{&1} with reference to this memory) is $a$ is
      the same as the probability of running \ec{$N$.$q$} in memory
      \ec{&2} and with arguments consisting of the values of its
      formal parameters in \ec{&2} and terminating in a memory in
      which the value of $e_2$ (replacing references to \ec{&2} with
      reference to this memory) is $a$.
    \end{itemize}

    \medskip For example, consider the modules
    \ecinput[linerange=3-20]{examps/tactics/bypr/1.ec}
    If the current goal is
    \ecinput{examps/parts/tactics/bypr/1-1.0.ec} then
    running \ecinput{examps/parts/tactics/bypr/1-1.ec}
    produces the goals
    \ecinput{examps/parts/tactics/bypr/1-1.1.ec}
    and
    \ecinput{examps/parts/tactics/bypr/1-1.2.ec}
  \end{tsyntax}

  \begin{tsyntax}{bypr}
    If the goal's conclusion has the form
    \begin{center}
      \ec{hoare[$M$.$p$ : $\;P$ ==> $\;Q$]},
    \end{center}
    then reduce the goal to one whose conclusion says that, for all
    memories \ec{&m} for \ec{$M$.$p$} such that \ec{$P$\{$m$\}} holds,
    the probability of running \ec{$M$.$p$} in memory \ec{&$m$} and
    with arguments consisting of the values of its formal parameters
    in \ec{&$m$} and terminating in a memory satisfying \ec{!$Q$} is $0$.

    \medskip For example, consider the module
    \ecinput[linerange=3-10]{examps/tactics/bypr/2.ec}
    If the current goal is
    \ecinput{examps/parts/tactics/bypr/2-1.0.ec} then
    running \ecinput{examps/parts/tactics/bypr/2-1.ec}
    produces the goal
    \ecinput{examps/parts/tactics/bypr/2-1.1.ec}
  \end{tsyntax}
\end{tactic}

%%  Derives a program judgment from a probability relation or an exact
%%  probability. Only applies to judgments on procedures.
%%
%%  \paragraph{Examples:}\strut
%%  
%%  \begin{cmathpar}
%%    \texample[\prhl{}]
%%      {\ec{bypr (r$_1$) (r$_2$)}}
%%      {\forall \mem{m_1}, \mem{m_2}, a.\,
%%          r_1 = a \Rightarrow
%%          r_2 = a \Rightarrow
%%          {\mem{m_1}} \rel{Q} {\mem{m_2}} \\
%%       \forall \vec{a}_1, \vec{a}_2, \mem{m_1}, \mem{m_2}, a.\,
%%         {\mem{m_1}[\Arg\mapsto\vec{a}_1]} \rel{P} {\mem{m_2}[\Arg\mapsto\vec{a}_2]} \Rightarrow \\
%%         \PR{f_1}{\vec{a}_1}{\mem{m_1}}{a = r_1} = \PR{f_2}{\vec{a}_2}{\mem{m_2}}{a = r_2}}
%%      {\pRHL{P}{f_1}{f_2}{Q}}
%%  \end{cmathpar}
%%
%%  \begin{cmathpar}
%%    \texample[\phl{}]
%%      {\ec{bypr}}
%%      {\forall \mem{m}, \vec{a}.\, P\ m[\Arg\mapsto\vec{a}] \Rightarrow
%%          \PR{f}{\vec{a}}{m}{E} \mathrel{\diamond} \delta}
%%      {\pHL{P}{f}{E}{\diamond}{\delta}}
%%  \end{cmathpar}
%%
%%  \begin{cmathpar}
%%    \texample[\hl{}]
%%      {\ec{bypr}}
%%      {\forall \mem{m}, \vec{a}.\, P\ m[\Arg\mapsto\vec{a}] \Rightarrow
%%          \PR{f}{\vec{a}}{m}{\neg E} \mathop{=}0\%r}
%%      {\HL{P}{f}{E}}
%%  \end{cmathpar}
%%  \end{tsyntax}
%%\end{tactic}

% --------------------------------------------------------------------
\begin{tactic}{exists*}
  \begin{tsyntax}[empty]{exists*}

  \end{tsyntax}
\end{tactic}

% --------------------------------------------------------------------
\begin{tactic}{elim*}
  \begin{tsyntax}[empty]{elim*}
  Destruct existential quantifications at the head of a
  precondition. Such existential quantifications may be introduced by
  \rtactic{sp} or \rtactic{exists*}.

  \paragraph{Examples:}\strut

  \begin{cmathpar}
  \texample[\prhl{}]{\ec{elim*}}%%
    {\forall x.\, \pRHL{x = \inmem{M.x}{1} \wedge P}{c_1}{c_2}{Q}}%%
    {\pRHL{\exists x, x = \inmem{M.x}{1} \wedge P}{c_1}{c_2}{Q}}

  \texample[\phl{}]{\ec{elim*}}%%
    {\forall x.\, \pHL{x = M.x \wedge P}{c}{Q}{\diamond}{\delta}}%%
    {\pHL{\exists x, x = M.x \wedge P}{c}{Q}{\diamond}{\delta}}

  \texample[\hl{}]{\ec{elim*}}%%
    {\forall x.\, \HL{x = M.x \wedge P}{c}{Q}}%%
    {\HL{\exists x, x = M.x \wedge P}{c}{Q}}  
  \end{cmathpar}

  \end{tsyntax}
\end{tactic}

% --------------------------------------------------------------------
\begin{tactic}{exfalso}
  \begin{tsyntax}{exfalso}
  Combines \rtactic{conseq}, \rtactic{byequiv}, \rtactic{byphoare},
  \rtactic{hoare} and \rtactic{bypr} to strengthen the precondition
  into $\mathsf{false}$ and to discharge the resulting trivial goal.

  \textbf{Examples:}
  \begin{mathpar}
  \inferrule%%
    {P \Rightarrow \mathsf{false}}%%
    {\pRHL{P}{c}{c'}{Q}}%%
    \quad\mbox{(\prhl)\quad\parbox{200pt}{\tct{exfalso}}} \\
  \inferrule%%
    {P \Rightarrow \mathsf{false}}%%
    {\pHL{P}{c}{Q}{\diamond}{\delta}}%%
    \quad\mbox{(\phl)\quad\parbox{200pt}{\tct{exfalso}}} \\
  \inferrule%%
    {P \Rightarrow \mathsf{false}}%%
    {\HL{P}{c}{Q}}%%
    \quad\mbox{(\hl)\quad\parbox{200pt}{\tct{exfalso}}} \\
  \end{mathpar}
  \fix{Move \tct{exfalso} to automatic tactics}?
  \end{tsyntax}
\end{tactic}


\subsection{Automated Tactics}
\label{subsec:automatedtactics}

% --------------------------------------------------------------------
\begin{tactic}{auto}
\end{tactic}

% --------------------------------------------------------------------
\begin{tactic}{sim}
  \ec{sim} attempts to solve a goal whose conclusion is a \prhl
  judgement or statement judgement by working backwards, propagating
  and extending a conjunction of equalties between variables of the
  two programs, verifying that the conclusion's precondition implies
  the final conjuction of equalities.  It's capable of working
  backwards through \ec{if} and \ec{while} statements and handing
  random assignments, but only when the programs are sufficiently
  similar (thus its name). Sometimes this process only partly
  succeeds, leaving a statement judgement whose programs are prefixes
  of the original programs.

  \begin{tsyntax}{sim}
    Without any arguments, \ec{sim} attemps to infer the conjuction of
    program variable equalities from the conclusion's postcondition.

    \medskip For example, if the current goal is
    \ecinput{examps/parts/tactics/sim/1-1.0.ec}{}{}{}{} then
    running \ecinput{examps/parts/tactics/sim/1-1.ec}{}{}{}{}
    produces the goal
    \ecinput{examps/parts/tactics/sim/1-1.1.ec}{}{}{}{}
    which \ec{auto} is able to solve.
  \end{tsyntax}

  \begin{tsyntax}{sim / $\;\phi$ : $\;\mathit{eqs}$}
    One may give the starting conjuction, $\mathit{eqs}$, of equalities
    explicitly, and may also specifiy an invariant $\phi$ on the
    global variables of the programs.

    \medskip For example, if the current goal is
    \ecinput{examps/parts/tactics/sim/1-2.0.ec}{}{}{}{} then
    running \ecinput{examps/parts/tactics/sim/1-2.ec}{}{}{}{}
    produces the goals
    \ecinput{examps/parts/tactics/sim/1-2.1.ec}{}{}{}{}
    and
    \ecinput{examps/parts/tactics/sim/1-2.2.ec}{}{}{}{}
    which \ec{smt} and \ec{auto;smt}, respectively, are
    able to solve.
  \end{tsyntax}

  \begin{tsyntax}{sim $\;\mathit{proceq}_1$ $\;\ldots$ $\;\mathit{proceq}_1$ / $\;\phi$ : $\;\mathit{eqs}$}
    In its most general form, one may also supply a sequence of
    procedure global equality specifications of the form
    \begin{center}
      \ec{($M$.$p$ ~ $\;N$.$q$ : $\;\mathit{eqs}$)},
    \end{center}
    where $\mathit{eqs}$ is a conjuction of global variable
    equalities. When \ec{sim} encounters a pair of procedure calls
    consisting of a call to \ec{$M$.$p$} in the first program and
    \ec{$N$.$q$} in the second program, it will generate a subgoal
    whose conclusion is a \prhl judgment between \ec{$M$.$p$} and
    \ec{$N$.$q$}, whose precondition assumes equality of its
    arguments, $\mathit{eqs}$ and $\phi$, and whose postcondition
    requires equality of the calls' results, $\mathit{eqs}$ and
    $\phi$.

    One may also replace \ec{$M$.$p$ ~ $\;N$.$q$} by \ec{_},
    meaning that the same conjunction of global variable equalities
    is used for all procedure calls.

    \medskip For example, if the current goal is
    \ecinput{examps/parts/tactics/sim/2-1.0.ec}{}{}{}{} then
    running \ecinput{examps/parts/tactics/sim/2-1.ec}{}{}{}{}
    produces the goals
    \ecinput{examps/parts/tactics/sim/2-1.1.ec}{}{}{}{}
    and
    \ecinput{examps/parts/tactics/sim/2-1.2.ec}{}{}{}{}
    and
    \ecinput{examps/parts/tactics/sim/2-1.3.ec}{}{}{}{}
    and
    \ecinput{examps/parts/tactics/sim/2-1.4.ec}{}{}{}{}
    which \ec{smt}, \ec{proc;auto;smt}, \ec{proc;auto;smt}
    and \ec{auto}, respectively, are able to solve.

    \medskip And, if the current goal is
    \ecinput{examps/parts/tactics/sim/2-2.0.ec}{}{}{}{} then
    running \ecinput{examps/parts/tactics/sim/2-2.ec}{}{}{}{}
    produces the goals
    \ecinput{examps/parts/tactics/sim/2-2.1.ec}{}{}{}{}
    and
    \ecinput{examps/parts/tactics/sim/2-2.2.ec}{}{}{}{}
    and
    \ecinput{examps/parts/tactics/sim/2-2.3.ec}{}{}{}{}
    and
    \ecinput{examps/parts/tactics/sim/2-2.4.ec}{}{}{}{}
    which \ec{smt}, \ec{proc;auto;smt}, \ec{proc;auto;smt}
    and \ec{auto}, respectively, are able to solve.
  \end{tsyntax}
\end{tactic}

%%  \begin{tsyntax}{sim <pos>? <hintgeqs>* <hintinv>? <eqs>?}\\
%%    where \begin{tabular}{lrl}
%%       <pos>      & = & <uint> <uint> \\
%%       <hintgeqs> & = & (<procname>? $\sim$ <procname>? : <formula>) \\
%%                  & | & ($\_$? : <formula>  \\
%%       <eqs>      & = & : <formula> \\
%%    \end{tabular}
%%  
%%  \fix{Missing description of sim}.
%%  \end{tsyntax}
%%\end{tactic}


\subsection{Advanced Tactics}
\label{subsec:advancedtactics}

% --------------------------------------------------------------------
\begin{tactic}{eager}
  \begin{tsyntax}[empty]{eager}
  \fix{Missing description of eager}.
  \fix{Missing descriptions for all eager <tactic> variants}.
  \end{tsyntax}
\end{tactic}

% --------------------------------------------------------------------
\begin{tactic}{fel}
  \begin{tsyntax}{fel $\;\mathit{init}$ $\;\mathit{ctr}$ $\;\mathit{stepub}$
                      $\;\mathit{bound}$ $\;\mathit{bad}$ $\;\mathit{conds}$
                      $\;\mathit{inv}$}
    ``fel'' stands for ``failure event lemma''. To use this tactic,
    one must load the theory \ec{FelTactic}. To be applicable, the
    current goal's conclusion must have the form
    \begin{center}
      \ec{Pr[$M$.$p$($a_1$, $\;\ldots$, $\;a_r$) @ &$m$ : $\;\phi$] <= $\;\mathit{ub}$}.
    \end{center}
    Here:
    \begin{itemize}
    \item $\mathit{ub}$ (``upper bound'') is an expression of type \ec{real}.

    \item $\mathit{ctr}$ is the \emph{counter}, an expression of
      type \ec{int} involving program variables.

    \item $\mathit{bad}$ is an expression of type \ec{bool} involving
      program variables. It is the ``bad'' or ``failure'' event.

    \item $\mathit{inv}$ is an optional invariant on program
      variables; if it's omitted, \ec{true} is used.

    \item $\mathit{init}$ is a natural number no bigger than the
      number of statements in $M$.$p$. It is the length of the initial
      part of the procedure that ``initializes'' the failure event
      lemma---causing $\mathit{ctr}$ to become $0$ and $\mathit{bad}$
      to become \ec{false} and establishing $\mathit{inv}$.  The
      non-initialization part of the procedure may not \emph{directly}
      use the program variables on which $\mathit{ctr}$,
      $\mathit{bad}$ and $\mathit{inv}$ depend. These variables may
      only be modified by concrete procedures \ec{$M$.$p$} may
      directly or indirectly call---such procedures are called
      \emph{oracle procedures}.  If \ec{$M$.$p$} directly or
      indirectly calls an abstract procedure, there must be a module
      constraint saying that the abstract procedure may not modify the
      program variables determining the values of $\mathit{ctr}$,
      $\mathit{bad}$ and $\mathit{inv}$ or that are used by the oracle
      procedures.

    \item $\mathit{bound}$ is an expression of type \ec{int}. It must
      be the case that
      \begin{center}
        \ec{$\phi$ /\\ $\;\mathit{inv}$ => $\;\mathit{bad}$ /\\
            $\;\mathit{ctr}$ <= $\;\mathit{bound}$}.
      \end{center}

    \item $\mathit{conds}$ is a list of \emph{procedure preconditions}
      \begin{center}
        \ec{[$N_1$.$p_1$ : $\;\phi_1$; $\;\ldots$; $\;N_l$.$p_l$ : $\;\phi_l$]},
      \end{center}
      where the $N_i$.$p_i$ are procedures, and the $\phi_i$ are
      expressions of type \ec{bool} involving program variables and
      procedure parameters.  When a procedure's precondition is true,
      it must increase the counter's value; when it isn't true, it
      must not decrease the counter's value, and must preserve the
      value of $\mathit{bad}$. Whether a procedure's precondition
      holds or not, the invariant $\mathit{inv}$ must be preserved.

    \item $\mathit{stepub}$ is a function of type \ec{int -> real},
      providing an upper bound as a function of the
      counter's current value. When a procedure's precondition, the
      invariant $\mathit{inv}$, \ec{!$\mathit{bad}$} and \ec{0 <=
        $\;\mathit{ctr}$ < $\;\mathit{bound}$} hold, the probability
      that $\mathit{bad}$ becomes set during that call must be
      upper-bounded by the application of $\mathit{stepub}$ to the
      counter's value.  In addition, it must be the case that the
      summation of \ec{$\mathit{stepub}$ $\;i$}, as $i$ ranges from
      $0$ to $\mathit{bound} - 1$, is upper-bounded by $\mathit{ub}$.

    \end{itemize}
    The subgoals generated by \ec{fel} enforce the above rules. The
    best way to understand the details is via an example.

    \medskip For example, consider the declarations
    \ecinput[linerange=30-76]{examps/tactics/fel/1.ec} Here, the oracle has
    a boolean variable \ec{won}, which is the bad event. It also has a
    list of integers \ec{gens}, all of which are within the range $1$
    to \ec{upp}, inclusive---the integers ``generated'' so far. The
    counter is the size of \ec{gens}. The procedure \ec{gen} randomly
    generates such an integer, setting \ec{won} to \ec{true} if the
    integer was previously generated. And the procedure \ec{add} adds
    a new integer to the list of generated integers, without possibily
    setting \ec{bad}. Both \ec{gen} and \ec{add} do nothing when the
    counter reaches the bound \ec{n}.  The adversary has access to both
    \ec{gen} and \ec{bad}.

    If the current goal is
    \ecinput{examps/parts/tactics/fel/1-1.0.ec} then
    running \ecinput{examps/parts/tactics/fel/1-1.ec}
    produces the goals
    \ecinput{examps/parts/tactics/fel/1-1.1.ec}
    and
    \ecinput{examps/parts/tactics/fel/1-1.2.ec}
    and
    \ecinput{examps/parts/tactics/fel/1-1.3.ec}
    and
    \ecinput{examps/parts/tactics/fel/1-1.4.ec}
    and
    \ecinput{examps/parts/tactics/fel/1-1.5.ec}
    and
    \ecinput{examps/parts/tactics/fel/1-1.6.ec}
    and
    \ecinput{examps/parts/tactics/fel/1-1.7.ec}
    and
    \ecinput{examps/parts/tactics/fel/1-1.8.ec}
    and
    \ecinput{examps/parts/tactics/fel/1-1.9.ec}
  \end{tsyntax}
\end{tactic}


% Libraries
% !TeX root = easycrypt.tex

%\chapter{Standard Library\label{chap:libraries}}
%\section{Theories}

%\section{Cryptographic Assumptions and Properties}

%%% Local Variables: 
%%% mode: latex
%%% TeX-master: "easycrypt"
%%% End: 

% Examples
% !TeX root = easycrypt.tex

\chapter{Advanced Examples}

\section{Public Key Encryption Schemes}
\subsection{Another Look at \citet{br93}}
In this Section, we return on the example proof discussed in
Section~\ref{sec:tutorial} and rewrite it making use of the standard library and
advanced features of \EC presented in this manual.

\subsection{El Gamal}

\subsection{Hashed El Gamal}

\subsection{OAEP}

\section{Public Key Signature Schemes}
\subsection{FDH}
\subsection{PSS}

%%% Local Variables: 
%%% mode: latex
%%% TeX-master: "easycrypt"
%%% End: 


\addcontentsline{toc}{chapter}{References}
\bibliographystyle{plainnat}
\bibliography{references}

%\part{Language Reference}

\cleardoublepage
\addcontentsline{toc}{part}{Index}
\printindex{easycrypt}{General Index}
\printindex{ambient}{Index of Ambient Logic Tactics}
\printindex{hoare}{Index of Possibilistic Hoare Logic Tactics}
\printindex{phl}{Index of Probabilistic Hoare Logic Tactics}
\printindex{prhl}{Index of Probabilistic Relational Hoare Logic Tactics}
\printindex{commonhl}{Index of Common Hoare Logic Tactics}

\end{document}


\lstnewenvironment{easycrypt}[3][]%
  {\lstset{language=easycrypt,caption={#2},label={#3},#1}}%
  {}

\newcommand{\ecinput}[4]{\lstinputlisting[language=easycrypt,linerange={#3},caption={#2},label={#4}]{#1}}

\def\ls{\lstinline}
\def\ec#1{\lstinline[language=easycrypt-math]"#1"}
\def\ecnocolors#1{\lstinline[language=easycrypt-math-nocolors]"#1"}

\def\Arg{\ensuretext{\ec{arg}}}

% --------------------------------------------------------------------
% Typesetting judgments
\newcommand{\pRHL}[4]{\{#1\}\; #2 \mathrel{\sim} #3\; \{#4\}}
\newcommand{\pHL}[5]{\{#1\}\; #2\; \{#3\} \mathrel{#4} #5}
\newcommand{\HL}[3]{\{#1\}\; #2\; \{#3\}}
\newcommand{\PR}[4]{\mathbf{Pr} [#3, #1(#2) : #4]}

\newcommand{\pRHLs}[4]{\ec{equiv [#2 ~ #3: #1 ==> #4]}}
\newcommand{\pHLs}[5]{\ec{phoare [#2: #1 ==> #3] #4 #5}}
\newcommand{\HLs}[3]{\ec{hoare [#2: #1 ==> #3]}}
\newcommand{\PRs}[4]{\ec{Pr[#1(#2) @ &#3: #4]}}

% --------------------------------------------------------------------
\newcommand{\mem}[1]{#1}
\newcommand{\inmem}[2]{#1\langle{#2}\rangle}

\newcommand{\backticksym}{\symbol{'096}}       % produces `
