% !TeX root = easycrypt.tex

Tactics can be combined together, composed and modified by
\define[tactical]{tacticals}. Tacticals do not correspond to any deduction rule
but make the proof process smoother, and sometimes permit the reuse of proofs
with similar patterns, but where the fine minutiae might differ..

\newcommand{\addTclTacticPlain}[2]{\addTacticPlain{#1}{#2}}
\newcommand{\addTclTacticIdx}[1]{\addTacticIdx{tactical}{#1}}
\newcommand{\addTclTactic}[2]{\addTactic{tactical}{#1}{#2}}

%seq
\addTclTacticPlain{Tactics sequencing}{\ec{t1; t2}}
Execute \ec{t1} and then \ec{t2} on all the subgoals generated by
\ec{t1}.

%try
\addTclTacticPlain{Failure recovering}{\ec{try t}}
\addTclTacticIdx{try}
Execute the tactic \ec{t} or do noting if \ec{t} failed.

%repetition
\addTclTacticPlain{Tactics repetition}{\ec{do !t}}
\addTclTacticIdx{do}
Apply \ec{t} to the current goal, then repeatedly apply it to all subgoals,
stopping only when it fails. An error is produced it \ec{t} does not apply to
the current goal.

\paragraph{Variants}\strut\\

\noindent\begin{tabularx}{\textwidth}{@{}ll@{}}
 {\ec{do ?t}} & apply {\ec{t}} 0 or more times, until it fails\\
 {\ec{do n !t}} & apply {\ec{t}} with depth exactly {\ec{n}}\\
 {\ec{do n ?t}} & apply {\ec{t}} with depth at most {\ec{n}}
\end{tabularx}

%selection
\addTclTacticPlain{Goal selection}{\ec{t1; first t2}}
\addTclTacticIdx{first}
\addTclTacticIdx{last}
Apply the tactic \ec{t1}, then apply \ec{t2} on the first subgoal
generated by \ec{1}. An error is produced if no subgoals have been
generated by \ec{1}.

\paragraph{Variants}\strut\\

\noindent\begin{tabularx}{\textwidth}{@{}ll@{}}
 {\ec{t1; first n t2}} & apply {\ec{t2}} on the first {\ec{n}} subgoals
   generated by {\ec{t1}}\\
 {\ec{t1; last t2}} & apply {\ec{t2}} on the last subgoal
   generated by {\ec{t1}}\\
 {\ec{t1; last n t2}} & apply {\ec{t2}} on the last {\ec{n}} subgoals
   generated by {\ec{t1}}
\end{tabularx}

