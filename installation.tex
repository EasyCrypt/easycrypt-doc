\section{Requirements}

We only list the versions of required software that are known to
work. Note that EasyCrypt may still compile and work as expected using
versions of packages other than those listed.

To compile EasyCrypt and run the examples you will need:

\begin{itemize}
\item GNU Automake

\item GNU Make 3.81

  Available at \url{http://www.gnu.org/software/make/}
  Version 3.82 will most probably work

\item Objective Caml >= 3.11
 
  Available at \url{http://caml.inria.fr/download.en.html}
  Older versions >= 3.08 will most probably work

\item Why3 0.71
 
  Install the version provided in the repository at trunk/why3-0.71
  Patched from the version available at \url{http://why3.lri.fr/}

\item CVC3 2.4.1

  Available at \url{http://www.cs.nyu.edu/acsys/cvc3/}

\item Alt-Ergo 0.94
 
  Available at \url{http://alt-ergo.lri.fr/}

The following automated theorem provers are supported by EasyCrypt,
but are not needed to reproduce the case studies:

\item Z3 

  Available at
  \url{http://research.microsoft.com/en-us/um/redmond/projects/z3/download.html}

\item Simplify

  Pre-compiled binaries for various architectures are available at
  \url{http://krakatoa.lri.fr/ws/Simplify-1.5.5-13-06-07-binary.zip}

\item Yices

  Available at \url{http://yices.csl.sri.com/}

\item Eprover

  Available at \url{http://www4.informatik.tu-muenchen.de/~schulz/E/}

\item Vampire

  Available at \url{http://www.vprover.org/}

\end{itemize}

To install the ProofGeneral front-end for EasyCrypt you will
additionally need:

\begin{itemize}
\item GNU Emacs 23.2
 
  Available at \url{http://www.gnu.org/software/emacs/}

\item  ProofGeneral 4.1

  Available at \url{http://proofgeneral.inf.ed.ac.uk/}
\end{itemize}


\section{Installing Why3}

To compile EasyCrypt you need to install the byte-compiled version of
the Why3 library. Follow first the standard installation instructions 
in the corresponding Why3 README file. If you do not plan to use
Why3 as a standalone tool it is recommended to invoke the Why3 configure
script with the --disable-ide option to avoid unnecessary library 
dependences:
\begin{verbatim}
  ./configure --disable-ide
  make
  make install
\end{verbatim}
After installing Why3 from source code, you must type

\begin{verbatim}
 make byte
 make install-lib
\end{verbatim}

to install the library.

Once you have installed Why3 and the automated provers of your choice,
please make sure that Why is correctly configured to use the provers
by running the command

\begin{verbatim}
 why3config --detect
\end{verbatim}

If everything is correct, you should see a table detailing the provers
that Why3 detected---you can safely ignore any "not know to be
supported" warnings. Remember that you need at least CVC3 and Alt-Ergo
to reproduce the case studies.


\section{Copmpilation}

When the contents of the package were extracted, you should have ended
up with a directory containing this README file and a sub-directory
"easycrypt". ("easycrypt/trunk" when installing from the SVN repository). 

To compile EasyCrypt, simply change to the sub-directory "easycrypt" and type
 
\begin{verbatim}
 ./configure --with-proof-general=PATH_TO_PROOFGENERAL
\end{verbatim}
The argument above is optional but recommended. For a list of additional 
options type

\begin{verbatim}
 ./configure --help
\end{verbatim}

Then type

\begin{verbatim}
 make
\end{verbatim}

and then 

\begin{verbatim}
 make install
\end{verbatim} 

with the appropriate access permissions.

If everything goes well, a binary named "easycrypt.top" will be
generated. To test the setup you may then run

\begin{verbatim}
 make test
\end{verbatim}

and verify that all tests pass.


\section{Running the examples}

Several examples are available under the directory
"easycrypt/examples". To compile them from the directory "easycrypt",
simply type

\begin{verbatim}
 ./easycrypt examples/elgamal.ec
 ./easycrypt examples/helgamal.ec
 ./easycrypt examples/fdh.ec
\end{verbatim}

\section{Installing the ProofGeneral front-end}

\subsection{ProofGeneral - Requirements}

\begin{itemize}
\item Proof general >=  4.1

  Available at http://proofgeneral.inf.ed.ac.uk/download

\item CertiCrypt

\end{itemize}

\subsection{ProofGeneral - Manual Installation}

Add the following line to <proof-general-home>/generic/proof-site.el
in the definition of `proof-assistant-table-default':

\begin{verbatim}
   (certicrypt "CertiCrypt" "ec" nil (".v" ".vo" ".glob" ".ml"))
\end{verbatim}

Copy the directory "certicrypt" and its contents to the directory
where ProofGeneral was installed, typically

\begin{verbatim}
   /usr/local/share/emacs/site-elisp/ProofGeneral/
\end{verbatim}

and check that the final directory has the appropriate access permissions.

\subsection{ProofGeneral - Automatic Installation}

The provided Makefile will install everything in default
locations (or the location specified to the ./configure script). Simply type

\begin{verbatim}
   make install_proofgeneral
\end{verbatim}

or 

\begin{verbatim}
   sudo make install_proofgeneral
\end{verbatim}

as appropriate.


\subsection{ProofGeneral - Configuration}

Add the following line to your emacs configuration file (typically ~/.emacs):

\begin{verbatim}
 (load-file "/usr/share/emacs/site-lisp/proofgeneral/generic/proof-site.el")
\end{verbatim}

Set the path to the EasyCrypt executable and the prelude file in your
Emacs configuration. This can be achieved either by modifying the
variable certicrypt-prog-name inside Emacs:

\begin{verbatim}
 Proof-General
   -> Advanced 
     -> Customize 
       -> Certicrypt 
         -> CertiCrypt prog name
\end{verbatim}

You should set its value to (modifying paths as appropriate):

\begin{verbatim}
 "<path-to-easycrypt>/easycrypt -emacs -prelude <path-to-prelude>/easycrypt_base.ec"
\end{verbatim}

The prelude file "easycrypt-base.ec" can be found at "easycrypt/src/".

Alternatively, you can modify your Emacs local configuration file
(typically ~/.emacs):

\begin{verbatim}
 (custom-set-variables
 ...
  '(certicrypt-prog-name 
    "<path-to-easycrypt>/easycrypt -emacs
        			   -prelude <path-to-prelude>/easycrypt_base.ec)
 ...)
\end{verbatim}





%%% Local Variables: 
%%% mode: latex
%%% TeX-master: "easycrypt"
%%% End: 
