\chapter{Getting Started}
\label{chap:started}

\section{Introduction}

\EasyCrypt~\cite{barthe-easycrypt-intro-2014,barthe-crypto-2011} is a
framework for interactively finding, constructing, and
machine-checking security proofs of cryptographic constructions and
protocols using the code-based sequence of games approach
\cite{bellare-rogaway-code-based-2004,%
  bellare-rogaway-triple-enc-2006,shoup-seq-games-2004}. In
\EasyCrypt, cryptographic games and algorithms are modeled as
\emph{modules}, which consist of procedures written in a simple
user-extensible imperative language featuring while loops and random
sampling operations. Adversaries are modeled by \emph{abstract}
modules---modules whose code is not known and can be quantified
over. Modules may be parameterized by abstract modules.

\EasyCrypt has four logics:
\begin{inparaitem}[]
\item a probabilistic, relational Hoare logic (\prhl), relating pairs
  of procedures;
\item a probabilistic Hoare logic (\phl) allowing one to carry out
  proofs about the probability of a procedure's execution resulting in
  a postcondition holding;
\item an ordinary (possibilistic) Hoare logic (\hl); and
\item an ambient higher-order logic for proving general mathematical
  facts and connecting judgments in the other logics.
\end{inparaitem}
Once lemmas are expressed, proofs are carried out using
\emph{tactics}, logical rules embodying general reasoning principles,
and which transform the current lemma (or \emph{goal}) into zero or
more \emph{subgoals}---sufficient conditions for the original lemma to
hold. Simple ambient logic goals may be automatically proved using SMT
solvers. Proofs may be structured as sequences of lemmas, and
\EasyCrypt's \emph{theories} may be used to group together related
types, predicates, operators, modules, axioms and lemmas. Theory
parameters that may be left abstract when proving its lemmas---types,
operators and predicates---may be instantiated via a \emph{cloning}
process, allowing the development of generic proofs that can later be
instantiated with concrete parameters.

\section{Installing \EasyCrypt}

\EasyCrypt may be found on GitHub.
\begin{center}
\url{https://github.com/EasyCrypt/easycrypt}
\end{center}
%%
Detailed building instructions for \EasyCrypt and its dependencies and
supporting tools can be found in the project's
\href{https://github.com/EasyCrypt/easycrypt/blob/1.0/README.md}{README
  file}.\footnote{\url{https://github.com/EasyCrypt/easycrypt/blob/1.0/README.md}}

\section{Running \EasyCrypt}

\EasyCrypt scripts resides in files with the \texttt{.ec} suffix. (As
we will see in Chapter~\ref{chap:structuring}, \EasyCrypt also has
\emph{abstract} theories, which must be cloned before being used. Such
theories reside in files with the \texttt{.eca} suffix.)

To run \EasyCrypt in \emph{batch} mode, simply invoke it from the
shell, giving it an \EasyCrypt script---with suffix \texttt{.ec}---as
argument:
\begin{lstlisting}[escapechar=\#]
easycrypt #$\mathit{file}\mathtt{.ec}$#
\end{lstlisting}
\EasyCrypt will display its progress as it checks the file.
Information about \EasyCrypt's command-line arguments can be found in
Chapter~\ref{chap:advanced}.

When developing \EasyCrypt scripts, though, \EasyCrypt can be run
\emph{interactively}, as a subprocess of the Emacs text editor. One's
interaction with \EasyCrypt is mediated by Proof General, a generic
Emacs front-end for proof assistants.  Upon visiting an \EasyCrypt
file, the ``Proof-General'' tab of the Emacs menu may be used execute
the file, step-by-step, as well as to undo steps, etc. Information
about the ``EasyCrypt'' menu tab may be found in
Chapter~\ref{chap:advanced}.

A sample \EasyCrypt script is shown in Listing~\ref{list:startedexamp}.
\ecinput[xleftmargin=.09\textwidth,xrightmargin=.09\textwidth]{examps/started-examp-1.ec}{Sample \EasyCrypt
  Script}{}{list:startedexamp}
%%
As can be inferred from the example, comments begin and end with
\ec{(*} and \ec{*)}, respectively; they may be nested. Each sentence
of an \EasyCrypt script is terminated with a dot (period, full stop).
Much can be learned by experimenting with this script, and in
particular by executing it step-by-step in Emacs.

\section{More Information}

More information about \EasyCrypt---and about the \EasyCrypt Team and
its work---may be found at
\begin{center}
  \url{https://www.easycrypt.info}
\end{center}
%%
The \EasyCrypt Club mailing list features discussion about
\EasyCrypt usage:
\begin{center}
  \url{https://lists.gforge.inria.fr/mailman/listinfo/easycrypt-club}
\end{center}
%%
Support requests should be sent to this list, as answers to questions
will be of use to other members of the \EasyCrypt community.

\section{Bug Reporting}

\EasyCrypt bugs should be reported using the Tracker:
\begin{center}
  \url{https://www.easycrypt.info/trac/report}
\end{center}
You can log into the Tracker to create tickets or comment on existing
ones using any GitHub account.

\section{About this Documentation}

The source for this document, along with the macros and language
definitions used, are available from its
\href{https://github.com/EasyCrypt/easycrypt-doc}{GitHub
  repository}.\footnote{\url{https://github.com/EasyCrypt/easycrypt-doc}}
Feel free to use the language definitions to typeset your
\EasyCrypt-related documents, and to contribute improvements to the
macros if you have any.

This document is intended as a reference manual for the \EasyCrypt
tool, and not as a tutorial on how to build a cryptographic proof, or
how to conduct interactive proofs. We provide some detailed examples
in Chapter~\ref{chap:examples}, but they may still seem obscure even with a
good understanding of cryptographic theory. We recommend
experimenting.
