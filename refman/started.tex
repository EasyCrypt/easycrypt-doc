\chapter{Getting Started}

\section{Introduction}

\EasyCrypt\ \cite{barthe-easycrypt-intro-2014,barthe-crypto-2011} is a
framework for interactively finding security proofs of cryptographic
constructions and protocols using the sequence of games approach
\cite{bellare-rogaway-code-based-2004,%
  bellare-rogaway-triple-enc-2006,shoup-seq-games-2004}.  In
\EasyCrypt, cryptographic games (probabilistic programs) are modeled
as \emph{modules}, which consist of procedures and global variables.
Procedures are written in a simple imperative language featuring while
loops and random assignments. Adversaries are modeled by \emph{abstract}
modules---modules whose code isn't known. Modules may be parameterized
by abstract modules.

\EasyCrypt\ has four logics: a probabilistic, relational Hoare logic
(\prhl), relating pairs of procedures; a probabilistic Hoare logic
(\phl) allowing one to carry out proofs about the probability of a
procedure's execution resulting in a postcondition holding; an
ordinary (possibilistic) Hoare logic (\hl); and an ambient
higher-order logic for proving general mathematical facts, as well as
for connecting judgements from the other logics.  Proofs are carried
out using \emph{tactics}, which transform the current proof goal into
zero or more subgoals. Simple ambient logic goals may be automatically
proved using SMT solvers. Proofs may be structured as sequences of
lemmas, and \EasyCrypt's \emph{theories} may be used to group together
types, predicates, operators, modules, axioms and lemmas. Theory
parameters---types, operators and predicates---may be instantiated via
a \emph{cloning} process.

\section{Installing \EasyCrypt}

\EasyCrypt may be downloaded from our GitHub repository:
\begin{center}
  \url{https://github.com/EasyCrypt/easycrypt}
\end{center}
After cloning the repository, one should follow the detailed
instructions for building \EasyCrypt and supporting tools.

\section{Running \EasyCrypt}

\EasyCrypt\ scripts resides in files with the \texttt{.ec} suffix. (As
we'll see in Chapter~\ref{Structuring}, \EasyCrypt\ also has
\emph{abstract} theories, which must be cloned before being used. Such
theories reside in files with the \texttt{.eca} suffix.)

To run
\EasyCrypt\ in \emph{batch} mode, simply invoke it from the shell,
giving it an \EasyCrypt\ script---with suffix \texttt{.ec}---as argument:
\begin{lstlisting}
easycrypt #$\mathit{file}\mathtt{.ec}$#
\end{lstlisting}
\EasyCrypt\ will display its progress as it checks the file.
Information about \EasyCrypt's command-line arguments can be found
in Chapter~\ref{Advanced}.

When developing \EasyCrypt\ scripts, though, \EasyCrypt\ should be run
\emph{interactively}, as a subprocess of the Emacs text editor. One's
interaction with \EasyCrypt\ is mediated by Proof General, a generic
Emacs front-end for proof assistants.  Upon visiting an \EasyCrypt\
file, the ``Proof-General'' tab of the Emacs menu may be used execute
the file, step-by-step, as well as to undo steps, etc. Information
about the ``EasyCrypt'' menu tab may be found in
Chapter~\ref{Advanced}.

An example \EasyCrypt\ script is displayed in
Listing~\ref{StartedExamp}.
\ecinput{examps/started-examp-1.ec}{Example \EasyCrypt\
  Script}{}{StartedExamp}
As can be inferred from the example, comments begin and end with
\ec{(*} and \ec{*)}, respectively; they may be nested. Each step of an
\EasyCrypt\ script is terminated with a dot (period, full stop).  Much
can be learned by experimenting with this script, e.g., by executing
it step-by-step in Emacs.

\section{More Information}

More information about \EasyCrypt---and about the \EasyCrypt\ Team
and its work---may be found at
\begin{center}
  \url{https://www.easycrypt.info}
\end{center}
The \EasyCrypt Club mailing list
\begin{center}
  \url{https://lists.gforge.inria.fr/mailman/listinfo/easycrypt-club}
\end{center}
features discussion about \EasyCrypt.
Support requests should be sent to this list, as answers to questions
will be of use to other members of the \EasyCrypt\ community.

\section{Bug Reporting}

\EasyCrypt\ Bugs should reported using the Tracker:
\begin{center}
  \url{https://www.easycrypt.info/trac/report}
\end{center}
You can log-into the Tracker using any GitHub account.

\section{About this Documentation}

The source for this document, along with the macros and language
definitions used, are available in the GitHub repository. Feel free to
use the language definitions to typeset your \EasyCrypt-related
documents, and to contribute improvements to the macros if you have
any.

This document is intended as a reference manual for the \EasyCrypt\
tool, and not as a tutorial on how to build a cryptographic proof, or
how to perform interactive proofs. We provide some detailed examples
in Chapter~\ref{Examples}, but they may still seem obscure
even with a good understanding of cryptographic theory. We recommend
experimenting.
