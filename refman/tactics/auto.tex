% --------------------------------------------------------------------
\begin{tactic}{auto}
  \begin{tsyntax}[empty]{auto}
    If the current goal is a \prhl, \hl or \phl statement judgement,
    uses various program logic tactics in an attempt to reduce the
    goal to a simpler one. Never fails, but may fail to make any
    progress.

    \fixme{Need better description of when the tactic is applicable and
           how the tactic works!}

    \medskip For example, if the current goal is
    \ecinput{examps/parts/tactics/auto/1-1.0.ec}{}{}{}{} then
    running \ecinput{examps/parts/tactics/auto/1-1.ec}{}{}{}{}
    produces the goal
    \ecinput{examps/parts/tactics/auto/1-1.1.ec}{}{}{}{}
    which \ec{progress;smt} is able to solve.
    If the current goal is
    \ecinput{examps/parts/tactics/auto/1-2.0.ec}{}{}{}{} then
    running \ecinput{examps/parts/tactics/auto/1-2.ec}{}{}{}{}
    produce a single goal, which \ec{smt} is able to solve.
    And, if the current goal is
    \ecinput{examps/parts/tactics/auto/1-3.0.ec}{}{}{}{} then
    running \ecinput{examps/parts/tactics/auto/1-3.ec}{}{}{}{}
    produce a single goal, which \ec{smt} is able to solve.
  \end{tsyntax}
\end{tactic}
