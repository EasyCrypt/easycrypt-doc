% --------------------------------------------------------------------
\begin{tactic}{byphoare}
  \begin{tsyntax}{byphoare [option]? <spec>}
  Derives a probability relation from a \phl judgement on the
  procedure involved. \ec{<spec>} can include wildcards when the
  tactic should infer the pre or postcondition.

  \textbf{Options:} By default, (\ec{eq} option) specification
  inference attempts to infer a conjunction of equalities sufficient
  to imply the desired relation. Passing the \ec{-eq} option
  overrides this behaviour, instead using the trivial relation on
  events.

  \paragraph{Examples:}\strut

  \begin{cmathpar}
    \texample
      {\ec{byphoare (_: P ==> Q)}}
      {\pHL{P}{f}{Q}{=}{\delta} \\
       P\ m[\Arg\mapsto\vec{a}] \\
       \forall \mem{m'}.\, Q\ m' \Leftrightarrow E\ m'}
      {\PR{f}{\vec{a}}{\mem{m}}{E} = \delta}
  \end{cmathpar}
  \end{tsyntax}

  \begin{tsyntax}{byphoare <lemma>}
  Same as \ec{byphoare <spec>}, but the specification to use is
  inferred from the lemma provided. Raises an error if the lemma does
  not refer to the expected procedure. Inference options have no
  effect in this setting.
  \end{tsyntax}
\end{tactic}
