% --------------------------------------------------------------------
\begin{tactic}{split}
  \begin{tsyntax}[empty]{split}
  Break an intrinsically conjunctive goal into its component subgoals.
  For instance, it can:
  \begin{itemize}
  \item close any goal that is convertible to \ec{true} or provable by
    \ec{reflexivity},
    \item replace a logical equivalence by the direct and indirect implications,
    \item replace a goal of the form \ec{$\phi_1\!$ /\\ $\;\,\phi_2$} by the two
      subgoals for $\phi_1$ and $\phi_2$. The same applies for a goal of
      the form \ec{$\phi_1\!$ && $\;\,\phi_2$},
    \item replace an equality between $n$-tuples by $n$ equalities
          on their components.
  \end{itemize}

  For example, if the current goal is
  \ecinput{../examps/parts/tactics/split/1-1.0.ec}{}{}{}{} then
  running \ecinput{../examps/parts/tactics/split/1-1.ec}{}{}{}{}
  produces the goals
  \ecinput{../examps/parts/tactics/split/1-1.1.ec}{}{}{}{}
  and
  \ecinput{../examps/parts/tactics/split/1-1.2.ec}{}{}{}{}
  And if the current goal is
  \ecinput{../examps/parts/tactics/split/2-1.0.ec}{}{}{}{} then
  running \ecinput{../examps/parts/tactics/split/2-1.ec}{}{}{}{}
  produces the goals
  \ecinput{../examps/parts/tactics/split/2-1.1.ec}{}{}{}{}
  and
  \ecinput{../examps/parts/tactics/split/2-1.2.ec}{}{}{}{}
  Repeating the last example with \ec{&&} rather than \ec{/\\},
  if the current goal is
  \ecinput{../examps/parts/tactics/split/2a-1.0.ec}{}{}{}{} then
  running \ecinput{../examps/parts/tactics/split/2a-1.ec}{}{}{}{}
  produces the goals
  \ecinput{../examps/parts/tactics/split/2a-1.1.ec}{}{}{}{}
  and
  \ecinput{../examps/parts/tactics/split/2a-1.2.ec}{}{}{}{}
  This illustrates the difference between \ec{/\\} and \ec{&&}.
  And if the current goal is
  \ecinput{../examps/parts/tactics/split/3-1.0.ec}{}{}{}{} then
  running \ecinput{../examps/parts/tactics/split/3-1.ec}{}{}{}{}
  produces the goals
  \ecinput{../examps/parts/tactics/split/3-1.1.ec}{}{}{}{}
  and
  \ecinput{../examps/parts/tactics/split/3-1.2.ec}{}{}{}{}
  \end{tsyntax}
\end{tactic}
