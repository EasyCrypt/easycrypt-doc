% --------------------------------------------------------------------
\begin{tactic}{inline}
  \begin{tsyntax}{inline $\;M_1$.$p_1$ $\;\cdots$ $\;M_n$.$p_n$}
    Inline the selected \emph{concrete} procedures in both programs,
    with \prhl, and in the program, with \hl and \phl, until no more
    inlining of these procedures is possible.

    To inline a procedure call, the procedure's parameters are
    assigned the values of their arguments (fresh parameter
    identifiers are used, as necessary, to avoid naming
    conflicts). This is followed by the body of the procedure. Finally,
    the procedure's return value is assigned to the identifiers (if
    any) to which the procedure call's result is assigned.
  \end{tsyntax}

  \begin{tsyntax}{inline\{1\} $\;M_1$.$p_1$ $\;\cdots$ $\;M_n$.$p_n$ | inline\{2\} $\;M_1$.$p_1$ $\;\cdots$ $\;M_n$.$p_n$}
    Do the inlining in just the first or second program, in the \prhl case.
  \end{tsyntax}

  \begin{tsyntax}{inline* | inline\{1\}* | inline\{2\}*}
    Inline all concrete procedures, continuing until no more inlining
    is possible.
  \end{tsyntax}

  \begin{tsyntax}{inline $\;\mathit{occs}$ $\;M$.$p$ | inline\{1\} $\;\mathit{occs}$ $\;M$.$p$ | inline\{2\} $\;\mathit{occs}$ $\;M$.$p$}
    Inline just the specified occurrences of $M$.$p$, where
    $\mathit{occs}$ is a parenthesized nonempty sequence of positive
    numbers \ec{($n_1$ $\;\cdots$ $\;n_l$)}. E.g., \ec{(1 3)} means the
    first and third occurrences of the procedure.  In the \prhl case,
    a side \ec{\{1\}} or \ec{\{2\}} must be specified.
  \end{tsyntax}

  \bigskip
  For example, given the declarations
  \ecinput{examps/tactics/inline/1.ec}{}{3-18}{}
  if the current goal is
  \ecinput{examps/parts/tactics/inline/1-1.0.ec}{}{}{}{} then
  running \ecinput{examps/parts/tactics/inline/1-1.ec}{}{}{}{}
  produces the goal
  \ecinput{examps/parts/tactics/inline/1-1.1.ec}{}{}{}{}
  From which running
  \ecinput{examps/parts/tactics/inline/1-2.ec}{}{}{}{}
  produces the goal
  \ecinput{examps/parts/tactics/inline/1-2.1.ec}{}{}{}{}
  From which running
  \ecinput{examps/parts/tactics/inline/1-3.ec}{}{}{}{}
  produces the goal
  \ecinput{examps/parts/tactics/inline/1-3.1.ec}{}{}{}{}
  And, if the current goal is
  \ecinput{examps/parts/tactics/inline/2-1.0.ec}{}{}{}{} then
  running \ecinput{examps/parts/tactics/inline/2-1.ec}{}{}{}{}
  produces the goal
  \ecinput{examps/parts/tactics/inline/2-1.1.ec}{}{}{}{}
\end{tactic}
