% --------------------------------------------------------------------
\begin{tactic}[smt $\textit{ smt-options}$]{smt}
  \begin{tsyntax}[empty]{smt}
  Try to solve the goal using SMT solvers. The goal is sent along with
  the local hypotheses plus selected axioms and lemmas. The SMT
  solvers used, their options, and the axiom selection algorithm are
  specified by $\textit{smt-option}$.

  For example, if the current goal is
  \ecinput{../examps/parts/tactics/smt/1-1.0.ec}{}{}{}{} then
  running \ecinput{../examps/parts/tactics/smt/1-1.ec}{}{}{}{}
  solves the goal.

  \textbf{Options}
  \begin{compactitem}
    \item \ec{timeout=}$n$: set the timeout for provers to $n$ (in seconds).
    \item \ec{maxprovers=}$n$: set the maximun number of prover runing in 
          parallele to $n$ 
    \item \ec{prover=[}\textit{prover-selector}\ec{]}: select the
      provers, where $\textit{prover-selector}$ is a list of modified
      prover names:
          \begin{compactitem}
            \item \ec{``}\textit{prover-name}\ec{''}: use the
              listed prover;
            \item \ec{+``}\textit{prover-name}\ec{''}: add
              \textit{prover-name} to the current list of provers;
            \item \ec{-``}\textit{prover-name}\ec{''}: remove
              \textit{prover-name} from the current list of provers;
          \end{compactitem}
          Examples:
          \begin{compactitem}
          \item \ec{[``Z3'' ``Alt-Ergo'']}: use only Z3 and Alt-Ergo;
          \item \ec{[``Z3'' ``Alt-Ergo'' -''Z3'']}: use only Alt-Ergo;
          \item \ec{[-''CVC4'']}: remove CVC4 form the current list of
            provers;
          \item \ec{[+''CVC4'']}: add CVC4 to the current list of provers;
          \end{compactitem}
      \fix{Describe failure states of prover selection.}
     \item Axiom selection: axioms and lemmas are not all send to smt
       provers, \EasyCrypt use a strategy to automatically select
       them. Lemmas and axioms marked with ``nosmt'' are not selected
       by default. This strategy can be parametrized using different
       options:
           \begin{compactitem}
             \item \ec{unwantedlemmas=}\textit{dbhint}: do not send
               axiom/lemma selected by \textit{dbhint}
             \item \ec{wantedlemmas=}\textit{dbhint}: send axiom/lemma
               selected by \textit{dbhint}
             \item \ec{all}: select all available axioms/lemmas
               execpted those specified by \ec{unwantedlemmas} (if
               any).
             \item \ec{maxlemmas=}$n$: set the maximun number of
               selected axioms/lemmas to $n$.  Keep this number small
               is generally more effienciant.  Variant: $n$
             \item \ec{iterate}: try to incrementally augment the
               number of selected axioms/lemmas. Last call will be
               equivalent to all.
           \end{compactitem}
       \fix{Describe \textit{dbhint} options.}
  \end{compactitem}

  \textbf{Variant}: Short options.\\
  Options can also be specified by short name, for example:
  \begin{center}\ec{smt 100 [+''Z3] tmo=4 mp=2}\end{center}
  is equivalent to 
  \begin{center}
  \ec{smt maxlemmas=100 prover=[+''Z3] timeout=4 maxprovers=2}
  \end{center}
  Short options can be any substring of the full option names that
  uniquely identifies the desired option: when several options match,
  their full names are given.

  Smt option can be set globally using the following syntax:
  \ec{prover} \textit{smt-options}
  \fix{Make this a pragma?}

  \textbf{Remark}: By default, \ec{smt} failures cannot be caught by
  the \rtactic{try} tactical.
  \end{tsyntax}
\end{tactic}
