% --------------------------------------------------------------------
\begin{tactic}{fission}
  \begin{tsyntax}{fission $\;c$!$l$ @ $\;m$, $\;n$}
    \hl statement judgement version.  Fails unless $0\leq l$ and
    $0\leq m\leq n$ and the $c$th statement of the program is a
    \ec{while} statement, and there are at least $l$ statements right
    before the \ec{while} statement, at its level, and the body of the
    \ec{while} statement has at least $n$ statements.

    Let
    \begin{itemize}
    \item $s_1$ be the $l$ statements before the \ec{while} statement
      at position $c$;

    \item $e$ be the boolean expression of the \ec{while} statement;

    \item $s_2$ be the first $m$ statements of the body of the
      \ec{while} statement;

    \item $s_3$ be the next $n - m$ statements of the body of the
      \ec{while} statement;

    \item $s_4$ be the rest of the body of the \ec{while} statement.
    \end{itemize}
    Fails unless:
    \begin{itemize}
    \item $e$ doesn't reference the variables written by $s_2$ and $s_3$;

    \item $s_1$ and $s_4$ don't read or write the variables written by
      $s_2$ and $s_3$;

    \item $s_2$ and $s_3$ don't write the variables written by $s_1$
      and $s_4$;

    \item $s_2$ and $s_3$ don't read or write the variables written by
      the other.
    \end{itemize}
    The tactic replaces
\begin{easycrypt}{}{}
#$s_1$# while (#$e$#) { #$s_2$# #$s_3$# #$s_4$# }
\end{easycrypt}
    by
\begin{easycrypt}{}{}
#$s_1$# while (#$e$#) { #$s_2$# #$s_4$# }
#$s_1$# while (#$e$#) { #$s_3$# #$s_4$# }
\end{easycrypt}

    \medskip For example, if the current goal is
    \ecinput{../examps/parts/tactics/fission/1-1.0.ec}{}{}{}{} then
    running \ecinput{../examps/parts/tactics/fission/1-1.ec}{}{}{}{}
    produces the goal
    \ecinput{../examps/parts/tactics/fission/1-1.1.ec}{}{}{}{}
  \end{tsyntax}

  \begin{tsyntax}{fission $\;c$ @ $\;m$, $\;n$}
    Equivalent to \ec{fission $\;c$!1 @ $\;m$, $\;n$}.
  \end{tsyntax}

  \begin{tsyntax}{fission\{1\} $\;\cdots$ | fission\{2\} $\;\cdots$}
    The \prhl versions of the above variants, working on the
    designated program.
  \end{tsyntax}
\end{tactic}
