% --------------------------------------------------------------------
\begin{tactic}{sim}
  \ec{sim} attempts to solve a goal whose conclusion is a \prhl
  judgement or statement judgement by working backwards, propagating
  and extending a conjunction of equalties between variables of the
  two programs, verifying that the conclusion's precondition implies
  the final conjuction of equalities.  It's capable of working
  backwards through \ec{if} and \ec{while} statements and handing
  random assignments, but only when the programs are sufficiently
  similar (thus its name). Sometimes this process only partly
  succeeds, leaving a statement judgement whose programs are prefixes
  of the original programs.

  \begin{tsyntax}{sim}
    Without any arguments, \ec{sim} attemps to infer the conjuction of
    program variable equalities from the conclusion's postcondition.

    \medskip For example, if the current goal is
    \ecinput{examps/parts/tactics/sim/1-1.0.ec}{}{}{}{} then
    running \ecinput{examps/parts/tactics/sim/1-1.ec}{}{}{}{}
    produces the goal
    \ecinput{examps/parts/tactics/sim/1-1.1.ec}{}{}{}{}
    which \ec{auto} is able to solve.
  \end{tsyntax}

  \begin{tsyntax}{sim / $\;\phi$ : $\;\mathit{eqs}$}
    One may give the starting conjuction, $\mathit{eqs}$, of equalities
    explicitly, and may also specifiy an invariant $\phi$ on the
    global variables of the programs.

    \medskip For example, if the current goal is
    \ecinput{examps/parts/tactics/sim/1-2.0.ec}{}{}{}{} then
    running \ecinput{examps/parts/tactics/sim/1-2.ec}{}{}{}{}
    produces the goals
    \ecinput{examps/parts/tactics/sim/1-2.1.ec}{}{}{}{}
    and
    \ecinput{examps/parts/tactics/sim/1-2.2.ec}{}{}{}{}
    which \ec{smt} and \ec{auto;smt}, respectively, are
    able to solve.
  \end{tsyntax}

  \begin{tsyntax}{sim $\;\mathit{proceq}_1$ $\;\ldots$ $\;\mathit{proceq}_1$ / $\;\phi$ : $\;\mathit{eqs}$}
    In its most general form, one may also supply a sequence of
    procedure global equality specifications of the form
    \begin{center}
      \ec{($M$.$p$ ~ $\;N$.$q$ : $\;\mathit{eqs}$)},
    \end{center}
    where $\mathit{eqs}$ is a conjuction of global variable
    equalities. When \ec{sim} encounters a pair of procedure calls
    consisting of a call to \ec{$M$.$p$} in the first program and
    \ec{$N$.$q$} in the second program, it will generate a subgoal
    whose conclusion is a \prhl judgment between \ec{$M$.$p$} and
    \ec{$N$.$q$}, whose precondition assumes equality of its
    arguments, $\mathit{eqs}$ and $\phi$, and whose postcondition
    requires equality of the calls' results, $\mathit{eqs}$ and
    $\phi$.

    One may also replace \ec{$M$.$p$ ~ $\;N$.$q$} by \ec{_},
    meaning that the same conjunction of global variable equalities
    is used for all procedure calls.

    \medskip For example, if the current goal is
    \ecinput{examps/parts/tactics/sim/2-1.0.ec}{}{}{}{} then
    running \ecinput{examps/parts/tactics/sim/2-1.ec}{}{}{}{}
    produces the goals
    \ecinput{examps/parts/tactics/sim/2-1.1.ec}{}{}{}{}
    and
    \ecinput{examps/parts/tactics/sim/2-1.2.ec}{}{}{}{}
    and
    \ecinput{examps/parts/tactics/sim/2-1.3.ec}{}{}{}{}
    and
    \ecinput{examps/parts/tactics/sim/2-1.4.ec}{}{}{}{}
    which \ec{smt}, \ec{proc;auto;smt}, \ec{proc;auto;smt}
    and \ec{auto}, respectively, are able to solve.

    \medskip And, if the current goal is
    \ecinput{examps/parts/tactics/sim/2-2.0.ec}{}{}{}{} then
    running \ecinput{examps/parts/tactics/sim/2-2.ec}{}{}{}{}
    produces the goals
    \ecinput{examps/parts/tactics/sim/2-2.1.ec}{}{}{}{}
    and
    \ecinput{examps/parts/tactics/sim/2-2.2.ec}{}{}{}{}
    and
    \ecinput{examps/parts/tactics/sim/2-2.3.ec}{}{}{}{}
    and
    \ecinput{examps/parts/tactics/sim/2-2.4.ec}{}{}{}{}
    which \ec{smt}, \ec{proc;auto;smt}, \ec{proc;auto;smt}
    and \ec{auto}, respectively, are able to solve.
  \end{tsyntax}
\end{tactic}

%%  \begin{tsyntax}{sim <pos>? <hintgeqs>* <hintinv>? <eqs>?}\\
%%    where \begin{tabular}{lrl}
%%       <pos>      & = & <uint> <uint> \\
%%       <hintgeqs> & = & (<procname>? $\sim$ <procname>? : <formula>) \\
%%                  & | & ($\_$? : <formula>  \\
%%       <eqs>      & = & : <formula> \\
%%    \end{tabular}
%%  
%%  \fix{Missing description of sim}.
%%  \end{tsyntax}
%%\end{tactic}
