% --------------------------------------------------------------------
\begin{tactic}{skip}
  \begin{tsyntax}{skip} Hoare logic rules for empty statement.

  \textbf{Examples:} 
  \begin{mathpar}
  \inferrule*[left=(\prhl),rightskip=10em]%%
    {P \Rightarrow Q }%%
    {\pRHL{P}{\Skip}{\Skip}{Q}}%%
    \quad\raisebox{.7em}{\tct{skip}} \\
  \inferrule*[left=(\phl),rightskip=10em]%%
    {P \Rightarrow Q }%%
    {\pHL{P'}{\Skip}{Q'}{\diamond}{1}}%%
    \quad\raisebox{.7em}{\tct{skip}} \\
  \inferrule*[left=(\hl),rightskip=10em]%%
    {P \Rightarrow Q}%%
    {\HL{P}{\Skip}{Q}}%%
    \quad\raisebox{.7em}{\tct{skip}} \\
  \end{mathpar}

  \textbf{Note:} Note that the \phl rule forces the bound of the goal
  to be 1. If you end up with an empty program and a bound other than
  1, you might want to use \rtactic{hoare} or \rtactic{conseq}. If
  neither of these work, you should probably have used \rtactic{seq}
  or \rtactic{phoare split} earlier on in your proof.
  \end{tsyntax}
\end{tactic}
