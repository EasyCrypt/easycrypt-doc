% --------------------------------------------------------------------
\begin{tactic}[rewrite $\;\pi_1 \cdots \pi_n$ | rewrite $\;\pi_1 \cdots \pi_n$ in $\;H$]{rewrite}
  \begin{tsyntax}{rewrite $\;\pi_1 \cdots \pi_n$}
  Rewrite the rewrite-pattern $\pi_1 \cdots \pi_n$ from left to right,
  where the $\pi_i$ can be of the following form:
  \begin{itemize}
  \item one of \ec{//}, \ec{/=}, \ec{//=},
  \item a proof-term $p$, or
  \item a pattern prefixed by \ec{/} (slash).
  \end{itemize}
  The two last forms can be prefixed by a direction indicator (the
  sign \ec{-}, see Subsection~\ref{subsec:occsels}), followed by an
  occurrence selector (see Subsection~\ref{subsec:occsels}), followed
  (for proof-terms only) by a repetition marker (\ec{!}, \ec{?},
  \ec{n!} or \ec{n?}). All these prefixes are optional.

  \smallskip
  Depending on the form of $\pi$, \ec{rewrite $\;\pi$} does the following:
    \begin{itemize}
    \item For \ec{//}, \ec{/=}, and \ec{//=}, see
      Subsection~\ref{subsec:introgen}.

    \item If $\pi$ is a proof-term with conclusion $f_1=f_2$, then
      \ec{rewrite} searches for the first subterm of the goal matching
      $f_1$ and resulting in the full instantiation of the pattern.
      It then replaces, after instantiation of the pattern, all the
      occurrences of $f_1$ by $f_2$ in the goal, and creates
      new subgoals for the instantiations of the assuptions of $p$.
      If no subterms of the goal match $f_1$ or if the pattern
      cannot be fully instantiated by matching, the tactic fails.  The
      tactic works the same if the pattern ends by \ec{$f_1\!$ <=> $\;f_2$}. If
      the direction indicator \ec{-} is given, \ec{rewrite} works in
      the reverse direction, searching for a match of $f_2$ and then
      replacing all occurrences of $f_2$ by $f_1$.

    \item If $\pi$ is a \ec{/}-prefixed pattern of the form
      $o\,p_1\,\cdots\,p_n$, with $o$ a defined symbol, then
      \ec{rewrite} searches for the first subterm of the goal matching
      $o\,p_1\,\cdots\,p_n$ and resulting in the full instantiation of
      the pattern. It then replaces, after instantiation of the
      pattern, all the occurrences of $o\,p_1\,\cdots\,p_n$ by the
      $\beta\delta$ head-normal form of $o\,p_1\,\cdots\,p_n$, where
      the $\delta$-reduction is restricted to subterms headed by the
      symbol $o$. If no subterms of the goal match
      $o\,p_1\,\cdots\,p_n$
      or if the pattern cannot be fully instantiated by matching, the
      tactic fails. If the direction indicator \ec{-} is given,
      \ec{rewrite} works in the reverse direction, searching for a
      match of the $\beta\delta_o$
      head-normal of $o\,p_1\,\cdots\,p_n$
      and then replacing all occurrences of this head-normal form with
      $o\,p_1\,\cdots\,p_n$.
    \end{itemize}
    
    \smallskip
    The occurrence selector restricts which occurrences of the
    matching pattern are replaced in the goal---see
    Subsection~\ref{subsec:occsels}.

    Repetition markers allow the repetition of the same rewriting. For
    instance, \ec{rewrite $\;\pi$} leads to \ec{do! rewrite
      $\;\pi$}. See the tactical \ec{do} for more information.
    
    Lastly, \ec{rewrite $\;\pi_1 \cdots \pi_n$} is equivalent to
    \ec{rewrite} $\;\pi_1$; ...; \ec{rewrite} $\;\pi_n$.

    \smallskip
    For example, if the current goal is
    \ecinput{../examps/parts/tactics/rewrite/1-1.0.ec}{}{}{}{} then
    running \ecinput{../examps/parts/tactics/rewrite/1-1.ec}{}{}{}{}
    produces \ecinput{../examps/parts/tactics/rewrite/1-1.1.ec}{}{}{}{}
    from which
    running \ecinput{../examps/parts/tactics/rewrite/1-2.ec}{}{}{}{}
    produces \ecinput{../examps/parts/tactics/rewrite/1-2.1.ec}{}{}{}{}
    from which
    running \ecinput{../examps/parts/tactics/rewrite/1-3.ec}{}{}{}{}
    produces \ecinput{../examps/parts/tactics/rewrite/1-3.1.ec}{}{}{}{}
  \end{tsyntax}

  \begin{tsyntax}{rewrite $\;\pi_1 \cdots \pi_n$ in $\;H$} Like the
    preceding case, except rewriting is done in the hypothesis
    $H$ instead of in the goal's conclusion.  Rewriting using a proof
    term is only allowed when the proof term was defined globally
    or before the assumption $H$.

    For example, if the current goal is
    \ecinput{../examps/parts/tactics/rewrite/2-1.0.ec}{}{}{}{} then
    running \ecinput{../examps/parts/tactics/rewrite/2-1.ec}{}{}{}{}
    produces \ecinput{../examps/parts/tactics/rewrite/2-1.1.ec}{}{}{}{}
  \end{tsyntax}
\end{tactic}
