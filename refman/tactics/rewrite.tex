% --------------------------------------------------------------------
\begin{tactic}[rewrite $\;\pi_1 \cdots \pi_n$]{rewrite}
  \begin{tsyntax}[empty]{rewrite}
  Rewrite the rewrite-pattern $\pi_1 \cdots \pi_n$ from left to right,
  where the $\pi_i$ can be of the following form:
  \begin{itemize}
  \item one of \tct{//}, \tct{/=}, \tct{//=},
  \item a proof-term, or
  \item a pattern prefixed by \tct{/} (slash).
  \end{itemize}
  The two last forms can be prefixed by a direction indicator (the sign
  \tct{-}), followed by an occurrence selector (\tct{\{i1 ... in\}}),
  followed (for proof-terms only) by a repetition marker
  (\tct{!}, \tct{?}, \tct{n!} or \tct{n?}). All these prefixes are optional.

  Depending on the form of $\pi$, \tct{rewrite $\;\pi$} does the following:
    \begin{itemize}
    \item For \tct{//}, \tct{/=}, and \tct{//=}, see \tct{intros}.
    \item If \tct{rw} is a proof-term for the pattern
      \begin{center}
	\tct{forall (x1 : t1) ... (xn : tn), A1 -> ... -> An -> f1 = f2}
      \end{center}
      \noindent then \tct{rewrite} searches for the first subterm of the goal
      matching \tct{f1} and resulting in the full instantiation of the pattern.
      It then replaces, after instantiation of the pattern, all the occurrences
      of \tct{f1} by \tct{f2} in the goal, and creates $n$ new subgoals for the
      \tct{Ai}'s. If no subterms of the goal match \tct{f1} or if the pattern
      cannot be fully instantiated by matching, the tactic fails.
      The tactic works the same if the pattern ends by \tct{f1 <=> f2}. If the
      direction indicator \tct{-} is given, \tct{rewrite} works in the reverse
      direction, searching for a match of \tct{f2} and then replacing all
      occurrences of \tct{f2} by \tct{f1}.
    \item If \tct{rw} is a \tct{/}-prefixed pattern of the form \tct{(o p1 ... pn)},
      with \tct{o} a defined symbol, then \tct{rewrite} searches for the first subterm
      of the goal matching \tct{(o p1 ... pn)} and resulting in the full instantiation
      of the pattern. It then replaces, after instantiation of the pattern, all
      the occurrences of \tct{(o p1 ... pn)} by the $\beta\delta$ head-normal form
      of \tct{(o p1 ... pn)}, where the $\delta$-reduction is restricted to subterms
      headed by the symbol \tct{o}. If no subterms of the goal match \tct{(o p1 ... pn)} or
      if the pattern cannot be fully instantiated by matching, the tactic fails. If the
      direction indicator \tct{-} is given, \tct{rewrite} works in the reverse
      direction, searching for a match of the $\beta\delta_{\rm o}$ head-normal
      of \tct{(o p1 ... pn)} and then replacing all occurrences of this head-normal
      form with \tct{(o p1 ... pn)}.
    \end{itemize}
    
    \smallskip
    
    The occurrence selector \tct{\{i1 ... in\}} restricts which occurrences
    of the matching pattern are replaced in the goal. If given, only the
    \tct{i1}-th, ..., \tct{in}-th ones are replaced (considering that the goal is
    traversed in DFS mode). Note that this selection applies after the matching has
    been done.
    
    \medskip
    
    Repetition markers allow the repetition of the same rewriting. For instance,
    \tct{rewrite $\;\pi$} leads to \tct{do! rewrite $\;\pi$}. See \tct{do} for
    more information.
    
    \medskip

    Last, \tct{rewrite} \tct{rw}${}_1$ ... \tct{rw}${}_n$ is equivalent to
    \tct{rewrite} \tct{rw}${}_1$; ...; \tct{rewrite} \tct{rw}${}_n$.
  \end{tsyntax}
\end{tactic}
