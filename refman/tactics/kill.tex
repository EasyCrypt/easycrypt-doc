% --------------------------------------------------------------------
\begin{tactic}{kill}
  \begin{tsyntax}{kill $\;c$ ! $\;m$}
    Fails unless $n\geq 1$ and $m\geq 0$.  If the goal's conclusion is
    an \hl statement judgement whose program has a statement block
    starting at position $c$ and having length $m$ (when $m = 0$, this
    block is empty), and the variables written by this statement block
    aren't used in the judgement's postcondition or read by the rest
    of the program, then reduce the goal to two subgoals.
    \begin{itemize}
    \item One whose conclusion is a \phl statement judgement whose pre-
       and postconditions are \ec{true}, whose program is the
       statement block, and whose bound part is \ec{= 1\%r}.

    \item One that's identical to the original goal except that the
      statement block has been removed.
    \end{itemize}

    \medskip For example, if the current goal is
    \ecinput{../examps/parts/tactics/kill/1-1.0.ec}{}{}{}{} then
    running \ecinput{../examps/parts/tactics/kill/1-1.ec}{}{}{}{}
    produces the goals
    \ecinput{../examps/parts/tactics/kill/1-1.1.ec}{}{}{}{}
    and
    \ecinput{../examps/parts/tactics/kill/1-1.2.ec}{}{}{}{}
  \end{tsyntax}

  \begin{tsyntax}{kill\{1\} $\;c$ ! $\;m$ | kill\{2\} $\;c$ ! $\;m$}
    Like the \hl case but for \prhl judgements, where the statement
    block to be killed is in the designated program.

    \medskip For example, if the current goal is
    \ecinput{../examps/parts/tactics/kill/1-2.0.ec}{}{}{}{} then
    running \ecinput{../examps/parts/tactics/kill/1-2.ec}{}{}{}{}
    produces the goals
    \ecinput{../examps/parts/tactics/kill/1-2.1.ec}{}{}{}{}
    and
    \ecinput{../examps/parts/tactics/kill/1-2.2.ec}{}{}{}{}
  \end{tsyntax}

  \begin{tsyntax}{kill $\;c$ | kill\{1\} $\;c$ | kill\{2\} $\;c$}
    Like the general cases, but with $m = 1$.
  \end{tsyntax}

  \begin{tsyntax}{kill $\;c$ ! * | kill\{1\} $\;c$ ! * | kill\{2\} $\;c$ ! *}
    Like the general cases, but with $m$ set so that the statement
    block to be killed is the rest of the current level of the
    (designated) program.
  \end{tsyntax}
\end{tactic}
