% --------------------------------------------------------------------
\begin{tactic}{exists*}
  \begin{tsyntax}{exists* $\;e_1$, $\;\ldots$, $\;e_n$}
    If the goal's conclusion is a \prhl, \hl or \phl judgment or
    statement judgements and the $e_i$ are well-typed expressions
    typically involving program variables (in the \prhl case,
    the expressions will refer to memories \ec{&1} and \ec{&2}),
    then change the conclusion's precondition $P$ to
    \begin{center}
      \ec{exists ($x_1$ $\;\ldots$ $\;x_n$),
          $\;x_1$ = $\;e_2$ /\\ $\;\ldots$ /\\ $\;x_n$ = $\;e_n$ /\\ $\;P$}.
    \end{center}

    The tactic can be used in conjunction with \rtactic{elim*} when
    handling a procedure call using a lemma that refers to initial
    values of program variables. See \rtactic{elim*} for an example
    of this.

    \medskip For example, if the current goal is
    \ecinput{examps/parts/tactics/exists-star/1-1.0.ec} then
    running \ecinput{examps/parts/tactics/exists-star/1-1.ec}
    produces the goal
    \ecinput{examps/parts/tactics/exists-star/1-1.1.ec}
    If the current goal is
    \ecinput{examps/parts/tactics/exists-star/1-2.0.ec} then
    running \ecinput{examps/parts/tactics/exists-star/1-2.ec}
    produces the goal
    \ecinput{examps/parts/tactics/exists-star/1-2.1.ec}
    If the current goal is
    \ecinput{examps/parts/tactics/exists-star/1-3.0.ec} then
    running \ecinput{examps/parts/tactics/exists-star/1-3.ec}
    produces the goal
    \ecinput{examps/parts/tactics/exists-star/1-3.1.ec}
    If the current goal is
    \ecinput{examps/parts/tactics/exists-star/1-4.0.ec} then
    running \ecinput{examps/parts/tactics/exists-star/1-4.ec}
    produces the goal
    \ecinput{examps/parts/tactics/exists-star/1-4.1.ec}
  \end{tsyntax}
\end{tactic}

%%  Introduce an existential quantification over the value of a program
%%  variable in the initial memory. This is particularly useful when
%%  dealing with a procedure call using a lemma that refers to initial
%%  values of arguments or state (using \rtactic{call}). Several program
%%  variables can be treated simultaneously by providing them in a
%%  comma-separated list.
%%
%%  \paragraph{Examples:}\strut
%%
%%  \begin{cmathpar}
%%  \texample[\prhl{}]{\ec{exists* M.x\{1\}}}%%
%%    {\pRHL{\exists x, x = \inmem{M.x}{1} \wedge P}{c_1}{c_2}{Q}}%%
%%    {\pRHL{P}{c_1}{c_2}{Q}}
%%
%%  \texample[\prhl{}]{\ec{exists* M.x\{1\}, M.x\{2\}}}%%
%%    {\pRHL{\exists x_1\ x_2, x1 = \inmem{M.x}{1} \wedge \inmem{M.x}{2} \wedge P}{c_1}{c_2}{Q}}%%
%%    {\pRHL{P}{c_1}{c_2}{Q}}
%%
%%  \texample[\phl{}]{\ec{exists* M.x}}%%
%%    {\pHL{\exists x, x = M.x \wedge P}{c}{Q}{\diamond}{\delta}}%%
%%    {\pHL{P}{c}{Q}{\diamond}{\delta}}
%%
%%  \texample[\hl{}]{\ec{exists* M.x}}%%
%%    {\HL{\exists x, x = M.x \wedge P}{c}{Q}}%%
%%    {\HL{P}{c}{Q}}
%%  \end{cmathpar}
%%
%%  \end{tsyntax}
