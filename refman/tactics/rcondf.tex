% --------------------------------------------------------------------
\begin{tactic}{rcondf}
  \begin{tsyntax}{rcondf $\;n$}
    If the goal's conclusion is an \hl statement judgement whose $n$th
    statement is an \ec{if} statement, reduce the goal to two
    subgoals.
    \begin{itemize}
    \item One whose concludion is an \hl statement judgement whose
      precondition is the original goal's precondition, program is the
      first $n-1$ statements of the original goal's program, and
      postcondition is the negation of the boolean expression of the
      \ec{if} statement.
   
    \item One whose conclusion is an \hl statement judgement that's
      the same as that of the original goal except that the \ec{if}
      statement has been replaced by its \ec{else} part.
    \end{itemize}

    \medskip For example, if the current goal is
    \ecinput{examps/parts/tactics/rcondf/1-1.0.ec}{}{}{}{} then
    running \ecinput{examps/parts/tactics/rcondf/1-1.ec}{}{}{}{}
    produces the goals
    \ecinput{examps/parts/tactics/rcondf/1-1.1.ec}{}{}{}{} and
    \ecinput{examps/parts/tactics/rcondf/1-1.2.ec}{}{}{}{}
  \end{tsyntax}

  \begin{tsyntax}{rcondf\{1\} $\;n$ | rcondf\{2\} $\;n$}
    If the goal's conclusion is a \prhl statement judgement where the
    $n$th statement of the designated program is an \ec{if} statement,
    reduce the goal to two subgoals.
    \begin{itemize}
    \item One whose conclusion is an \hl statement judgement whose
      precondition is the original goal's precondition, program is the
      first $n-1$ statements of the original goal's designated
      program, and postcondition is the negation of the boolean
      expression of the \ec{if} statement. Actually, the \hl statement
      judgement is universally quantified by a memory of the
      non-designated program, and references in the precondition to
      variables of the non-designated program are interpreted in that
      memory.
   
    \item One whose conclusion is a \prhl statement judgement that's
      the same as that of the original goal except that the \ec{if}
      statement has been replaced by its \ec{else} part.
    \end{itemize}

  \medskip
  For example, if the current goal is
  \ecinput{examps/parts/tactics/rcondf/1-2.0.ec}{}{}{}{} then
  running \ecinput{examps/parts/tactics/rcondf/1-2.ec}{}{}{}{}
  produces the goals
  \ecinput{examps/parts/tactics/rcondf/1-2.1.ec}{}{}{}{}
  and
  \ecinput{examps/parts/tactics/rcondf/1-2.2.ec}{}{}{}{}
  \end{tsyntax}

  \begin{tsyntax}{rcondf $\;n$}
    If the goal's conclusion is an \hl statement judgement whose $n$th
    statement is a \ec{while} statement, reduce the goal to two
    subgoals.
    \begin{itemize}
    \item One whose concludion is an \hl statement judgement whose
      precondition is the original goal's precondition, program is the
      first $n-1$ statements of the original goal's program, and
      postcondition is the negation of the boolean expression of the
      \ec{while} statement.
   
    \item One whose conclusion is an \hl statement judgement that's
      the same as that of the original goal except that the \ec{while}
      statement has been removed.
    \end{itemize}

    \medskip For example, if the current goal is
    \ecinput{examps/parts/tactics/rcondf/2-1.0.ec}{}{}{}{} then
    running \ecinput{examps/parts/tactics/rcondf/2-1.ec}{}{}{}{}
    produces the goals
    \ecinput{examps/parts/tactics/rcondf/2-1.1.ec}{}{}{}{} and
    \ecinput{examps/parts/tactics/rcondf/2-1.2.ec}{}{}{}{}
  \end{tsyntax}

  \begin{tsyntax}{rcondf\{1\} $\;n$ | rcondf\{2\} $\;n$}
    If the goal's conclusion is a \prhl statement judgement where the
    $n$th statement of the designated program is a \ec{while} statement,
    reduce the goal to two subgoals.
    \begin{itemize}
    \item One whose conclusion is an \hl statement judgement whose
      precondition is the original goal's precondition, program is the
      first $n-1$ statements of the original goal's designated
      program, and postcondition is the negation of the boolean
      expression of the \ec{while} statement. Actually, the \hl
      statement judgement is universally quantified by a memory of the
      non-designated program, and references in the precondition to
      variables of the non-designated program are interpreted in that
      memory.
   
    \item One whose conclusion is a \prhl statement judgement that's
      the same as that of the original goal except that the \ec{while}
      statement has been removed.
    \end{itemize}

  \medskip
  For example, if the current goal is
  \ecinput{examps/parts/tactics/rcondf/2-2.0.ec}{}{}{}{} then
  running \ecinput{examps/parts/tactics/rcondf/2-2.ec}{}{}{}{}
  produces the goals
  \ecinput{examps/parts/tactics/rcondf/2-2.1.ec}{}{}{}{}
  and
  \ecinput{examps/parts/tactics/rcondf/2-2.2.ec}{}{}{}{}
  \end{tsyntax}
\end{tactic}
