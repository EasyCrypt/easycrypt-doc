% --------------------------------------------------------------------
\begin{tactic}{bypr}
  \begin{tsyntax}{bypr $\;e_1$ $\;e_2$}
    If the goal's conclusion has the form
    \begin{center}
      \ec{equiv[$M$.$p$ ~ $\;N$.$q$ : $\;P$ ==> $\;Q$]},
    \end{center}
    and the $e_i$ are expressions of the same type possibily
    involving memories \ec{&1} and \ec{&2} for \ec{$M$.$p$} and
    \ec{$N$.$q$}, respectively, then reduce the goal to two subgoals:
    \begin{itemize}
    \item One whose conclusion says that for all memories \ec{&1}
      and \ec{&2} for \ec{$M$.$p$} and \ec{$N$.$q$}, if $e_1 = e_2$,
      then $Q$ holds; and

    \item One whose conclusion says that, for all memories \ec{&1} and
      \ec{&2} for \ec{$M$.$p$} and \ec{$N$.$q$} and values $a$ of the
      common type of the $e_i$, if $P$ holds, then the probability of
      running \ec{$M$.$p$} in memory \ec{&1} and with arguments
      consisting of the values of its formal parameters in \ec{&1} and
      terminating in a memory in which the value of $e_1$ (replacing
      references to \ec{&1} with reference to this memory) is $a$ is
      the same as the probability of running \ec{$N$.$q$} in memory
      \ec{&2} and with arguments consisting of the values of its
      formal parameters in \ec{&2} and terminating in a memory in
      which the value of $e_2$ (replacing references to \ec{&2} with
      reference to this memory) is $a$.
    \end{itemize}

    \medskip For example, consider the modules
    \ecinput[linerange=3-20]{examps/tactics/bypr/1.ec}
    If the current goal is
    \ecinput{examps/parts/tactics/bypr/1-1.0.ec} then
    running \ecinput{examps/parts/tactics/bypr/1-1.ec}
    produces the goals
    \ecinput{examps/parts/tactics/bypr/1-1.1.ec}
    and
    \ecinput{examps/parts/tactics/bypr/1-1.2.ec}
  \end{tsyntax}

  \begin{tsyntax}{bypr}
    If the goal's conclusion has the form
    \begin{center}
      \ec{hoare[$M$.$p$ : $\;P$ ==> $\;Q$]},
    \end{center}
    then reduce the goal to one whose conclusion says that, for all
    memories \ec{&m} for \ec{$M$.$p$} such that \ec{$P$\{$m$\}} holds,
    the probability of running \ec{$M$.$p$} in memory \ec{&$m$} and
    with arguments consisting of the values of its formal parameters
    in \ec{&$m$} and terminating in a memory satisfying \ec{!$Q$} is $0$.

    \medskip For example, consider the module
    \ecinput[linerange=3-10]{examps/tactics/bypr/2.ec}
    If the current goal is
    \ecinput{examps/parts/tactics/bypr/2-1.0.ec} then
    running \ecinput{examps/parts/tactics/bypr/2-1.ec}
    produces the goal
    \ecinput{examps/parts/tactics/bypr/2-1.1.ec}
  \end{tsyntax}
\end{tactic}

%%  Derives a program judgment from a probability relation or an exact
%%  probability. Only applies to judgments on procedures.
%%
%%  \paragraph{Examples:}\strut
%%  
%%  \begin{cmathpar}
%%    \texample[\prhl{}]
%%      {\ec{bypr (r$_1$) (r$_2$)}}
%%      {\forall \mem{m_1}, \mem{m_2}, a.\,
%%          r_1 = a \Rightarrow
%%          r_2 = a \Rightarrow
%%          {\mem{m_1}} \rel{Q} {\mem{m_2}} \\
%%       \forall \vec{a}_1, \vec{a}_2, \mem{m_1}, \mem{m_2}, a.\,
%%         {\mem{m_1}[\Arg\mapsto\vec{a}_1]} \rel{P} {\mem{m_2}[\Arg\mapsto\vec{a}_2]} \Rightarrow \\
%%         \PR{f_1}{\vec{a}_1}{\mem{m_1}}{a = r_1} = \PR{f_2}{\vec{a}_2}{\mem{m_2}}{a = r_2}}
%%      {\pRHL{P}{f_1}{f_2}{Q}}
%%  \end{cmathpar}
%%
%%  \begin{cmathpar}
%%    \texample[\phl{}]
%%      {\ec{bypr}}
%%      {\forall \mem{m}, \vec{a}.\, P\ m[\Arg\mapsto\vec{a}] \Rightarrow
%%          \PR{f}{\vec{a}}{m}{E} \mathrel{\diamond} \delta}
%%      {\pHL{P}{f}{E}{\diamond}{\delta}}
%%  \end{cmathpar}
%%
%%  \begin{cmathpar}
%%    \texample[\hl{}]
%%      {\ec{bypr}}
%%      {\forall \mem{m}, \vec{a}.\, P\ m[\Arg\mapsto\vec{a}] \Rightarrow
%%          \PR{f}{\vec{a}}{m}{\neg E} \mathop{=}0\%r}
%%      {\HL{P}{f}{E}}
%%  \end{cmathpar}
%%  \end{tsyntax}
%%\end{tactic}
