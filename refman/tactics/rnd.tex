% --------------------------------------------------------------------
\begin{tactic}{rnd}
  \begin{tsyntax}{rnd | rnd $\;f$ | rnd $\;f$ $\;g$} If the conclusion
    is a \prhl statement judgement whose programs end with random
    assignments \ec{$x_1\!$ <$\$$ $\;d_1$} and \ec{$x_2\!$ <$\$$
      $\;d_2$}, and $f$ and $g$ are functions between the types of
    $x_1$ and $x_2$, then consume those random assignments, replacing
    the conclusion's postcondition by the probabilistic weakest
    precondition of the random assignments wrt.\ $f$ and $g$.  The new
    postcondition checks that:
    \begin{itemize}
    \item $f$ and $g$ are an isomorphism between the distributions
      $d_1$ and $d_2$;

    \item for all elements $u$ in the support of $d_1$, the result
      of substituting $u$ and $f\,u$ for \ec{$x_1$\{1\}} and
      \ec{$x_2$\{2\}} in the conclusion's original postcondition
      holds.
    \end{itemize}
    When $g$ is $f$, it can be omitted. When $f$ is the identity, it
    can be omitted.

    \bigskip For example, if the current goal is
    \ecinput{../examps/parts/tactics/rnd/2-1.0.ec}{}{}{}{} then
    running \ecinput{../examps/parts/tactics/rnd/2-1.ec}{}{}{}{}
    produces the goal
    \ecinput{../examps/parts/tactics/rnd/2-1.1.ec}{}{}{}{}
    Note that if one uses the other isomorphism between \ec{\{0,1\}} and
    \ec{[2..3]} the generated subgoal will be false.
  \end{tsyntax}

  \begin{tsyntax}{rnd\{1\} | rnd\{2\}}
    If the conclusion is a \prhl statement judgement whose designated
    program (1 or 2) ends with a random assignment 
    \ec{$x\!$ <$\$$ $\;d$}, then consume that random assignment,
    replacing the conclusion's postcondition with a check that:
    \begin{itemize}
    \item the weight of $d$ is $1$ (so the random assignment can't fail);

    \item for all elements $u$ in the support of $d$, the result
      of substituting $u$ for \ec{$x$\{$i$\}}---where $i$ is the
      selected side---in the conclusion's original
      postcondition holds.
    \end{itemize}

    \bigskip For example, if the current goal is
    \ecinput{../examps/parts/tactics/rnd/3-1.0.ec}{}{}{}{} then
    running \ecinput{../examps/parts/tactics/rnd/3-1.ec}{}{}{}{}
    produces the (false!) goal
    \ecinput{../examps/parts/tactics/rnd/3-1.1.ec}{}{}{}{}
  \end{tsyntax}

  \begin{tsyntax}{rnd}
    If the conclusion is an \hl statement judgement whose program ends
    with a random assignment, then consume that random assignment,
    replacing the conclusion's postcondition by the possibilistic
    weakest precondition of the random assignment.

    \bigskip For example, if the current goal is
    \ecinput{../examps/parts/tactics/rnd/1-1.0.ec}{}{}{}{} then
    running \ecinput{../examps/parts/tactics/rnd/1-1.ec}{}{}{}{}
    produces the goal
    \ecinput{../examps/parts/tactics/rnd/1-1.1.ec}{}{}{}{}
  \end{tsyntax}

  \begin{tsyntax}{rnd | rnd $\;E$}
    In \phl, compute the probabilistic weakest precondition of a
    random sampling with respect to event $E$. When $E$ is not
    specified, it is inferred from the current postcondition.
  \end{tsyntax}
\end{tactic}
