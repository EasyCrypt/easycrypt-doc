% --------------------------------------------------------------------
\begin{tactic}{byequiv}

  \begin{tsyntax}{byequiv [option]? <spec>}
  Derives a probability relation from a \prhl judgement on the
  procedures involved. \ec{<spec>} can include wildcards when the
  tactic should infer the pre or postcondition. In addition,
  \ec{<spec>} can be extended with a failure event to infer precise
  applications of the Fundamental Lemma.

  \textbf{Options:} By default, (\ec{eq} option) specification
  inference attempts to infer a conjunction of equalities sufficient
  to imply the desired relation. Passing the \ec{-eq} option
  overrides this behaviour, instead using the trivial relation on
  events.

  \textbf{Examples:}
  \begin{mathpar}
    \inferrule%%
      {\pRHL{P}{f_1}{f_2}{Q} \\%
       {m_1[\Arg\mapsto\vec{a}_1]} \rel{P} {m_2[\Arg\mapsto\vec{a}_2]} \\%
       Q \Rightarrow E_1\{1\}  \Leftrightarrow E_2\{2\}}%%
      {\PR{f_1}{\vec{a}_1}{\mem{m_1}}{E_1} = \PR{f_2}{\vec{a}_2}{m_2}{E_2}}%%
      \quad\mbox{\parbox{200pt}{\ec{byequiv (_: P ==> Q)}}} \\
    \inferrule%%
      {\pRHL{P}{f_1}{f_2}{Q} \\% 
       {m_1[\Arg\mapsto\vec{a}_1]} \rel{P} {m_2[\Arg\mapsto\vec{a}_2]} \\%
       Q \Rightarrow E_1\{1\}  \Rightarrow E_2\{2\}}%%
      {\PR{f_1}{\vec{a}_1}{\mem{m_1}}{E_1} \leq \PR{f_2}{\vec{a}_2}{m_2}{E_2}}%%
      \quad\mbox{\parbox{200pt}{\ec{byequiv (_: P ==> Q)}}} \\
    \inferrule%%
      {\pRHL{P}{f_1}{f_2}{Q} \\%
       {m_1[\Arg\mapsto\vec{a}_1]} \rel{P} {m_2[\Arg\mapsto\vec{a}_2]} \\%
       Q \Rightarrow E_2\{2\}  \Rightarrow E_1\{1\}} %%
      {\PR{f_1}{\vec{a}_1}{\mem{m_1}}{E_1} \geq \PR{f_2}{\vec{a}_2}{m_2}{E_2}}%%
      \quad\mbox{\parbox{200pt}{\ec{byequiv (_: P ==> Q)}}} \\
    \inferrule%%
      {\pRHL{P}{f_1}{f_2}{Q} \\% 
       {m_1[\Arg\mapsto\vec{a}_1]} \rel{P} {m_2[\Arg\mapsto\vec{a}_2]} \\%
       Q \Rightarrow \neg B_2\{2\} \Rightarrow E_1\{1\}  \Rightarrow E_2\{2\}}%%
      {\PR{f_1}{\vec{a}_1}{\mem{m_1}}{E_1}
       \leq \PR{f_2}{\vec{a}_2}{m_2}{E_2}
          + \PR{f_2}{\vec{a}_2}{m_2}{B_2}}%% 
      \quad\mbox{\parbox{200pt}{\ec{byequiv (_: P ==> Q)}}} \\
    \inferrule%%
      {\pRHL{P}{f_1}{f_2}{Q} \\% 
       {m_1[\Arg\mapsto\vec{a}_1]} \rel{P} {m_2[\Arg\mapsto\vec{a}_2]} \\%
       Q \Rightarrow
         (B_1\{1\} \Leftrightarrow B_2\{2\})
         \wedge (\neg B_2\{2\} \Rightarrow E_1\{1\} \Leftrightarrow E_2\{2\})}%%
      {| \PR{f_1}{\vec{a}_1}{\mem{m_1}}{E_1} - \PR{f_2}{\vec{a}_2}{m_2}{E_2} |
       \leq \PR{f_2}{\vec{a}_2}{m_2}{B_2}}%% 
      \quad\mbox{\parbox{200pt}{\ec{byequiv (_: P ==> Q) : B$_1$}}} \\%%
    \inferrule%%
      {\pRHL{P}{f_1}{f_2}{E_1\{1\} \Leftrightarrow E_2\{2\}} \\%
       {m_1[\Arg\mapsto\vec{a}_1]} \rel{P} {m_2[\Arg\mapsto\vec{a}_2]}}%%
      {\PR{f_1}{\vec{a}_1}{\mem{m_1}}{E_1} = \PR{f_2}{\vec{a}_2}{m_2}{E_2}}%%
      \quad\mbox{\parbox{200pt}{\ec{byequiv [-eq] (_: P ==> _)}}}
  \end{mathpar}
 \end{tsyntax}

  \begin{tsyntax}{byequiv <lemma>}
  Same as \ec{byequiv <spec>}, but the specification to use
  is inferred from the lemma provided. Raises an error if the lemma
  does not refer to the expected procedures. Inference options have no
  effect in this setting.
  \end{tsyntax}
\end{tactic}
