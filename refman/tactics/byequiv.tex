% --------------------------------------------------------------------
\begin{tactic}{byequiv}
  \begin{tsyntax}{byequiv (_ : $\;P$ ==> $\;Q$)}
    If the goal's conclusion has the form
    \begin{center}
      \ec{Pr[$M_1$.$p_1$($a_{1,1}$, $\ldots$, $a_{1,n_1}$) @ &$m_1$ : $\;E_1$] =
          Pr[$M_2$.$p_2$($a_{2,1}$, $\ldots$, $a_{2,n_2}$) @ &$m_2$ : $\;E_2$]},
    \end{center}
    reduce the goal to three subgoals:
    \begin{itemize}
    \item One with conclusion
          \ec{equiv[$M_1$.$p_1$ ~ $\;M_2$.$p_2$ : $\;P$ ==> $\;Q$]};

    \item One whose conclusion says that $P$ holds, where
      references to memories \ec{&1} and \ec{&2} have been replaced
      by \ec{&$m_1$} and \ec{&$m_2$}, respectively, and references
      to the formal parameters of \ec{$M_1$.$p_1$} and
      \ec{$M_2$.$p_2$} have been replaced by their arguments;

    \item One whose conclusion says that $Q$ implies that
      \ec{$E_1$\{1\} <=> $\;E_2$\{2\}}.
    \end{itemize}

    The argument to \ec{byequiv} may be replaced by a proof term for
    \ec{equiv[$M_1$.$p_1$ ~ $\;M_2$.$p_2$ : $\;P$ ==> $\;Q$]}, in which
    case the first subgoal isn't generated.
    Furthermore, either or both of $P$ and $Q$ may be replaced by
    \ec{_}, asking that the pre- or postcondition be inferred.
    Supplying no argument to \ec{equiv} is the same as replacing
    both $P$ and $Q$ by \ec{_}. By default, inference of $Q$ attempts
    to infer a conjuction of equalities implying   
    \ec{$E_1$\{1\} <=> $\;E_2$\{2\}}. Passing the \ec{[-eq]} option to
    \ec{conseq} takes $Q$ to \emph{be} \ec{$E_1$\{1\} <=> $\;E_2$\{2\}}.

    \medskip
    \emph{The other variants of the tactic behave similarly with
    regards to the use of proof terms and specification inference.}

    \medskip For example, consider the module
    \ecinput{examps/tactics/byequiv/1.ec}{}{3-10}{}
    If the current goal is
    \ecinput{examps/parts/tactics/byequiv/1-1.0.ec}{}{}{}{} then
    running \ecinput{examps/parts/tactics/byequiv/1-1.ec}{}{}{}{}
    produces the goals
    \ecinput{examps/parts/tactics/byequiv/1-1.1.ec}{}{}{}{}
    and
    \ecinput{examps/parts/tactics/byequiv/1-1.2.ec}{}{}{}{}
    and
    \ecinput{examps/parts/tactics/byequiv/1-1.3.ec}{}{}{}{}
    Given the lemma
    \ecinput{examps/tactics/byequiv/1.ec}{}{24-24}{}
    if the current goal is
    \ecinput{examps/parts/tactics/byequiv/1-2.0.ec}{}{}{}{} then
    running \ecinput{examps/parts/tactics/byequiv/1-2.ec}{}{}{}{}
    produces the goals
    \ecinput{examps/parts/tactics/byequiv/1-2.1.ec}{}{}{}{}
    and
    \ecinput{examps/parts/tactics/byequiv/1-2.2.ec}{}{}{}{}
    And, if the current goal is
    \ecinput{examps/parts/tactics/byequiv/1-3.0.ec}{}{}{}{} then
    running \ecinput{examps/parts/tactics/byequiv/1-3.ec}{}{}{}{}
    produces the goals
    \ecinput{examps/parts/tactics/byequiv/1-3.1.ec}{}{}{}{}
    and
    \ecinput{examps/parts/tactics/byequiv/1-3.2.ec}{}{}{}{}
    and
    \ecinput{examps/parts/tactics/byequiv/1-3.3.ec}{}{}{}{}
  \end{tsyntax}

  \begin{tsyntax}{byequiv (_ : $\;P$ ==> $\;Q$)}
    If the goal's conclusion has the form
    \begin{center}
      \ec{Pr[$M_1$.$p_1$($a_{1,1}$, $\ldots$, $a_{1,n_1}$) @ &$m_1$ : $\;E_1$] <=
          Pr[$M_2$.$p_2$($a_{2,1}$, $\ldots$, $a_{2,n_2}$) @ &$m_2$ : $\;E_2$]},
    \end{center}
    then \ec{conseq} behaves the same as in the first variant
    except that the conclusion of the third subgoal says that $Q$
    implies \ec{$E_1$\{1\} => $\;E_2$\{2\}}.

    \medskip For example, if the current goal is
    \ecinput{examps/parts/tactics/byequiv/2-1.0.ec}{}{}{}{} then
    running \ecinput{examps/parts/tactics/byequiv/2-1.ec}{}{}{}{}
    produces the goals
    \ecinput{examps/parts/tactics/byequiv/2-1.1.ec}{}{}{}{}
    and
    \ecinput{examps/parts/tactics/byequiv/2-1.2.ec}{}{}{}{}
    and
    \ecinput{examps/parts/tactics/byequiv/2-1.3.ec}{}{}{}{}
    And, if the current goal is
    \ecinput{examps/parts/tactics/byequiv/2-2.0.ec}{}{}{}{} then
    running \ecinput{examps/parts/tactics/byequiv/2-2.ec}{}{}{}{}
    produces the goals
    \ecinput{examps/parts/tactics/byequiv/2-2.1.ec}{}{}{}{}
    and
    \ecinput{examps/parts/tactics/byequiv/2-2.2.ec}{}{}{}{}
    and
    \ecinput{examps/parts/tactics/byequiv/2-2.3.ec}{}{}{}{}
  \end{tsyntax}

  \begin{tsyntax}{byequiv (_ : $\;P$ ==> $\;Q$)}
    If the goal's conclusion has the form
    \begin{center}
      \ec{Pr[$M_1$.$p_1$($a_{1,1}$, $\ldots$, $a_{1,n_1}$) @ &$m_1$ : $\;E_1$] <=} \\
      \ec{Pr[$M_2$.$p_2$($a_{2,1}$, $\ldots$, $a_{2,n_2}$) @ &$m_2$ : $\;E_2$] +
          Pr[$M_2$.$p_2$($a_{2,1}$, $\ldots$, $a_{2,n_2}$) @ &$m_2$ : $\;B_2$]},
    \end{center}
    then \ec{conseq} behaves the same as in the first variant
    except that the conclusion of the third subgoal says that $Q$
    implies \ec{!$B_2$\{2\} => $\;E_1$\{1\} => $\;E_2$\{2\}}.

    \medskip For example, if the current goal is
    \ecinput{examps/parts/tactics/byequiv/3-1.0.ec}{}{}{}{} then
    running \ecinput{examps/parts/tactics/byequiv/3-1.ec}{}{}{}{}
    produces the goals
    \ecinput{examps/parts/tactics/byequiv/3-1.1.ec}{}{}{}{}
    and
    \ecinput{examps/parts/tactics/byequiv/3-1.2.ec}{}{}{}{}
    \fixme{Why is the second subgoal pruned? (Compare with first and
    second variants, where the corresponding subgoal isn't pruned.)}
    And, if the current goal is
    \ecinput{examps/parts/tactics/byequiv/3-2.0.ec}{}{}{}{} then
    running \ecinput{examps/parts/tactics/byequiv/3-2.ec}{}{}{}{}
    produces the goals
    \ecinput{examps/parts/tactics/byequiv/3-2.1.ec}{}{}{}{}
    and
    \ecinput{examps/parts/tactics/byequiv/3-2.2.ec}{}{}{}{}
    \fixme{Why is the second subgoal pruned?}
  \end{tsyntax}

  \begin{tsyntax}{byequiv (_ : $\;P$ ==> $\;Q$) : $\;B_1$}
    If the goal's conclusion has the form
    \begin{center}
      \ec{`| Pr[$M_1$.$p_1$($a_{1,1}$, $\ldots$, $a_{1,n_1}$) @ &$m_1$ : $\;E_1$] -
          Pr[$M_2$.$p_2$($a_{2,1}$, $\ldots$, $a_{2,n_2}$) @ &$m_2$ : $\;E_2$] | <=} \\
      \ec{Pr[$M_2$.$p_2$($a_{2,1}$, $\ldots$, $a_{2,n_2}$) @ &$m_2$ : $\;B_2$]},
    \end{center}
    then \ec{conseq} behaves the same as in the first variant
    except that the conclusion of the third subgoal says that $Q$
    implies
    \begin{center}
      \ec{($B_1$\{1\} <=> $\;B_2$\{2\}) /\\ ! $\;B_2$\{2\} => ($\;E_1$\{1\} <=> $\;E_2$\{2\})}
    \end{center}

    \medskip For example, if the current goal is
    \ecinput{examps/parts/tactics/byequiv/4-1.0.ec}{}{}{}{} then
    running \ecinput{examps/parts/tactics/byequiv/4-1.ec}{}{}{}{}
    produces the goals
    \ecinput{examps/parts/tactics/byequiv/4-1.1.ec}{}{}{}{}
    and
    \ecinput{examps/parts/tactics/byequiv/4-1.2.ec}{}{}{}{}
    and
    \ecinput{examps/parts/tactics/byequiv/4-1.3.ec}{}{}{}{}
    Given the lemma
    \ecinput{examps/tactics/byequiv/4.ec}{}{37-38}{}
    if the current goal is
    \ecinput{examps/parts/tactics/byequiv/4-2.0.ec}{}{}{}{} then
    running \ecinput{examps/parts/tactics/byequiv/4-2.ec}{}{}{}{}
    produces the goals
    \ecinput{examps/parts/tactics/byequiv/4-2.1.ec}{}{}{}{}
    and
    \ecinput{examps/parts/tactics/byequiv/4-2.2.ec}{}{}{}{}
  \end{tsyntax}
\end{tactic}

%%  \begin{tsyntax}{byequiv [option]? <spec>}
%%  Derives a probability relation from a \prhl judgement on the
%%  procedures involved. \ec{<spec>} can include wildcards when the
%%  tactic should infer the pre or postcondition. In addition,
%%  \ec{<spec>} can be extended with a failure event to infer precise
%%  applications of the Fundamental Lemma.
%%
%%  \textbf{Options:} By default, (\ec{eq} option) specification
%%  inference attempts to infer a conjunction of equalities sufficient
%%  to imply the desired relation. Passing the \ec{-eq} option
%%  overrides this behaviour, instead using the trivial relation on
%%  events.
%%
%%  \paragraph{Examples:}\strut
%%
%%  \begin{cmathpar}
%%    \texample
%%      {\ec{byequiv (_: P ==> Q)}}
%%      {\pRHL{P}{f_1}{f_2}{Q} \\\\
%%       {m_1[\Arg\mapsto\vec{a}_1]} \rel{P} {m_2[\Arg\mapsto\vec{a}_2]} \\
%%       Q \Rightarrow E_1\{1\}  \Leftrightarrow E_2\{2\}}
%%      {\PR{f_1}{\vec{a}_1}{\mem{m_1}}{E_1} = \PR{f_2}{\vec{a}_2}{m_2}{E_2}}
%%  \end{cmathpar}
%%
%%  \begin{cmathpar}
%%    \texample
%%      {\ec{byequiv (_: P ==> Q)}}
%%      {\pRHL{P}{f_1}{f_2}{Q} \\\\
%%       {m_1[\Arg\mapsto\vec{a}_1]} \rel{P} {m_2[\Arg\mapsto\vec{a}_2]} \\
%%       Q \Rightarrow E_1\{1\}  \Rightarrow E_2\{2\}}
%%      {\PR{f_1}{\vec{a}_1}{\mem{m_1}}{E_1} \leq \PR{f_2}{\vec{a}_2}{m_2}{E_2}}
%%  \end{cmathpar}
%%
%%  \begin{cmathpar}
%%    \texample
%%      {\ec{byequiv (_: P ==> Q)}}
%%      {\pRHL{P}{f_1}{f_2}{Q} \\\\
%%       {m_1[\Arg\mapsto\vec{a}_1]} \rel{P} {m_2[\Arg\mapsto\vec{a}_2]} \\
%%       Q \Rightarrow E_2\{2\}  \Rightarrow E_1\{1\}}
%%      {\PR{f_1}{\vec{a}_1}{\mem{m_1}}{E_1} \geq \PR{f_2}{\vec{a}_2}{m_2}{E_2}}
%%  \end{cmathpar}
%%
%%  \begin{cmathpar}
%%    \texample
%%      {\ec{byequiv (_: P ==> Q)}}
%%      {\pRHL{P}{f_1}{f_2}{Q} \\
%%       {m_1[\Arg\mapsto\vec{a}_1]} \rel{P} {m_2[\Arg\mapsto\vec{a}_2]} \\
%%       Q \Rightarrow \neg B_2\{2\} \Rightarrow E_1\{1\}  \Rightarrow E_2\{2\}}
%%      {\PR{f_1}{\vec{a}_1}{\mem{m_1}}{E_1}
%%       \leq \PR{f_2}{\vec{a}_2}{m_2}{E_2}
%%          + \PR{f_2}{\vec{a}_2}{m_2}{B_2}}
%%  \end{cmathpar}
%%
%%  \begin{cmathpar}
%%    \texample
%%      {\ec{byequiv (_: P ==> Q) : B$_1$}}
%%      {\pRHL{P}{f_1}{f_2}{Q} \\
%%       {m_1[\Arg\mapsto\vec{a}_1]} \rel{P} {m_2[\Arg\mapsto\vec{a}_2]} \\
%%       Q \Rightarrow
%%         (B_1\{1\} \Leftrightarrow B_2\{2\})
%%         \wedge (\neg B_2\{2\} \Rightarrow E_1\{1\} \Leftrightarrow E_2\{2\})}
%%      {| \PR{f_1}{\vec{a}_1}{\mem{m_1}}{E_1} - \PR{f_2}{\vec{a}_2}{m_2}{E_2} |
%%       \leq \PR{f_2}{\vec{a}_2}{m_2}{B_2}}
%%  \end{cmathpar}
%%
%%  \begin{cmathpar}
%%    \texample
%%      {\ec{byequiv [-eq] (_: P ==> _)}}
%%      {\pRHL{P}{f_1}{f_2}{E_1\{1\} \Leftrightarrow E_2\{2\}} \\
%%       {m_1[\Arg\mapsto\vec{a}_1]} \rel{P} {m_2[\Arg\mapsto\vec{a}_2]}}
%%      {\PR{f_1}{\vec{a}_1}{\mem{m_1}}{E_1} = \PR{f_2}{\vec{a}_2}{m_2}{E_2}}
%%  \end{cmathpar}
%% \end{tsyntax}
%%
%%  \begin{tsyntax}{byequiv <lemma>}
%%  Same as \ec{byequiv <spec>}, but the specification to use
%%  is inferred from the lemma provided. Raises an error if the lemma
%%  does not refer to the expected procedures. Inference options have no
%%  effect in this setting.
%%  \end{tsyntax}
