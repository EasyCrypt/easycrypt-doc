% --------------------------------------------------------------------
\begin{tactic}{while}
  \begin{tsyntax}{while $\;I$}
    Here $I$ is an \emph{invariant} (formula), which may reference
    variables of the two programs, interpreted in their memories.  If
    the goal's conclusion is a \prhl statement judgement whose
    programs both end with \ec{while} statements, reduce the goal to
    two subgoals whose conclusions are \prhl statement judgements:
    \begin{itemize}
    \item One whose first and second programs are the bodies of the
      first and second while statements, whose precondition is the
      conjunction of $I$ and the while statements' boolean expressions (the
      first of which is interpreted in memory \ec{&1}, and the second
      of which is interpreted in \ec{&2}) and whose postcondition is
      the conjunction of $I$ and the assertion that the while statements'
      boolean expressions (interpreted in the appropriate memories)
      are equivalent.

    \item One whose precondition is the original goal's precondition,
      whose first and second programs are all the results of removing
      the while statements from the two programs, and whose postcondition is
      the conjunction of:
      \begin{itemize}
      \item the conjunction of $I$ and the assertion that the while statements'
        boolean expressions are equivalent; and

      \item the assertion that, for all values of the variables
        \emph{modified} by the while statements, if the while statements'
        boolean expressions don't hold, but $I$ holds, then the
        original goal's postcondition holds (in $I$, the while statements'
        boolean expressions, and the postcondition, variables modified
        by the while statements are replaced by universally quantified
        identifiers; otherwise, the boolean expressions are
        interpreted in the program's respective memories, and the
        memory references of $I$ and the postcondition are
        maintained).
      \end{itemize}
    \end{itemize}

  \medskip
  For example, if the current goal is
  \ecinput{examps/parts/tactics/while/1-1.0.ec} then
  running \ecinput{examps/parts/tactics/while/1-1.ec}
  produces the goals
  \ecinput{examps/parts/tactics/while/1-1.1.ec}
  and
  \ecinput{examps/parts/tactics/while/1-1.2.ec}
  \end{tsyntax}

  \begin{tsyntax}{while\{1\} $\;I$ $\;v$ | while\{2\} $\;I$ $\;v$}
    Here $I$ is an \emph{invariant} (formula) and $v$ is a
    \emph{termination variant} integer expression, both of which may reference
    variables of the two programs, interpreted in their memories.  If
    the goal's conclusion is a \prhl statement judgement whose
    designated program (1 or 2) ends with a \ec{while} statement,
    reduce the goal to two subgoals;
    \begin{itemize}
    \item One whose conclusion is a \phl statement judgement, saying that
      running the body of the while statement in a memory in which
      $I$ holds and the while statement's boolean expression is true
      is guaranteed to result in termination in a memory in which
      $I$ holds and in which the value of the variant expression $v$
      has decreased by at least $1$. (More precisely, the \phl statement
      judgment is universally quantified by the memory of the non-designated
      program and the initial value of $v$. References to the variables
      of the nondesignated program in $I$ and $v$ are interpreted in this
      memory; reference to the variables of the designed program have
      their memory references removed.)

    \item One whose conclusion is a \prhl statement judgement whose
      precondition is the original goal's precondition, whose
      designated program is the result of removing the while statement
      from the original designated program, whose other program is
      unchanged, and whose postcondition is the conjunction of $I$ and
      the assertion that, for all values of the variables modified by
      the while statement, that the conjunction of the following
      formulas holds:
      \begin{itemize}
      \item the assertion that, if $I$ holds, but the variant
        expression $v$ is not positive, then the while statement's
        boolean expression is false;

      \item the assertion that, if the while statement's boolean expression
        doesn't hold, but $I$ holds, then the original goal's postcondition
        holds.
      \end{itemize}
    \end{itemize}

  \bigskip
  For example, if the current goal is
  \ecinput{examps/parts/tactics/while/3-1.0.ec} then
  running \ecinput{examps/parts/tactics/while/3-1.ec}
  produces the goals
  \ecinput{examps/parts/tactics/while/3-1.1.ec}
  and
  \ecinput{examps/parts/tactics/while/3-1.2.ec}
  \end{tsyntax}

  \begin{tsyntax}{while $\;I$}
    Here $I$ is an \emph{invariant} (formula), which may reference
    variables of the program.  If the goal's conclusion is an \hl
    statement judgement ending with a \ec{while} statement, reduce the
    goal to two subgoals whose conclusions are \hl statement
    judgements:
    \begin{itemize}
    \item One whose program is the body of the while statement, whose
      precondition is the conjunction of $I$ and the while statement's
      boolean expression, and whose postcondition is $I$.

    \item One whose precondition is the original goal's precondition,
      whose program is the result of removing the while statement from
      the original program, and whose postcondition is the conjunction
      of:
      \begin{itemize}
      \item $I$; and

      \item the assertion that, for all values of the variables
        \emph{modified} by the while statement, if the while statement's boolean
        expression doesn't hold, but $I$ holds, then the original
        goal's postcondition holds (in $I$, the while statement's boolean
        expression, and the postcondition, variables modified by the
        while statement are replaced by universally quantified
        identifiers).
      \end{itemize}
    \end{itemize}

  \bigskip
  For example, if the current goal is
  \ecinput{examps/parts/tactics/while/2-1.0.ec} then
  running \ecinput{examps/parts/tactics/while/2-1.ec}
  produces the goals
  \ecinput{examps/parts/tactics/while/2-1.1.ec}
  and
  \ecinput{examps/parts/tactics/while/2-1.2.ec}
  \end{tsyntax}

  \begin{tsyntax}{while $\;I$ $\;v$}
  \phl version...

%  Where \ec{v} is an integer-valued expression. In \phl, performs a
%  weakest precondition computation over a loop, using \ec{I} as
%  invariant and \ec{v} as a decreasing variant to prove
%  termination. In addition to the two invariant-related subgoals (see
%  above), two subgoals regarding the variant are generated; one
%  requiring that the variant be less than 0 exactly when the loop
%  condition is false, and the other requiring that the variant
%  decreases strictly.
  \end{tsyntax}
\end{tactic}
