% --------------------------------------------------------------------
\begin{tactic}{while}
  \begin{tsyntax}{while I}
  In \prhl and \hl, as well as upper-bounding \phl judgments, performs
  a weakest precondition computation over a loop using \ec{I} as
  invariant. This generates two subgoals: one explaining that \ec{I}
  is a valid loop invariant, and the other explaining that the
  invariant is initially true and that it is sufficient to establish
  the current postcondition.
  \end{tsyntax}

  \begin{tsyntax}{while I v}
  Where \ec{v} is an integer-valued expression. In \phl, performs a
  weakest precondition computation over a loop, using \ec{I} as
  invariant and \ec{v} as a decreasing variant to prove
  termination. In addition to the two invariant-related subgoals (see
  above), two subgoals regarding the variant are generated; one
  requiring that the variant be less than 0 exactly when the loop
  condition is false, and the other requiring that the variant
  decreases strictly.
  \end{tsyntax}

  \textbf{Note:} More complex variants of the \ec{while} tactic
  exist, useful in particular for dealing in \phl with loops whose
  termination is not variant-based. However, these advanced variants
  are currently undocumented.
\end{tactic}
