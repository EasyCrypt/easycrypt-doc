% --------------------------------------------------------------------
\begin{tactic}[move | move: $\;\pi_1 \cdots \pi_n$]{move}
  \begin{tsyntax}{move}
     Does nothing, equivalent to \rtactic{idtac}. This form is mainly
     used in conjunction with the introduction tactical, as in
     \ec{move=>$\;\iota_1 \cdots \iota_n$}; see
     Section~\ref{subsec:intropatterns}
     for more information about the introduction tactical.
  \end{tsyntax}

  \begin{tsyntax}{move: $\;\pi_1 \cdots \pi_n$}
    This is equivalent to \ec{generalize $\;\pi_1 \cdots \pi_n$}.
    See Section~\ref{subsec:intropatterns} for more
    information on the generalization mechanism.
  \end{tsyntax}
\end{tactic}
