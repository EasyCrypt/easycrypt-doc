% --------------------------------------------------------------------
\begin{tactic}[move | move: $\;\pi_1 \cdots \pi_n$]{move}
  \begin{tsyntax}{move}
     Does nothing, equivalent to \rtactic{idtac}. This form is mainly
     used in conjunction with an introduction pattern, as in
     \ls!move=> $\iota_1 \cdots \iota_n$! -- see Section~\ref{subsec:intropatterns})
     for more information about introduction patterns.
  \end{tsyntax}

  \begin{tsyntax}{move: $\;\pi_1 \cdots \pi_n$}
    Generalize the patterns $\pi_1, \cdots, \pi_n$, starting from
    $\pi_n$ and going back. See Section~\ref{subsec:intropatterns} for more
    information on the generalization mechanism.
  \end{tsyntax}
\end{tactic}
