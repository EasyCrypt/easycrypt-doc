% --------------------------------------------------------------------
\begin{tactic}{exfalso}
  \begin{tsyntax}{exfalso}
  Combines \rtactic{conseq}, \rtactic{byequiv}, \rtactic{byphoare},
  \rtactic{hoare} and \rtactic{bypr} to strengthen the precondition
  into $\mathsf{false}$ and discharge the resulting trivial goal.

  For example, if the current goal is
  \ecinput{examps/parts/tactics/exfalso/1-1.0.ec}{}{}{}{} then
  running \ecinput{examps/parts/tactics/exfalso/1-1.ec}{}{}{}{}
  produces the goals
  \ecinput{examps/parts/tactics/exfalso/1-1.1.ec}{}{}{}{}
  and
  \ecinput{examps/parts/tactics/exfalso/1-1.2.ec}{}{}{}{}
  The first of these goals is solved by
  running \ecinput{examps/parts/tactics/exfalso/1-2.ec}{}{}{}{}
  And running \ecinput{examps/parts/tactics/exfalso/1-3.ec}{}{}{}{}
  reduces the second of these goals to
  \ecinput{examps/parts/tactics/exfalso/1-3.1.ec}{}{}{}{}
  which \ec{smt} solves.
  \end{tsyntax}

  \fixme{Perhaps need other examples?}
\end{tactic}

%%  \paragraph{Examples:}\strut
%%  
%%  \begin{cmathpar}
%%  \texample[\prhl{}]
%%    {\ec{exfalso}}
%%    {P \Rightarrow \mathsf{false}}
%%    {\pRHL{P}{c}{c'}{Q}}
%%
%%  \texample[\phl{}]
%%    {\ec{exfalso}}
%%    {P \Rightarrow \mathsf{false}}
%%    {\pHL{P}{c}{Q}{\diamond}{\delta}}
%%
%%  \texample[\hl{}]
%%    {\ec{exfalso}}
%%    {P \Rightarrow \mathsf{false}}
%%    {\HL{P}{c}{Q}}
%%  \end{cmathpar}
%%  \end{tsyntax}
%%\end{tactic}
