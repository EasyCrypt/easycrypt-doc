% --------------------------------------------------------------------
\begin{tactic}{unroll}
  \begin{tsyntax}{unroll $\;c$}
    If the goal's conclusion is an \hl statement judgement whose $c$th
    statement is a \ec{while} statement, then insert before that
    statement an \ec{if} statement whose boolean expression is the
    \ec{while} statement's boolean expression, whose \ec{then} part is
    the \ec{while} statements's body, and whose \ec{else} part is
    empty.

    \medskip For example, if the current goal is
    \ecinput{examps/parts/tactics/unroll/1-1.0.ec} then
    running \ecinput{examps/parts/tactics/unroll/1-1.ec}
    produces the goal
    \ecinput{examps/parts/tactics/unroll/1-1.1.ec}
    And, if the current goal is
    \ecinput{examps/parts/tactics/unroll/2-1.0.ec} then
    running \ecinput{examps/parts/tactics/unroll/2-1.ec}
    produces the goal
    \ecinput{examps/parts/tactics/unroll/2-1.1.ec}
  \end{tsyntax}

  \begin{tsyntax}{unroll\{1\} $\;c$ | unroll\{2\} $\;c$}
    If the goal's conclusion is an \prhl statement judgement where the
    $c$th statement of the designated program is a \ec{while}
    statement, then insert before that statement an \ec{if} statement
    whose boolean expression is the \ec{while} statement's boolean
    expression, whose \ec{then} part is the \ec{while} statements's
    body, and whose \ec{else} part is empty.

    \medskip For example, if the current goal is
    \ecinput{examps/parts/tactics/unroll/1-2.0.ec} then
    running \ecinput{examps/parts/tactics/unroll/1-2.ec}
    produces the goal
    \ecinput{examps/parts/tactics/unroll/1-2.1.ec}
    from which
    running \ecinput{examps/parts/tactics/unroll/1-3.ec}
    produces the goal
    \ecinput{examps/parts/tactics/unroll/1-3.1.ec}
  \end{tsyntax}
\end{tactic}
