% --------------------------------------------------------------------
\begin{tactic}[simplify | simplify $\;x_1 \cdots x_n$ | simplify delta]{simplify}
  \begin{tsyntax}[empty]{simplify}
    Reduce the goal's conclusion to its $\beta\iota\zeta\Lambda$-head
    normal-form, followed by one step of parallel, strong
    $\delta$-reduction if \ec{delta} is given.  The $\delta$-reduction
    can be restricted to a set of defined symbols by replacing
    \ec{delta} by a non-empty sequence of targeted symbols. You can
    reduce the conclusion to its $\beta$-head normal form
    (resp. $\iota$, $\zeta$, $\Lambda$-head normal form) by using the
    tactic \ec{beta} (resp. \ec{iota}, \ec{zeta}, \ec{logic}). These
    tactics can be combined together, separated by spaces, to perform
    head reduction by any combination of the rule sets.

  For example, if the current goal is
  \ecinput{../examps/parts/tactics/simplify/1-1.0.ec}{}{}{}{} then
  running \ecinput{../examps/parts/tactics/simplify/1-1.ec}{}{}{}{}
  produces the goal
  \ecinput{../examps/parts/tactics/simplify/1-1.1.ec}{}{}{}{}
  And if the current goal is
  \ecinput{../examps/parts/tactics/simplify/1-2.0.ec}{}{}{}{} then
  running \ecinput{../examps/parts/tactics/simplify/1-2.ec}{}{}{}{}
  produces the goal
  \ecinput{../examps/parts/tactics/simplify/1-2.1.ec}{}{}{}{}
  \end{tsyntax}
  \fix{Is this the right place to define ``convertible''?}
\end{tactic}
