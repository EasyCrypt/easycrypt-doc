% --------------------------------------------------------------------
\begin{tactic}{conseq}
  \begin{tsyntax}{conseq (_ : $\;P$ ==> $\;Q$)}

    If the goal's conclusion is a \prhl or \hl judgement or statement
    judgement, weaken the conclusion's precondition to $P$, and
    strengthen its postcondition to $Q$, generating initial auxilary
    subgoals checking that this reduction is sound. Fails if $P$ and
    $Q$ aren't compatible with the judgement type. The conclusion of
    the first auxiliary subgoal checks that the precondition $P'$ of
    the original goal's conclusion implies $P$. The conclusion of the
    second auxiliary subgoal checks that $Q$ implies the postcondition
    $Q'$ of the original goal's conclusion, except that any memory
    references to variables that may be modified by the conclusion's
    program(s) are universally quantified in $Q$ and $Q'$, and $P$ is
    also included as an assumption.

    $P$ or $Q$ may be replaced by \ec{_}, in which case the pre- or
    postcondition is left unchanged. When a pre- or postcondition is
    unchanged, the corresponding auxilary subgoal isn't generated (as
    its proof would be trivial). When the goal's conclusion is a
    judgement (not a statement judgement), a proof term whose
    conclusion is a judgement on the procedure(s) of the original
    goals's conclusion may be supplied as the argument to \ec{conseq},
    in which case $P$ and $Q$ are taken to be the pre- and
    postconditions of this judgement, and only the auxilary subgoals
    are generated.

    \medskip For example, if the current goal is
    \ecinput{examps/parts/tactics/conseq/1-1.0.ec}{}{}{}{} then
    running \ecinput{examps/parts/tactics/conseq/1-1.ec}{}{}{}{}
    produces the goals
    \ecinput{examps/parts/tactics/conseq/1-1.1.ec}{}{}{}{}
    and
    \ecinput{examps/parts/tactics/conseq/1-1.2.ec}{}{}{}{}
    Continuing from the last of these goals,
    running \ecinput{examps/parts/tactics/conseq/1-2.ec}{}{}{}{}
    produces the goals
    \ecinput{examps/parts/tactics/conseq/1-2.1.ec}{}{}{}{}
    and
    \ecinput{examps/parts/tactics/conseq/1-2.2.ec}{}{}{}{}
    Continuing from the last of these goals,
    running \ecinput{examps/parts/tactics/conseq/1-3.ec}{}{}{}{}
    produces the goals
    \ecinput{examps/parts/tactics/conseq/1-3.1.ec}{}{}{}{}
    and
    \ecinput{examps/parts/tactics/conseq/1-3.2.ec}{}{}{}{}
    and
    \ecinput{examps/parts/tactics/conseq/1-3.3.ec}{}{}{}{}

    \medskip If the current goal is
    \ecinput{examps/parts/tactics/conseq/1-4.0.ec}{}{}{}{} then
    running \ecinput{examps/parts/tactics/conseq/1-1.ec}{}{}{}{}
    produces the goals
    \ecinput{examps/parts/tactics/conseq/1-4.1.ec}{}{}{}{}
    and
    \ecinput{examps/parts/tactics/conseq/1-4.2.ec}{}{}{}{}
    Continuing from the last of these goals,
    running \ecinput{examps/parts/tactics/conseq/1-5.ec}{}{}{}{}
    produces the goals
    \ecinput{examps/parts/tactics/conseq/1-5.1.ec}{}{}{}{}
    and
    \ecinput{examps/parts/tactics/conseq/1-5.2.ec}{}{}{}{}
    Continuing from the last of these goals,
    running \ecinput{examps/parts/tactics/conseq/1-6.ec}{}{}{}{}
    produces the goals
    \ecinput{examps/parts/tactics/conseq/1-6.1.ec}{}{}{}{}
    and
    \ecinput{examps/parts/tactics/conseq/1-6.2.ec}{}{}{}{}
    and
    \ecinput{examps/parts/tactics/conseq/1-6.3.ec}{}{}{}{}

    Given lemma
    \ecinput{examps/tactics/conseq/1.ec}{}{66-69}{}
    if the current goal is
    \ecinput{examps/parts/tactics/conseq/1-7.0.ec}{}{}{}{} then
    running \ecinput{examps/parts/tactics/conseq/1-7.ec}{}{}{}{}
    produces the goals
    \ecinput{examps/parts/tactics/conseq/1-7.1.ec}{}{}{}{}
    and
    \ecinput{examps/parts/tactics/conseq/1-7.2.ec}{}{}{}{}

    \medskip If the current goal is
    \ecinput{examps/parts/tactics/conseq/1-8.0.ec}{}{}{}{} then
    running \ecinput{examps/parts/tactics/conseq/1-8.ec}{}{}{}{}
    produces the goals
    \ecinput{examps/parts/tactics/conseq/1-8.1.ec}{}{}{}{}
    and
    \ecinput{examps/parts/tactics/conseq/1-8.2.ec}{}{}{}{}
    and
    \ecinput{examps/parts/tactics/conseq/1-8.3.ec}{}{}{}{}
    If the current goal is
    \ecinput{examps/parts/tactics/conseq/1-9.0.ec}{}{}{}{} then
    running \ecinput{examps/parts/tactics/conseq/1-9.ec}{}{}{}{}
    produces the goals
    \ecinput{examps/parts/tactics/conseq/1-9.1.ec}{}{}{}{}
    and
    \ecinput{examps/parts/tactics/conseq/1-9.2.ec}{}{}{}{}
    and
    \ecinput{examps/parts/tactics/conseq/1-9.3.ec}{}{}{}{}

    Given lemma
    \ecinput{examps/tactics/conseq/1.ec}{}{102-103}{}
    if the current goal is
    \ecinput{examps/parts/tactics/conseq/1-10.0.ec}{}{}{}{} then
    running \ecinput{examps/parts/tactics/conseq/1-10.ec}{}{}{}{}
    produces the goals
    \ecinput{examps/parts/tactics/conseq/1-10.1.ec}{}{}{}{}
    and
    \ecinput{examps/parts/tactics/conseq/1-10.2.ec}{}{}{}{}
  \end{tsyntax}

  \begin{tsyntax}{conseq (_ : $\;P$ ==> $\;Q$) (_ : $\;P_1$ ==>
      $\;Q_1$) (_ : $\;P_2$ ==> $\;Q_2$)}

    This form only applies to \prhl judgements and statement
    judgements, reducing the goal's conclusion to
    \begin{itemize}
    \item an \hl (statement) judgement for the left program with
      precondition $P_1$ and postcondition $Q_1$;

    \item an \hl (statement) judgement for the right program with
      precondition $P_2$ and postcondition $Q_2$; and

    \item a \prhl (statement) judgement whose precondition is $P$,
      postcondition is $Q$, and programs are as in the original
      judgement;

    \end{itemize}
    As before, auxiliary goals are generated whose conclusions
    check the validity of the reduction: that the original
    conclusion's precondition $P'$ implies the conjunction of
    $P$, $P_1$ and $P_2$; and that the conjunction of
    $Q$, $Q_1$ and $Q_2$ implies the original conclusion's
    postcondition $Q'$. One of the \hl specifications may be
    replaced by \ec{_}, which is equivalent to
    \ec{conseq (_ : true ==> true)}. And in the case of
    a \prhl judgement (not a statement judgement), proof
    terms may be used as the arguments to \ec{conseq}.

    \medskip For example, if the current goal is
    \ecinput{examps/parts/tactics/conseq/2-1.0.ec}{}{}{}{} then
    running \ecinput{examps/parts/tactics/conseq/2-1.ec}{}{}{}{}
    produces the goals
    \ecinput{examps/parts/tactics/conseq/2-1.1.ec}{}{}{}{}
    and
    \ecinput{examps/parts/tactics/conseq/2-1.2.ec}{}{}{}{}
    and
    \ecinput{examps/parts/tactics/conseq/2-1.3.ec}{}{}{}{}
    and
    \ecinput{examps/parts/tactics/conseq/2-1.4.ec}{}{}{}{}
    and
    \ecinput{examps/parts/tactics/conseq/2-1.5.ec}{}{}{}{}
    If the current goal is
    \ecinput{examps/parts/tactics/conseq/2-2.0.ec}{}{}{}{} then
    running \ecinput{examps/parts/tactics/conseq/2-2.ec}{}{}{}{}
    produces the goals
    \ecinput{examps/parts/tactics/conseq/2-2.1.ec}{}{}{}{}
    and
    \ecinput{examps/parts/tactics/conseq/2-2.2.ec}{}{}{}{}
    and
    \ecinput{examps/parts/tactics/conseq/2-2.3.ec}{}{}{}{}
    and
    \ecinput{examps/parts/tactics/conseq/2-2.4.ec}{}{}{}{}
    and
    \ecinput{examps/parts/tactics/conseq/2-2.5.ec}{}{}{}{}

    And given lemmas
    \ecinput{examps/tactics/conseq/2.ec}{}{60-60}{}
    and
    \ecinput{examps/tactics/conseq/2.ec}{}{63-63}{}
    and
    \ecinput{examps/tactics/conseq/2.ec}{}{66-66}{}
    if the current goal is
    \ecinput{examps/parts/tactics/conseq/2-3.0.ec}{}{}{}{} then
    running \ecinput{examps/parts/tactics/conseq/2-3.ec}{}{}{}{}
    produces the goals
    \ecinput{examps/parts/tactics/conseq/2-3.1.ec}{}{}{}{}
    and
    \ecinput{examps/parts/tactics/conseq/2-3.2.ec}{}{}{}{}
  \end{tsyntax}
\end{tactic}

%%  \paragraph{Examples:}\strut
%%  
%%  \begin{cmathpar}
%%  \texample[\prhl{}]
%%    {\ec{conseq (_: P ==> Q)}}
%%    {P' \Rightarrow P \\ Q \Rightarrow Q' \\ \pRHL{P}{c}{c'}{Q}}
%%    {\pRHL{P'}{c}{c'}{Q'}}
%%
%%  \texample[\phl{}]
%%    {\ec{conseq (_: P ==> Q: $\;\delta$)}}
%%    {P' \Rightarrow \delta \mathrel{\diamond} \delta' \\
%%     P' \Rightarrow P \\
%%     Q \mathrel{\diamond^\uparrow} Q' \\
%%     \pHL{P}{c}{Q}{\diamond}{\delta}}
%%    {\pHL{P'}{c}{Q'}{\diamond}{\delta'}}
%%
%%  \texample[\hl{}]
%%    {\ec{conseq (_: P ==> Q)}}
%%    {P' \Rightarrow P \\ Q \Rightarrow Q' \\ \HL{P}{c}{Q}}
%%    {\HL{P'}{c}{Q'}}
%%  \end{cmathpar}
%%
%%  \textbf{Note:} The \phl variant can also be used to strengthen the
%%  relation $\diamond$ into an equality by forcing the equality into
%%  the specification. For example, the following is a valid application
%%  of \ec{conseq}.
%%
%%  \begin{cmathpar}
%%  \texample[\phl{}]
%%    {\ec{conseq (_: P ==> Q: =$\;\delta$)}}
%%    {P' \Rightarrow \delta \mathrel{=} \delta' \\
%%     P' \Rightarrow P \\
%%     Q \mathrel{\Leftrightarrow} Q' \
%%     \pHL{P}{c}{Q}{=}{\delta}}
%%    {\pHL{P'}{c}{Q'}{\leq}{\delta'}}
%%  \end{cmathpar}  
%%  \end{tsyntax}
%%
%%  \begin{tsyntax}{conseq H}
%%  Only applies to judgments on procedures. Same as \ec{conseq} with a
%%  specification, but the specification to use is inferred from the
%%  lemma \ec{H} provided. Raises an error if the lemma does not refer
%%  to the expected procedure(s). All variants of \ec{conseq} may take
%%  lemmas in place of explicit specifications with the same effect, in
%%  which case they must be applied to judgments on procedures.
%%  \end{tsyntax}
%%
%%  \begin{tsyntax}{conseq*}
%%  Same as \ec{conseq}, but the subgoal corresponding to the
%%  postcondition is refined by a ``may modify'' analysis. All variants
%%  of \ec{conseq} can be refined using the \ec{*}, with the same
%%  effect.
%%  \end{tsyntax}
%%
%%  \begin{tsyntax}{conseq <prhl> <hl> <hl>}
%%  Combine relational and non-relational specifications to prove a
%%  relational specification. Either one of the Hoare logic
%%  specifications can be replaced with a wildcard.
%%
%%  \paragraph{Examples:}\strut
%%
%%  \begin{cmathpar}
%%  \texample[\prhl{}]
%%    {\ec{conseq (_: P ==> Q) (_: P$_1$ ==> Q$_1$) (_: P$_2$ ==> Q$_2$)}}
%%    {P' \Rightarrow P \wedge \inmem{P_1}{1} \wedge \inmem{P_2}{2} \\
%%     Q \wedge \inmem{Q_1}{1} \wedge \inmem{Q_2}{2} \Rightarrow Q' \\
%%     \HL{P_1}{c_1}{Q_1} \\
%%     \HL{P_2}{c_2}{Q_2} \\
%%     \pRHL{P}{c_1}{c_2}{Q}}
%%    {\pRHL{P'}{c_1}{c_2}{Q'}}
%%  \end{cmathpar}
%%  \end{tsyntax}
%%
%%  \begin{tsyntax}{conseq <prhl> <phl> | conseq <prhl> _ <phl>}
%%  Strengthen a relational specification where one of the programs is
%%  empty into a non-relational specification about the non-empty
%%  program. Fails if the program expected to be empty is not. For
%%  uniformity and simplicity of use, this variant also allows the user
%%  to strengthen the \prhl judgment before abstracting its relational
%%  aspects. The general rule below can be understood more clearly when
%%  $P = P'$ and $Q = Q'$.
%%
%%  \paragraph{Examples:}\strut
%%
%%  \begin{cmathpar}
%%  \texample[\prhl{}]
%%    {\ec{conseq (_: P ==> Q) (_: P$_1$ ==> Q$_1$: =1\%r)}}
%%    {P' \Rightarrow P   \\ Q   \Rightarrow Q' \\
%%     P  \Rightarrow P_1 \\ Q_1 \Rightarrow Q  \\
%%     \pHL{P_1}{c}{Q_1}{=}{1}}
%%    {\pRHL{P'}{c}{\Skip}{Q'}}
%%
%%  \texample[\prhl{}]
%%    {\ec{conseq (_: _ ==> _) _ (_: P$_2$ ==> Q$_2$: =1\%r)}}
%%    {P \Rightarrow P_2 \\ Q_2 \Rightarrow Q \\ \pHL{P_2}{c}{Q_2}{=}{1}}
%%    {\pRHL{P}{\Skip}{c}{Q}}
%%  \end{cmathpar}
%%  \end{tsyntax}
