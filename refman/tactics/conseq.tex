% --------------------------------------------------------------------
\begin{tactic}{conseq}
  \begin{tsyntax}{conseq <specification>}
  Rule of consequence. Proves a specification by weakening of a
  stronger result. Any one of the specification places can be filled
  with a wildcard \tct{_} to keep the value it contains in the current
  goal and trivially discharge the corresponding subgoal.

  \textbf{Examples:} In the following, $\leq^\uparrow$ (resp. $=^\uparrow$,
  $\geq^\uparrow$) is $\Leftarrow$ (resp. $\Leftrightarrow$ and
  $\Rightarrow$).
  \begin{mathpar}
  \inferrule*[left=(\prhl),rightskip=10em]%%
    {P' \Rightarrow P \\%
     Q \Rightarrow Q' \\%
     \pRHL{P}{c}{c'}{Q}}%%
    {\pRHL{P'}{c}{c'}{Q'}}%%
    \quad\raisebox{.7em}{\tct{conseq (_: P ==> Q)}} \\
  \inferrule*[left=(\prhl),rightskip=10em]%%
    {Q \Rightarrow Q' \\%
     \pRHL{P'}{c}{c'}{Q}}%%
    {\pRHL{P'}{c}{c'}{Q'}}%%
    \quad\raisebox{.7em}{\tct{conseq (_: _ ==> Q)}} \\
  \inferrule*[left=(\phl),rightskip=10em]%%
    {P' \Rightarrow \delta \mathrel{\diamond} \delta' \\%
     P' \Rightarrow P \\%
     Q \mathrel{\diamond^\uparrow} Q' \\%
     \pHL{P}{c}{Q}{\diamond}{\delta}}%%
    {\pHL{P'}{c}{Q'}{\diamond}{\delta'}}%%
    \quad\raisebox{.7em}{\tct{conseq (_: P ==> Q: $\delta$)}} \\
  \inferrule*[left=(\hl),rightskip=10em]%%
    {P' \Rightarrow P \\%
     Q \Rightarrow Q' \\%
     \HL{P}{c}{Q}}%%
    {\HL{P'}{c}{Q'}}%%
    \quad\raisebox{.7em}{\tct{conseq (_: P ==> Q)}} \\
  \end{mathpar}
  \end{tsyntax}

  \begin{tsyntax}{conseq <lemma>}
  Only applies to judgments on procedures. Same as \tct{conseq
  <specification>}, but the specification to use is inferred from
  the lemma provided. Raises an error if the lemma does not refer to
  the expected procedure(s). All variants of \tct{conseq} may take
  lemmas in place of explicit specifications with the same effect, in
  which case they must be applied to judgments on procedures.
  \end{tsyntax}

  \begin{tsyntax}{conseq* <spec>}
  Same as \tct{conseq}, but the subgoal corresponding to the
  postcondition is refined by a ``may modify'' analysis. All variants
  of \tct{conseq} can be refined using the \tct{*}, with the same
  effect.
  \end{tsyntax}

  \begin{tsyntax}{conseq <prhl> <hl> <hl>}
  Combine relational and non-relational specifications to prove a
  relational specification. Either one of the Hoare logic
  specifications can be replaced with a wildcard.

  \textbf{Examples:}
  \begin{mathpar}
  \inferrule*[left=(\prhl),rightskip=5em]%%
    {P' \Rightarrow P \wedge P_1\{1\} \wedge P_2\{2\} \\%
     Q \wedge Q_1\{1\} \wedge Q_2\{2\} \Rightarrow Q' \\%
     \HL{P_1}{c_1}{Q_1} \\%
     \HL{P_2}{c_2}{Q_2} \\%
     \pRHL{P}{c_1}{c_2}{Q}}%%
    {\pRHL{P'}{c_1}{c_2}{Q'}}%%
    \quad\raisebox{.7em}{\tct{conseq (_: P ==> Q) (_: P$_1$ ==> Q$_1$) (_: P$_2$ ==> Q$_2$)}}
  \end{mathpar}
  \end{tsyntax}

  \fix{Missing descriptions of combining variants of conseq}.
\end{tactic}
