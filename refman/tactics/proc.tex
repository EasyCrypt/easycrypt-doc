% --------------------------------------------------------------------
\begin{tactic}{proc}
  \begin{tsyntax}{proc}
    Turn a goal whose conclusion is a \prhl (resp., \phl) judgement
    involving concrete procedures (resp., a concrete procedure)
    into one whose conclusion is a \prhl (resp., \phl) statement
    judgement by replacing the \emph{concrete} procedures (resp.,
    procedure) by their (resp., its) bodies (resp., body).

  \bigskip
  For example, if the current goal is
  \ecinput{../examps/parts/tactics/proc/1-1.0.ec}{}{}{}{} then
  running \ecinput{../examps/parts/tactics/proc/1-1.ec}{}{}{}{}
  produces the goal
  \ecinput{../examps/parts/tactics/proc/1-1.1.ec}{}{}{}{}
  \end{tsyntax}

  \begin{tsyntax}{proc $\;I$}
    Reduce a goal whose conclusion is a \prhl (resp., \phl) judgement
    involving an abstract procedure to \prhl (resp., \phl) judgements
    on the oracles the procedure may query.

  \bigskip
  For example, given the declarations
  \ecinput{examps/tactics/proc/2.ec}{}{3-20}{}
  if the current goal is
  \ecinput{../examps/parts/tactics/proc/2-1.0.ec}{}{}{}{} then
  running \ecinput{../examps/parts/tactics/proc/2-1.ec}{}{}{}{}
  produces the goals
  \ecinput{../examps/parts/tactics/proc/2-1.1.ec}{}{}{}{}
  and
  \ecinput{../examps/parts/tactics/proc/2-1.2.ec}{}{}{}{}
  and
  \ecinput{../examps/parts/tactics/proc/2-1.3.ec}{}{}{}{}
  and
  \ecinput{../examps/parts/tactics/proc/2-1.4.ec}{}{}{}{}
  The tactic would fail without the module restriction \ec{T\{Or\}} on
  \ec{M}, as then \ec{M} could directly manipulate \ec{Or.x}.
  \end{tsyntax}

  \begin{tsyntax}{proc B I}
  Derive a specification for an \emph{abstract} procedure from an
  ``upto-failure'' invariant on the oracles it may query. The failure
  event \ec{B} is evaluated in the right memory. The left oracles
  must be lossless once the bad event occurs. The right oracles must
  guarantee the stability of the failure event with probability 1.
  \end{tsyntax}

  \begin{tsyntax}{proc B I I'}
  Similar to \ec{proc B I}, with an additional invariant once the bad
  event occurs. This is particularly useful when additional facts
  about the state need to be known to prove the losslessness and
  stability conditions.
  \end{tsyntax}

  \begin{tsyntax}{proc*}
  Derive a specification on procedures from a specification on a
  program whose code consists in a call to that procedure. This tactic
  is particularly useful in combination with \rtactic{inline} when
  faced with a \prhl judgment where one of the procedures is concrete
  and the other is abstract.
  \end{tsyntax}
\end{tactic}
