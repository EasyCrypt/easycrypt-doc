% --------------------------------------------------------------------
\begin{tactic}{proc}
  \begin{tsyntax}{proc}
    Turn a goal whose conclusion is a \prhl, \phl or \hl judgement
    involving \emph{concrete} procedure(s) into one whose conclusion
    is a statement judgement by replacing the concrete procedure(s) by
    their body/ies. Assertions about \ec{res}/\ec{res\{$i$\}} are turned
    into ones about the value(s) returned by the procedure(s).

  \bigskip
  For example, if the current goal is
  \ecinput{examps/parts/tactics/proc/1-1.0.ec}{}{}{}{} then
  running \ecinput{examps/parts/tactics/proc/1-1.ec}{}{}{}{}
  produces the goal
  \ecinput{examps/parts/tactics/proc/1-1.1.ec}{}{}{}{}
  \end{tsyntax}

  \begin{tsyntax}{proc $\;I$}
    Reduce a goal whose conclusion is a \prhl judgement involving the
    same \emph{abstract} procedure (but perhaps using different
    implementations of its oracles) (resp., an \hl judgement involving
    an abstract procedure) to goals whose conclusions are \prhl (resp.,
    \hl) judgements on those oracles, plus goals with ambient logic
    conclusions checking the original judgement's pre- and
    postconditions allow such a reduction (the preconditions
    \emph{must} assume $I$ and, in the \prhl case, the equality of the
    abstract procedure's parameter(s) and the global variables of the
    module in which the procedure is contained (except in the case
    when the module's type specifies that the abstract procedure
    initializes its global variables; the postconditions \emph{may}
    assume $I$ and, in the \prhl case, the equality of the results of
    the procedure call(s) and the values of the global variables). The
    generated \prhl/\hl subgoals have pre- and postconditions
    assuming/asserting $I$; in the \prhl case, the preconditions also
    assume the equality of the oracles' parameters, and their
    postconditions also assert the equality of the oracles' results).

  \bigskip
  For example, given the declarations
  \ecinput{examps/tactics/proc/2.ec}{}{3-24}{}
  if the current goal is
  \ecinput{examps/parts/tactics/proc/2-1.0.ec}{}{}{}{} then
  running \ecinput{examps/parts/tactics/proc/2-1.ec}{}{}{}{}
  produces the goals
  \ecinput{examps/parts/tactics/proc/2-1.1.ec}{}{}{}{}
  and
  \ecinput{examps/parts/tactics/proc/2-1.2.ec}{}{}{}{}
  and
  \ecinput{examps/parts/tactics/proc/2-1.3.ec}{}{}{}{}
  and
  \ecinput{examps/parts/tactics/proc/2-1.4.ec}{}{}{}{}
  The tactic would fail without the module restriction \ec{T\{Or\}} on
  \ec{M}, as then \ec{M} could directly manipulate \ec{Or.x}.
  It would also fail if, in the declaration of the module type \ec{T},
  \ec{g} were given access to \ec{O.f3}.
  \end{tsyntax}

  \begin{tsyntax}{proc $\;B$ $\;I$}
    Like \ec{proc $\;I$}, but just for \prhl judgements and uses
    ``upto-bad'' (upto-failure) reasoning, where the \emph{bad}
    (failure) event, $B$, is evaluated in the second program's memory,
    and the invariant $I$ only holds up to the point when failure
    occurs.  In addition to subgoals whose conclusions are \prhl
    judgments involving the oracles the abstract procedure may query
    (their preconditions assume $I$ and the equality of oracles'
    parameters, as well as that $B$ is false; their
    postconditions assert $I$ and the equality of the oracles'
    results---but only when $B$ does not hold), subgoals are generated
    that check that: the original judgement's pre- and postconditions
    support the reduction; the abstract procedure is lossless,
    assuming the losslessness of the oracles it may query; the oracles
    used by the abstract procedure in the first program are lossless
    once the bad event occurs; and the oracles used by the abstract
    procedure in the second program guarantee the stability of the
    failure event with probability 1.

  \bigskip
  For example, suppose we have the following declarations
  \ecinput{examps/tactics/proc/3.ec}{}{3-40}{}
  Then, if the current goal is
  \ecinput{examps/parts/tactics/proc/3-1.0.ec}{}{}{}{}
  running \ecinput{examps/parts/tactics/proc/3-1.ec}{}{}{}{}
  produces the goals
  \ecinput{examps/parts/tactics/proc/3-1.1.ec}{}{}{}{}
  and
  \ecinput{examps/parts/tactics/proc/3-1.2.ec}{}{}{}{}
  and
  \ecinput{examps/parts/tactics/proc/3-1.3.ec}{}{}{}{}
  and
  \ecinput{examps/parts/tactics/proc/3-1.4.ec}{}{}{}{}
  and
  \ecinput{examps/parts/tactics/proc/3-1.5.ec}{}{}{}{}
  and
  \ecinput{examps/parts/tactics/proc/3-1.6.ec}{}{}{}{}
  \end{tsyntax}

  \begin{tsyntax}{proc $\;B$ $\;I$ $\;J$}
    Like \ec{proc $\;B$ $\;I$}, but where the extra invariant, $J$,
    holds \emph{after} failure has occurred.  In the \prhl subgoals
    involving oracles called by the abstract procedure: the
    preconditions assume $I$ and the equality of the oracles'
    parameters, as well as that $B$ is
    false; and the postconditions assert
    \begin{itemize}
    \item $I$ and the equality of the oracles' results---when $B$ does
      not hold; and

    \item $J$---when $B$ does hold.
    \end{itemize}

  \medskip
  For example, given the declarations of the \ec{proc $\;B$ $\;I$} example,
  if the current goal is
  \ecinput{examps/parts/tactics/proc/3-2.0.ec}{}{}{}{} then
  running \ecinput{examps/parts/tactics/proc/3-2.ec}{}{}{}{}
  produces the goals
  \ecinput{examps/parts/tactics/proc/3-2.1.ec}{}{}{}{}
  and
  \ecinput{examps/parts/tactics/proc/3-2.2.ec}{}{}{}{}
  and
  \ecinput{examps/parts/tactics/proc/3-2.3.ec}{}{}{}{}
  and
  \ecinput{examps/parts/tactics/proc/3-2.4.ec}{}{}{}{}
  and
  \ecinput{examps/parts/tactics/proc/3-2.5.ec}{}{}{}{}
  and
  \ecinput{examps/parts/tactics/proc/3-2.6.ec}{}{}{}{}
  \end{tsyntax}

  \begin{tsyntax}{proc*}
    Reduce a \prhl (resp., \hl) judgement to a \prhl (resp., \hl)
    statement judgement involving calls (resp., a call) to the
    procedures (resp., procedure).

  \bigskip
  For example, if the current goal is
  \ecinput{examps/parts/tactics/proc/4-1.0.ec}{}{}{}{} then
  running \ecinput{examps/parts/tactics/proc/4-1.ec}{}{}{}{}
  produces the goal
  \ecinput{examps/parts/tactics/proc/4-1.1.ec}{}{}{}{}

  \paragraph{Remark.}
  This tactic is particularly useful in combination with
  \rtactic{inline} when faced with a \prhl judgment where one of the
  procedures is concrete and the other is abstract.
  \end{tsyntax}
\end{tactic}
