% --------------------------------------------------------------------
\begin{tactic}{transitivity}
  \begin{tsyntax}{transitivity $\;N$.$r$ ($P_1$ ==> $\ Q_1$) ($P_2$ ==> $\ Q_2$)}
    Reduces a goal whose conclusion is a \prhl judgement
    (\emph{not} statement judgement)
    \begin{center}
      \ec{equiv[$M_1$.$p_1$ ~ $\;M_2$.$p_2$ : $\;P$ ==> $\;Q$]}
    \end{center}
    to goals whose conclusions are
    \begin{itemize}
    \item \ec{equiv[$M_1$.$p_1$ ~ $\;N$.$r$ : $\;P_1$ ==> $\;Q_1$]} and
    \item \ec{equiv[$N$.$r$ ~ $\;M_2$.$p_2$ : $\;P_2$ ==> $\;Q_2$]},
    \end{itemize}
    preceded by two auxiliary goals. The tactic fails if the $P_i$ and
    $Q_i$ aren't compatible with these left and right programs.  The
    conclusion of the first auxiliary goal checks that $P$ implies the
    conjunction of $P_1$ and $P_2$, where each corresponding pair of
    \ec{&2} memory references in $P_1$ and \ec{&1} reference in $P_2$
    is existentially quantified.  And the conclusion of the second
    auxilary goal checks that the conjuction of $Q_1$ and $Q_2$ implies
    $Q$, where each corresponding pair of \ec{&2} references in $Q_1$
    and \ec{&1} references in $Q_2$ are universally quantified.

  \medskip For example, consider the modules
  \ecinput{examps/tactics/transitivity/1.ec}{}{6-30}{}
  If the current goal is
  \ecinput{examps/parts/tactics/transitivity/1-1.0.ec}{}{}{}{} then
  running \ecinput{examps/parts/tactics/transitivity/1-1.ec}{}{}{}{}
  produces the four goals
  \ecinput{examps/parts/tactics/transitivity/1-1.1.ec}{}{}{}{}
  and
  \ecinput{examps/parts/tactics/transitivity/1-1.2.ec}{}{}{}{}
  and
  \ecinput{examps/parts/tactics/transitivity/1-1.3.ec}{}{}{}{}
  and
  \ecinput{examps/parts/tactics/transitivity/1-1.4.ec}{}{}{}{}
  \end{tsyntax}

  \begin{tsyntax}{transitivity\{$i$\} \{$s$\} ($P_1$ ==> $\ Q_1$) ($P_2$ ==> $\ Q_2$)}
    
    Reduces a goal whose conclusion is a \prhl \emph{statement}
    judgement with precondition $P$, postcondition $Q$, left program
    (statement sequence) $s_1$ and right program $s_2$ to goals whose
    conclusions are \prhl statement judgements:
    \begin{itemize}
    \item \ec{equiv[$s_1$ ~ $\;s$ : $\;P_1$ ==> $\;Q_1$]} and
    \item \ec{equiv[$s$ ~ $\;s_2$ : $\;P_2$ ==> $\;Q_2$]},
    \end{itemize}
    preceded by two auxiliary goals. If the side $i=1$, then the
    statement sequence $s$ may only use variables and unqualified
    procedures of the left program; when $i=2$, it may only use
    variables and unqualified procedures of the right program. The
    tactic fails if the $P_i$ and $Q_i$ aren't compatible with these
    left and right programs.  The conclusion of the first auxiliary
    goal checks that $P$ implies the conjunction of $P_1$ and $P_2$,
    where each corresponding pair of \ec{&2} memory references in
    $P_1$ and \ec{&1} reference in $P_2$ is existentially quantified.
    And the conclusion of the second auxilary goal checks that the
    conjuction of $Q_1$ and $Q_2$ implies $Q$, where each
    corresponding pair of \ec{&2} references in $Q_1$ and \ec{&1}
    references in $Q_2$ are universally quantified.

  \medskip Consider the modules \ec{M}, \ec{N} and \ec{R} of the
  preceding case.  If the current goal is
  \ecinput{examps/parts/tactics/transitivity/1-2.0.ec}{}{}{}{} then
  running \ecinput{examps/parts/tactics/transitivity/1-2.ec}{}{}{}{}
  produces the four goals
  \ecinput{examps/parts/tactics/transitivity/1-2.1.ec}{}{}{}{} and
  \ecinput{examps/parts/tactics/transitivity/1-2.2.ec}{}{}{}{} and
  \ecinput{examps/parts/tactics/transitivity/1-2.3.ec}{}{}{}{} and
  \ecinput{examps/parts/tactics/transitivity/1-2.4.ec}{}{}{}{} And, if
  the current goal is
  \ecinput{examps/parts/tactics/transitivity/1-3.0.ec}{}{}{}{} then
  running \ecinput{examps/parts/tactics/transitivity/1-3.ec}{}{}{}{}
  produces the four goals
  \ecinput{examps/parts/tactics/transitivity/1-3.1.ec}{}{}{}{} and
  \ecinput{examps/parts/tactics/transitivity/1-3.2.ec}{}{}{}{} and
  \ecinput{examps/parts/tactics/transitivity/1-3.3.ec}{}{}{}{} and
  \ecinput{examps/parts/tactics/transitivity/1-3.4.ec}{}{}{}{}
  \end{tsyntax}
\end{tactic}

%%  \paragraph{Examples:}\strut
%%
%%  \begin{cmathpar}
%%  \texample[\prhl{}]{\ec{transitivity f ($P_1$ ==> $\ Q_1$) ($P_2$ ==> $\ Q_2$)}}
%%    {\forall \mem{m_1}\ \mem{m_2}.\, \mem{m_1} \rel{P} \mem{m_2} \Rightarrow
%%        \exists \mem{m}.\, \mem{m_1} \rel{P_1} \mem{m}
%%                           \wedge \mem{m} \rel{P_2} \mem{m_2} \\
%%     \forall \mem{m_1}\ \mem{m}\ \mem{m_2}.\,
%%        \mem{m_1} \rel{Q_1} \mem{m} \Rightarrow
%%        \mem{m}   \rel{Q_2} \mem{m_2} \Rightarrow
%%        \mem{m_1} \rel{Q}   \mem{m_2} \\
%%     \pRHL{P_1}{f_1}{f}{Q_1} \\
%%     \pRHL{P_2}{f}{f_2}{Q_2}}
%%    {\pRHL{P}{f_1}{f_2}{Q}}
%%
%%  \texample[\prhl{}]
%%    {\ec{transitivity \{1\} \{ s \} ($P_1$ ==> $\ Q_1$) ($P_2$ ==> $\ Q_2$)}}
%%    {\forall \mem{m_1}\ \mem{m_2}.\, \mem{m_1} \rel{P} \mem{m_2} \Rightarrow
%%        \exists \mem{m}.\, \mem{m_1} \rel{P_1} \mem{m}
%%                           \wedge \mem{m} \rel{P_2} \mem{m_2} \\
%%     \forall \mem{m_1}\ \mem{m}\ \mem{m_2}.\,
%%        \mem{m_1} \rel{Q_1} \mem{m} \Rightarrow
%%        \mem{m}   \rel{Q_2} \mem{m_2} \Rightarrow
%%        \mem{m_1} \rel{Q}   \mem{m_2} \\
%%     \pRHL{P_1}{s_1}{s}{Q_1} \\
%%     \pRHL{P_2}{s}{s_2}{Q_2}}
%%    {\pRHL{P}{s_1}{s_2}{Q}}
%%  \end{cmathpar}
%%
%%  \textbf{Note:} In practice, the existential quantification over
%%  memory $\mem{m}$ in the first generated subgoal is replaced with an
%%  existential quantification over the program variables appearing in $P$,
%%  $P_1$, or $P_2$.
%%  \end{tsyntax}
