% --------------------------------------------------------------------
\begin{tactic}[case $\;\phi$ | case]{case}
  \begin{tsyntax}{case $\;\phi$}
    Do an excluded-middle case analysis on $\phi$, substituting $\phi$
    in the goal's conclusion.

    For example, if the current goal is
    \ecinput{../examps/parts/tactics/case/1-1.0.ec}{}{}{}{} then
    running \ecinput{../examps/parts/tactics/case/1-1.ec}{}{}{}{}
    produces the goals
    \ecinput{../examps/parts/tactics/case/1-1.1.ec}{}{}{}{}
    and
    \ecinput{../examps/parts/tactics/case/1-1.2.ec}{}{}{}{}
  \end{tsyntax}

  \begin{tsyntax}{case}
    Destruct the top assumption of the goal's conclusion, generating
    subgoals that are dependent upon the kind of assumption
    destructed. \emph{This form of the tactic can be followed by
    the generalization tactical---see Subsection~\ref{subsec:introgen}.}

    \begin{itemize}
    \item (\textbf{conjunction})
    For example, if the current goal is
    \ecinput{../examps/parts/tactics/case/2-1.0.ec}{}{}{}{} then
    running \ecinput{../examps/parts/tactics/case/2-1.ec}{}{}{}{}
    produces the goal
    \ecinput{../examps/parts/tactics/case/2-1.1.ec}{}{}{}{}
    \ec{&&} works identically.

    \item (\textbf{disjunction})
    For example, if the current goal is
    \ecinput{../examps/parts/tactics/case/3-1.0.ec}{}{}{}{} then
    running \ecinput{../examps/parts/tactics/case/3-1.ec}{}{}{}{}
    produces the goals
    \ecinput{../examps/parts/tactics/case/3-1.1.ec}{}{}{}{}
    and
    \ecinput{../examps/parts/tactics/case/3-1.2.ec}{}{}{}{}
    \ec{||} works identically.

    \item (\textbf{existential})
    For example, if the current goal is
    \ecinput{../examps/parts/tactics/case/4-1.0.ec}{}{}{}{} then
    running \ecinput{../examps/parts/tactics/case/4-1.ec}{}{}{}{}
    produces the goal
    \ecinput{../examps/parts/tactics/case/4-1.1.ec}{}{}{}{}

    \item (\textbf{unit}) Substitutes \ec{tt} for the assumption in
      the remainder of the conclusion.

    \item (\textbf{bool})
    For example, if the current goal is
    \ecinput{../examps/parts/tactics/case/5-1.0.ec}{}{}{}{} then
    running \ecinput{../examps/parts/tactics/case/5-1.ec}{}{}{}{}
    produces the goals
    \ecinput{../examps/parts/tactics/case/5-1.1.ec}{}{}{}{}
    and
    \ecinput{../examps/parts/tactics/case/5-1.2.ec}{}{}{}{}

    \item (\textbf{product type})
    For example, if the current goal is
    \ecinput{../examps/parts/tactics/case/6-1.0.ec}{}{}{}{} then
    running \ecinput{../examps/parts/tactics/case/6-1.ec}{}{}{}{}
    produces the goal
    \ecinput{../examps/parts/tactics/case/6-1.1.ec}{}{}{}{}

    \item (\textbf{inductive datatype})
    Consider the inductive datatype declaration:
\begin{easycrypt}{}{}
type tree = [Leaf | Node of bool & tree & tree].
\end{easycrypt}
    Then, if the current goal is
    \ecinput{../examps/parts/tactics/case/7-1.0.ec}{}{}{}{} then
    running \ecinput{../examps/parts/tactics/case/7-1.ec}{}{}{}{}
    produces the goals
    \ecinput{../examps/parts/tactics/case/7-1.1.ec}{}{}{}{}
    and
    \ecinput{../examps/parts/tactics/case/7-1.2.ec}{}{}{}{}
    \end{itemize}
  \end{tsyntax}
\end{tactic}
