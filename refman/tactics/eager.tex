% --------------------------------------------------------------------
\begin{tactic}{eager}
  \ec{eager} is a family of tactics for proving \prhl \emph{statement}
  judgements of the form
  \begin{center}
    \ec{equiv[$s_1$ $\;t_1$ ~ $\;t_2$ $\;s_2$ : $\;P$ ==> $\;Q$]},
  \end{center}
  where the pre- and postconditions $P$ and $Q$ are conjunctions of
  equalities between program variables, and the statement sequences
  $s_i$ only read and write global variables.  Here $s_1$ is in the
  ``eager'' position, and its replacement, $s_2$, is in the ``lazy''
  position. Some of the tactics work with \emph{eager judgements} of
  the form
  \begin{center}
    \ec{eager[$s_1$, $\;M$.$p$ ~ $\;N$.$q$, $\;s_2$ : $\;P$ ==> $\;Q$]},
  \end{center}
  where, again, the $s_i$ only involve global variables, but where
  $P$ and $Q$ may talk about the parameters and results of
  \ec{$M$.$p$} and \ec{$N$.$q$} in the usual way.

  \medskip The context of our examples is the following \EasyCrypt
  script involving variable incrementation oracles \ec{Or1} and
  \ec{Or2}, which only differ in that \ec{Or2} keeps a ``transcript''
  of its operation.
  \ecinput{examps/tactics/eager/1-nodumps.ec}{}{5-200}{}

  \begin{tsyntax}{eager proc}
  Turn a goal whose conclusion is an eager judgement into one
  whose conclusion is a \prhl statement judgement in which
  the eager judgement's procedures have been replaced by their
  bodies.
  \medskip
  For example, if the current goal is
  \ecinput{examps/parts/tactics/eager/1-1.0.ec}{}{}{}{} then
  running \ecinput{examps/parts/tactics/eager/1-1.ec}{}{}{}{}
  produces the goal
  \ecinput{examps/parts/tactics/eager/1-1.1.ec}{}{}{}{}
  \end{tsyntax}

  \begin{tsyntax}{proc*}
  Turn a goal whose conclusion is an eager judgement into one
  whose conclusion is a \prhl statement judgement in which
  the eager judgement's procedures are called, as opposed to
  being inlined.
  \medskip
  For example, if the current goal is
  \ecinput{examps/parts/tactics/eager/1-2.0.ec}{}{}{}{} then
  running \ecinput{examps/parts/tactics/eager/1-2.ec}{}{}{}{}
  produces the goal
  \ecinput{examps/parts/tactics/eager/1-2.1.ec}{}{}{}{}
  \end{tsyntax}

  \begin{tsyntax}{eager call $\;p$}
    Here $p$ is a proof term for an eager judgement. If the goal's
    conclusion is a \prhl statement judgement whose programs' suffixes
    match the left and right sides of the eager judgment, then consume
    those suffixes.

  \medskip
  For example, if the current goal is
  \ecinput{examps/parts/tactics/eager/1-4.0.ec}{}{}{}{} then
  running \ecinput{examps/parts/tactics/eager/1-4.ec}{}{}{}{}
  (see the statement of \ec{eager_incr}, above)
  produces the goal
  \ecinput{examps/parts/tactics/eager/1-4.1.ec}{}{}{}{}
  \end{tsyntax}

  \begin{tsyntax}{eager seq $\;n_1$ $\;n_2$ ($H$ : $\;s_1$ ~ $\;s_2$ :
    $\;A$ ==> $\;B$) : $\;C$}

   Here, the goal's conclusion must be a \prhl statement
  judgement whose left and right programs begin and end with $s_1$ and
  $s_2$, respectively. A first subgoal is generated whose conclusion
  is the specified \prhl statement judgement, which is made available
  as $H$ for the subsequent subgoals's use.  The $n_1$ and $n_2$ must
  be natural numbers saying how many statements from the part of the
  first program following $s_1$ and from the beginning of the second
  program to put together with the $s_i$ in a second \prhl statement
  judgement subgoal. The remaining statements are put together with
  the $s_i$ in a third \prhl statement judgement subgoal. And there is
  a final subgoal whose conclusion is a \prhl statement judgment whose
  left and right sides are those remaining statements of the second
  program only.

  \medskip
  For example, if the current goal is
  \ecinput{examps/parts/tactics/eager/1-3.0.ec}{}{}{}{} then
  running \ecinput{examps/parts/tactics/eager/1-3.ec}{}{}{}{}
  produces the goals
  \ecinput{examps/parts/tactics/eager/1-3.1.ec}{}{}{}{}
  and
  \ecinput{examps/parts/tactics/eager/1-3.2.ec}{}{}{}{}
  and
  \ecinput{examps/parts/tactics/eager/1-3.3.ec}{}{}{}{}
  and
  \ecinput{examps/parts/tactics/eager/1-3.4.ec}{}{}{}{}
  \end{tsyntax}

  \begin{tsyntax}{eager if}
    If the goal's conclusion is a \prhl statement judgement whose left
    program consists of $s_1$ followed by a conditional, and whose
    right program consists of a conditional followed by $s_2$, reduce
    the goal to two subgoals using the $s_i$ together with the ``then''
    and ``else'' parts of the conditionals, along with auxiliary subgoals
    verifying that---even after running $s_1$---the conditionals'
    boolean expressions are equivalent.

  \medskip
  For example, if the current goal is
  \ecinput{examps/parts/tactics/eager/1-6.0.ec}{}{}{}{} then
  running \ecinput{examps/parts/tactics/eager/1-6.ec}{}{}{}{}
  produces the goals
  \ecinput{examps/parts/tactics/eager/1-6.1.ec}{}{}{}{}
  and
  \ecinput{examps/parts/tactics/eager/1-6.2.ec}{}{}{}{}
  and
  \ecinput{examps/parts/tactics/eager/1-6.3.ec}{}{}{}{}
  and
  \ecinput{examps/parts/tactics/eager/1-6.4.ec}{}{}{}{}
  \end{tsyntax}

  \begin{tsyntax}{eager while ($H$ : $\;s_1$ ~ $\;s_2$ :
    $\;A$ ==> $\;B$)}
  Like \ec{eager if}, but working with while loops instead of
  conditionals and featuring an explicit, named \prhl statement
  judgement involving the $s_i$, available as $H$ to the subgoals.
  The subgoal involving the bodies of the while loops uses $B$ as its
  invariant. There is also a subgoal whose conclusion is a \prhl
  statement judgement both of whose sides are the body of the second
  program's while loop.

  \medskip
  For example, if the current goal is
  \ecinput{examps/parts/tactics/eager/1-5.0.ec}{}{}{}{} then
  running \ecinput{examps/parts/tactics/eager/1-5.ec}{}{}{}{}
  produces the goals
  \ecinput{examps/parts/tactics/eager/1-5.1.ec}{}{}{}{}
  and
  \ecinput{examps/parts/tactics/eager/1-5.2.ec}{}{}{}{}
  and
  \ecinput{examps/parts/tactics/eager/1-5.3.ec}{}{}{}{}
  and
  \ecinput{examps/parts/tactics/eager/1-5.4.ec}{}{}{}{}
  \end{tsyntax}

  \begin{tsyntax}{eager proc ($H$ : $\;s_1$ ~ $\;s_2$ :
    $\;A$ ==> $\;B$) $\;C$ | eager $\;H$ $\;C$}
  This is the form of \ec{eager proc} that applies when the procedures
  \ec{$M$.$p$} and \ec{$N$.$q$} are abstract.  The second variant is
  where the specified \prhl statement judgement has already been
  introduced by another \ec{eager} tactic. There must be a module
  restriction saying that the abstract procedures can't directly
  interfere with the global variables on which the $s_i$
  depend. Subgoals are generated for each pair of ``oracles'' the
  abstract procedures are capable of calling, i.e., for each of the
  procedures that may read/write the global variables used by the
  $s_i$.

  \medskip
  For example, if the current goal is
  \ecinput{examps/parts/tactics/eager/1-7.0.ec}{}{}{}{} then
  running \ecinput{examps/parts/tactics/eager/1-7.ec}{}{}{}{}
  produces the goals
  \ecinput{examps/parts/tactics/eager/1-7.1.ec}{}{}{}{}
  and
  \ecinput{examps/parts/tactics/eager/1-7.2.ec}{}{}{}{}
  and
  \ecinput{examps/parts/tactics/eager/1-7.3.ec}{}{}{}{}
  and
  \ecinput{examps/parts/tactics/eager/1-7.4.ec}{}{}{}{}
  and
  \ecinput{examps/parts/tactics/eager/1-7.5.ec}{}{}{}{}
  and
  \ecinput{examps/parts/tactics/eager/1-7.6.ec}{}{}{}{}
  and
  \ecinput{examps/parts/tactics/eager/1-7.7.ec}{}{}{}{}
  and
  \ecinput{examps/parts/tactics/eager/1-7.8.ec}{}{}{}{}
  and
  \ecinput{examps/parts/tactics/eager/1-7.9.ec}{}{}{}{}
  and
  \ecinput{examps/parts/tactics/eager/1-7.10.ec}{}{}{}{}
  and
  \ecinput{examps/parts/tactics/eager/1-7.11.ec}{}{}{}{}
  and
  \ecinput{examps/parts/tactics/eager/1-7.12.ec}{}{}{}{}
  and
  \ecinput{examps/parts/tactics/eager/1-7.13.ec}{}{}{}{}
  \end{tsyntax}

  \begin{tsyntax}{eager ($H$ : $\;s_1$ ~ $\;s_2$ :
    $\;A$ ==> $\;B$) : $\;C$}
   Reduces a goal whose conclusion is a \prhl statement judgement of
   the form
  \begin{center}
    \ec{equiv[$s_1$ $\;t_1$ ~ $\;t_2$ $\;s_2$ : $\;P$ ==> $\;Q$]},
  \end{center}
  to an eager judgement, along with a subgoal for the specified
  \prhl statement judgement $H$, plus an auxiliary goal.

  \medskip
  For example, if the current goal is
  \ecinput{examps/parts/tactics/eager/1-8.0.ec}{}{}{}{} then
  running \ecinput{examps/parts/tactics/eager/1-8.ec}{}{}{}{}
  produces the goals
  \ecinput{examps/parts/tactics/eager/1-8.1.ec}{}{}{}{}
  and
  \ecinput{examps/parts/tactics/eager/1-8.2.ec}{}{}{}{}
  and
  \ecinput{examps/parts/tactics/eager/1-8.3.ec}{}{}{}{}
  \end{tsyntax}
\end{tactic}
