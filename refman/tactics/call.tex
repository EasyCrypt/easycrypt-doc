% --------------------------------------------------------------------
\begin{tactic}{call}
  \begin{tsyntax}{call (_ : $\;P$ ==> $\;Q$)}
    If the goal's conclusion is a \prhl or \hl statement judgement
    whose program(s) end(s) with (a) procedure call(s) or (a) procedure
    call assignment(s), then generate two subgoals:
  \begin{itemize}
  \item One whose conclusion is a judgement of the same kind whose
    precondition is $P$, whose procedure(s) are/is the procedure(s)
    being called, and whose postcondition is $Q$.

  \item One whose conclusion is a statement judgement of the same
    kind whose precondition is the original goal's precondition,
    whose program(s) are/is the result of removing the procedure
    call(s) from the program(s), and whose postcondition is the
    conjunction of
    \begin{itemize}
    \item the result of replacing the procedure's/procedures'
      parameter(s) by their actual argument(s) in $P$; and

    \item the assertion that, for all values of the global variable(s)
      modified by the procedure(s) and the result(s) of the procedure
      call(s), if $Q$ holds (where these quantified identifiers have
      been substituted for the modified variables and procedure
      results), then the original goal's postcondition holds (where
      the modified global variables and occurrences of the variable(s)
      (if any) to which the result(s) of the call(s) of the
      procedure(s) are/is assigned have been replaced by the
      appropriate quantified identifiers).
    \end{itemize}
  \end{itemize}

  \medskip
  For example, if the current goal is
  \ecinput{../examps/parts/tactics/call/1-1.0.ec}{}{}{}{}
  and the procedures \ec{M.f} and \ec{N.f} have a single parameter,
  \ec{y}, then running
  \ecinput{../examps/parts/tactics/call/1-1.ec}{}{}{}{} produces the
  goals \ecinput{../examps/parts/tactics/call/1-1.1.ec}{}{}{}{} and
  \ecinput{../examps/parts/tactics/call/1-1.2.ec}{}{}{}{}

  \bigskip
  Alternatively, a proof term whose conclusion is a \prhl or
  \hl judgement involving the procedure(s) called at the
  end(s) of the program(s) may be supplied as the argument to
  \ec{call}, in which case only the second subgoal need be
  generated.

  \medskip
  For example, in the start-goal of the preceding example,
  if the lemma \ec{M_N_f} is
  \ecinput{examps/tactics/call/1.ec}{}{31-33}{}
  then running
  \ecinput{../examps/parts/tactics/call/1-2.ec}{}{}{}{} produces the
  goal \ecinput{../examps/parts/tactics/call/1-2.1.ec}{}{}{}{}
  \end{tsyntax}

  \begin{tsyntax}{call\{1\} (_ : $\;P$ ==> $\;Q$) | call\{2\} (_ : $\;P$ ==> $\;Q$)}
    If the goal's conclusion is a \prhl statement judgement whose
    designated program ends with a procedure call, then generate two
    subgoals:
  \begin{itemize}
  \item One whose conclusion is a \phl judgement whose precondition is
    $P$, whose procedure is the procedure being called, whose
    postcondition is $Q$, and whose bound part specifies equality with
    probability 1.
    (Consequently, $P$ and $Q$ may not mention \ec{&1} and
    \ec{&2}.)

  \item One whose conclusion is a \prhl statement judgement whose
    precondition is the original goal's precondition, whose programs
    are the result of removing the procedure call from the designated
    program, and leaving the other program unchanged, and whose
    postcondition is the conjunction of
    \begin{itemize}
    \item the result of replacing the procedure's
      parameter(s) by their actual argument(s) in $P$; and

    \item the assertion that, for all values of the global variable(s)
      modified by the procedure and the result of the procedure call,
      if $Q$ holds (where these quantified identifiers have been
      substituted for the modified variables and procedure result),
      then the original goal's postcondition holds (where the modified
      global variables and occurrences the variable (if any) to which
      the result of the procedure call is assigned have been replaced
      by the appropriate quantified identifiers).
    \end{itemize}
  \end{itemize}

  For example, if the current goal is
  \ecinput{../examps/parts/tactics/call/2-1.0.ec}{}{}{}{}
  then running
  \ecinput{../examps/parts/tactics/call/2-1.ec}{}{}{}{} produces the
  goals \ecinput{../examps/parts/tactics/call/2-1.1.ec}{}{}{}{} and
  \ecinput{../examps/parts/tactics/call/2-1.2.ec}{}{}{}{}

  \bigskip Alternatively, a proof term whose conclusion is a \phl
  judgement specifying equality with probability 1 and involving the
  procedure called at the end of the designated program may be
  supplied as the argument to \ec{call}, in which case only the second
  subgoal need be generated.

  \medskip
  For example, in the start-goal of the preceding example,
  if the lemma \ec{M_f} is
  \ecinput{examps/tactics/call/2.ec}{}{23-24}{}
  then running
  \ecinput{../examps/parts/tactics/call/2-2.ec}{}{}{}{} produces the
  goal \ecinput{../examps/parts/tactics/call/2-2.1.ec}{}{}{}{}
  \end{tsyntax}

  \begin{tsyntax}{call (_ : $\;I$)}
    If the conclusion is a \prhl or \hl statement judgement whose
    program(s) end(s) with (a) call(s) of (a) \emph{concrete}
    procedure(s), then use the specification argument to \ec{call}
    generated from the \emph{invariant} $I$, and automatically apply
    \ec{proc} to its first subgoal.  In the \prhl case, its
    precondition will assume equality of the procedures' parameters,
    and its postcondition will assert equality of the results of the
    procedure calls.

    \medskip
    For example, if the current goal is
    \ecinput{../examps/parts/tactics/call/1-3.0.ec}{}{}{}{}
    and modules \ec{M} and \ec{N} contain
    \ecinput{examps/tactics/call/1.ec}{}{4-8}{} and
    \ecinput{examps/tactics/call/1.ec}{}{18-22}{}
    respectively, then
    running \ecinput{../examps/parts/tactics/call/1-3.ec}{}{}{}{}
    produces the goals
    \ecinput{../examps/parts/tactics/call/1-3.1.ec}{}{}{}{} and
    \ecinput{../examps/parts/tactics/call/1-3.2.ec}{}{}{}{}
  \end{tsyntax}

  \begin{tsyntax}{call (_ : $\;I$)}
    If the conclusion is a \prhl or \hl statement judgement whose
    program(s) end(s) with (a) call(s) of the same \emph{abstract}
    procedure, then use the specification argument to \ec{call}
    generated from the \emph{invariant} $I$, and automatically apply
    \ec{proc $\;I$} to its first subgoal, pruning the first two
    subgoals the application generates, because their conclusions
    consist of ambient logic formulas that are true by construction.
    In the \prhl case, its precondition will assume equality of the
    procedure's parameters and of the global variables of the module
    containing the procedure, and its postcondition will assume
    equality of the results of the procedure calls and of the global
    variables of the containing module.

    \medskip
    For example, given the declarations
    \ecinput{examps/tactics/call/3.ec}{}{3-24}{}
    if the current goal is
    \ecinput{../examps/parts/tactics/call/3-1.0.ec}{}{}{}{}
    then running
    \ecinput{../examps/parts/tactics/call/3-1.ec}{}{}{}{}
    produces the goals
    \ecinput{../examps/parts/tactics/call/3-1.1.ec}{}{}{}{} and
    \ecinput{../examps/parts/tactics/call/3-1.2.ec}{}{}{}{}
  \end{tsyntax}

  \begin{tsyntax}{call (_ : $\;B$, $\;I$)}
    If the conclusion is a \prhl statement judgement whose programs
    end with calls of the same \emph{abstract} procedure, then use the
    specification argument to \ec{call} generated from the \emph{bad
      event} $B$ and \emph{invariant} $I$, and automatically apply
    \ec{proc $\;B$ $\;I$} to its first subgoal, pruning the first two
    subgoals the application generates, because their conclusions
    consist of ambient logic formulas that are true by construction.
    The specification's precondition will assume equality of the
    procedure's parameters and of the global variables of the module
    containing the procedure as well as $I$, and its postcondition
    will assert $I$ and the equality of the results of the procedure
    calls and of the global variables of the containing module---but
    only when $B$ does not hold.

    \medskip
    For example, given the declarations
    \ecinput{examps/tactics/call/4.ec}{}{3-51}{}
    if the current goal is
    \ecinput{../examps/parts/tactics/call/4-1.0.ec}{}{}{}{}
    then running
    \ecinput{../examps/parts/tactics/call/4-1.ec}{}{}{}{}
    produces the goals
    \ecinput{../examps/parts/tactics/call/4-1.1.ec}{}{}{}{} and
    \ecinput{../examps/parts/tactics/call/4-1.2.ec}{}{}{}{} and
    \ecinput{../examps/parts/tactics/call/4-1.3.ec}{}{}{}{} and
    \ecinput{../examps/parts/tactics/call/4-1.4.ec}{}{}{}{}
  \end{tsyntax}

  \begin{tsyntax}{call (_ : $\;B$, $\;I$, $\;J$)}
    If the conclusion is a \prhl statement judgement whose programs
    end with calls of the same \emph{abstract} procedure, then use the
    specification argument to \ec{call} generated from the \emph{bad
      event} $B$ and \emph{invariants} $I$ and $J$, and automatically
    apply \ec{proc $\;B$ $\;I$ $\;J$} to its first subgoal, pruning the
    first two subgoals the application generates, because their
    conclusions consist of ambient logic formulas that are true by
    construction.  The specification's precondition will assume
    equality of the procedure's parameters and of the global variables
    of the module containing the procedure as well as $I$, and its
    postcondition will assert
    \begin{itemize}
    \item $I$ and the equality of the results of the procedure calls
      and of the global variables of the containing module---if $B$
      does not hold; and

    \item $J$---if $B$ does hold.
    \end{itemize}

    \medskip
    For example, given the declarations of the preceding example
    if the current goal is
    \ecinput{../examps/parts/tactics/call/4-2.0.ec}{}{}{}{}
    then running
    \ecinput{../examps/parts/tactics/call/4-2.ec}{}{}{}{}
    produces the goals
    \ecinput{../examps/parts/tactics/call/4-2.1.ec}{}{}{}{} and
    \ecinput{../examps/parts/tactics/call/4-2.2.ec}{}{}{}{} and
    \ecinput{../examps/parts/tactics/call/4-2.3.ec}{}{}{}{} and
    \ecinput{../examps/parts/tactics/call/4-2.4.ec}{}{}{}{}
  \end{tsyntax}
\end{tactic}
