% --------------------------------------------------------------------
\begin{tactic}{seq}
  \begin{tsyntax}{seq $\;n_1$ $\;n_2$ : $\;R$}
    \textbf{\prhl sequence rule.} If $n_1$ and $n_2$ are natural
    numbers and the goal's conclusion is a \prhl statement judgement
    with precondition $P$, postcondition $Q$ and such that the lengths
    of the first and second programs are at least $n_1$ and $n_2$,
    respectively, then reduce the goal to two subgoals:
    \begin{itemize}
    \item A first goal whose conclusion has precondition $P$,
      postcondition $R$, first program consisting of the first $n_1$
      statements of the original goal's first program, and second
      program consisting of the first $n_2$ statements of the original
      goal's second program.

    \item A second goal whose conclusion has precondition $R$,
      postcondition $Q$, first program consisting of all but the first
      $n_1$ statements of the original goal's first program, and
      second program consisting of all but the first $n_2$ statements
      of the original goal's second program.
    \end{itemize}

  \bigskip
  For example, if the current goal is
  \ecinput{examps/parts/tactics/seq/1-1.0.ec}{}{}{}{} then
  running \ecinput{examps/parts/tactics/seq/1-1.ec}{}{}{}{}
  produces the goals
  \ecinput{examps/parts/tactics/seq/1-1.1.ec}{}{}{}{}
  and
  \ecinput{examps/parts/tactics/seq/1-1.2.ec}{}{}{}{}
  \end{tsyntax}

  \begin{tsyntax}{seq $\;n$ : $\;R$}
  \textbf{\hl sequence rule.} If $n$ is a natural
    number and the goal's conclusion is an \hl statement judgement
    with precondition $P$, postcondition $Q$ and such that the length
    of the program is at least $n$, then reduce the goal to two subgoals:
    \begin{itemize}
    \item A first goal whose conclusion has precondition $P$,
      postcondition $R$, and program consisting of the first $n$
      statements of the original goal's program.

    \item A second goal whose conclusion has precondition $R$,
      postcondition $Q$, and program consisting of all but the first
      $n$ statements of the original goal's program.
    \end{itemize}

  \bigskip
  For example, if the current goal is
  \ecinput{examps/parts/tactics/seq/2-1.0.ec}{}{}{}{} then
  running \ecinput{examps/parts/tactics/seq/2-1.ec}{}{}{}{}
  produces the goals
  \ecinput{examps/parts/tactics/seq/2-1.1.ec}{}{}{}{}
  and
  \ecinput{examps/parts/tactics/seq/2-1.2.ec}{}{}{}{}
  \end{tsyntax}

  \begin{tsyntax}{seq $\ n$: C $\ \delta_1\ \delta_2\ \rho_1\ \rho_2$ R}
    \textbf{\phl sequence rule.} If $n$ is a natural number and the
    goal's conclusion is a \phl statement of the form \ec{phoare[$c$ :
      $\,P$ ==> $\,Q$] $\,\diamond\,$ $e$} where the program $c$ has
    length at least $n$ and $\diamond$ is one of $<$, $\leq$, $=$,
    $>$, or $\geq$, then reduce the goal to the following sequence of
    goals where $c_1$ denotes the first $n$ statements of $c$ and
    $c_2$ denotes the remainder of $c$.
    \begin{itemize}
    \item \ec{hoare[$c_1$ : $\,P$ ==> $\,R$]}
    \item \ec{phoare[$c_1$ : $\,P$ ==> $\,C$] $\,\diamond\,$ $\delta_1$}
    \item \ec{phoare[$c_2$ : $\,R \land C$ ==> $\,Q$] $\,\diamond\,$ $\delta_2$}
    \item \ec{phoare[$c_1$ : $\,P$ ==> $\,!C$] $\,\diamond\,$ $\rho_1$}
    \item \ec{phoare[$c_2$ : $\,R\, \land\, !C$ ==> $\,Q$] $\,\diamond\,$ $\rho_2$}
    \item A goal asking to prove $\delta_1 \delta_2 + \rho_1 \rho_2 \mathrel{\diamond} e$
    \end{itemize}
    When one of $\delta_1,\delta_2$ (resp. $\rho_1,\rho_2$) is 0,
    the other can be replaced with a wildcard \ec{_}, and the
    corresponding goal is not generated, as it is not relevant to the
    proof. When none of $\delta_i$ or $\rho_i$ are given, the
    following default values are used: $\delta_1 = e$, $\delta_2 = 1$,
    $\rho_1 = 0$, $\rho_2 = \_$. \ec{R} is optional and defaults to \ec{true}.

    \fix{Add some Example(s).}

%  \paragraph{Examples:}\strut
%  
%  \begin{cmathpar}
%  \texample[\phl{}]
%    {\ec{seq $\ \left|c\right|$: R $\ \delta_1$ $\ \delta_2$ $\ \delta_3$ $\ \delta_4$ I}}
%    {\HL{P}{c}{I} \\
%     \pHL{P}{c}{R}{\diamond}{\delta_1}  \\
%     \pHL{R \wedge I}{c'}{Q}{\diamond}{\delta_2} \\
%     \pHL{P}{c}{\neg R}{\diamond}{\delta_3} \\
%     \pHL{\neg R \wedge I}{c'}{Q}{\diamond}{\delta_4} \\
%     \delta_1 \delta_2 + \delta_3 \delta_4 \diamond \delta}
%    {\pHL{P}{c;c'}{Q}{\diamond}{\delta}}
%  \end{cmathpar}
%
%  \begin{cmathpar}
%  \texample[\phl{}]
%    {\ec{seq $\ \left|c\right|$: R $\ \delta_1$ $\ \delta_2\ \_\ 0$}}
%    {\HL{P}{c}{\mathsf{true}} \\
%     \pHL{P}{c}{R}{\diamond}{\delta_1} \\\\
%     \pHL{R \wedge I}{c'}{Q}{\diamond}{\delta_2} \\
%     \pHL{\neg R \wedge I}{c'}{Q}{\diamond}{0} \\
%     \delta_1 \delta_2 \diamond \delta}
%    {\pHL{P}{c;c'}{Q}{\diamond}{\delta}}
%  \end{cmathpar}
%
%  \textbf{Note:} Since most tactics implicitly apply the \rtactic{seq}
%  rule, most \phl tactics take optional final arguments corresponding
%  to the $\delta$s and \ec{I}.
 \end{tsyntax}
\end{tactic}
