\section{Program Logics}
\label{sec:programlogics}

In this section, we describe the tactics of \EasyCrypt's three program
logics: \prhl, \phl and \hl.  There are five rough classes of program
logic tactics:
\begin{enumerate}
\item those that actually reason about the program in Hoare logic
  style;

\item those that correspond to semantics-preserving program
  transformations or compiler optimizations;

\item those that operate at the level of specifications,
  strenghtening, combining or splitting goals without modifying the
  program;

\item tactics that automate the application of other tactics;

\item advanced tactics for handling eager/lazy sampling and bounding
  the probability of failure.
\end{enumerate}
We discuss these five classes in turn.

\subsection{Tactics for Reasoning about Programs}
\label{subsec:reasoningprograms}

Unless specified, the following program logic tactics operate on a
program's last statement. Although we describe these tactics as if
they operated on single statements, their practical implementation
automatically and implicitly applies tactic \rtactic{seq} to deal with
context when necessary.

For simple proofs, it is often enough to simply apply the program
tactic corresponding to the last statement in the program and let
\ec{smt} deal with the residual program-free formula, once the program
has been consumed.

Most of the program reasoning tactics discussed in this subsection
have two modes when used on \prhl proof obligations. Their default
mode is to operate on both programs at once. When a side is specified
(using \ec{<tactic>\{1\}} or \ec{<tactic>\{2\}}), a one-sided variant
is used. Apart from the \rtactic{if} tactic, the one-sided variant is
in fact a combination of the \phl tactic and \rtactic{conseq}.

\medskip

% --------------------------------------------------------------------
\begin{tactic}{skip}
  \begin{tsyntax}[empty]{skip} If the goal's conclusion is a statement
    judgement whose program(s) are empty, reduce it to the goal whose
    conclusion is the ambient logic formula \ec{$P$ => $\;Q$}, where
    $P$ is the original conclusion's precondition, and $Q$ is its
    postcondition.

  \bigskip
  For example, if the current goal is
  \ecinput{../examps/parts/tactics/skip/1-1.0.ec}{}{}{}{} then
  running \ecinput{../examps/parts/tactics/skip/1-1.ec}{}{}{}{}
  produces the goal
  \ecinput{../examps/parts/tactics/skip/1-1.1.ec}{}{}{}{}

%  \textbf{Note:} Note that the \phl rule forces the bound of the goal
%  to be 1. If you end up with an empty program and a bound other than
%  1, you might want to use \rtactic{hoare} or \rtactic{conseq}. If
%  neither of these work, you should probably have used \rtactic{seq}
%  or \rtactic{phoare split} earlier on in your proof.
  \end{tsyntax}
\end{tactic}

% --------------------------------------------------------------------
\begin{tactic}{seq}
  \begin{tsyntax}{seq $\;n_1$ $\;n_2$ : $\;R$}
    \textbf{\prhl sequence rule.} If $n_1$ and $n_2$ are natural
    numbers and the goal's conclusion is a \prhl statement judgement
    with precondition $P$, postcondition $Q$ and such that the lengths
    of the first and second programs are at least $n_1$ and $n_2$,
    respectively, then reduce the goal to two subgoals:
    \begin{itemize}
    \item A first goal whose conclusion has precondition $P$,
      postcondition $R$, first program consisting of the first $n_1$
      statements of the original goal's first program, and second
      program consisting of the first $n_2$ statements of the original
      goal's second program.

    \item A second goal whose conclusion has precondition $R$,
      postcondition $Q$, first program consisting of all but the first
      $n_1$ statements of the original goal's first program, and
      second program consisting of all but the first $n_2$ statements
      of the original goal's second program.
    \end{itemize}

  \bigskip
  For example, if the current goal is
  \ecinput{examps/parts/tactics/seq/1-1.0.ec}{}{}{}{} then
  running \ecinput{examps/parts/tactics/seq/1-1.ec}{}{}{}{}
  produces the goals
  \ecinput{examps/parts/tactics/seq/1-1.1.ec}{}{}{}{}
  and
  \ecinput{examps/parts/tactics/seq/1-1.2.ec}{}{}{}{}
  \end{tsyntax}

  \begin{tsyntax}{seq $\;n$ : $\;R$}
  \textbf{\hl sequence rule.} If $n$ is a natural
    number and the goal's conclusion is an \hl statement judgement
    with precondition $P$, postcondition $Q$ and such that the length
    of the program is at least $n$, then reduce the goal to two subgoals:
    \begin{itemize}
    \item A first goal whose conclusion has precondition $P$,
      postcondition $R$, and program consisting of the first $n$
      statements of the original goal's program.

    \item A second goal whose conclusion has precondition $R$,
      postcondition $Q$, and program consisting of all but the first
      $n$ statements of the original goal's program.
    \end{itemize}

  \bigskip
  For example, if the current goal is
  \ecinput{examps/parts/tactics/seq/2-1.0.ec}{}{}{}{} then
  running \ecinput{examps/parts/tactics/seq/2-1.ec}{}{}{}{}
  produces the goals
  \ecinput{examps/parts/tactics/seq/2-1.1.ec}{}{}{}{}
  and
  \ecinput{examps/parts/tactics/seq/2-1.2.ec}{}{}{}{}
  \end{tsyntax}

%  \begin{tsyntax}{seq $\ n$: R $\ \delta_1\ \delta_2\ \delta_3\ \delta_4$ I}
%  Non-relational probabilistic sequence rule. Argument \ec{I} is
%  optional (and defaults to $\mathsf{true}$). When one of
%  $(\delta_1,\delta_2)$ (resp. $(\delta_3,\delta_4)$) is 0, the other
%  can be replaced with a wildcard \ec{_}, and the corresponding goal
%  is not generated, as it is not relevant to the proof. When none of
%  the $\delta$s are given, the following default values are used:
%  $\delta_1 = 1$, $\delta_2 = \delta$, $\delta_3 = 0$.
%
%  \paragraph{Examples:}\strut
%  
%  \begin{cmathpar}
%  \texample[\phl{}]
%    {\ec{seq $\ \left|c\right|$: R $\ \delta_1$ $\ \delta_2$ $\ \delta_3$ $\ \delta_4$ I}}
%    {\HL{P}{c}{I} \\
%     \pHL{P}{c}{R}{\diamond}{\delta_1}  \\
%     \pHL{R \wedge I}{c'}{Q}{\diamond}{\delta_2} \\
%     \pHL{P}{c}{\neg R}{\diamond}{\delta_3} \\
%     \pHL{\neg R \wedge I}{c'}{Q}{\diamond}{\delta_4} \\
%     \delta_1 \delta_2 + \delta_3 \delta_4 \diamond \delta}
%    {\pHL{P}{c;c'}{Q}{\diamond}{\delta}}
%  \end{cmathpar}
%
%  \begin{cmathpar}
%  \texample[\phl{}]
%    {\ec{seq $\ \left|c\right|$: R $\ \delta_1$ $\ \delta_2\ \_\ 0$}}
%    {\HL{P}{c}{\mathsf{true}} \\
%     \pHL{P}{c}{R}{\diamond}{\delta_1} \\\\
%     \pHL{R \wedge I}{c'}{Q}{\diamond}{\delta_2} \\
%     \pHL{\neg R \wedge I}{c'}{Q}{\diamond}{0} \\
%     \delta_1 \delta_2 \diamond \delta}
%    {\pHL{P}{c;c'}{Q}{\diamond}{\delta}}
%  \end{cmathpar}
%
%  \textbf{Note:} Since most tactics implicitly apply the \rtactic{seq}
%  rule, most \phl tactics take optional final arguments corresponding
%  to the $\delta$s and \ec{I}.
%  \end{tsyntax}
\end{tactic}

% --------------------------------------------------------------------
\begin{tactic}{sp}
  \begin{tsyntax}{sp}
    If the goal's conclusion is a \prhl, \phl or \hl statement
    judgement, consume the longest prefix(es) of the conclusion's
    program(s) consisting entirely of statements built-up from
    ordinary assignments (not random assignments or procedure call
    assignments) and \ec{if} statements, replacing the conclusion's
    precondition by the strongest postcondition $R$ such that the
    statement judgement consisting of the conclusion's original
    precondition, the consumed prefix(es) and $R$
    holds.

    \bigskip For example, if the current goal is
    \ecinput{../examps/parts/tactics/sp/1-1.0.ec}{}{}{}{} then
    running \ecinput{../examps/parts/tactics/sp/1-1.ec}{}{}{}{}
    produces the goal
    \ecinput{../examps/parts/tactics/sp/1-1.1.ec}{}{}{}{}
  \end{tsyntax}

  \begin{tsyntax}{sp $\;n_1$ $\;n_2$}
    In \prhl, let \ec{sp} consume \emph{exactly} $n_1$ statements of
    the first program and $n_2$ statements of the second
    program. Fails if this isn't possible.
  \end{tsyntax}

  \begin{tsyntax}{sp $\;n$}
    In \phl and \hl, let \ec{sp} consume \emph{exactly} $n$ statements
    of the program. Fails if this isn't possible.
  \end{tsyntax}
\end{tactic}

% --------------------------------------------------------------------
\begin{tactic}{wp}
  \begin{tsyntax}{wp}
    If the goal's conclusion is a \prhl, \phl or \hl statement
    judgement, consume the longest suffix(es) of the conclusion's
    program(s) consisting entirely of statements built-up from
    ordinary assignments (not random assignments or procedure call
    assignments) and \ec{if} statements, replacing the conclusion's
    postcondition by the weakest precondition $R$ such that the
    statement judgement consisting of $R$, the consumed suffix(es)
    and the conclusion's original postcondition holds.

    \bigskip For example, if the current goal is
    \ecinput{../examps/parts/tactics/wp/1-1.0.ec}{}{}{}{} then
    running \ecinput{../examps/parts/tactics/wp/1-1.ec}{}{}{}{}
    produces the goal
    \ecinput{../examps/parts/tactics/wp/1-1.1.ec}{}{}{}{}
  \end{tsyntax}

  \begin{tsyntax}{wp $\;n_1$ $\;n_2$}
    In \prhl, let \ec{wp} consume \emph{exactly} $n_1$ statements of
    the first program and $n_2$ statements of the second
    program. Fails if this isn't possible.
  \end{tsyntax}

  \begin{tsyntax}{wp $\;n$}
    In \phl and \hl, let \ec{wp} consume \emph{exactly} $n$ statements
    of the program. Fails if this isn't possible.
  \end{tsyntax}
\end{tactic}

% --------------------------------------------------------------------
\begin{tactic}{rnd}
  When describing the variants of this tactic, we only consider random
  assignments whose left-hand sides consist of single identifiers.
  The generalization to multiple assignment, when distributions over
  tuple types are sampled, is straightforward.

  \bigskip
  \begin{tsyntax}{rnd | rnd $\;f$ | rnd $\;f$ $\;g$} If the conclusion
    is a \prhl statement judgement whose programs end with random
    assignments \ec{$x_1\!$ <$\$$ $\;d_1$} and \ec{$x_2\!$ <$\$$
      $\;d_2$}, and $f$ and $g$ are functions between the types of
    $x_1$ and $x_2$, then consume those random assignments, replacing
    the conclusion's postcondition by the probabilistic weakest
    precondition of the random assignments wrt.\ $f$ and $g$.  The new
    postcondition checks that:
    \begin{itemize}
    \item $f$ and $g$ are an isomorphism between the distributions
      $d_1$ and $d_2$;

    \item for all elements $u$ in the support of $d_1$, the result
      of substituting $u$ and $f\,u$ for \ec{$x_1$\{1\}} and
      \ec{$x_2$\{2\}} in the conclusion's original postcondition
      holds.
    \end{itemize}
    When $g$ is $f$, it can be omitted. When $f$ is the identity, it
    can be omitted.

    \bigskip For example, if the current goal is
    \ecinput{examps/parts/tactics/rnd/2-1.0.ec} then
    running \ecinput{examps/parts/tactics/rnd/2-1.ec}
    produces the goal
    \ecinput{examps/parts/tactics/rnd/2-1.1.ec}
    Note that if one uses the other isomorphism between \ec{\{0,1\}} and
    \ec{[2..3]} the generated subgoal will be false.
  \end{tsyntax}

  \begin{tsyntax}{rnd\{1\} | rnd\{2\}}
    If the conclusion is a \prhl statement judgement whose designated
    program (1 or 2) ends with a random assignment 
    \ec{$x\!$ <$\$$ $\;d$}, then consume that random assignment,
    replacing the conclusion's postcondition with a check that:
    \begin{itemize}
    \item the weight of $d$ is $1$ (so the random assignment can't fail);

    \item for all elements $u$ in the support of $d$, the result
      of substituting $u$ for \ec{$x$\{$i$\}}---where $i$ is the
      selected side---in the conclusion's original
      postcondition holds.
    \end{itemize}

    \bigskip For example, if the current goal is
    \ecinput{examps/parts/tactics/rnd/3-1.0.ec} then
    running \ecinput{examps/parts/tactics/rnd/3-1.ec}
    produces the (false!) goal
    \ecinput{examps/parts/tactics/rnd/3-1.1.ec}
  \end{tsyntax}

  \begin{tsyntax}{rnd}
    If the conclusion is an \hl statement judgement whose program ends
    with a random assignment, then consume that random assignment,
    replacing the conclusion's postcondition by the possibilistic
    weakest precondition of the random assignment.

    \bigskip For example, if the current goal is
    \ecinput{examps/parts/tactics/rnd/1-1.0.ec} then
    running \ecinput{examps/parts/tactics/rnd/1-1.ec}
    produces the goal
    \ecinput{examps/parts/tactics/rnd/1-1.1.ec}
  \end{tsyntax}

  \begin{tsyntax}{rnd | rnd $\;E$}
    In \phl, compute the probabilistic weakest precondition of a
    random sampling with respect to event $E$. When $E$ is not
    specified, it is inferred from the current postcondition.
  \end{tsyntax}
\end{tactic}

% --------------------------------------------------------------------
\begin{tactic}{if}
  \begin{tsyntax}{if}
  If the goal's conclusion is a \prhl statement judgement whose programs
  both \emph{begin} with \ec{if} statements, reduce the goal to
  three subgoals:
  \begin{itemize}
  \item One whose conclusion is the ambient logic formula asserting that
    the equivalence of the boolean expressions of the \ec{if} statements
    in their respective memories holds given that the statement judgement's
    precondition holds in those memories.

  \item One in which the \ec{if} statements have been replaced by
    their ``then'' parts, and where the assertion of the truth of
    the first \ec{if} statement's boolean expression in the first
    program's memory has been added to the conclusion's precondition.

  \item One in which the \ec{if} statements have been replaced by
    their ``else'' parts, and where the assertion of the falsity of
    the first \ec{if} statement's boolean expression in the first
    program's memory has been added to the conclusion's precondition.
  \end{itemize}

  \bigskip
  For example, if the current goal is
  \ecinput{../examps/parts/tactics/if/1-1.0.ec}{}{}{}{} then
  running \ecinput{../examps/parts/tactics/if/1-1.ec}{}{}{}{}
  produces the goals
  \ecinput{../examps/parts/tactics/if/1-1.1.ec}{}{}{}{}
  and
  \ecinput{../examps/parts/tactics/if/1-1.2.ec}{}{}{}{}
  and
  \ecinput{../examps/parts/tactics/if/1-1.3.ec}{}{}{}{}
  \end{tsyntax}

  \begin{tsyntax}{if\{1\} | if\{2\}}
    If the goal's conclusion is a \prhl judgement in which the first
    statement of the specified program is an \ec{if} statement,
    then reduce the goal to two subgoals:
    \begin{itemize}
    \item One where the \ec{if} statement has been replaced by its
      ``then'' part, and the precondition has been augmented by the
      assertion that the \ec{if} statement's boolean expression is true
      in the specified program's memory.

    \item One where the \ec{if} statement has been replaced by its
      ``else'' part, and the precondition has been augmented by the
      assertion that the \ec{if} statement's boolean expression is false
      in the specified program's memory.
    \end{itemize}

    For example, if the current goal is
    \ecinput{../examps/parts/tactics/if/2-1.0.ec}{}{}{}{} then running
    \ecinput{../examps/parts/tactics/if/2-1.ec}{}{}{}{} produces the
    goals \ecinput{../examps/parts/tactics/if/2-1.1.ec}{}{}{}{} and
    \ecinput{../examps/parts/tactics/if/2-1.2.ec}{}{}{}{}
  \end{tsyntax}

  \begin{tsyntax}{if}
    If the goal's conclusion is an \hl judgement whose first statement
    is an \ec{if} statement, then reduce the goal to two subgoals:
    \begin{itemize}
    \item One where the \ec{if} statement has been replaced by its
      ``then'' part, and the precondition has been augmented by the
      assertion that the \ec{if} statement's boolean expression is true.

    \item One where the \ec{if} statement has been replaced by its
      ``else'' part, and the precondition has been augmented by the
      assertion that the \ec{if} statement's boolean expression is false.
    \end{itemize}

    For example, if the current goal is
    \ecinput{../examps/parts/tactics/if/3-1.0.ec}{}{}{}{} then running
    \ecinput{../examps/parts/tactics/if/3-1.ec}{}{}{}{} produces the
    goals \ecinput{../examps/parts/tactics/if/3-1.1.ec}{}{}{}{} and
    \ecinput{../examps/parts/tactics/if/3-1.2.ec}{}{}{}{}
  \end{tsyntax}
\end{tactic}

% --------------------------------------------------------------------
\begin{tactic}{while}
  \begin{tsyntax}{while $\;I$}
  In \prhl and \hl, as well as upper-bounding \phl judgments, performs
  a weakest precondition computation over a loop using \ec{I} as
  invariant. This generates two subgoals: one explaining that \ec{I}
  is a valid loop invariant, and the other explaining that the
  invariant is initially true and that it is sufficient to establish
  the current postcondition.

  \bigskip
  For example, if the current goal is
  \ecinput{../examps/parts/tactics/while/1-1.0.ec}{}{}{}{} then
  running \ecinput{../examps/parts/tactics/while/1-1.ec}{}{}{}{}
  produces the goals
  \ecinput{../examps/parts/tactics/while/1-1.1.ec}{}{}{}{}
  and
  \ecinput{../examps/parts/tactics/while/1-1.2.ec}{}{}{}{}
  \end{tsyntax}

  \begin{tsyntax}{while $\;I$}
  In \prhl and \hl, as well as upper-bounding \phl judgments, performs
  a weakest precondition computation over a loop using \ec{I} as
  invariant. This generates two subgoals: one explaining that \ec{I}
  is a valid loop invariant, and the other explaining that the
  invariant is initially true and that it is sufficient to establish
  the current postcondition.

  \bigskip
  For example, if the current goal is
  \ecinput{../examps/parts/tactics/while/2-1.0.ec}{}{}{}{} then
  running \ecinput{../examps/parts/tactics/while/2-1.ec}{}{}{}{}
  produces the goals
  \ecinput{../examps/parts/tactics/while/2-1.1.ec}{}{}{}{}
  and
  \ecinput{../examps/parts/tactics/while/2-1.2.ec}{}{}{}{}
  \end{tsyntax}

  \begin{tsyntax}{while $\;I$ $\;v$}
  Where \ec{v} is an integer-valued expression. In \phl, performs a
  weakest precondition computation over a loop, using \ec{I} as
  invariant and \ec{v} as a decreasing variant to prove
  termination. In addition to the two invariant-related subgoals (see
  above), two subgoals regarding the variant are generated; one
  requiring that the variant be less than 0 exactly when the loop
  condition is false, and the other requiring that the variant
  decreases strictly.
  \end{tsyntax}
\end{tactic}

% --------------------------------------------------------------------
\begin{tactic}{call}
  When describing the variants of this tactic, we ony consider
  procedure call assignments whose left-hand sides consist of single
  identifiers.  The generalization to multiple assignment, when values
  of tuple types are returned, is straightforward.

  \bigskip
  \begin{tsyntax}{call (_ : $\;P$ ==> $\;Q$)}
    If the goal's conclusion is a \prhl statement judgement whose
    programs end with procedure calls or procedure call assignments
    (resp., an \hl statement judgement whose program ends with a
    procedure call or procedure call assignment), then generate two
    subgoals:
  \begin{itemize}
  \item One whose conclusion is a \prhl judgement (resp., \hl
    judgement) whose precondition is $P$, whose procedures are the
    procedures being called (resp., procedure is the procedure being
    called), and whose postcondition is $Q$.

  \item One whose conclusion is a \prhl statement judgement (resp.,
    \hl statement judgement) whose precondition is the original goal's
    precondition, whose programs are (resp., program is) the result of
    removing the procedure calls (resp., call) from the programs
    (resp., program), and whose postcondition is the conjunction of
    \begin{itemize}
    \item the result of replacing the procedures' (resp., procedure's)
      parameter(s) by their actual argument(s) in $P$; and

    \item the assertion that, for all values of the global variable(s)
      modified by the procedures (resp., procedure) and the results
      (resp., result) of the procedure calls (resp., procedure call),
      if $Q$ holds (where these quantified identifiers have been
      substituted for the modified variables and procedure results),
      then the original goal's postcondition holds (where the modified
      global variables and occurrences of the variables (resp.,
      variable) (if any) to which the results of the procedure calls
      are (resp., result of the procedure call is) assigned have been
      replaced by the appropriate quantified identifiers).
    \end{itemize}
  \end{itemize}

  \medskip
  For example, if the current goal is
  \ecinput{../examps/parts/tactics/call/1-1.0.ec}{}{}{}{}
  and the procedures \ec{M.f} and \ec{N.f} have a single parameter,
  \ec{y}, then running
  \ecinput{../examps/parts/tactics/call/1-1.ec}{}{}{}{} produces the
  goals \ecinput{../examps/parts/tactics/call/1-1.1.ec}{}{}{}{} and
  \ecinput{../examps/parts/tactics/call/1-1.2.ec}{}{}{}{}

  \bigskip
  Alternatively, a proof term whose conclusion is a \prhl or
  \hl judgement involving the procedure(s) called at the
  end(s) of the program(s) may be supplied as the argument to
  \ec{call}, in which case only the second subgoal need be
  generated.

  \medskip
  For example, in the start-goal of the preceding example,
  if the lemma \ec{M_N_f} is
  \ecinput{examps/tactics/call/1.ec}{}{31-33}{}
  then running
  \ecinput{../examps/parts/tactics/call/1-2.ec}{}{}{}{} produces the
  goal \ecinput{../examps/parts/tactics/call/1-2.1.ec}{}{}{}{}
  \end{tsyntax}

  \begin{tsyntax}{call\{1\} (_ : $\;P$ ==> $\;Q$) | call\{2\} (_ : $\;P$ ==> $\;Q$)}
    If the goal's conclusion is a \prhl statement judgement whose
    designated program ends with a procedure call, then generate two
    subgoals:
  \begin{itemize}
  \item One whose conclusion is a \phl judgement whose precondition is
    $P$, whose procedure is the procedure being called, whose
    postcondition is $Q$, and whose bound part specifies equality with
    probability 1.
    (Consequently, $P$ and $Q$ may not mention \ec{&1} and
    \ec{&2}.)

  \item One whose conclusion is a \prhl statement judgement whose
    precondition is the original goal's precondition, whose programs
    are the result of removing the procedure call from the designated
    program, and leaving the other program unchanged, and whose
    postcondition is the conjunction of
    \begin{itemize}
    \item the result of replacing the procedure's
      parameter(s) by their actual argument(s) in $P$; and

    \item the assertion that, for all values of the global variable(s)
      modified by the procedure and the result of the procedure call,
      if $Q$ holds (where these quantified identifiers have been
      substituted for the modified variables and procedure result),
      then the original goal's postcondition holds (where the modified
      global variables and occurrences the variable (if any) to which
      the result of the procedure call is assigned have been replaced
      by the appropriate quantified identifiers).
    \end{itemize}
  \end{itemize}

  For example, if the current goal is
  \ecinput{../examps/parts/tactics/call/2-1.0.ec}{}{}{}{}
  then running
  \ecinput{../examps/parts/tactics/call/2-1.ec}{}{}{}{} produces the
  goals \ecinput{../examps/parts/tactics/call/2-1.1.ec}{}{}{}{} and
  \ecinput{../examps/parts/tactics/call/2-1.2.ec}{}{}{}{}

  \bigskip Alternatively, a proof term whose conclusion is a \phl
  judgement specifying equality with probability 1 and involving the
  procedure called at the end of the designated program may be
  supplied as the argument to \ec{call}, in which case only the second
  subgoal need be generated.

  \medskip
  For example, in the start-goal of the preceding example,
  if the lemma \ec{M_f} is
  \ecinput{examps/tactics/call/2.ec}{}{23-24}{}
  then running
  \ecinput{../examps/parts/tactics/call/2-2.ec}{}{}{}{} produces the
  goal \ecinput{../examps/parts/tactics/call/2-2.1.ec}{}{}{}{}
  \end{tsyntax}

  \begin{tsyntax}{call (_ : $\;I$)}
    If the conclusion is a \prhl statement judgement whose programs
    end with calls to \emph{concrete} procedures (resp., an \hl
    statement judgement whose program ends with a call to a concrete
    procedure), then use the specification argument to \ec{call}
    generated from the \emph{invariant} $I$, and automatically apply
    \ec{proc} to its first subgoal.  In the \prhl case, its
    precondition will assume equality of the procedures' parameters,
    and its postcondition will assert equality of the results of the
    procedure calls.

    \medskip
    For example, if the current goal is
    \ecinput{../examps/parts/tactics/call/1-3.0.ec}{}{}{}{}
    and modules \ec{M} and \ec{N} contain
    \ecinput{examps/tactics/call/1.ec}{}{4-8}{} and
    \ecinput{examps/tactics/call/1.ec}{}{18-22}{}
    respectively, then
    running \ecinput{../examps/parts/tactics/call/1-3.ec}{}{}{}{}
    produces the goals
    \ecinput{../examps/parts/tactics/call/1-3.1.ec}{}{}{}{} and
    \ecinput{../examps/parts/tactics/call/1-3.2.ec}{}{}{}{}
  \end{tsyntax}

  \begin{tsyntax}{call (_ : $\;I$)}
    If the conclusion is a \prhl statement judgement whose programs
    end with calls of the same \emph{abstract} procedure (resp., an
    \hl statement judgement whose program ends with a call to an
    abstract procedure), then use the specification argument to
    \ec{call} generated from the \emph{invariant} $I$, and
    automatically apply \ec{proc $\;I$} to its first subgoal, pruning
    the first two subgoals the application generates, because their
    conclusions consist of ambient logic formulas that are true by
    construction.  In the \prhl case, its precondition will assume
    equality of the procedure's parameters and of the global variables
    of the module containing the procedure, and its postcondition will
    assume equality of the results of the procedure calls and of the
    global variables of the containing module.

    \medskip
    For example, given the declarations
    \ecinput{examps/tactics/call/3.ec}{}{3-24}{}
    if the current goal is
    \ecinput{../examps/parts/tactics/call/3-1.0.ec}{}{}{}{}
    then running
    \ecinput{../examps/parts/tactics/call/3-1.ec}{}{}{}{}
    produces the goals
    \ecinput{../examps/parts/tactics/call/3-1.1.ec}{}{}{}{} and
    \ecinput{../examps/parts/tactics/call/3-1.2.ec}{}{}{}{}
  \end{tsyntax}

  \begin{tsyntax}{call (_ : $\;B$, $\;I$)}
    If the conclusion is a \prhl statement judgement whose programs
    end with calls of the same \emph{abstract} procedure, then use the
    specification argument to \ec{call} generated from the \emph{bad
      event} $B$ and \emph{invariant} $I$, and automatically apply
    \ec{proc $\;B$ $\;I$} to its first subgoal, pruning the first two
    subgoals the application generates, because their conclusions
    consist of ambient logic formulas that are true by construction,
    and pruning the next goal (showing the losslessness of the abstract
    procedure given the losslessness of the abstract oracles it uses), if
    trivial suffices to solve it.
    The specification's precondition will assume equality of the
    procedure's parameters and of the global variables of the module
    containing the procedure as well as $I$, and its postcondition
    will assert $I$ and the equality of the results of the procedure
    calls and of the global variables of the containing module---but
    only when $B$ does not hold.

    \medskip
    For example, given the declarations
    \ecinput{examps/tactics/call/4.ec}{}{3-51}{}
    if the current goal is
    \ecinput{../examps/parts/tactics/call/4-1.0.ec}{}{}{}{}
    then running
    \ecinput{../examps/parts/tactics/call/4-1.ec}{}{}{}{}
    produces the goals
    \ecinput{../examps/parts/tactics/call/4-1.1.ec}{}{}{}{} and
    \ecinput{../examps/parts/tactics/call/4-1.2.ec}{}{}{}{} and
    \ecinput{../examps/parts/tactics/call/4-1.3.ec}{}{}{}{} and
    \ecinput{../examps/parts/tactics/call/4-1.4.ec}{}{}{}{}
  \end{tsyntax}

  \begin{tsyntax}{call (_ : $\;B$, $\;I$, $\;J$)}
    If the conclusion is a \prhl statement judgement whose programs
    end with calls of the same \emph{abstract} procedure, then use the
    specification argument to \ec{call} generated from the \emph{bad
      event} $B$ and \emph{invariants} $I$ and $J$, and automatically
    apply \ec{proc $\;B$ $\;I$ $\;J$} to its first subgoal, pruning
    the first two subgoals the application generates, because their
    conclusions consist of ambient logic formulas that are true by
    construction, and pruning the next goal (showing the losslessness
    of the abstract procedure given the losslessness of the abstract
    oracles it uses), if trivial suffices to solve it.  The
    specification's precondition will assume equality of the
    procedure's parameters and of the global variables of the module
    containing the procedure as well as $I$, and its postcondition
    will assert
    \begin{itemize}
    \item $I$ and the equality of the results of the procedure calls
      and of the global variables of the containing module---if $B$
      does not hold; and

    \item $J$---if $B$ does hold.
    \end{itemize}

    \medskip
    For example, given the declarations of the preceding example
    if the current goal is
    \ecinput{../examps/parts/tactics/call/4-2.0.ec}{}{}{}{}
    then running
    \ecinput{../examps/parts/tactics/call/4-2.ec}{}{}{}{}
    produces the goals
    \ecinput{../examps/parts/tactics/call/4-2.1.ec}{}{}{}{} and
    \ecinput{../examps/parts/tactics/call/4-2.2.ec}{}{}{}{} and
    \ecinput{../examps/parts/tactics/call/4-2.3.ec}{}{}{}{} and
    \ecinput{../examps/parts/tactics/call/4-2.4.ec}{}{}{}{}
  \end{tsyntax}
\end{tactic}

% --------------------------------------------------------------------
\begin{tactic}{proc}
  \begin{tsyntax}{proc}
    Turn a goal whose conclusion is a \prhl, \phl or \hl judgement
    involving \emph{concrete} procedure(s) into one whose conclusion
    is a statement judgement by replacing the concrete procedure(s) by
    their body/ies. Assertions about \ec{res}/\ec{res\{$i$\}} are turned
    into ones about the value(s) returned by the procedure(s).

  \bigskip
  For example, if the current goal is
  \ecinput{../examps/parts/tactics/proc/1-1.0.ec}{}{}{}{} then
  running \ecinput{../examps/parts/tactics/proc/1-1.ec}{}{}{}{}
  produces the goal
  \ecinput{../examps/parts/tactics/proc/1-1.1.ec}{}{}{}{}
  \end{tsyntax}

  \begin{tsyntax}{proc $\;I$}
    Reduce a goal whose conclusion is a \prhl (resp., \hl) judgement
    involving an \emph{abstract} procedure to goals whose conclusions
    are \prhl (resp., \hl) judgements on the oracles the procedure may
    query, plus goals with ambient logic conclusions checking the
    original judgement's pre- and postconditions allow such
    a reduction.

  \bigskip
  For example, given the declarations
  \ecinput{examps/tactics/proc/2.ec}{}{3-24}{}
  if the current goal is
  \ecinput{../examps/parts/tactics/proc/2-1.0.ec}{}{}{}{} then
  running \ecinput{../examps/parts/tactics/proc/2-1.ec}{}{}{}{}
  produces the goals
  \ecinput{../examps/parts/tactics/proc/2-1.1.ec}{}{}{}{}
  and
  \ecinput{../examps/parts/tactics/proc/2-1.2.ec}{}{}{}{}
  and
  \ecinput{../examps/parts/tactics/proc/2-1.3.ec}{}{}{}{}
  and
  \ecinput{../examps/parts/tactics/proc/2-1.4.ec}{}{}{}{}
  The tactic would fail without the module restriction \ec{T\{Or\}} on
  \ec{M}, as then \ec{M} could directly manipulate \ec{Or.x}.
  \end{tsyntax}

  \begin{tsyntax}{proc $\;B$ $\;I$}
    Like \ec{proc $\;I$}, but just for \prhl judgements and uses
    ``upto-bad'' (upto-failure) reasoning, where the ``bad'' (failure)
    event, $B$, is evaluated in the second program's memory, and the
    invariant $I$ only holds up to the point when failure occurs.  In
    addition to subgoals whose conclusions are \prhl judgments
    involving the oracles the abstract procedure may query, subgoals
    are generated that check that: the original judgement's pre- and
    postconditions support the reduction; the abstract procedure is
    lossless, assuming the losslessness of the oracles it may query;
    the oracles used by the abstract procedure in the first program
    are lossless once the bad event occurs; and the oracles used by
    the abstract procedure in the second program guarantee the
    stability of the failure event with probability 1.

  \bigskip
  For example, suppose we have the following declarations
  \ecinput{examps/tactics/proc/3.ec}{}{3-40}{}
  Then, if the current goal is
  \ecinput{../examps/parts/tactics/proc/3-1.0.ec}{}{}{}{}
  running \ecinput{../examps/parts/tactics/proc/3-1.ec}{}{}{}{}
  produces the goals
  \ecinput{../examps/parts/tactics/proc/3-1.1.ec}{}{}{}{}
  and
  \ecinput{../examps/parts/tactics/proc/3-1.2.ec}{}{}{}{}
  and
  \ecinput{../examps/parts/tactics/proc/3-1.3.ec}{}{}{}{}
  and
  \ecinput{../examps/parts/tactics/proc/3-1.4.ec}{}{}{}{}
  and
  \ecinput{../examps/parts/tactics/proc/3-1.5.ec}{}{}{}{}
  and
  \ecinput{../examps/parts/tactics/proc/3-1.6.ec}{}{}{}{}
  \end{tsyntax}

  \begin{tsyntax}{proc $\;B$ $\;I$ $\;J$}
    Like \ec{proc $\;B$ $\;I$}, but where the extra invariant, $J$,
    holds \emph{after} failure has occurred.

  \bigskip
  For example, given the declarations of the \ec{proc $\;B$ $\;I$} example,
  if the current goal is
  \ecinput{../examps/parts/tactics/proc/3-2.0.ec}{}{}{}{} then
  running \ecinput{../examps/parts/tactics/proc/3-2.ec}{}{}{}{}
  produces the goals
  \ecinput{../examps/parts/tactics/proc/3-2.1.ec}{}{}{}{}
  and
  \ecinput{../examps/parts/tactics/proc/3-2.2.ec}{}{}{}{}
  and
  \ecinput{../examps/parts/tactics/proc/3-2.3.ec}{}{}{}{}
  and
  \ecinput{../examps/parts/tactics/proc/3-2.4.ec}{}{}{}{}
  and
  \ecinput{../examps/parts/tactics/proc/3-2.5.ec}{}{}{}{}
  and
  \ecinput{../examps/parts/tactics/proc/3-2.6.ec}{}{}{}{}
  \end{tsyntax}

  \begin{tsyntax}{proc*}
    Reduce a \prhl (resp., \phl) judgement to a \prhl (resp., \phl)
    statement judgement involving calls (resp., a call) to the
    procedures (resp., procedure).

  \bigskip
  For example, if the current goal is
  \ecinput{../examps/parts/tactics/proc/4-1.0.ec}{}{}{}{} then
  running \ecinput{../examps/parts/tactics/proc/4-1.ec}{}{}{}{}
  produces the goal
  \ecinput{../examps/parts/tactics/proc/4-1.1.ec}{}{}{}{}

  \paragraph{Remark.}
  This tactic is particularly useful in combination with
  \rtactic{inline} when faced with a \prhl judgment where one of the
  procedures is concrete and the other is abstract.
  \end{tsyntax}
\end{tactic}


\subsection{Tactics for Transforming Programs}
\label{subsec:transformingprograms}

% --------------------------------------------------------------------
\begin{tactic}{swap}
All versions of the tactic work for \prhl (an optional side can be given),
\phl and \hl statement judgements. We'll describe their operation
in terms of a single program (list of statements).

\medskip
\begin{tsyntax}{swap $\;n$ $\;m$ $\;l$}
  Fails unless $1\leq n < m \leq l$ and the program has at least $l$
  statements. Swaps the statement block from positions $n$ through
  $m-1$ with the statement block from $m$ through $l$, failing if these
  blocks of statements aren't independent.
\end{tsyntax}

\begin{tsyntax}{swap [$n$..$m$] $\;k$}
  Fails unless $1\leq n \leq m$ and the program has at
  least $m$ statements.
  \begin{itemize}
  \item If $k$ is non-negative, move the statement block from $n$
    through $m$ forward $k$ positions, failing if the program doesn't
    have at least $m + k$ statements or if the swapped statements
    blocks aren't independent.

  \item If $k$ is negative, move the statement block from $n$ through
    $m$ backward $-k$ positions, failing if $n + k < 1$ or if the
    swapped statement blocks aren't independent.
  \end{itemize}
\end{tsyntax}

\begin{tsyntax}{swap $\;n$\ $k$}
  Equivalent to \ec{swap [$n$..$n$] $\;k$}.
\end{tsyntax}

\begin{tsyntax}{swap $\;k$}
  If $k$ is non-negative, equivalent to \ec{swap 1 $\;k$}.
  If $k$ is negative, equivalent to \ec{swap $\;n$ $\;k$},
  where $n$ is the length of the program.
\end{tsyntax}

\medskip For example, suppose the current goal is
  \ecinput{examps/parts/tactics/swap/1-1.0.ec}
  Then running
  \ecinput{examps/parts/tactics/swap/1-1.ec}
  produces goal
  \ecinput{examps/parts/tactics/swap/1-1.1.ec}
  From which running
  \ecinput{examps/parts/tactics/swap/1-2.ec}
  produces goal
  \ecinput{examps/parts/tactics/swap/1-2.1.ec}
  From which running
  \ecinput{examps/parts/tactics/swap/1-3.ec}
  produces goal
  \ecinput{examps/parts/tactics/swap/1-3.1.ec}
  From which running
  \ecinput{examps/parts/tactics/swap/1-4.ec}
  produces goal
  \ecinput{examps/parts/tactics/swap/1-4.1.ec}
  From which running
  \ecinput{examps/parts/tactics/swap/1-5.ec}
  produces goal
  \ecinput{examps/parts/tactics/swap/1-5.1.ec}
\end{tactic}

% --------------------------------------------------------------------
\begin{tactic}{inline}
  \begin{tsyntax}{inline $\;M_1$.$p_1$ $\;\cdots$ $\;M_n$.$p_n$}
    Inline the selected \emph{concrete} procedures in both programs,
    with \prhl, and in the program, with \hl and \phl, until no more
    inlining of these procedures is possible.

    To inline a procedure call, the procedure's parameters are
    assigned the values of their arguments (fresh parameter
    identifiers are used, as necessary, to avoid naming
    conflicts). This is followed by the body of the procedure. Finally,
    the procedure's return value is assigned to the identifiers (if
    any) to which the procedure call's result is assigned.
  \end{tsyntax}

  \begin{tsyntax}{inline\{1\} $\;M_1$.$p_1$ $\;\cdots$ $\;M_n$.$p_n$ | inline\{2\} $\;M_1$.$p_1$ $\;\cdots$ $\;M_n$.$p_n$}
    Do the inlining in just the first or second program, in the \prhl case.
  \end{tsyntax}

  \begin{tsyntax}{inline* | inline\{1\}* | inline\{2\}*}
    Inline all concrete procedures, continuing until no more inlining
    is possible.
  \end{tsyntax}

  \begin{tsyntax}{inline $\;\mathit{occs}$ $\;M$.$p$ | inline\{1\} $\;\mathit{occs}$ $\;M$.$p$ | inline\{2\} $\;\mathit{occs}$ $\;M$.$p$}
    Inline just the specified occurrences of $M$.$p$, where
    $\mathit{occs}$ is a parenthesized nonempty sequence of positive
    numbers \ec{($n_1$ $\;\cdots$ $\;n_l$)}. E.g., \ec{(1 3)} means the
    first and third occurrences of the procedure.  In the \prhl case,
    a side \ec{\{1\}} or \ec{\{2\}} must be specified.
  \end{tsyntax}

  \bigskip
  For example, given the declarations
  \ecinput{examps/tactics/inline/1.ec}{}{3-18}{}
  if the current goal is
  \ecinput{examps/parts/tactics/inline/1-1.0.ec}{}{}{}{} then
  running \ecinput{examps/parts/tactics/inline/1-1.ec}{}{}{}{}
  produces the goal
  \ecinput{examps/parts/tactics/inline/1-1.1.ec}{}{}{}{}
  From which running
  \ecinput{examps/parts/tactics/inline/1-2.ec}{}{}{}{}
  produces the goal
  \ecinput{examps/parts/tactics/inline/1-2.1.ec}{}{}{}{}
  From which running
  \ecinput{examps/parts/tactics/inline/1-3.ec}{}{}{}{}
  produces the goal
  \ecinput{examps/parts/tactics/inline/1-3.1.ec}{}{}{}{}
  And, if the current goal is
  \ecinput{examps/parts/tactics/inline/2-1.0.ec}{}{}{}{} then
  running \ecinput{examps/parts/tactics/inline/2-1.ec}{}{}{}{}
  produces the goal
  \ecinput{examps/parts/tactics/inline/2-1.1.ec}{}{}{}{}
\end{tactic}


% --------------------------------------------------------------------
\begin{tactic}{rcondf}
  \begin{tsyntax}{rcondf $\;n$}
    If the goal's conclusion is an \hl statement judgement whose $n$th
    statement is an \ec{if} statement, reduce the goal to two
    subgoals.
    \begin{itemize}
    \item One whose concludion is an \hl statement judgement whose
      precondition is the original goal's precondition, program is the
      first $n-1$ statements of the original goal's program, and
      postcondition is the negation of the boolean expression of the
      \ec{if} statement.
   
    \item One whose conclusion is an \hl statement judgement that's
      the same as that of the original goal except that the \ec{if}
      statement has been replaced by its \ec{else} part.
    \end{itemize}

    \medskip For example, if the current goal is
    \ecinput{examps/parts/tactics/rcondf/1-1.0.ec}{}{}{}{} then
    running \ecinput{examps/parts/tactics/rcondf/1-1.ec}{}{}{}{}
    produces the goals
    \ecinput{examps/parts/tactics/rcondf/1-1.1.ec}{}{}{}{} and
    \ecinput{examps/parts/tactics/rcondf/1-1.2.ec}{}{}{}{}
  \end{tsyntax}

  \begin{tsyntax}{rcondf\{1\} $\;n$ | rcondf\{2\} $\;n$}
    If the goal's conclusion is a \prhl statement judgement where the
    $n$th statement of the designated program is an \ec{if} statement,
    reduce the goal to two subgoals.
    \begin{itemize}
    \item One whose conclusion is an \hl statement judgement whose
      precondition is the original goal's precondition, program is the
      first $n-1$ statements of the original goal's designated
      program, and postcondition is the negation of the boolean
      expression of the \ec{if} statement. Actually, the \hl statement
      judgement is universally quantified by a memory of the
      non-designated program, and references in the precondition to
      variables of the non-designated program are interpreted in that
      memory.
   
    \item One whose conclusion is a \prhl statement judgement that's
      the same as that of the original goal except that the \ec{if}
      statement has been replaced by its \ec{else} part.
    \end{itemize}

  \medskip
  For example, if the current goal is
  \ecinput{examps/parts/tactics/rcondf/1-2.0.ec}{}{}{}{} then
  running \ecinput{examps/parts/tactics/rcondf/1-2.ec}{}{}{}{}
  produces the goals
  \ecinput{examps/parts/tactics/rcondf/1-2.1.ec}{}{}{}{}
  and
  \ecinput{examps/parts/tactics/rcondf/1-2.2.ec}{}{}{}{}
  \end{tsyntax}

  \begin{tsyntax}{rcondf $\;n$}
    If the goal's conclusion is an \hl statement judgement whose $n$th
    statement is a \ec{while} statement, reduce the goal to two
    subgoals.
    \begin{itemize}
    \item One whose concludion is an \hl statement judgement whose
      precondition is the original goal's precondition, program is the
      first $n-1$ statements of the original goal's program, and
      postcondition is the negation of the boolean expression of the
      \ec{while} statement.
   
    \item One whose conclusion is an \hl statement judgement that's
      the same as that of the original goal except that the \ec{while}
      statement has been removed.
    \end{itemize}

    \medskip For example, if the current goal is
    \ecinput{examps/parts/tactics/rcondf/2-1.0.ec}{}{}{}{} then
    running \ecinput{examps/parts/tactics/rcondf/2-1.ec}{}{}{}{}
    produces the goals
    \ecinput{examps/parts/tactics/rcondf/2-1.1.ec}{}{}{}{} and
    \ecinput{examps/parts/tactics/rcondf/2-1.2.ec}{}{}{}{}
  \end{tsyntax}

  \begin{tsyntax}{rcondf\{1\} $\;n$ | rcondf\{2\} $\;n$}
    If the goal's conclusion is a \prhl statement judgement where the
    $n$th statement of the designated program is a \ec{while} statement,
    reduce the goal to two subgoals.
    \begin{itemize}
    \item One whose conclusion is an \hl statement judgement whose
      precondition is the original goal's precondition, program is the
      first $n-1$ statements of the original goal's designated
      program, and postcondition is the negation of the boolean
      expression of the \ec{while} statement. Actually, the \hl
      statement judgement is universally quantified by a memory of the
      non-designated program, and references in the precondition to
      variables of the non-designated program are interpreted in that
      memory.
   
    \item One whose conclusion is a \prhl statement judgement that's
      the same as that of the original goal except that the \ec{while}
      statement has been removed.
    \end{itemize}

  \medskip
  For example, if the current goal is
  \ecinput{examps/parts/tactics/rcondf/2-2.0.ec}{}{}{}{} then
  running \ecinput{examps/parts/tactics/rcondf/2-2.ec}{}{}{}{}
  produces the goals
  \ecinput{examps/parts/tactics/rcondf/2-2.1.ec}{}{}{}{}
  and
  \ecinput{examps/parts/tactics/rcondf/2-2.2.ec}{}{}{}{}
  \end{tsyntax}
\end{tactic}

% --------------------------------------------------------------------
\begin{tactic}{rcondt}
  \begin{tsyntax}{rcondt $\;n$}
    If the goal's conclusion is an \hl statement judgement whose $n$th
    statement is an \ec{if} statement, reduce the goal to two
    subgoals.
    \begin{itemize}
    \item One whose concludion is an \hl statement judgement whose
      precondition is the original goal's precondition, program is the
      first $n-1$ statements of the original goal's program, and
      postcondition is the boolean expression of the \ec{if}
      statement.
   
    \item One whose conclusion is an \hl statement judgement that's
      the same as that of the original goal except that the \ec{if}
      statement has been replaced by its then part.
    \end{itemize}

    \medskip For example, if the current goal is
    \ecinput{../examps/parts/tactics/rcondt/1-1.0.ec}{}{}{}{} then
    running \ecinput{../examps/parts/tactics/rcondt/1-1.ec}{}{}{}{}
    produces the goals
    \ecinput{../examps/parts/tactics/rcondt/1-1.1.ec}{}{}{}{} and
    \ecinput{../examps/parts/tactics/rcondt/1-1.2.ec}{}{}{}{}
  \end{tsyntax}

  \begin{tsyntax}{rcondt\{1\} $\;n$ | rcondt\{2\} $\;n$}
    If the goal's conclusion is a \prhl statement judgement where the
    $n$th statement of the designated program is an \ec{if} statement,
    reduce the goal to two subgoals.
    \begin{itemize}
    \item One whose conclusion is an \hl statement judgement whose
      precondition is the original goal's precondition, program is the
      first $n-1$ statements of the original goal's designated
      program, and postcondition is the boolean expression of the
      \ec{if} statement. Actually, the \hl statement judgement is
      universally quantified by a memory of the non-designated
      program, and references in the precondition to variables of the
      non-designated program are interpreted in that memory.
   
    \item One whose conclusion is a \prhl statement judgement that's
      the same as that of the original goal except that the \ec{if}
      statement has been replaced by its then part.
    \end{itemize}

  \medskip
  For example, if the current goal is
  \ecinput{../examps/parts/tactics/rcondt/1-2.0.ec}{}{}{}{} then
  running \ecinput{../examps/parts/tactics/rcondt/1-2.ec}{}{}{}{}
  produces the goals
  \ecinput{../examps/parts/tactics/rcondt/1-2.1.ec}{}{}{}{}
  and
  \ecinput{../examps/parts/tactics/rcondt/1-2.2.ec}{}{}{}{}
  \end{tsyntax}
\end{tactic}


% --------------------------------------------------------------------
\begin{tactic}{splitwhile}
  \begin{tsyntax}{splitwhile $\;n$ : $\;e$}
    If the goal's conclusion is an \hl statement judgement whose $n$th
    statement is a \ec{while} statement and $e$ is a well-typed
    boolean expression in the \ec{while} statement's context, then
    insert before the \ec{while} statement a copy of the \ec{while}
    statement in which $e$ is added as a conjunct of the statement's
    boolean expression.

    \medskip For example, if the current goal is
    \ecinput{../examps/parts/tactics/splitwhile/1-1.0.ec}{}{}{}{} then
    running \ecinput{../examps/parts/tactics/splitwhile/1-1.ec}{}{}{}{}
    produces the goal
    \ecinput{../examps/parts/tactics/splitwhile/1-1.1.ec}{}{}{}{}
  \end{tsyntax}

  \begin{tsyntax}{splitwhile\{1\} $\;n$ : $\;e$ | splitwhile\{2\} $\;n$ : $\;e$}
    If the goal's conclusion is a \prhl statement judgement where the
    $n$th statement of the designated program is a \ec{while}
    statement and $e$ is a well-typed boolean expression in the
    \ec{while} statement's context, then insert before the \ec{while}
    statement a copy of the \ec{while} statement in which $e$ is added
    as a conjunct of the statement's boolean expression.

    \medskip For example, if the current goal is
    \ecinput{../examps/parts/tactics/splitwhile/1-2.0.ec}{}{}{}{} then
    running \ecinput{../examps/parts/tactics/splitwhile/1-2.ec}{}{}{}{}
    produces the goal
    \ecinput{../examps/parts/tactics/splitwhile/1-2.1.ec}{}{}{}{}
    from which
    running \ecinput{../examps/parts/tactics/splitwhile/1-3.ec}{}{}{}{}
    produces the goal
    \ecinput{../examps/parts/tactics/splitwhile/1-3.1.ec}{}{}{}{}
  \end{tsyntax}
\end{tactic}

% --------------------------------------------------------------------
\begin{tactic}{unroll}
  \begin{tsyntax}{unroll $\;c$}
    If the goal's conclusion is an \hl statement judgement whose $c$th
    statement is a \ec{while} statement, then insert before that
    statement an \ec{if} statement whose boolean expression is the
    \ec{while} statement's boolean expression, whose \ec{then} part is
    the \ec{while} statements's body, and whose \ec{else} part is
    empty.

    \medskip For example, if the current goal is
    \ecinput{examps/parts/tactics/unroll/1-1.0.ec} then
    running \ecinput{examps/parts/tactics/unroll/1-1.ec}
    produces the goal
    \ecinput{examps/parts/tactics/unroll/1-1.1.ec}
    And, if the current goal is
    \ecinput{examps/parts/tactics/unroll/2-1.0.ec} then
    running \ecinput{examps/parts/tactics/unroll/2-1.ec}
    produces the goal
    \ecinput{examps/parts/tactics/unroll/2-1.1.ec}
  \end{tsyntax}

  \begin{tsyntax}{unroll\{1\} $\;c$ | unroll\{2\} $\;c$}
    If the goal's conclusion is an \prhl statement judgement where the
    $c$th statement of the designated program is a \ec{while}
    statement, then insert before that statement an \ec{if} statement
    whose boolean expression is the \ec{while} statement's boolean
    expression, whose \ec{then} part is the \ec{while} statements's
    body, and whose \ec{else} part is empty.

    \medskip For example, if the current goal is
    \ecinput{examps/parts/tactics/unroll/1-2.0.ec} then
    running \ecinput{examps/parts/tactics/unroll/1-2.ec}
    produces the goal
    \ecinput{examps/parts/tactics/unroll/1-2.1.ec}
    from which
    running \ecinput{examps/parts/tactics/unroll/1-3.ec}
    produces the goal
    \ecinput{examps/parts/tactics/unroll/1-3.1.ec}
  \end{tsyntax}
\end{tactic}

% --------------------------------------------------------------------
\begin{tactic}{fission}
  \begin{tsyntax}{fission $\;c$!$l$ @ $\;m$, $\;n$}
    \hl statement judgement version.  Fails unless $0\leq l$ and
    $0\leq m\leq n$ and the $c$th statement of the program is a
    \ec{while} statement, and there are at least $l$ statements right
    before the \ec{while} statement, at its level, and the body of the
    \ec{while} statement has at least $n$ statements.

    Let
    \begin{itemize}
    \item $s_1$ be the $l$ statements before the \ec{while} statement
      at position $c$;

    \item $e$ be the boolean expression of the \ec{while} statement;

    \item $s_2$ be the first $m$ statements of the body of the
      \ec{while} statement;

    \item $s_3$ be the next $n - m$ statements of the body of the
      \ec{while} statement;

    \item $s_4$ be the rest of the body of the \ec{while} statement.
    \end{itemize}
    Fails unless:
    \begin{itemize}
    \item $e$ doesn't reference the variables written by $s_2$ and $s_3$;

    \item $s_1$ and $s_4$ don't read or write the variables written by
      $s_2$ and $s_3$;

    \item $s_2$ and $s_3$ don't write the variables written by $s_1$
      and $s_4$;

    \item $s_2$ and $s_3$ don't read or write the variables written by
      the other.
    \end{itemize}
    The tactic replaces
\begin{easycrypt}{}{}
#$s_1$# while (#$e$#) { #$s_2$# #$s_3$# #$s_4$# }
\end{easycrypt}
    by
\begin{easycrypt}{}{}
#$s_1$# while (#$e$#) { #$s_2$# #$s_4$# }
#$s_1$# while (#$e$#) { #$s_3$# #$s_4$# }
\end{easycrypt}

    \medskip For example, if the current goal is
    \ecinput{../examps/parts/tactics/fission/1-1.0.ec}{}{}{}{} then
    running \ecinput{../examps/parts/tactics/fission/1-1.ec}{}{}{}{}
    produces the goal
    \ecinput{../examps/parts/tactics/fission/1-1.1.ec}{}{}{}{}
  \end{tsyntax}

  \begin{tsyntax}{fission $\;c$ @ $\;m$, $\;n$}
    Equivalent to \ec{fission $\;c$!1 @ $\;m$, $\;n$}.
  \end{tsyntax}

  \begin{tsyntax}{fission\{1\} $\;\cdots$ | fission\{2\} $\;\cdots$}
    The \prhl versions of the above variants, working on the
    designated program.
  \end{tsyntax}
\end{tactic}

% --------------------------------------------------------------------
\begin{tactic}{fusion}
  \begin{tsyntax}{fusion $\;c$!$l$ @ $\;m$, $\;n$}
    \hl statement judgement version.  Fails unless $0\leq l$ and
    $0\leq m$ and $0\leq n$ and the $c$th statement of the program
    is a \ec{while} statement, and there are at least $l$ statements right
    before the \ec{while} statement, at its level,
    and the part of the program beginning from the $l$ statements
    before the while loop may be uniquely matched against
\begin{easycrypt}{}{}
#$s_1$# while (#$e$#) { #$s_2$# #$s_4$# }
#$s_1$# while (#$e$#) { #$s_3$# #$s_4$# }
\end{easycrypt}
    where:
    \begin{itemize}
    \item $s_1$ has length $l$;

    \item $s_2$ has length $m$;

    \item $s_3$ has length $n$;

    \item $e$ doesn't reference the variables written by $s_2$ and $s_3$;

    \item $s_1$ and $s_4$ don't read or write the variables written by
      $s_2$ and $s_3$;

    \item $s_2$ and $s_3$ don't write the variables written by $s_1$
      and $s_4$;

    \item $s_2$ and $s_3$ don't read or write the variables written by
      the other.
    \end{itemize}
    The tactic replaces
\begin{easycrypt}{}{}
#$s_1$# while (#$e$#) { #$s_2$# #$s_4$# }
#$s_1$# while (#$e$#) { #$s_3$# #$s_4$# }
\end{easycrypt}
by
\begin{easycrypt}{}{}
#$s_1$# while (#$e$#) { #$s_2$# #$s_3$# #$s_4$# }
\end{easycrypt}

    \medskip For example, if the current goal is
    \ecinput{examps/parts/tactics/fusion/1-1.0.ec}{}{}{}{} then
    running \ecinput{examps/parts/tactics/fusion/1-1.ec}{}{}{}{}
    produces the goal
    \ecinput{examps/parts/tactics/fusion/1-1.1.ec}{}{}{}{}
  \end{tsyntax}

  \begin{tsyntax}{fusion $\;c$ @ $\;m$, $\;n$}
    Equivalent to \ec{fusion $\;c$!1 @ $\;m$, $\;n$}.
  \end{tsyntax}

  \begin{tsyntax}{fusion\{1\} $\;\cdots$ | fusion\{2\} $\;\cdots$}
    The \prhl versions of the above variants, working on the
    designated program.
  \end{tsyntax}
\end{tactic}


% --------------------------------------------------------------------
\begin{tactic}{alias}
  \begin{tsyntax}{alias $\;c$ with $\;x$}
    If the goal's conclusion is an \hl statement judgement whose
    program's $c$th statement is an assignment statement, and $x$ is
    an identifier, then replace the assignment statement by the
    following two statements:
    \begin{itemize}
    \item an assignment statement of the same kind as the original
      assignment statement (ordinary, random, procedure call) whose
      left-hand side is $x$, and whose right-hand side is the
      right-hand side of the original assignment statement;

    \item an ordinary assignment statement whose left-hand side is
      the left-hand side of the original assignment statement,
      and whose right-hand side is $x$.
    \end{itemize}
    If $x$ is a local variable of the program, a fresh name is
    generated by adding digits to the end of $x$.
  \end{tsyntax}

  \begin{tsyntax}{alias $\;c$}
    Equivalent to \ec{alias $\;c$ with x}.
  \end{tsyntax}

  \begin{tsyntax}{alias $\;c$ $\;x$ = $\;e$}
    If the program has an $c$th statement, and the expression $e$ is
    well-typed in the context of the program, insert before the $c$th
    statement an ordinary assignment statement whose left-hand side is
    $x$ and whose right-hand side is $e$.
    If $x$ is a local variable of the program, a fresh name is
    generated by adding digits to the end of $x$.
  \end{tsyntax}

  \begin{tsyntax}{alias\{1\} $\;\cdots$ | alias\{2\} $\;\cdots$}
    The \prhl versions of the preceding forms, where the aliasing
    is done in the designated program.
  \end{tsyntax}

  \bigskip For example, if the current goal is
  \ecinput{examps/parts/tactics/alias/1-1.0.ec} then running
  \ecinput{examps/parts/tactics/alias/1-1.ec} produces the
  goal \ecinput{examps/parts/tactics/alias/1-1.1.ec}
  from which running
  \ecinput{examps/parts/tactics/alias/1-2.ec} produces the
  goal \ecinput{examps/parts/tactics/alias/1-2.1.ec}
  from which running
  \ecinput{examps/parts/tactics/alias/1-3.ec} produces the
  goal \ecinput{examps/parts/tactics/alias/1-3.1.ec}
\end{tactic}

% --------------------------------------------------------------------
\begin{tactic}{cfold}
  \begin{tsyntax}{cfold $\;c$ ! $\;m$}
    Fails unless $n\geq 1$ and $m\geq 0$.  If the goal's conclusion is
    an \hl statement judgement in which statement $c$ of the
    judgement's program is an ordinary assignment statement in which
    constant values are assigned to local identifiers, and the
    following statement block of length $m$ does not write any of
    those identifiers, then replace all occurrences of the assigned
    identifiers in that statement block by the constants assigned to
    them, and move the assignment statement to after the modified
    statement block.

    \medskip For example, if the current goal is
    \ecinput{../examps/parts/tactics/cfold/1-1.0.ec}{}{}{}{} then
    running \ecinput{../examps/parts/tactics/cfold/1-1.ec}{}{}{}{}
    produces the goal
    \ecinput{../examps/parts/tactics/cfold/1-1.1.ec}{}{}{}{}
    from which
    running \ecinput{../examps/parts/tactics/cfold/1-2.ec}{}{}{}{}
    produces the goal
    \ecinput{../examps/parts/tactics/cfold/1-2.1.ec}{}{}{}{}
    from which
    running \ecinput{../examps/parts/tactics/cfold/1-3.ec}{}{}{}{}
    produces the goal
    \ecinput{../examps/parts/tactics/cfold/1-3.1.ec}{}{}{}{}
    from which
    running \ecinput{../examps/parts/tactics/cfold/1-4.ec}{}{}{}{}
    produces the goal
    \ecinput{../examps/parts/tactics/cfold/1-4.1.ec}{}{}{}{}
  \end{tsyntax}

  \begin{tsyntax}{cfold\{1\} $\;c$ ! $\;m$ | cfold\{2\} $\;c$ ! $\;m$}
    Like the \hl version, but operating on the designed program of
    a \prhl judgement's conclusion.
  \end{tsyntax}

  \begin{tsyntax}{cfold\{1\} $\;c$ | cfold\{2\} $\;c$}
    Like the general cases, but where $m$ is set so as to be the
    number of statements after the assignment statement.
  \end{tsyntax}
\end{tactic}

% --------------------------------------------------------------------
\begin{tactic}{kill}
  \begin{tsyntax}{kill $\;c$ ! $\;m$}
    Fails unless $m\geq 0$.  If the goal's conclusion is
    an \hl statement judgement whose program has a statement block
    starting at position $c$ and having length $m$ (when $m = 0$, this
    block is empty), and the variables written by this statement block
    aren't used in the judgement's postcondition or read by the rest
    of the program, then reduce the goal to two subgoals.
    \begin{itemize}
    \item One whose conclusion is a \phl statement judgement whose pre-
       and postconditions are \ec{true}, whose program is the
       statement block, and whose bound part is \ec{= 1\%r}.

    \item One that's identical to the original goal except that the
      statement block has been removed.
    \end{itemize}

    \medskip For example, if the current goal is
    \ecinput{examps/parts/tactics/kill/1-1.0.ec} then
    running \ecinput{examps/parts/tactics/kill/1-1.ec}
    produces the goals
    \ecinput{examps/parts/tactics/kill/1-1.1.ec}
    and
    \ecinput{examps/parts/tactics/kill/1-1.2.ec}
  \end{tsyntax}

  \begin{tsyntax}{kill\{1\} $\;c$ ! $\;m$ | kill\{2\} $\;c$ ! $\;m$}
    Like the \hl case but for \prhl judgements, where the statement
    block to be killed is in the designated program.

    \medskip For example, if the current goal is
    \ecinput{examps/parts/tactics/kill/1-2.0.ec} then
    running \ecinput{examps/parts/tactics/kill/1-2.ec}
    produces the goals
    \ecinput{examps/parts/tactics/kill/1-2.1.ec}
    and
    \ecinput{examps/parts/tactics/kill/1-2.2.ec}
  \end{tsyntax}

  \begin{tsyntax}{kill $\;c$ | kill\{1\} $\;c$ | kill\{2\} $\;c$}
    Like the general cases, but with $m = 1$.
  \end{tsyntax}

  \begin{tsyntax}{kill $\;c$ ! * | kill\{1\} $\;c$ ! * | kill\{2\} $\;c$ ! *}
    Like the general cases, but with $m$ set so that the statement
    block to be killed is the rest of the current level of the
    (designated) program.
  \end{tsyntax}
\end{tactic}

\input{tactics/modpath.tex}

\subsection{Tactics for Reasoning about Specifications}

% --------------------------------------------------------------------
\begin{tactic}{symmetry}
  \begin{tsyntax}{symmetry}
    If the goal's conclusion is a \prhl (statement) judgement, swap
    the two programs, transforming the pre- and postconditions by
    swapping the memories they refer to.

    \medskip For example, if the current goal is
    \ecinput{examps/parts/tactics/symmetry/1-1.0.ec} then
    running \ecinput{examps/parts/tactics/symmetry/1-1.ec}
    produces the goal
    \ecinput{examps/parts/tactics/symmetry/1-1.1.ec}
    And, if the current goal is
    \ecinput{examps/parts/tactics/symmetry/1-2.0.ec} then
    running \ecinput{examps/parts/tactics/symmetry/1-2.ec}
    produces the goal
    \ecinput{examps/parts/tactics/symmetry/1-2.1.ec}.
  \end{tsyntax}
\end{tactic}

% --------------------------------------------------------------------
\begin{tactic}{transitivity}
  \begin{tsyntax}{transitivity $\;N$.$r$ ($P_1$ ==> $\ Q_1$) ($P_2$ ==> $\ Q_2$)}
    Reduces a goal whose conclusion is a \prhl judgement
    (\emph{not} statement judgement)
    \begin{center}
      \ec{equiv[$M_1$.$p_1$ ~ $\;M_2$.$p_2$ : $\;P$ ==> $\;Q$]}
    \end{center}
    to goals whose conclusions are
    \begin{itemize}
    \item \ec{equiv[$M_1$.$p_1$ ~ $\;N$.$r$ : $\;P_1$ ==> $\;Q_1$]} and
    \item \ec{equiv[$N$.$r$ ~ $\;M_2$.$p_2$ : $\;P_2$ ==> $\;Q_2$]},
    \end{itemize}
    preceded by two auxiliary goals. The tactic fails if the $P_i$ and
    $Q_i$ aren't compatible with these left and right programs.  The
    conclusion of the first auxiliary goal checks that $P$ implies the
    conjunction of $P_1$ and $P_2$, where each corresponding pair of
    \ec{&2} memory references in $P_1$ and \ec{&1} reference in $P_2$
    is existentially quantified.  And the conclusion of the second
    auxilary goal checks that the conjuction of $Q_1$ and $Q_2$ implies
    $Q$, where each corresponding pair of \ec{&2} references in $Q_1$
    and \ec{&1} references in $Q_2$ are universally quantified.

  \medskip For example, consider the modules
  \ecinput{examps/tactics/transitivity/1.ec}{}{6-30}{}
  If the current goal is
  \ecinput{examps/parts/tactics/transitivity/1-1.0.ec}{}{}{}{} then
  running \ecinput{examps/parts/tactics/transitivity/1-1.ec}{}{}{}{}
  produces the four goals
  \ecinput{examps/parts/tactics/transitivity/1-1.1.ec}{}{}{}{}
  and
  \ecinput{examps/parts/tactics/transitivity/1-1.2.ec}{}{}{}{}
  and
  \ecinput{examps/parts/tactics/transitivity/1-1.3.ec}{}{}{}{}
  and
  \ecinput{examps/parts/tactics/transitivity/1-1.4.ec}{}{}{}{}
  \end{tsyntax}

  \begin{tsyntax}{transitivity\{$i$\} \{$s$\} ($P_1$ ==> $\ Q_1$) ($P_2$ ==> $\ Q_2$)}
    
    Reduces a goal whose conclusion is a \prhl \emph{statement}
    judgement with precondition $P$, postcondition $Q$, left program
    (statement sequence) $s_1$ and right program $s_2$ to goals whose
    conclusions are \prhl statement judgements:
    \begin{itemize}
    \item \ec{equiv[$s_1$ ~ $\;s$ : $\;P_1$ ==> $\;Q_1$]} and
    \item \ec{equiv[$s$ ~ $\;s_2$ : $\;P_2$ ==> $\;Q_2$]},
    \end{itemize}
    preceded by two auxiliary goals. If the side $i=1$, then the
    statement sequence $s$ may only use variables and unqualified
    procedures of the left program; when $i=2$, it may only use
    variables and unqualified procedures of the right program. The
    tactic fails if the $P_i$ and $Q_i$ aren't compatible with these
    left and right programs.  The conclusion of the first auxiliary
    goal checks that $P$ implies the conjunction of $P_1$ and $P_2$,
    where each corresponding pair of \ec{&2} memory references in
    $P_1$ and \ec{&1} reference in $P_2$ is existentially quantified.
    And the conclusion of the second auxilary goal checks that the
    conjuction of $Q_1$ and $Q_2$ implies $Q$, where each
    corresponding pair of \ec{&2} references in $Q_1$ and \ec{&1}
    references in $Q_2$ are universally quantified.

  \medskip Consider the modules \ec{M}, \ec{N} and \ec{R} of the
  preceding case.  If the current goal is
  \ecinput{examps/parts/tactics/transitivity/1-2.0.ec}{}{}{}{} then
  running \ecinput{examps/parts/tactics/transitivity/1-2.ec}{}{}{}{}
  produces the four goals
  \ecinput{examps/parts/tactics/transitivity/1-2.1.ec}{}{}{}{} and
  \ecinput{examps/parts/tactics/transitivity/1-2.2.ec}{}{}{}{} and
  \ecinput{examps/parts/tactics/transitivity/1-2.3.ec}{}{}{}{} and
  \ecinput{examps/parts/tactics/transitivity/1-2.4.ec}{}{}{}{} And, if
  the current goal is
  \ecinput{examps/parts/tactics/transitivity/1-3.0.ec}{}{}{}{} then
  running \ecinput{examps/parts/tactics/transitivity/1-3.ec}{}{}{}{}
  produces the four goals
  \ecinput{examps/parts/tactics/transitivity/1-3.1.ec}{}{}{}{} and
  \ecinput{examps/parts/tactics/transitivity/1-3.2.ec}{}{}{}{} and
  \ecinput{examps/parts/tactics/transitivity/1-3.3.ec}{}{}{}{} and
  \ecinput{examps/parts/tactics/transitivity/1-3.4.ec}{}{}{}{}
  \end{tsyntax}
\end{tactic}

%%  \paragraph{Examples:}\strut
%%
%%  \begin{cmathpar}
%%  \texample[\prhl{}]{\ec{transitivity f ($P_1$ ==> $\ Q_1$) ($P_2$ ==> $\ Q_2$)}}
%%    {\forall \mem{m_1}\ \mem{m_2}.\, \mem{m_1} \rel{P} \mem{m_2} \Rightarrow
%%        \exists \mem{m}.\, \mem{m_1} \rel{P_1} \mem{m}
%%                           \wedge \mem{m} \rel{P_2} \mem{m_2} \\
%%     \forall \mem{m_1}\ \mem{m}\ \mem{m_2}.\,
%%        \mem{m_1} \rel{Q_1} \mem{m} \Rightarrow
%%        \mem{m}   \rel{Q_2} \mem{m_2} \Rightarrow
%%        \mem{m_1} \rel{Q}   \mem{m_2} \\
%%     \pRHL{P_1}{f_1}{f}{Q_1} \\
%%     \pRHL{P_2}{f}{f_2}{Q_2}}
%%    {\pRHL{P}{f_1}{f_2}{Q}}
%%
%%  \texample[\prhl{}]
%%    {\ec{transitivity \{1\} \{ s \} ($P_1$ ==> $\ Q_1$) ($P_2$ ==> $\ Q_2$)}}
%%    {\forall \mem{m_1}\ \mem{m_2}.\, \mem{m_1} \rel{P} \mem{m_2} \Rightarrow
%%        \exists \mem{m}.\, \mem{m_1} \rel{P_1} \mem{m}
%%                           \wedge \mem{m} \rel{P_2} \mem{m_2} \\
%%     \forall \mem{m_1}\ \mem{m}\ \mem{m_2}.\,
%%        \mem{m_1} \rel{Q_1} \mem{m} \Rightarrow
%%        \mem{m}   \rel{Q_2} \mem{m_2} \Rightarrow
%%        \mem{m_1} \rel{Q}   \mem{m_2} \\
%%     \pRHL{P_1}{s_1}{s}{Q_1} \\
%%     \pRHL{P_2}{s}{s_2}{Q_2}}
%%    {\pRHL{P}{s_1}{s_2}{Q}}
%%  \end{cmathpar}
%%
%%  \textbf{Note:} In practice, the existential quantification over
%%  memory $\mem{m}$ in the first generated subgoal is replaced with an
%%  existential quantification over the program variables appearing in $P$,
%%  $P_1$, or $P_2$.
%%  \end{tsyntax}

% --------------------------------------------------------------------
\begin{tactic}{conseq}
  \begin{tsyntax}{conseq (_ : $\;P$ ==> $\;Q$)}

    If the goal's conclusion is a \prhl or \hl judgement or statement
    judgement, weaken the conclusion's precondition to $P$, and
    strengthen its postcondition to $Q$, generating initial auxilary
    subgoals checking that this reduction is sound. Fails if $P$ and
    $Q$ aren't compatible with the judgement type. The conclusion of
    the first auxiliary subgoal checks that the precondition $P'$ of
    the original goal's conclusion implies $P$. The conclusion of the
    second auxiliary subgoal checks that $Q$ implies the postcondition
    $Q'$ of the original goal's conclusion, except that any memory
    references to variables that may be modified by the conclusion's
    program(s) are universally quantified in $Q$ and $Q'$, and $P$ is
    also included as an assumption.

    $P$ or $Q$ may be replaced by \ec{_}, in which case the pre- or
    postcondition is left unchanged. When a pre- or postcondition is
    unchanged, the corresponding auxilary subgoal isn't generated (as
    its proof would be trivial). When the goal's conclusion is a
    judgement (not a statement judgement), a proof term whose
    conclusion is a judgement on the procedure(s) of the original
    goals's conclusion may be supplied as the argument to \ec{conseq},
    in which case $P$ and $Q$ are taken to be the pre- and
    postconditions of this judgement, and only the auxilary subgoals
    are generated.

    \medskip For example, if the current goal is
    \ecinput{examps/parts/tactics/conseq/1-1.0.ec}{}{}{}{} then
    running \ecinput{examps/parts/tactics/conseq/1-1.ec}{}{}{}{}
    produces the goals
    \ecinput{examps/parts/tactics/conseq/1-1.1.ec}{}{}{}{}
    and
    \ecinput{examps/parts/tactics/conseq/1-1.2.ec}{}{}{}{}
    Continuing from the last of these goals,
    running \ecinput{examps/parts/tactics/conseq/1-2.ec}{}{}{}{}
    produces the goals
    \ecinput{examps/parts/tactics/conseq/1-2.1.ec}{}{}{}{}
    and
    \ecinput{examps/parts/tactics/conseq/1-2.2.ec}{}{}{}{}
    Continuing from the last of these goals,
    running \ecinput{examps/parts/tactics/conseq/1-3.ec}{}{}{}{}
    produces the goals
    \ecinput{examps/parts/tactics/conseq/1-3.1.ec}{}{}{}{}
    and
    \ecinput{examps/parts/tactics/conseq/1-3.2.ec}{}{}{}{}
    and
    \ecinput{examps/parts/tactics/conseq/1-3.3.ec}{}{}{}{}

    \medskip If the current goal is
    \ecinput{examps/parts/tactics/conseq/1-4.0.ec}{}{}{}{} then
    running \ecinput{examps/parts/tactics/conseq/1-1.ec}{}{}{}{}
    produces the goals
    \ecinput{examps/parts/tactics/conseq/1-4.1.ec}{}{}{}{}
    and
    \ecinput{examps/parts/tactics/conseq/1-4.2.ec}{}{}{}{}
    Continuing from the last of these goals,
    running \ecinput{examps/parts/tactics/conseq/1-5.ec}{}{}{}{}
    produces the goals
    \ecinput{examps/parts/tactics/conseq/1-5.1.ec}{}{}{}{}
    and
    \ecinput{examps/parts/tactics/conseq/1-5.2.ec}{}{}{}{}
    Continuing from the last of these goals,
    running \ecinput{examps/parts/tactics/conseq/1-6.ec}{}{}{}{}
    produces the goals
    \ecinput{examps/parts/tactics/conseq/1-6.1.ec}{}{}{}{}
    and
    \ecinput{examps/parts/tactics/conseq/1-6.2.ec}{}{}{}{}
    and
    \ecinput{examps/parts/tactics/conseq/1-6.3.ec}{}{}{}{}

    Given lemma
    \ecinput{examps/tactics/conseq/1.ec}{}{66-69}{}
    if the current goal is
    \ecinput{examps/parts/tactics/conseq/1-7.0.ec}{}{}{}{} then
    running \ecinput{examps/parts/tactics/conseq/1-7.ec}{}{}{}{}
    produces the goals
    \ecinput{examps/parts/tactics/conseq/1-7.1.ec}{}{}{}{}
    and
    \ecinput{examps/parts/tactics/conseq/1-7.2.ec}{}{}{}{}

    \medskip If the current goal is
    \ecinput{examps/parts/tactics/conseq/1-8.0.ec}{}{}{}{} then
    running \ecinput{examps/parts/tactics/conseq/1-8.ec}{}{}{}{}
    produces the goals
    \ecinput{examps/parts/tactics/conseq/1-8.1.ec}{}{}{}{}
    and
    \ecinput{examps/parts/tactics/conseq/1-8.2.ec}{}{}{}{}
    and
    \ecinput{examps/parts/tactics/conseq/1-8.3.ec}{}{}{}{}
    If the current goal is
    \ecinput{examps/parts/tactics/conseq/1-9.0.ec}{}{}{}{} then
    running \ecinput{examps/parts/tactics/conseq/1-9.ec}{}{}{}{}
    produces the goals
    \ecinput{examps/parts/tactics/conseq/1-9.1.ec}{}{}{}{}
    and
    \ecinput{examps/parts/tactics/conseq/1-9.2.ec}{}{}{}{}
    and
    \ecinput{examps/parts/tactics/conseq/1-9.3.ec}{}{}{}{}

    Given lemma
    \ecinput{examps/tactics/conseq/1.ec}{}{102-103}{}
    if the current goal is
    \ecinput{examps/parts/tactics/conseq/1-10.0.ec}{}{}{}{} then
    running \ecinput{examps/parts/tactics/conseq/1-10.ec}{}{}{}{}
    produces the goals
    \ecinput{examps/parts/tactics/conseq/1-10.1.ec}{}{}{}{}
    and
    \ecinput{examps/parts/tactics/conseq/1-10.2.ec}{}{}{}{}
  \end{tsyntax}

  \begin{tsyntax}{conseq (_ : $\;P$ ==> $\;Q$) (_ : $\;P_1$ ==>
      $\;Q_1$) (_ : $\;P_2$ ==> $\;Q_2$)}

    This form only applies to \prhl judgements and statement
    judgements, reducing the goal's conclusion to
    \begin{itemize}
    \item an \hl (statement) judgement for the left program with
      precondition $P_1$ and postcondition $Q_1$;

    \item an \hl (statement) judgement for the right program with
      precondition $P_2$ and postcondition $Q_2$; and

    \item a \prhl (statement) judgement whose precondition is $P$,
      postcondition is $Q$, and programs are as in the original
      judgement;

    \end{itemize}
    As before, auxiliary goals are generated whose conclusions
    check the validity of the reduction: that the original
    conclusion's precondition $P'$ implies the conjunction of
    $P$, $P_1$ and $P_2$; and that the conjunction of
    $Q$, $Q_1$ and $Q_2$ implies the original conclusion's
    postcondition $Q'$. One of the \hl specifications may be
    replaced by \ec{_}, which is equivalent to
    \ec{conseq (_ : true ==> true)}. And in the case of
    a \prhl judgement (not a statement judgement), proof
    terms may be used as the arguments to \ec{conseq}.

    \medskip For example, if the current goal is
    \ecinput{examps/parts/tactics/conseq/2-1.0.ec}{}{}{}{} then
    running \ecinput{examps/parts/tactics/conseq/2-1.ec}{}{}{}{}
    produces the goals
    \ecinput{examps/parts/tactics/conseq/2-1.1.ec}{}{}{}{}
    and
    \ecinput{examps/parts/tactics/conseq/2-1.2.ec}{}{}{}{}
    and
    \ecinput{examps/parts/tactics/conseq/2-1.3.ec}{}{}{}{}
    and
    \ecinput{examps/parts/tactics/conseq/2-1.4.ec}{}{}{}{}
    and
    \ecinput{examps/parts/tactics/conseq/2-1.5.ec}{}{}{}{}
    If the current goal is
    \ecinput{examps/parts/tactics/conseq/2-2.0.ec}{}{}{}{} then
    running \ecinput{examps/parts/tactics/conseq/2-2.ec}{}{}{}{}
    produces the goals
    \ecinput{examps/parts/tactics/conseq/2-2.1.ec}{}{}{}{}
    and
    \ecinput{examps/parts/tactics/conseq/2-2.2.ec}{}{}{}{}
    and
    \ecinput{examps/parts/tactics/conseq/2-2.3.ec}{}{}{}{}
    and
    \ecinput{examps/parts/tactics/conseq/2-2.4.ec}{}{}{}{}
    and
    \ecinput{examps/parts/tactics/conseq/2-2.5.ec}{}{}{}{}

    And given lemmas
    \ecinput{examps/tactics/conseq/2.ec}{}{60-60}{}
    and
    \ecinput{examps/tactics/conseq/2.ec}{}{63-63}{}
    and
    \ecinput{examps/tactics/conseq/2.ec}{}{66-66}{}
    if the current goal is
    \ecinput{examps/parts/tactics/conseq/2-3.0.ec}{}{}{}{} then
    running \ecinput{examps/parts/tactics/conseq/2-3.ec}{}{}{}{}
    produces the goals
    \ecinput{examps/parts/tactics/conseq/2-3.1.ec}{}{}{}{}
    and
    \ecinput{examps/parts/tactics/conseq/2-3.2.ec}{}{}{}{}
  \end{tsyntax}
\end{tactic}

%%  \paragraph{Examples:}\strut
%%  
%%  \begin{cmathpar}
%%  \texample[\prhl{}]
%%    {\ec{conseq (_: P ==> Q)}}
%%    {P' \Rightarrow P \\ Q \Rightarrow Q' \\ \pRHL{P}{c}{c'}{Q}}
%%    {\pRHL{P'}{c}{c'}{Q'}}
%%
%%  \texample[\phl{}]
%%    {\ec{conseq (_: P ==> Q: $\;\delta$)}}
%%    {P' \Rightarrow \delta \mathrel{\diamond} \delta' \\
%%     P' \Rightarrow P \\
%%     Q \mathrel{\diamond^\uparrow} Q' \\
%%     \pHL{P}{c}{Q}{\diamond}{\delta}}
%%    {\pHL{P'}{c}{Q'}{\diamond}{\delta'}}
%%
%%  \texample[\hl{}]
%%    {\ec{conseq (_: P ==> Q)}}
%%    {P' \Rightarrow P \\ Q \Rightarrow Q' \\ \HL{P}{c}{Q}}
%%    {\HL{P'}{c}{Q'}}
%%  \end{cmathpar}
%%
%%  \textbf{Note:} The \phl variant can also be used to strengthen the
%%  relation $\diamond$ into an equality by forcing the equality into
%%  the specification. For example, the following is a valid application
%%  of \ec{conseq}.
%%
%%  \begin{cmathpar}
%%  \texample[\phl{}]
%%    {\ec{conseq (_: P ==> Q: =$\;\delta$)}}
%%    {P' \Rightarrow \delta \mathrel{=} \delta' \\
%%     P' \Rightarrow P \\
%%     Q \mathrel{\Leftrightarrow} Q' \
%%     \pHL{P}{c}{Q}{=}{\delta}}
%%    {\pHL{P'}{c}{Q'}{\leq}{\delta'}}
%%  \end{cmathpar}  
%%  \end{tsyntax}
%%
%%  \begin{tsyntax}{conseq H}
%%  Only applies to judgments on procedures. Same as \ec{conseq} with a
%%  specification, but the specification to use is inferred from the
%%  lemma \ec{H} provided. Raises an error if the lemma does not refer
%%  to the expected procedure(s). All variants of \ec{conseq} may take
%%  lemmas in place of explicit specifications with the same effect, in
%%  which case they must be applied to judgments on procedures.
%%  \end{tsyntax}
%%
%%  \begin{tsyntax}{conseq*}
%%  Same as \ec{conseq}, but the subgoal corresponding to the
%%  postcondition is refined by a ``may modify'' analysis. All variants
%%  of \ec{conseq} can be refined using the \ec{*}, with the same
%%  effect.
%%  \end{tsyntax}
%%
%%  \begin{tsyntax}{conseq <prhl> <hl> <hl>}
%%  Combine relational and non-relational specifications to prove a
%%  relational specification. Either one of the Hoare logic
%%  specifications can be replaced with a wildcard.
%%
%%  \paragraph{Examples:}\strut
%%
%%  \begin{cmathpar}
%%  \texample[\prhl{}]
%%    {\ec{conseq (_: P ==> Q) (_: P$_1$ ==> Q$_1$) (_: P$_2$ ==> Q$_2$)}}
%%    {P' \Rightarrow P \wedge \inmem{P_1}{1} \wedge \inmem{P_2}{2} \\
%%     Q \wedge \inmem{Q_1}{1} \wedge \inmem{Q_2}{2} \Rightarrow Q' \\
%%     \HL{P_1}{c_1}{Q_1} \\
%%     \HL{P_2}{c_2}{Q_2} \\
%%     \pRHL{P}{c_1}{c_2}{Q}}
%%    {\pRHL{P'}{c_1}{c_2}{Q'}}
%%  \end{cmathpar}
%%  \end{tsyntax}
%%
%%  \begin{tsyntax}{conseq <prhl> <phl> | conseq <prhl> _ <phl>}
%%  Strengthen a relational specification where one of the programs is
%%  empty into a non-relational specification about the non-empty
%%  program. Fails if the program expected to be empty is not. For
%%  uniformity and simplicity of use, this variant also allows the user
%%  to strengthen the \prhl judgment before abstracting its relational
%%  aspects. The general rule below can be understood more clearly when
%%  $P = P'$ and $Q = Q'$.
%%
%%  \paragraph{Examples:}\strut
%%
%%  \begin{cmathpar}
%%  \texample[\prhl{}]
%%    {\ec{conseq (_: P ==> Q) (_: P$_1$ ==> Q$_1$: =1\%r)}}
%%    {P' \Rightarrow P   \\ Q   \Rightarrow Q' \\
%%     P  \Rightarrow P_1 \\ Q_1 \Rightarrow Q  \\
%%     \pHL{P_1}{c}{Q_1}{=}{1}}
%%    {\pRHL{P'}{c}{\Skip}{Q'}}
%%
%%  \texample[\prhl{}]
%%    {\ec{conseq (_: _ ==> _) _ (_: P$_2$ ==> Q$_2$: =1\%r)}}
%%    {P \Rightarrow P_2 \\ Q_2 \Rightarrow Q \\ \pHL{P_2}{c}{Q_2}{=}{1}}
%%    {\pRHL{P}{\Skip}{c}{Q}}
%%  \end{cmathpar}
%%  \end{tsyntax}

% --------------------------------------------------------------------
\begin{tactic}[case]{case-pl}
  \begin{tsyntax}{case $\;e$}
    If the goal's conclusion is a \prhl, \hl or \phl \emph{statement}
    judgement and $e$ is well-typed in the goal's context, split the
    goal into two goals:
    \begin{itemize}
    \item a first goal in which $e$ is added as a conjunct to the
      conclusion's precondition; and

    \item a second goal in which $!e$ is added as a conjunct to the
      conclusion's precondition.
    \end{itemize}

    \medskip For example, if the current goal is
    \ecinput{examps/parts/tactics/case_pl/1-1.0.ec}{}{}{}{} then
    running \ecinput{examps/parts/tactics/case_pl/1-1.ec}{}{}{}{}
    produces the goals
    \ecinput{examps/parts/tactics/case_pl/1-1.1.ec}{}{}{}{}
    and
    \ecinput{examps/parts/tactics/case_pl/1-1.2.ec}{}{}{}{}
    And if the current goal is
    \ecinput{examps/parts/tactics/case_pl/2-1.0.ec}{}{}{}{} then
    running \ecinput{examps/parts/tactics/case_pl/2-1.ec}{}{}{}{}
    produces the goals
    \ecinput{examps/parts/tactics/case_pl/2-1.1.ec}{}{}{}{}
    and
    \ecinput{examps/parts/tactics/case_pl/2-1.2.ec}{}{}{}{}
  \end{tsyntax}
\end{tactic}

\input{tactics/phoare_split.tex}
% --------------------------------------------------------------------
\begin{tactic}{byequiv}
  \begin{tsyntax}{byequiv (_ : $\;P$ ==> $\;Q$)}
    If the goal's conclusion has the form
    \begin{center}
      \ec{Pr[$M_1$.$p_1$($a_{1,1}$, $\ldots$, $a_{1,n_1}$) @ &$m_1$ : $\;E_1$] =
          Pr[$M_2$.$p_2$($a_{2,1}$, $\ldots$, $a_{2,n_2}$) @ &$m_2$ : $\;E_2$]},
    \end{center}
    reduce the goal to three subgoals:
    \begin{itemize}
    \item One with conclusion
          \ec{equiv[$M_1$.$p_1$ ~ $\;M_2$.$p_2$ : $\;P$ ==> $\;Q$]};

    \item One whose conclusion says that $P$ holds, where
      references to memories \ec{&1} and \ec{&2} have been replaced
      by \ec{&$m_1$} and \ec{&$m_2$}, respectively, and references
      to the formal parameters of \ec{$M_1$.$p_1$} and
      \ec{$M_2$.$p_2$} have been replaced by their arguments;

    \item One whose conclusion says that $Q$ implies that
      \ec{$E_1$\{1\} <=> $\;E_2$\{2\}}.
    \end{itemize}

    The argument to \ec{byequiv} may be replaced by a proof term for
    \ec{equiv[$M_1$.$p_1$ ~ $\;M_2$.$p_2$ : $\;P$ ==> $\;Q$]}, in which
    case the first subgoal isn't generated.
    Furthermore, either or both of $P$ and $Q$ may be replaced by
    \ec{_}, asking that the pre- or postcondition be inferred.
    Supplying no argument to \ec{byequiv} is the same as replacing
    both $P$ and $Q$ by \ec{_}. By default, inference of $Q$ attempts
    to infer a conjuction of equalities implying   
    \ec{$E_1$\{1\} <=> $\;E_2$\{2\}}. Passing the \ec{[-eq]} option to
    \ec{byequiv} takes $Q$ to \emph{be} \ec{$E_1$\{1\} <=> $\;E_2$\{2\}}.

    \medskip
    \emph{The other variants of the tactic behave similarly with
    regards to the use of proof terms and specification inference.}

    \medskip For example, consider the module
    \ecinput[linerange=3-10]{examps/tactics/byequiv/1.ec}
    If the current goal is
    \ecinput{examps/parts/tactics/byequiv/1-1.0.ec} then
    running \ecinput{examps/parts/tactics/byequiv/1-1.ec}
    produces the goals
    \ecinput{examps/parts/tactics/byequiv/1-1.1.ec}
    and
    \ecinput{examps/parts/tactics/byequiv/1-1.2.ec}
    and
    \ecinput{examps/parts/tactics/byequiv/1-1.3.ec}
    Given the lemma
    \ecinput[linerange=24-24]{examps/tactics/byequiv/1.ec}
    if the current goal is
    \ecinput{examps/parts/tactics/byequiv/1-2.0.ec} then
    running \ecinput{examps/parts/tactics/byequiv/1-2.ec}
    produces the goals
    \ecinput{examps/parts/tactics/byequiv/1-2.1.ec}
    and
    \ecinput{examps/parts/tactics/byequiv/1-2.2.ec}
    And, if the current goal is
    \ecinput{examps/parts/tactics/byequiv/1-3.0.ec} then
    running \ecinput{examps/parts/tactics/byequiv/1-3.ec}
    produces the goals
    \ecinput{examps/parts/tactics/byequiv/1-3.1.ec}
    and
    \ecinput{examps/parts/tactics/byequiv/1-3.2.ec}
    and
    \ecinput{examps/parts/tactics/byequiv/1-3.3.ec}
  \end{tsyntax}

  \begin{tsyntax}{byequiv (_ : $\;P$ ==> $\;Q$)}
    If the goal's conclusion has the form
    \begin{center}
      \ec{Pr[$M_1$.$p_1$($a_{1,1}$, $\ldots$, $a_{1,n_1}$) @ &$m_1$ : $\;E_1$] <=
          Pr[$M_2$.$p_2$($a_{2,1}$, $\ldots$, $a_{2,n_2}$) @ &$m_2$ : $\;E_2$]},
    \end{center}
    then \ec{byequiv} behaves the same as in the first variant
    except that the conclusion of the third subgoal says that $Q$
    implies \ec{$E_1$\{1\} => $\;E_2$\{2\}}.

    \medskip For example, if the current goal is
    \ecinput{examps/parts/tactics/byequiv/2-1.0.ec} then
    running \ecinput{examps/parts/tactics/byequiv/2-1.ec}
    produces the goals
    \ecinput{examps/parts/tactics/byequiv/2-1.1.ec}
    and
    \ecinput{examps/parts/tactics/byequiv/2-1.2.ec}
    and
    \ecinput{examps/parts/tactics/byequiv/2-1.3.ec}
    And, if the current goal is
    \ecinput{examps/parts/tactics/byequiv/2-2.0.ec} then
    running \ecinput{examps/parts/tactics/byequiv/2-2.ec}
    produces the goals
    \ecinput{examps/parts/tactics/byequiv/2-2.1.ec}
    and
    \ecinput{examps/parts/tactics/byequiv/2-2.2.ec}
    and
    \ecinput{examps/parts/tactics/byequiv/2-2.3.ec}
  \end{tsyntax}

  \begin{tsyntax}{byequiv (_ : $\;P$ ==> $\;Q$)}
    If the goal's conclusion has the form
    \begin{center}
      \ec{Pr[$M_1$.$p_1$($a_{1,1}$, $\ldots$, $a_{1,n_1}$) @ &$m_1$ : $\;E_1$] <=} \\
      \ec{Pr[$M_2$.$p_2$($a_{2,1}$, $\ldots$, $a_{2,n_2}$) @ &$m_2$ : $\;E_2$] +
          Pr[$M_2$.$p_2$($a_{2,1}$, $\ldots$, $a_{2,n_2}$) @ &$m_2$ : $\;B_2$]},
    \end{center}
    then \ec{byequiv} behaves the same as in the first variant
    except that the conclusion of the third subgoal says that $Q$
    implies \ec{!$B_2$\{2\} => $\;E_1$\{1\} => $\;E_2$\{2\}}.

    \medskip For example, if the current goal is
    \ecinput{examps/parts/tactics/byequiv/3-1.0.ec} then
    running \ecinput{examps/parts/tactics/byequiv/3-1.ec}
    produces the goals
    \ecinput{examps/parts/tactics/byequiv/3-1.1.ec}
    and
    \ecinput{examps/parts/tactics/byequiv/3-1.2.ec}
    \fxfatal{Why is the second subgoal pruned? (Compare with first and
    second variants, where the corresponding subgoal isn't pruned.)}
    And, if the current goal is
    \ecinput{examps/parts/tactics/byequiv/3-2.0.ec} then
    running \ecinput{examps/parts/tactics/byequiv/3-2.ec}
    produces the goals
    \ecinput{examps/parts/tactics/byequiv/3-2.1.ec}
    and
    \ecinput{examps/parts/tactics/byequiv/3-2.2.ec}
    \fxfatal{Why is the second subgoal pruned?}
  \end{tsyntax}

  \begin{tsyntax}{byequiv (_ : $\;P$ ==> $\;Q$) : $\;B_1$}
    If the goal's conclusion has the form
    \begin{center}
      \ec{`| Pr[$M_1$.$p_1$($a_{1,1}$, $\ldots$, $a_{1,n_1}$) @ &$m_1$ : $\;E_1$] -
          Pr[$M_2$.$p_2$($a_{2,1}$, $\ldots$, $a_{2,n_2}$) @ &$m_2$ : $\;E_2$] | <=} \\
      \ec{Pr[$M_2$.$p_2$($a_{2,1}$, $\ldots$, $a_{2,n_2}$) @ &$m_2$ : $\;B_2$]},
    \end{center}
    then \ec{byequiv} behaves the same as in the first variant
    except that the conclusion of the third subgoal says that $Q$
    implies
    \begin{center}
      \ec{($B_1$\{1\} <=> $\;B_2$\{2\}) /\\ ! $\;B_2$\{2\} => ($\;E_1$\{1\} <=> $\;E_2$\{2\})}
    \end{center}

    \medskip For example, if the current goal is
    \ecinput{examps/parts/tactics/byequiv/4-1.0.ec} then
    running \ecinput{examps/parts/tactics/byequiv/4-1.ec}
    produces the goals
    \ecinput{examps/parts/tactics/byequiv/4-1.1.ec}
    and
    \ecinput{examps/parts/tactics/byequiv/4-1.2.ec}
    and
    \ecinput{examps/parts/tactics/byequiv/4-1.3.ec}
    Given the lemma
    \ecinput[linerange=37-38]{examps/tactics/byequiv/4.ec}
    if the current goal is
    \ecinput{examps/parts/tactics/byequiv/4-2.0.ec} then
    running \ecinput{examps/parts/tactics/byequiv/4-2.ec}
    produces the goals
    \ecinput{examps/parts/tactics/byequiv/4-2.1.ec}
    and
    \ecinput{examps/parts/tactics/byequiv/4-2.2.ec}
  \end{tsyntax}
\end{tactic}

%%  \begin{tsyntax}{byequiv [option]? <spec>}
%%  Derives a probability relation from a \prhl judgement on the
%%  procedures involved. \ec{<spec>} can include wildcards when the
%%  tactic should infer the pre or postcondition. In addition,
%%  \ec{<spec>} can be extended with a failure event to infer precise
%%  applications of the Fundamental Lemma.
%%
%%  \textbf{Options:} By default, (\ec{eq} option) specification
%%  inference attempts to infer a conjunction of equalities sufficient
%%  to imply the desired relation. Passing the \ec{-eq} option
%%  overrides this behaviour, instead using the trivial relation on
%%  events.
%%
%%  \paragraph{Examples:}\strut
%%
%%  \begin{cmathpar}
%%    \texample
%%      {\ec{byequiv (_: P ==> Q)}}
%%      {\pRHL{P}{f_1}{f_2}{Q} \\\\
%%       {m_1[\Arg\mapsto\vec{a}_1]} \rel{P} {m_2[\Arg\mapsto\vec{a}_2]} \\
%%       Q \Rightarrow E_1\{1\}  \Leftrightarrow E_2\{2\}}
%%      {\PR{f_1}{\vec{a}_1}{\mem{m_1}}{E_1} = \PR{f_2}{\vec{a}_2}{m_2}{E_2}}
%%  \end{cmathpar}
%%
%%  \begin{cmathpar}
%%    \texample
%%      {\ec{byequiv (_: P ==> Q)}}
%%      {\pRHL{P}{f_1}{f_2}{Q} \\\\
%%       {m_1[\Arg\mapsto\vec{a}_1]} \rel{P} {m_2[\Arg\mapsto\vec{a}_2]} \\
%%       Q \Rightarrow E_1\{1\}  \Rightarrow E_2\{2\}}
%%      {\PR{f_1}{\vec{a}_1}{\mem{m_1}}{E_1} \leq \PR{f_2}{\vec{a}_2}{m_2}{E_2}}
%%  \end{cmathpar}
%%
%%  \begin{cmathpar}
%%    \texample
%%      {\ec{byequiv (_: P ==> Q)}}
%%      {\pRHL{P}{f_1}{f_2}{Q} \\\\
%%       {m_1[\Arg\mapsto\vec{a}_1]} \rel{P} {m_2[\Arg\mapsto\vec{a}_2]} \\
%%       Q \Rightarrow E_2\{2\}  \Rightarrow E_1\{1\}}
%%      {\PR{f_1}{\vec{a}_1}{\mem{m_1}}{E_1} \geq \PR{f_2}{\vec{a}_2}{m_2}{E_2}}
%%  \end{cmathpar}
%%
%%  \begin{cmathpar}
%%    \texample
%%      {\ec{byequiv (_: P ==> Q)}}
%%      {\pRHL{P}{f_1}{f_2}{Q} \\
%%       {m_1[\Arg\mapsto\vec{a}_1]} \rel{P} {m_2[\Arg\mapsto\vec{a}_2]} \\
%%       Q \Rightarrow \neg B_2\{2\} \Rightarrow E_1\{1\}  \Rightarrow E_2\{2\}}
%%      {\PR{f_1}{\vec{a}_1}{\mem{m_1}}{E_1}
%%       \leq \PR{f_2}{\vec{a}_2}{m_2}{E_2}
%%          + \PR{f_2}{\vec{a}_2}{m_2}{B_2}}
%%  \end{cmathpar}
%%
%%  \begin{cmathpar}
%%    \texample
%%      {\ec{byequiv (_: P ==> Q) : B$_1$}}
%%      {\pRHL{P}{f_1}{f_2}{Q} \\
%%       {m_1[\Arg\mapsto\vec{a}_1]} \rel{P} {m_2[\Arg\mapsto\vec{a}_2]} \\
%%       Q \Rightarrow
%%         (B_1\{1\} \Leftrightarrow B_2\{2\})
%%         \wedge (\neg B_2\{2\} \Rightarrow E_1\{1\} \Leftrightarrow E_2\{2\})}
%%      {| \PR{f_1}{\vec{a}_1}{\mem{m_1}}{E_1} - \PR{f_2}{\vec{a}_2}{m_2}{E_2} |
%%       \leq \PR{f_2}{\vec{a}_2}{m_2}{B_2}}
%%  \end{cmathpar}
%%
%%  \begin{cmathpar}
%%    \texample
%%      {\ec{byequiv [-eq] (_: P ==> _)}}
%%      {\pRHL{P}{f_1}{f_2}{E_1\{1\} \Leftrightarrow E_2\{2\}} \\
%%       {m_1[\Arg\mapsto\vec{a}_1]} \rel{P} {m_2[\Arg\mapsto\vec{a}_2]}}
%%      {\PR{f_1}{\vec{a}_1}{\mem{m_1}}{E_1} = \PR{f_2}{\vec{a}_2}{m_2}{E_2}}
%%  \end{cmathpar}
%% \end{tsyntax}
%%
%%  \begin{tsyntax}{byequiv <lemma>}
%%  Same as \ec{byequiv <spec>}, but the specification to use
%%  is inferred from the lemma provided. Raises an error if the lemma
%%  does not refer to the expected procedures. Inference options have no
%%  effect in this setting.
%%  \end{tsyntax}

% --------------------------------------------------------------------
\begin{tactic}{byphoare}
  \begin{tsyntax}{byphoare (_ : $\;P$ ==> $\;Q$)}
    If the goal's conclusion has the form
    \begin{center}
      \ec{Pr[$M$.$p$($a_1$, $\ldots$, $a_n$) @ &$m$ : $\;E$] = $\;e$},
    \end{center}
    reduce the goal to three subgoals:
    \begin{itemize}
    \item One with conclusion
      \ec{phoare[$M$.$p$ : $\;P$ ==> $\;Q$] = $\;e$};

    \item One whose conclusion says that $P$ holds, where
      variables references are looked-up in \ec{&$m$} and
      references to the formal parameters of \ec{$M$.$p$}
      have been replaced by its arguments; and

    \item One whose conclusion says that \ec{$Q$ <=> $\;E$}.
    \end{itemize}

    The argument to \ec{byphoare} may be replaced by a proof term for
    \ec{phoare[$M$.$p$ : $\;P$ ==> $\;Q$] = $\;e$}, in which case the
    first subgoal isn't generated.  Furthermore, either or both of $P$
    and $Q$ may be replaced by \ec{_}, asking that the pre- or
    postcondition be inferred.  Supplying no argument to \ec{byphoare}
    is the same as replacing both $P$ and $Q$ by \ec{_}.

    \medskip
    \emph{The other variants of the tactic behave similarly with
    regards to the use of proof terms and specification inference.}

    \medskip For example, consider the module
    \ecinput[linerange=3-9]{examps/tactics/byphoare/1.ec}
    If the current goal is
    \ecinput{examps/parts/tactics/byphoare/1-1.0.ec} then
    running \ecinput{examps/parts/tactics/byphoare/1-1.ec}
    produces the goals
    \ecinput{examps/parts/tactics/byphoare/1-1.1.ec}
    and
    \ecinput{examps/parts/tactics/byphoare/1-1.2.ec}
    and
    \ecinput{examps/parts/tactics/byphoare/1-1.3.ec}
    Given the lemma
    \ecinput[linerange=27-28]{examps/tactics/byphoare/1.ec}
    if the current goal is
    \ecinput{examps/parts/tactics/byphoare/1-2.0.ec} then
    running \ecinput{examps/parts/tactics/byphoare/1-2.ec}
    produces the goals
    \ecinput{examps/parts/tactics/byphoare/1-2.1.ec}
    and
    \ecinput{examps/parts/tactics/byphoare/1-2.2.ec}
  \end{tsyntax}

  \begin{tsyntax}{byphoare (_ : $\;P$ ==> $\;Q$)}
    If the goal's conclusion has the form
    \begin{center}
      \ec{$e$ <= Pr[$M$.$p$($a_1$, $\ldots$, $a_n$) @ &$m$ : $\;E$]},
    \end{center}
    then \ec{byphoare} behaves as in the first variant except
    the conclusion of the first subgoal is
    \ec{phoare[$M$.$p$ : $\;P$ ==> $\;Q$] >= $\;e$}.

    \medskip For example, if the current goal is
    \ecinput{examps/parts/tactics/byphoare/2-1.0.ec} then
    running \ecinput{examps/parts/tactics/byphoare/2-1.ec}
    produces the goals
    \ecinput{examps/parts/tactics/byphoare/2-1.1.ec}
    and
    \ecinput{examps/parts/tactics/byphoare/2-1.2.ec}
    and
    \ecinput{examps/parts/tactics/byphoare/2-1.3.ec}
    \fxfatal{It's confusing how the third goal has been simplified,
    but not pruned.}
    Given the lemma
    \ecinput[linerange=19-19]{examps/tactics/byphoare/2.ec}
    if the current goal is
    \ecinput{examps/parts/tactics/byphoare/2-2.0.ec} then
    running \ecinput{examps/parts/tactics/byphoare/2-2.ec}
    produces the goals
    \ecinput{examps/parts/tactics/byphoare/2-2.1.ec}
    and
    \ecinput{examps/parts/tactics/byphoare/2-2.2.ec}
    And, if the current goal is
    \ecinput{examps/parts/tactics/byphoare/2-3.0.ec} then
    running \ecinput{examps/parts/tactics/byphoare/2-3.ec}
    produces the goals
    \ecinput{examps/parts/tactics/byphoare/2-3.1.ec}
    and
    \ecinput{examps/parts/tactics/byphoare/2-3.2.ec}
    and
    \ecinput{examps/parts/tactics/byphoare/2-3.3.ec}
  \end{tsyntax}

  \begin{tsyntax}{byphoare (_ : $\;P$ ==> $\;Q$)}
    If the goal's conclusion has the form
    \begin{center}
      \ec{Pr[$M$.$p$($a_1$, $\ldots$, $a_n$) @ &$m$ : $\;E$] <= $\;e$},
    \end{center}
    then \ec{byphoare} behaves as in the first variant except
    the conclusion of the first subgoal is
    \ec{phoare[$M$.$p$ : $\;P$ ==> $\;Q$] <= $\;e$}.

    \medskip For example, if the current goal is
    \ecinput{examps/parts/tactics/byphoare/3-1.0.ec} then
    running \ecinput{examps/parts/tactics/byphoare/3-1.ec}
    produces the goals
    \ecinput{examps/parts/tactics/byphoare/3-1.1.ec}
    and
    \ecinput{examps/parts/tactics/byphoare/3-1.2.ec}
    and
    \ecinput{examps/parts/tactics/byphoare/3-1.3.ec}
    \fxfatal{It's confusing how the third goal has been simplified,
    but not pruned.}
    Given the lemma
    \ecinput[linerange=19-19]{examps/tactics/byphoare/3.ec}
    if the current goal is
    \ecinput{examps/parts/tactics/byphoare/3-2.0.ec} then
    running \ecinput{examps/parts/tactics/byphoare/3-2.ec}
    produces the goals
    \ecinput{examps/parts/tactics/byphoare/3-2.1.ec}
    and
    \ecinput{examps/parts/tactics/byphoare/3-2.2.ec}
    And, if the current goal is
    \ecinput{examps/parts/tactics/byphoare/3-3.0.ec} then
    running \ecinput{examps/parts/tactics/byphoare/3-3.ec}
    produces the goals
    \ecinput{examps/parts/tactics/byphoare/3-3.1.ec}
    and
    \ecinput{examps/parts/tactics/byphoare/3-3.2.ec}
    and
    \ecinput{examps/parts/tactics/byphoare/3-3.3.ec}
  \end{tsyntax}
\end{tactic}  

%%  \begin{tsyntax}{byphoare [option]? <spec>}
%%  Derives a probability relation from a \phl judgement on the
%%  procedure involved. \ec{<spec>} can include wildcards when the
%%  tactic should infer the pre or postcondition.
%%
%%  \textbf{Options:} By default, (\ec{eq} option) specification
%%  inference attempts to infer a conjunction of equalities sufficient
%%  to imply the desired relation. Passing the \ec{-eq} option
%%  overrides this behaviour, instead using the trivial relation on
%%  events.
%%
%%  \paragraph{Examples:}\strut
%%
%%  \begin{cmathpar}
%%    \texample
%%      {\ec{byphoare (_: P ==> Q)}}
%%      {\pHL{P}{f}{Q}{=}{\delta} \\
%%       P\ m[\Arg\mapsto\vec{a}] \\
%%       \forall \mem{m'}.\, Q\ m' \Leftrightarrow E\ m'}
%%      {\PR{f}{\vec{a}}{\mem{m}}{E} = \delta}
%%  \end{cmathpar}
%%  \end{tsyntax}
%%
%%  \begin{tsyntax}{byphoare <lemma>}
%%  Same as \ec{byphoare <spec>}, but the specification to use is
%%  inferred from the lemma provided. Raises an error if the lemma does
%%  not refer to the expected procedure. Inference options have no
%%  effect in this setting.
%%  \end{tsyntax}
%%\end{tactic}

\input{tactics/hoare.tex}
% --------------------------------------------------------------------
\begin{tactic}{bypr}
  \begin{tsyntax}{bypr $\;e_1$ $\;e_2$}
    If the goal's conclusion has the form
    \begin{center}
      \ec{equiv[$M$.$p$ ~ $\;N$.$q$ : $\;P$ ==> $\;Q$]},
    \end{center}
    and the $e_i$ are expressions of the same type possibily
    involving memories \ec{&1} and \ec{&2} for \ec{$M$.$p$} and
    \ec{$N$.$q$}, respectively, then reduce the goal to two subgoals:
    \begin{itemize}
    \item One whose conclusion says that for all memories \ec{&1}
      and \ec{&2} for \ec{$M$.$p$} and \ec{$N$.$q$}, if $e_1 = e_2$,
      then $Q$ holds; and

    \item One whose conclusion says that, for all memories \ec{&1} and
      \ec{&2} for \ec{$M$.$p$} and \ec{$N$.$q$} and values $a$ of the
      common type of the $e_i$, if $P$ holds, then the probability of
      running \ec{$M$.$p$} in memory \ec{&1} and with arguments
      consisting of the values of its formal parameters in \ec{&1} and
      terminating in a memory in which the value of $e_1$ (replacing
      references to \ec{&1} with reference to this memory) is $a$ is
      the same as the probability of running \ec{$N$.$q$} in memory
      \ec{&2} and with arguments consisting of the values of its
      formal parameters in \ec{&2} and terminating in a memory in
      which the value of $e_2$ (replacing references to \ec{&2} with
      reference to this memory) is $a$.
    \end{itemize}

    \medskip For example, consider the modules
    \ecinput{examps/tactics/bypr/1.ec}{}{3-20}{}
    If the current goal is
    \ecinput{examps/parts/tactics/bypr/1-1.0.ec}{}{}{}{} then
    running \ecinput{examps/parts/tactics/bypr/1-1.ec}{}{}{}{}
    produces the goals
    \ecinput{examps/parts/tactics/bypr/1-1.1.ec}{}{}{}{}
    and
    \ecinput{examps/parts/tactics/bypr/1-1.2.ec}{}{}{}{}
  \end{tsyntax}

  \begin{tsyntax}{bypr}
    If the goal's conclusion has the form
    \begin{center}
      \ec{hoare[$M$.$p$ : $\;P$ ==> $\;Q$]},
    \end{center}
    then reduce the goal to one whose conclusion says that, for all
    memories \ec{&m} for \ec{$M$.$p$} such that \ec{$P$\{$m$\}} holds,
    the probability of running \ec{$M$.$p$} in memory \ec{&$m$} and
    with arguments consisting of the values of its formal parameters
    in \ec{&$m$} and terminating in a memory satisfying \ec{!$Q$} is $0$.

    \medskip For example, consider the module
    \ecinput{examps/tactics/bypr/2.ec}{}{3-10}{}
    If the current goal is
    \ecinput{examps/parts/tactics/bypr/2-1.0.ec}{}{}{}{} then
    running \ecinput{examps/parts/tactics/bypr/2-1.ec}{}{}{}{}
    produces the goal
    \ecinput{examps/parts/tactics/bypr/2-1.1.ec}{}{}{}{}
  \end{tsyntax}
\end{tactic}

%%  Derives a program judgment from a probability relation or an exact
%%  probability. Only applies to judgments on procedures.
%%
%%  \paragraph{Examples:}\strut
%%  
%%  \begin{cmathpar}
%%    \texample[\prhl{}]
%%      {\ec{bypr (r$_1$) (r$_2$)}}
%%      {\forall \mem{m_1}, \mem{m_2}, a.\,
%%          r_1 = a \Rightarrow
%%          r_2 = a \Rightarrow
%%          {\mem{m_1}} \rel{Q} {\mem{m_2}} \\
%%       \forall \vec{a}_1, \vec{a}_2, \mem{m_1}, \mem{m_2}, a.\,
%%         {\mem{m_1}[\Arg\mapsto\vec{a}_1]} \rel{P} {\mem{m_2}[\Arg\mapsto\vec{a}_2]} \Rightarrow \\
%%         \PR{f_1}{\vec{a}_1}{\mem{m_1}}{a = r_1} = \PR{f_2}{\vec{a}_2}{\mem{m_2}}{a = r_2}}
%%      {\pRHL{P}{f_1}{f_2}{Q}}
%%  \end{cmathpar}
%%
%%  \begin{cmathpar}
%%    \texample[\phl{}]
%%      {\ec{bypr}}
%%      {\forall \mem{m}, \vec{a}.\, P\ m[\Arg\mapsto\vec{a}] \Rightarrow
%%          \PR{f}{\vec{a}}{m}{E} \mathrel{\diamond} \delta}
%%      {\pHL{P}{f}{E}{\diamond}{\delta}}
%%  \end{cmathpar}
%%
%%  \begin{cmathpar}
%%    \texample[\hl{}]
%%      {\ec{bypr}}
%%      {\forall \mem{m}, \vec{a}.\, P\ m[\Arg\mapsto\vec{a}] \Rightarrow
%%          \PR{f}{\vec{a}}{m}{\neg E} \mathop{=}0\%r}
%%      {\HL{P}{f}{E}}
%%  \end{cmathpar}
%%  \end{tsyntax}
%%\end{tactic}

% --------------------------------------------------------------------
\begin{tactic}{exists*}
  \begin{tsyntax}{exists* $\;e_1$, $\;\ldots$, $\;e_n$}
    If the goal's conclusion is a \prhl, \hl or \phl judgment or
    statement judgements and the $e_i$ are well-typed expressions
    typically involving program variables (in the \prhl case,
    the expressions will refer to memories \ec{&1} and \ec{&2}),
    then change the conclusion's precondition $P$ to
    \begin{center}
      \ec{exists ($x_1$ $\;\ldots$ $\;x_n$),
          $\;x_1$ = $\;e_2$ /\\ $\;\ldots$ /\\ $\;x_n$ = $\;e_n$ /\\ $\;P$}.
    \end{center}

    The tactic can be used in conjunction with \rtactic{elim*} when
    handling a procedure call using a lemma that refers to initial
    values of program variables. See \rtactic{elim*} for an example
    of this.

    \medskip For example, if the current goal is
    \ecinput{examps/parts/tactics/exists-star/1-1.0.ec}{}{}{}{} then
    running \ecinput{examps/parts/tactics/exists-star/1-1.ec}{}{}{}{}
    produces the goal
    \ecinput{examps/parts/tactics/exists-star/1-1.1.ec}{}{}{}{}
    If the current goal is
    \ecinput{examps/parts/tactics/exists-star/1-2.0.ec}{}{}{}{} then
    running \ecinput{examps/parts/tactics/exists-star/1-2.ec}{}{}{}{}
    produces the goal
    \ecinput{examps/parts/tactics/exists-star/1-2.1.ec}{}{}{}{}
    If the current goal is
    \ecinput{examps/parts/tactics/exists-star/1-3.0.ec}{}{}{}{} then
    running \ecinput{examps/parts/tactics/exists-star/1-3.ec}{}{}{}{}
    produces the goal
    \ecinput{examps/parts/tactics/exists-star/1-3.1.ec}{}{}{}{}
    If the current goal is
    \ecinput{examps/parts/tactics/exists-star/1-4.0.ec}{}{}{}{} then
    running \ecinput{examps/parts/tactics/exists-star/1-4.ec}{}{}{}{}
    produces the goal
    \ecinput{examps/parts/tactics/exists-star/1-4.1.ec}{}{}{}{}
  \end{tsyntax}
\end{tactic}

%%  Introduce an existential quantification over the value of a program
%%  variable in the initial memory. This is particularly useful when
%%  dealing with a procedure call using a lemma that refers to initial
%%  values of arguments or state (using \rtactic{call}). Several program
%%  variables can be treated simultaneously by providing them in a
%%  comma-separated list.
%%
%%  \paragraph{Examples:}\strut
%%
%%  \begin{cmathpar}
%%  \texample[\prhl{}]{\ec{exists* M.x\{1\}}}%%
%%    {\pRHL{\exists x, x = \inmem{M.x}{1} \wedge P}{c_1}{c_2}{Q}}%%
%%    {\pRHL{P}{c_1}{c_2}{Q}}
%%
%%  \texample[\prhl{}]{\ec{exists* M.x\{1\}, M.x\{2\}}}%%
%%    {\pRHL{\exists x_1\ x_2, x1 = \inmem{M.x}{1} \wedge \inmem{M.x}{2} \wedge P}{c_1}{c_2}{Q}}%%
%%    {\pRHL{P}{c_1}{c_2}{Q}}
%%
%%  \texample[\phl{}]{\ec{exists* M.x}}%%
%%    {\pHL{\exists x, x = M.x \wedge P}{c}{Q}{\diamond}{\delta}}%%
%%    {\pHL{P}{c}{Q}{\diamond}{\delta}}
%%
%%  \texample[\hl{}]{\ec{exists* M.x}}%%
%%    {\HL{\exists x, x = M.x \wedge P}{c}{Q}}%%
%%    {\HL{P}{c}{Q}}
%%  \end{cmathpar}
%%
%%  \end{tsyntax}

% --------------------------------------------------------------------
\begin{tactic}{elim*}
  \begin{tsyntax}[empty]{elim*}
     If the goal's conclusion is a \prhl, \hl or \phl judgement or
     statement judgement whose precondition has the form
     \begin{center}
       \ec{exists ($x_1$ $\;\ldots$ $\;x_n$), $\;P$},
     \end{center}
    then remove the existential quantification from the precondition,
    and universally quantify the judgement or statement judgement
    by the $x_i$.

    Such existential quantifications may be introduced by \rtactic{sp}
    or \rtactic{exists*}.

    \medskip For example, if the current goal is
    \ecinput{examps/parts/tactics/elim-star/1-1.0.ec}{}{}{}{} then
    running \ecinput{examps/parts/tactics/elim-star/1-1.ec}{}{}{}{}
    produces the goal
    \ecinput{examps/parts/tactics/elim-star/1-1.1.ec}{}{}{}{}
    If the current goal is
    \ecinput{examps/parts/tactics/elim-star/1-2.0.ec}{}{}{}{} then
    running \ecinput{examps/parts/tactics/elim-star/1-2.ec}{}{}{}{}
    produces the goal
    \ecinput{examps/parts/tactics/elim-star/1-2.1.ec}{}{}{}{}
    If the current goal is
    \ecinput{examps/parts/tactics/elim-star/1-3.0.ec}{}{}{}{} then
    running \ecinput{examps/parts/tactics/elim-star/1-3.ec}{}{}{}{}
    produces the goal
    \ecinput{examps/parts/tactics/elim-star/1-3.1.ec}{}{}{}{}
    If the current goal is
    \ecinput{examps/parts/tactics/elim-star/1-4.0.ec}{}{}{}{} then
    running \ecinput{examps/parts/tactics/elim-star/1-4.ec}{}{}{}{}
    produces the goal
    \ecinput{examps/parts/tactics/elim-star/1-4.1.ec}{}{}{}{}

    \medskip As a more realistic example, consider the module
    \ecinput{examps/tactics/elim-star/2.ec}{}{3-12}{}
    and lemma
    \ecinput{examps/tactics/elim-star/2.ec}{}{14-14}{}
    If the current goal is
    \ecinput{examps/parts/tactics/elim-star/2-1.0.ec}{}{}{}{} then
    running \ecinput{examps/parts/tactics/elim-star/2-1.ec}{}{}{}{}
    produces the goal
    \ecinput{examps/parts/tactics/elim-star/2-1.1.ec}{}{}{}{}
    from which
    running \ecinput{examps/parts/tactics/elim-star/2-2.ec}{}{}{}{}
    produces the goal
    \ecinput{examps/parts/tactics/elim-star/2-2.1.ec}{}{}{}{}
    from which
    running \ecinput{examps/parts/tactics/elim-star/2-3.ec}{}{}{}{}
    produces the goal
    \ecinput{examps/parts/tactics/elim-star/2-3.1.ec}{}{}{}{}
  \end{tsyntax}
\end{tactic}

%%  \paragraph{Examples:}\strut
%%
%%  \begin{cmathpar}
%%  \texample[\prhl{}]{\ec{elim*}}%%
%%    {\forall x.\, \pRHL{x = \inmem{M.x}{1} \wedge P}{c_1}{c_2}{Q}}%%
%%    {\pRHL{\exists x, x = \inmem{M.x}{1} \wedge P}{c_1}{c_2}{Q}}
%%
%%  \texample[\phl{}]{\ec{elim*}}%%
%%    {\forall x.\, \pHL{x = M.x \wedge P}{c}{Q}{\diamond}{\delta}}%%
%%    {\pHL{\exists x, x = M.x \wedge P}{c}{Q}{\diamond}{\delta}}
%%
%%  \texample[\hl{}]{\ec{elim*}}%%
%%    {\forall x.\, \HL{x = M.x \wedge P}{c}{Q}}%%
%%    {\HL{\exists x, x = M.x \wedge P}{c}{Q}}  
%%  \end{cmathpar}
%%
%%  \end{tsyntax}

% --------------------------------------------------------------------
\begin{tactic}{exfalso}
  \begin{tsyntax}{exfalso}
  Combines \rtactic{conseq}, \rtactic{byequiv}, \rtactic{byphoare},
  \rtactic{hoare} and \rtactic{bypr} to strengthen the precondition
  into $\mathsf{false}$ and discharge the resulting trivial goal.

  For example, if the current goal is
  \ecinput{examps/parts/tactics/exfalso/1-1.0.ec}{}{}{}{} then
  running \ecinput{examps/parts/tactics/exfalso/1-1.ec}{}{}{}{}
  produces the goals
  \ecinput{examps/parts/tactics/exfalso/1-1.1.ec}{}{}{}{}
  and
  \ecinput{examps/parts/tactics/exfalso/1-1.2.ec}{}{}{}{}
  The first of these goals is solved by
  running \ecinput{examps/parts/tactics/exfalso/1-2.ec}{}{}{}{}
  And running \ecinput{examps/parts/tactics/exfalso/1-3.ec}{}{}{}{}
  reduces the second of these goals to
  \ecinput{examps/parts/tactics/exfalso/1-3.1.ec}{}{}{}{}
  which \ec{smt} solves.
  \end{tsyntax}

  \fixme{Perhaps need other examples?}
\end{tactic}

%%  \paragraph{Examples:}\strut
%%  
%%  \begin{cmathpar}
%%  \texample[\prhl{}]
%%    {\ec{exfalso}}
%%    {P \Rightarrow \mathsf{false}}
%%    {\pRHL{P}{c}{c'}{Q}}
%%
%%  \texample[\phl{}]
%%    {\ec{exfalso}}
%%    {P \Rightarrow \mathsf{false}}
%%    {\pHL{P}{c}{Q}{\diamond}{\delta}}
%%
%%  \texample[\hl{}]
%%    {\ec{exfalso}}
%%    {P \Rightarrow \mathsf{false}}
%%    {\HL{P}{c}{Q}}
%%  \end{cmathpar}
%%  \end{tsyntax}
%%\end{tactic}


\subsection{Automated Tactics}
\label{subsec:automatedtactics}

% --------------------------------------------------------------------
\begin{tactic}{auto}
  \begin{tsyntax}[empty]{auto}
    If the current goal is a \prhl, \hl or \phl statement judgement,
    uses various program logic tactics in an attempt to reduce the
    goal to a simpler one. Never fails, but may fail to make any
    progress.

    \fixme{Need better description of when the tactic is applicable and
           how the tactic works!}

    \medskip For example, if the current goal is
    \ecinput{examps/parts/tactics/auto/1-1.0.ec}{}{}{}{} then
    running \ecinput{examps/parts/tactics/auto/1-1.ec}{}{}{}{}
    produces the goal
    \ecinput{examps/parts/tactics/auto/1-1.1.ec}{}{}{}{}
    which \ec{progress;smt} is able to solve.
    If the current goal is
    \ecinput{examps/parts/tactics/auto/1-2.0.ec}{}{}{}{} then
    running \ecinput{examps/parts/tactics/auto/1-2.ec}{}{}{}{}
    produce a single goal, which \ec{smt} is able to solve.
    And, if the current goal is
    \ecinput{examps/parts/tactics/auto/1-3.0.ec}{}{}{}{} then
    running \ecinput{examps/parts/tactics/auto/1-3.ec}{}{}{}{}
    produce a single goal, which \ec{smt} is able to solve.
  \end{tsyntax}
\end{tactic}

% --------------------------------------------------------------------
\begin{tactic}{sim}
  \ec{sim} attempts to solve a goal whose conclusion is a \prhl
  judgement or statement judgement by working backwards, propagating
  and extending a conjunction of equalties between variables of the
  two programs, verifying that the conclusion's precondition implies
  the final conjuction of equalities.  It's capable of working
  backwards through \ec{if} and \ec{while} statements and handing
  random assignments, but only when the programs are sufficiently
  similar (thus its name). Sometimes this process only partly
  succeeds, leaving a statement judgement whose programs are prefixes
  of the original programs.

  \begin{tsyntax}{sim}
    Without any arguments, \ec{sim} attemps to infer the conjuction of
    program variable equalities from the conclusion's postcondition.

    \medskip For example, if the current goal is
    \ecinput{examps/parts/tactics/sim/1-1.0.ec}{}{}{}{} then
    running \ecinput{examps/parts/tactics/sim/1-1.ec}{}{}{}{}
    produces the goal
    \ecinput{examps/parts/tactics/sim/1-1.1.ec}{}{}{}{}
    which \ec{auto} is able to solve.
  \end{tsyntax}

  \begin{tsyntax}{sim / $\;\phi$ : $\;\mathit{eqs}$}
    One may give the starting conjuction, $\mathit{eqs}$, of equalities
    explicitly, and may also specifiy an invariant $\phi$ on the
    global variables of the programs.

    \medskip For example, if the current goal is
    \ecinput{examps/parts/tactics/sim/1-2.0.ec}{}{}{}{} then
    running \ecinput{examps/parts/tactics/sim/1-2.ec}{}{}{}{}
    produces the goals
    \ecinput{examps/parts/tactics/sim/1-2.1.ec}{}{}{}{}
    and
    \ecinput{examps/parts/tactics/sim/1-2.2.ec}{}{}{}{}
    which \ec{smt} and \ec{auto;smt}, respectively, are
    able to solve.
  \end{tsyntax}

  \begin{tsyntax}{sim $\;\mathit{proceq}_1$ $\;\ldots$ $\;\mathit{proceq}_1$ / $\;\phi$ : $\;\mathit{eqs}$}
    In its most general form, one may also supply a sequence of
    procedure global equality specifications of the form
    \begin{center}
      \ec{($M$.$p$ ~ $\;N$.$q$ : $\;\mathit{eqs}$)},
    \end{center}
    where $\mathit{eqs}$ is a conjuction of global variable
    equalities. When \ec{sim} encounters a pair of procedure calls
    consisting of a call to \ec{$M$.$p$} in the first program and
    \ec{$N$.$q$} in the second program, it will generate a subgoal
    whose conclusion is a \prhl judgment between \ec{$M$.$p$} and
    \ec{$N$.$q$}, whose precondition assumes equality of its
    arguments, $\mathit{eqs}$ and $\phi$, and whose postcondition
    requires equality of the calls' results, $\mathit{eqs}$ and
    $\phi$.

    One may also replace \ec{$M$.$p$ ~ $\;N$.$q$} by \ec{_},
    meaning that the same conjunction of global variable equalities
    is used for all procedure calls.

    \medskip For example, if the current goal is
    \ecinput{examps/parts/tactics/sim/2-1.0.ec}{}{}{}{} then
    running \ecinput{examps/parts/tactics/sim/2-1.ec}{}{}{}{}
    produces the goals
    \ecinput{examps/parts/tactics/sim/2-1.1.ec}{}{}{}{}
    and
    \ecinput{examps/parts/tactics/sim/2-1.2.ec}{}{}{}{}
    and
    \ecinput{examps/parts/tactics/sim/2-1.3.ec}{}{}{}{}
    and
    \ecinput{examps/parts/tactics/sim/2-1.4.ec}{}{}{}{}
    which \ec{smt}, \ec{proc;auto;smt}, \ec{proc;auto;smt}
    and \ec{auto}, respectively, are able to solve.

    \medskip And, if the current goal is
    \ecinput{examps/parts/tactics/sim/2-2.0.ec}{}{}{}{} then
    running \ecinput{examps/parts/tactics/sim/2-2.ec}{}{}{}{}
    produces the goals
    \ecinput{examps/parts/tactics/sim/2-2.1.ec}{}{}{}{}
    and
    \ecinput{examps/parts/tactics/sim/2-2.2.ec}{}{}{}{}
    and
    \ecinput{examps/parts/tactics/sim/2-2.3.ec}{}{}{}{}
    and
    \ecinput{examps/parts/tactics/sim/2-2.4.ec}{}{}{}{}
    which \ec{smt}, \ec{proc;auto;smt}, \ec{proc;auto;smt}
    and \ec{auto}, respectively, are able to solve.
  \end{tsyntax}
\end{tactic}

%%  \begin{tsyntax}{sim <pos>? <hintgeqs>* <hintinv>? <eqs>?}\\
%%    where \begin{tabular}{lrl}
%%       <pos>      & = & <uint> <uint> \\
%%       <hintgeqs> & = & (<procname>? $\sim$ <procname>? : <formula>) \\
%%                  & | & ($\_$? : <formula>  \\
%%       <eqs>      & = & : <formula> \\
%%    \end{tabular}
%%  
%%  \fix{Missing description of sim}.
%%  \end{tsyntax}
%%\end{tactic}


\subsection{Advanced Tactics}
\label{subsec:advancedtactics}

% --------------------------------------------------------------------
\begin{tactic}{eager}
  \ec{eager} is a family of tactics for proving \prhl \emph{statement}
  judgements of the form
  \begin{center}
    \ec{equiv[$s_1$ $\;t_1$ ~ $\;t_2$ $\;s_2$ : $\;P$ ==> $\;Q$]},
  \end{center}
  where the pre- and postconditions $P$ and $Q$ are conjunctions of
  equalities between program variables, and the statement sequences
  $s_i$ only read and write global variables.  Here $s_1$ is in the
  ``eager'' position, and its replacement, $s_2$, is in the ``lazy''
  position. Some of the tactics work with \emph{eager judgements} of
  the form
  \begin{center}
    \ec{eager[$s_1$, $\;M$.$p$ ~ $\;N$.$q$, $\;s_2$ : $\;P$ ==> $\;Q$]},
  \end{center}
  where, again, the $s_i$ only involve global variables, but where
  $P$ and $Q$ may talk about the parameters and results of
  \ec{$M$.$p$} and \ec{$N$.$q$} in the usual way.

  \medskip The context of our examples is the following \EasyCrypt
  script involving variable incrementation oracles \ec{Or1} and
  \ec{Or2}, which only differ in that \ec{Or2} keeps a ``transcript''
  of its operation.
  \ecinput{examps/tactics/eager/1-nodumps.ec}{}{5-200}{}

  \begin{tsyntax}{eager proc}
  Turn a goal whose conclusion is an eager judgement into one
  whose conclusion is a \prhl statement judgement in which
  the eager judgement's procedures have been replaced by their
  bodies.
  \medskip
  For example, if the current goal is
  \ecinput{examps/parts/tactics/eager/1-1.0.ec}{}{}{}{} then
  running \ecinput{examps/parts/tactics/eager/1-1.ec}{}{}{}{}
  produces the goal
  \ecinput{examps/parts/tactics/eager/1-1.1.ec}{}{}{}{}
  \end{tsyntax}

  \begin{tsyntax}{proc*}
  Turn a goal whose conclusion is an eager judgement into one
  whose conclusion is a \prhl statement judgement in which
  the eager judgement's procedures are called, as opposed to
  being inlined.
  \medskip
  For example, if the current goal is
  \ecinput{examps/parts/tactics/eager/1-2.0.ec}{}{}{}{} then
  running \ecinput{examps/parts/tactics/eager/1-2.ec}{}{}{}{}
  produces the goal
  \ecinput{examps/parts/tactics/eager/1-2.1.ec}{}{}{}{}
  \end{tsyntax}

  \begin{tsyntax}{eager call $\;p$}
    Here $p$ is a proof term for an eager judgement. If the goal's
    conclusion is a \prhl statement judgement whose programs' suffixes
    match the left and right sides of the eager judgment, then consume
    those suffixes.

  \medskip
  For example, if the current goal is
  \ecinput{examps/parts/tactics/eager/1-4.0.ec}{}{}{}{} then
  running \ecinput{examps/parts/tactics/eager/1-4.ec}{}{}{}{}
  (see the statement of \ec{eager_incr}, above)
  produces the goal
  \ecinput{examps/parts/tactics/eager/1-4.1.ec}{}{}{}{}
  \end{tsyntax}

  \begin{tsyntax}{eager seq $\;n_1$ $\;n_2$ ($H$ : $\;s_1$ ~ $\;s_2$ :
    $\;A$ ==> $\;B$) : $\;C$}

   Here, the goal's conclusion must be a \prhl statement
  judgement whose left and right programs begin and end with $s_1$ and
  $s_2$, respectively. A first subgoal is generated whose conclusion
  is the specified \prhl statement judgement, which is made available
  as $H$ for the subsequent subgoals's use.  The $n_1$ and $n_2$ must
  be natural numbers saying how many statements from the part of the
  first program following $s_1$ and from the beginning of the second
  program to put together with the $s_i$ in a second \prhl statement
  judgement subgoal. The remaining statements are put together with
  the $s_i$ in a third \prhl statement judgement subgoal. And there is
  a final subgoal whose conclusion is a \prhl statement judgment whose
  left and right sides are those remaining statements of the second
  program only.

  \medskip
  For example, if the current goal is
  \ecinput{examps/parts/tactics/eager/1-3.0.ec}{}{}{}{} then
  running \ecinput{examps/parts/tactics/eager/1-3.ec}{}{}{}{}
  produces the goals
  \ecinput{examps/parts/tactics/eager/1-3.1.ec}{}{}{}{}
  and
  \ecinput{examps/parts/tactics/eager/1-3.2.ec}{}{}{}{}
  and
  \ecinput{examps/parts/tactics/eager/1-3.3.ec}{}{}{}{}
  and
  \ecinput{examps/parts/tactics/eager/1-3.4.ec}{}{}{}{}
  \end{tsyntax}

  \begin{tsyntax}{eager if}
    If the goal's conclusion is a \prhl statement judgement whose left
    program consists of $s_1$ followed by a conditional, and whose
    right program consists of a conditional followed by $s_2$, reduce
    the goal to two subgoals using the $s_i$ together with the ``then''
    and ``else'' parts of the conditionals, along with auxiliary subgoals
    verifying that---even after running $s_1$---the conditionals'
    boolean expressions are equivalent.

  \medskip
  For example, if the current goal is
  \ecinput{examps/parts/tactics/eager/1-6.0.ec}{}{}{}{} then
  running \ecinput{examps/parts/tactics/eager/1-6.ec}{}{}{}{}
  produces the goals
  \ecinput{examps/parts/tactics/eager/1-6.1.ec}{}{}{}{}
  and
  \ecinput{examps/parts/tactics/eager/1-6.2.ec}{}{}{}{}
  and
  \ecinput{examps/parts/tactics/eager/1-6.3.ec}{}{}{}{}
  and
  \ecinput{examps/parts/tactics/eager/1-6.4.ec}{}{}{}{}
  \end{tsyntax}

  \begin{tsyntax}{eager while ($H$ : $\;s_1$ ~ $\;s_2$ :
    $\;A$ ==> $\;B$)}
  Like \ec{eager if}, but working with while loops instead of
  conditionals and featuring an explicit, named \prhl statement
  judgement involving the $s_i$, available as $H$ to the subgoals.
  The subgoal involving the bodies of the while loops uses $B$ as its
  invariant. There is also a subgoal whose conclusion is a \prhl
  statement judgement both of whose sides are the body of the second
  program's while loop.

  \medskip
  For example, if the current goal is
  \ecinput{examps/parts/tactics/eager/1-5.0.ec}{}{}{}{} then
  running \ecinput{examps/parts/tactics/eager/1-5.ec}{}{}{}{}
  produces the goals
  \ecinput{examps/parts/tactics/eager/1-5.1.ec}{}{}{}{}
  and
  \ecinput{examps/parts/tactics/eager/1-5.2.ec}{}{}{}{}
  and
  \ecinput{examps/parts/tactics/eager/1-5.3.ec}{}{}{}{}
  and
  \ecinput{examps/parts/tactics/eager/1-5.4.ec}{}{}{}{}
  \end{tsyntax}

  \begin{tsyntax}{eager proc ($H$ : $\;s_1$ ~ $\;s_2$ :
    $\;A$ ==> $\;B$) $\;C$ | eager $\;H$ $\;C$}
  This is the form of \ec{eager proc} that applies when the procedures
  \ec{$M$.$p$} and \ec{$N$.$q$} are abstract.  The second variant is
  where the specified \prhl statement judgement has already been
  introduced by another \ec{eager} tactic. There must be a module
  restriction saying that the abstract procedures can't directly
  interfere with the global variables on which the $s_i$
  depend. Subgoals are generated for each pair of ``oracles'' the
  abstract procedures are capable of calling, i.e., for each of the
  procedures that may read/write the global variables used by the
  $s_i$.

  \medskip
  For example, if the current goal is
  \ecinput{examps/parts/tactics/eager/1-7.0.ec}{}{}{}{} then
  running \ecinput{examps/parts/tactics/eager/1-7.ec}{}{}{}{}
  produces the goals
  \ecinput{examps/parts/tactics/eager/1-7.1.ec}{}{}{}{}
  and
  \ecinput{examps/parts/tactics/eager/1-7.2.ec}{}{}{}{}
  and
  \ecinput{examps/parts/tactics/eager/1-7.3.ec}{}{}{}{}
  and
  \ecinput{examps/parts/tactics/eager/1-7.4.ec}{}{}{}{}
  and
  \ecinput{examps/parts/tactics/eager/1-7.5.ec}{}{}{}{}
  and
  \ecinput{examps/parts/tactics/eager/1-7.6.ec}{}{}{}{}
  and
  \ecinput{examps/parts/tactics/eager/1-7.7.ec}{}{}{}{}
  and
  \ecinput{examps/parts/tactics/eager/1-7.8.ec}{}{}{}{}
  and
  \ecinput{examps/parts/tactics/eager/1-7.9.ec}{}{}{}{}
  and
  \ecinput{examps/parts/tactics/eager/1-7.10.ec}{}{}{}{}
  and
  \ecinput{examps/parts/tactics/eager/1-7.11.ec}{}{}{}{}
  and
  \ecinput{examps/parts/tactics/eager/1-7.12.ec}{}{}{}{}
  and
  \ecinput{examps/parts/tactics/eager/1-7.13.ec}{}{}{}{}
  \end{tsyntax}

  \begin{tsyntax}{eager ($H$ : $\;s_1$ ~ $\;s_2$ :
    $\;A$ ==> $\;B$) : $\;C$}
   Reduces a goal whose conclusion is a \prhl statement judgement of
   the form
  \begin{center}
    \ec{equiv[$s_1$ $\;t_1$ ~ $\;t_2$ $\;s_2$ : $\;P$ ==> $\;Q$]},
  \end{center}
  to an eager judgement, along with a subgoal for the specified
  \prhl statement judgement $H$, plus an auxiliary goal.

  \medskip
  For example, if the current goal is
  \ecinput{examps/parts/tactics/eager/1-8.0.ec}{}{}{}{} then
  running \ecinput{examps/parts/tactics/eager/1-8.ec}{}{}{}{}
  produces the goals
  \ecinput{examps/parts/tactics/eager/1-8.1.ec}{}{}{}{}
  and
  \ecinput{examps/parts/tactics/eager/1-8.2.ec}{}{}{}{}
  and
  \ecinput{examps/parts/tactics/eager/1-8.3.ec}{}{}{}{}
  \end{tsyntax}
\end{tactic}

% --------------------------------------------------------------------
\begin{tactic}{fel}
  \begin{tsyntax}{fel $\;\mathit{init}$ $\;\mathit{ctr}$ $\;\mathit{stepub}$
                      $\;\mathit{bound}$ $\;\mathit{bad}$ $\;\mathit{conds}$}
    ``fel'' stands for ``failure event lemma''. To use this tactic,
    one must load the theory \ec{FelTactic}. To be applicable, the
    current goal's conclusion must have the form
    \begin{center}
      \ec{Pr[$M$.$p$($a_1$, $\;\ldots$, $\;a_r$) @ &$m$ : $\;\phi$] <= $\;\mathit{ub}$}.
    \end{center}
    Here:
    \begin{itemize}
    \item $\mathit{ub}$ (``upper bound'') is an expression of type \ec{real}.

    \item $\mathit{ctr}$ is the \emph{counter}, an expression of
      type \ec{int} involving program variables.

    \item $\mathit{bad}$ is an expression of type \ec{bool} involving
      program variables. It is the ``bad'' or ``failure'' event.

    \item $\mathit{init}$ is a natural number no bigger than the
      number of statements in $M$.$p$. It is length of the initial
      part of the procedure that ``initializes'' the failure event
      lemma---causing $\mathit{ctr}$ to become $0$ and $\mathit{bad}$
      to become \ec{false}. The non-initialization part of the
      procedure may not \emph{directly} use the program variables on
      which $\mathit{ctr}$ and $\mathit{bad}$ depend. These variables
      may only be modified by concrete procedures \ec{$M$.$p$} may
      directly or indirectly call---such procedures are called
      \emph{oracle procedures}.  If \ec{$M$.$p$} directly or
      indirectly calls an abstract procedure, there must be a module
      constraint saying that the abstract procedure may not modify the
      program variables determining the values of $\mathit{ctr}$ and
      $\mathit{bad}$ or that are used by the oracle procedures.

    \item $\mathit{bound}$ is an expression of type \ec{int} not
      involving program variables. It is an upper bound on the value
      of the counter. The oracle procedures must not increase the
      counter beyond this bound. It must be the case that $\phi$
      implies \ec{$\mathit{bad}$ /\\ $\;\mathit{ctr}$ <=
        $\;\mathit{bound}$}.

    \item $\mathit{conds}$ is a list of \emph{procedure preconditions}
      \begin{center}
        \ec{[$N_1$.$p_1$ : $\;\phi_1$; $\;\ldots$ $\;N_l$.$p_l$ : $\;\phi_l$]},
      \end{center}
      where the $N_i$.$p_i$ are procedures, and the $\phi_i$ are
      expressions of type \ec{bool} involving program variables and
      procedure parameters.  When a procedure's precondition is true,
      it must increase the counter's value (and may cause
      $\mathit{bad}$ to become true); when it isn't true, it must
      preserve the counter's value (as well as the value of
      $\mathit{bad}$).

    \item $\mathit{stepub}$ is a function of type \ec{int -> real},
      intuitively providing an upper bound as a function of the
      counter's current value. When a procedure's precondition holds,
      the probability that $\mathit{bad}$ becomes set during that
      single call must be upper-bounded by the application of
      $\mathit{stepub}$ to the counter's value. It must be the case
      that the summation of \ec{$\mathit{stepub}$ $\;i$}, as $i$
      ranges from $0$ to $\mathit{bound} - 1$, is upper-bounded by
      $\mathit{ub}$.

    \end{itemize}
    The subgoals generated by \ec{fel} enforce the above rules. The
    best way to understand the details is by an example.

    \medskip For example, consider the declarations
    \ecinput{examps/tactics/fel/1.ec}{}{30-76}{} Here, the oracle has
    a boolean variable \ec{won}, which is the bad event. It also has a
    list of integers \ec{gens}, all of which are within the range $1$
    to \ec{upp}, inclusive---the integers ``generated'' so far. The
    counter is the size of \ec{gens}. The procedure \ec{gen} randomly
    generates such an integer, setting \ec{won} to \ec{true} if the
    integer was previously generated. And the procedure \ec{add} adds
    a new integer to the list of generated integers, without possibily
    setting \ec{bad}. Both \ec{gen} and \ec{add} do nothing when the
    counter reaches the bound \ec{n}.  The adversary has access to both
    \ec{gen} and \ec{bad}.

    Suppose that our goal is to prove
    \ecinput{examps/tactics/fel/1.ec}{}{111-112}{}
    We can't prove this directly, using \ec{fel}. Instead, we
    must first prove
    \ecinput{examps/tactics/fel/1.ec}{}{78-80}{}
    and then use \rtactic{byequiv} to obtain \ec{G_UB}.

    If the current goal is
    \ecinput{examps/parts/tactics/fel/1-1.0.ec}{}{}{}{} then
    running \ecinput{examps/parts/tactics/fel/1-1.ec}{}{}{}{}
    produces the goals
    \ecinput{examps/parts/tactics/fel/1-1.1.ec}{}{}{}{}
    and
    \ecinput{examps/parts/tactics/fel/1-1.2.ec}{}{}{}{}
    and
    \ecinput{examps/parts/tactics/fel/1-1.3.ec}{}{}{}{}
    and
    \ecinput{examps/parts/tactics/fel/1-1.4.ec}{}{}{}{}
    and
    \ecinput{examps/parts/tactics/fel/1-1.5.ec}{}{}{}{}
    and
    \ecinput{examps/parts/tactics/fel/1-1.6.ec}{}{}{}{}
    and
    \ecinput{examps/parts/tactics/fel/1-1.7.ec}{}{}{}{}
    and
    \ecinput{examps/parts/tactics/fel/1-1.8.ec}{}{}{}{}
    and
    \ecinput{examps/parts/tactics/fel/1-1.9.ec}{}{}{}{}
  \end{tsyntax}

  \begin{tsyntax}{fel $\;\mathit{init}$ $\;\mathit{ctr}$ $\;\mathit{stepub}$
                      $\;\mathit{bound}$ $\;\mathit{bad}$ $\;\mathit{conds}$
                      $\;\mathit{inv}$}
    This variant of \ec{fel} is like the preceding one except for
    the addition of an invariant, $\mathit{inv}$, argument, which must be
    established by \ec{$M$.$p$} and which the oracle procedures must
    preserve.

    \medskip For example, consider the declarations
    \ecinput{examps/tactics/fel/2.ec}{}{30-80}{}
    These are similar to the ones used to illustrate the preceding
    variant of \ec{fel}, except that this time an explicit counter
    variable \ec{Or.ctr} is used, and \ec{fel}'s optional invariant
    is used to ensure the counter's value is continually equal
    to the size of \ec{Or.gens}.

    Suppose that our goal is to prove
    \ecinput{examps/tactics/fel/2.ec}{}{116-117}{} As with the
    preceduing variant, we can't prove this directly, using
    \ec{fel}. Instead, we must first prove
    \ecinput{examps/tactics/fel/2.ec}{}{82-84}{} and then use
    \rtactic{byequiv} to obtain \ec{G_UB}.

    If the current goal is
    \ecinput{examps/parts/tactics/fel/2-1.0.ec}{}{}{}{} then
    running \ecinput{examps/parts/tactics/fel/2-1.ec}{}{}{}{}
    produces the goals
    \ecinput{examps/parts/tactics/fel/2-1.1.ec}{}{}{}{}
    and
    \ecinput{examps/parts/tactics/fel/2-1.2.ec}{}{}{}{}
    and
    \ecinput{examps/parts/tactics/fel/2-1.3.ec}{}{}{}{}
    and
    \ecinput{examps/parts/tactics/fel/2-1.4.ec}{}{}{}{}
    and
    \ecinput{examps/parts/tactics/fel/2-1.5.ec}{}{}{}{}
    and
    \ecinput{examps/parts/tactics/fel/2-1.6.ec}{}{}{}{}
    and
    \ecinput{examps/parts/tactics/fel/2-1.7.ec}{}{}{}{}
    and
    \ecinput{examps/parts/tactics/fel/2-1.8.ec}{}{}{}{}
    and
    \ecinput{examps/parts/tactics/fel/2-1.9.ec}{}{}{}{}
  \end{tsyntax}
\end{tactic}

