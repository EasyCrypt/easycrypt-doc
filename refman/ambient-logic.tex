\section{Ambient logic}
\label{sec:ambientlogic}

In this section, we describe the proof terms, tactics and tacticals of
\EasyCrypt's ambient logic.

\subsection{Proof Terms}
\label{subsec:proofterms}

Formulas introduce identifier and formula assumptions using universal
quantifiers and implications. For example, the formula
\begin{easycrypt}{}{}
forall (x y : bool), x = y => forall (z : bool), y = z => x = z.
\end{easycrypt}
introduces the assumptions
\begin{easycrypt}{}{}
x     : bool
y     : bool
eq_xy : x = y
z     : bool
eq_yz : y = z
\end{easycrypt}
(where the names of the two formulas were chosen to be meaningful),
and has \ec{x = z} as its conclusion. We refer to the first assumption
of a formula as the formula's \emph{top assumption}. E.g., the top
assumption of the preceding formula is \ec{x : bool}.

\EasyCrypt has \emph{proof terms}, which partially describe how
to prove a formula.  Their syntax is described in Figure~\ref{fig:proofterms},
where $X$ ranges over lemma (or formula assumption) names.
\begin{figure}
  \begin{center}
  \begin{tabular}{rcl>{\bf}l}
    $p$ & ::=
      & {\ec{_}} & proof hole \\
     && {\ec{($X$, $\;q_1$, $\;\ldots$, $\;q_n$)}} & lemma application \\[.2cm]
    $q$ & ::=
      & {$e$} & expression \\
      && {$p$} & proof term \\
  \end{tabular}
  \end{center}
  \caption{\label{fig:proofterms}Proof Terms}
\end{figure}
A proof term for a lemma (or formula assumption) $X$ has components
corresponding to the assumptions introduced by $X$.  A component
corresponding to a variable consists of an expression of the
variable's type. The proof term is explaining how the instantiation of
the lemma's conclusion with these expressions may be proved.  A
formula component consists of a proof term explaining how the
instantiation of the formula may be proved.  Proof holes will get
turned into subgoals when a proof term is used in backward reasoning,
e.g., by the \rtactic{apply} tactic. \fix{Need explanation of how a
  proof term may be used in forward reasoning.}

Consider, e.g., the following declarations and axioms
\begin{easycrypt}{}{}
pred P : int.
pred Q : int.
pred R : int.
axiom P (x : int) : P x.
axiom Q (x : int) : P x => Q x.
axiom R (x : int) : P(x + 1) => Q x => R x.
\end{easycrypt}
Then, given that \ec{x : int} is an assumption,
\begin{easycrypt}{}{}
(R x (P(x + 1)) (Q x (P x)))
\end{easycrypt}
is a proof term proving the conclusion \ec{R x}. And
\begin{easycrypt}{}{}
(R x _ (Q x _))
\end{easycrypt}
is a proof term that turns proofs of \ec{P(x + 1)} and \ec{P x}
into proofs of \ec{R x}. When used in backward reasoning, it
will reduce a goal with conclusion \ec{R x} to subgoals with
conclusions \ec{P(x + 1)} and \ec{P x}.
\fix{Can it be used in forward reasoning?}

Some of a proof term's expressions may be replaced by \ec{_}, asking
\EasyCrypt to infer them from the context.  Going even further, one
may abbreviate a one-level proof term all of whose components are
\ec{_} to just its lemma name. For example, we can write \ec{R} for
\ec{(R _ _ _)}.  When used in backward reasoning, it will reduce a
goal with conclusion \ec{R x} to subgoals with conclusions \ec{P(x +
  1)} and \ec{Q x}. \fix{In forward reasoning they aren't equivalent---why?}

\subsection{Occurrence Selectors and Rewriting Directions}
\label{subsec:occsels}

Some ambient logic tactics use \emph{occurrence selectors} to restrict
their operation to certain occurrences of a term or formula in a
goal's conclusion or formula assumption. The syntax is \ec{\{$i_1$,
  $\;\ldots$, $\;i_n$\}}, specifying that only occurrences $i_1$
throught $i_n$ of the term/formula in a depth-first, left-to-right
traversal of the goal's conclusion or formula assumption should be
operated on. Specifying \ec{\{- $i_1$, $\;\ldots$, $\;i_n$\}}
restricts attention to all occurrences \emph{not} in the following
list. They may also be empty, meaning that all applicable occurrences
should be operated on.

Some ambient logic tactics use \emph{rewriting directions}, $\mathit{dir}$,
which may either be empty (meaning rewriting from left to right), or \ec{-},
meaning rewriting from right to left.

\subsection{Introduction and Generalization}
\label{subsec:introgen}

\paragraph{Introduction.}

One moves the assumptions of a goal's conclusion into the goal's
context using the introduction tactical.  This tactical uses
introduction patterns, which are defined in Figure~\ref{fig:intropat}.
\begin{figure}
  \begin{center}
  \begin{tabular}{rcl>{\bf}l}
    $\iota$ & ::=
      & {$b$} & name \\
      && {\ec{$b\,$!}} & rename \\
      && {\ec{_}} & no name \\
      && {\ec{+}} & auto revert \\
      && {\ec{?}} & find name \\
      && {$\mathit{occ}\;$\ec{->}} & rewrite using assumption \\
      && {$\mathit{occ}\;$\ec{<-}} & rewrite in reverse using assumption \\
      && {\ec{->>}} & substitute using assumption \\
      && {\ec{<<-}} & substitute in reverse using assumption \\
      && {\ec{/$p$}} & replace assumption by applying proof term \\
      && {\ec{\{$a_1\cdots a_n$\}}} & clear introduced assumptions \\
      && {\ec{/=}} & simplify \\
      && {\ec{//}} & trivial \\
      && {\ec{//=}} & simplify then trivial \\
      && {\ec{$\mathit{dir}\;\mathit{occ}\;$@/$\mathit{op}$}} & unfold definition of operator \\
      && {\ec{[$\iota_{11}\cdots\iota_{1{m_1}}$ | $\;\cdots$ | $\;\iota_{r1}\cdots\iota_{r{m_r}}$]}} & case pattern \\[.2cm]
    $\mathit{b}$ & ::=
      & {$x$} & identifier \\
      && {$M$} & module name \\
      && {\ec{&$m$}} & memory name \\
  \end{tabular}
  \end{center}
  \caption{\label{fig:intropat} Introduction Patterns}
\end{figure}
In this definition, $\mathit{occ}$ ranges over occurrence selectors,
and $\mathit{dir}$ ranges over directions---see
Subsection~\ref{subsec:occsels}).

If a list $\iota_1,\ldots,\iota_n$ of introduction patterns consists
entirely of \ec{//} (apply the \rtactic{trivial} tactic), \ec{/=}
(apply the \rtactic{simplify} tactic) and \ec{//=} (apply the
\ec{simplify} and then \ec{trivial}), then \emph{applying}
$\iota_1,\ldots,\iota_n$ \emph{to} a list of goals $G_1,\ldots,G_m$ is
done by applying the tactics corresponding to the $\iota_i$ in order
to each $G_j$, causing some of the goals to be solved and thus
disappear and some of the goals to be simplified.

\begin{tactic}[$\tau$=>$\;\iota_1 \cdots \iota_n$]{introduction}
  \begin{tsyntax}[empty]{}
    Runs the tactic $\tau$, matching the resulting goals, $G_1,\ldots,G_l$,
    with the introduction patterns $\iota_1,\ldots,\iota_n$:
    \begin{itemize}
    \item Suppose $k$ is such that all of $\iota_1,\ldots,\iota_{k-1}$
      are \ec{//}, \ec{/=} and \ec{//=}, and either $k>n$ or $\iota_k$
      is not \ec{//}, \ec{/=} or \ec{//=}.

    \item Let $G'_1,\ldots,G'_{l'}$ be the goals resulting from
      applying $\iota_1,\ldots,\iota_{k-1}$ to $G_1,\ldots,G_l$.

    \item If $l'=0$, the tactical produces no subgoals.

    \item Otherwise, if $k>n$, the tactical's result is
      $G'_1,\ldots,G'_{l'}$.

    \item Otherwise, if $\iota_k$ is not a case pattern, each subgoal
      $G'_i$ is matched against $\iota_k,\ldots,\iota_n$ by the
      procedure described below, with the resulting subgoals being
      collected into a list of goals (maintaining order viz a viz the
      indices $i$) as the tactical's result.

    \item Otherwise, $\iota_k$ is a case pattern
          \ec{[$\iota_{11}\cdots\iota_{1{m_1}}$ | $\;\cdots$ | $\;\iota_{r1}\cdots\iota_{r{m_r}}$]}.

     \item If $\tau$ is not equivalent to \rtactic{idtac}, the tactic fails
       unless $r = l'$, in which case each $G'_i$ is matched against 
        \begin{displaymath}
          \iota_{i1}\cdots\iota_{i{m_i}}\iota_{k+1}\cdots \iota_n
        \end{displaymath}
       by the procedure described below, with the resulting subgoals
       being collected into the tactical's result.

     \item Otherwise, $\tau$ is equivalent to \rtactic{idtac} (and so
       $l'=1$). In this case $G'_1$ is matched against
       $\iota_k,\ldots,\iota_n$ by the procedure described below, with
       the resulting subgoals being collected into a list of goals as
       the tactical's result.
    \end{itemize}

    \paragraph{Matching a single goal against a list of patterns:}

    To match a goal $G$ against a list of introduction patterns
    $\iota_1,\ldots,\iota_n$, the introduction patterns are processed
    from left-to-right, as follows:
    \begin{itemize}
    \item ($b$)\quad The top assumption (universally quantified
      identifier, module name or memory; or left side of implication)
      is consumed, and introduced with this name. Fails if the top
      assumption has neither of these forms.

    \item (\ec{$b\,$!})\quad Same as the preceding case, except that
      $b$ is used as the base of the introduced name, extending the
      base to avoid naming conflicts.

    \item (\ec{_})\quad Same as the preceding case, except the
      assumption is introduced with an anonymous name (which can't be
      uttered by the user).

    \item (\ec{+})\quad Same as the preceding case, except that after
      a branch of the procedure completes, yielding a goal, the
      assumption will be reverted, i.e., un-introduced (using a
      universal quantifier or implication as appropriate).

    \item (\ec{?})\quad Same as the preceding case, except \EasyCrypt
      chooses the name by which the assumption is introduced (using
      universally quantified names as assumption bases).

    \item ($\mathit{occ}\;$\ec{->})\quad Consume the top assumption,
      which must be an equality, and use it as a left-to-right rewriting
      rule in the remainder of the goal's conclusion, restricting rewriting
      to the specified occurrences of the equality's left side.

    \item ($\mathit{occ}\;$\ec{<-})\quad The same as the preceding case,
      except the rewriting is from right-to-left.

    \item (\ec{->>})\quad The same as \ec{->}, except the consumed
      equality assuption is used to perform a left-to-right substitution
      in the entire goal, i.e., in its assumptions, as well as its
      conclusion.

    \item (\ec{<<-})\quad The same as the preceding case, except
      the substitution is from right-to-left.

    \item (\ec{/$p$})\quad Replace the top assumption by the result
    of applying the proof term $p$ to it using forward reasoning.

    \item (\ec{\{$a_1\cdots a_n$\}})\quad Doesn't affect the goal's
      conclusion, but clears the specified assumptions, i.e., removes
      them. Fails if one or more of the assumptions can't be cleared,
      because a remaining assumption depends upon it.

    \item (\ec{/=})\quad Apply \rtactic{simplify} to goal's conclusion.

    \item (\ec{//})\quad Apply \rtactic{trivial} to goal's conclusion;
      this may solve the goal, i.e., so that the procedure's current
      branch yields no goals.

    \item (\ec{/=})\quad Apply \rtactic{simplify} and then \rtactic{trivial}
      to goal's conclusion; this may solve the goal, so that the
      procedure's current branch yields no goals.

    \item ({\ec{$\mathit{dir}\;\mathit{occ}\;$@/$\mathit{op}$}})\quad
      Unfold (fold, if the direction is \ec{-}) the definition of
      operator $\mathit{op}$ at the specified occurrences of the
      goal's conclusion. See the \rtactic{rewrite} tactic for the
      details.

    \item (\ec{[$\iota_{11}\cdots\iota_{1{m_1}}$ | $\;\cdots$ | $\;\iota_{r1}\cdots\iota_{r{m_r}}$]})\quad
      \begin{itemize}
      \item If $r=0$, then the top assumption of the goal is destructed
        using the \rtactic{case} tactic, the resulting goals are
        matched against $\iota_2,\ldots,\iota_n$, and their subgoals
        are assembled into a list of goals.

      \item Otherwise $r>0$. The goal's top assumption is destructed
        using the \rtactic{case} tactic, yielding subgoals
        $H_1,\ldots H_p$.  If $p\neq r$, the procedure fails. Otherwise
        each subgoal $H_i$ is matched against
        \begin{displaymath}
          \iota_{i1}\cdots\iota_{i{m_i}}\iota_2\cdots \iota_n
        \end{displaymath}
        with the resulting goals being collected into a list as
        the procedure's result.
      \end{itemize}
    \end{itemize}

    The following examples use the tactic \rtactic{move}, which is
    equivalent to \rtactic{idtac}.
    In its simplest form, the introduction tactical simply gives names
    to assumptions.  For example, if the current goal is
    \ecinput{examps/parts/tactics/introduction/1-1.0.ec}{}{}{}{}
    then running
    \ecinput{examps/parts/tactics/introduction/1-1.ec}{}{}{}{}
    produces
    \ecinput{examps/parts/tactics/introduction/1-1.1.ec}{}{}{}{}
    Alternatively, we can use the introduction pattern \ec{?}
    to let \EasyCrypt choose the assumption names, using
    \ec{H} as a base for formula assumptions and starting
    from the identifier names given in universal quantifiers:
    \ecinput{examps/parts/tactics/introduction/2-1.ec}{}{}{}{}
    produces
    \ecinput{examps/parts/tactics/introduction/2-1.1.ec}{}{}{}{}
    Or we can use the \ec{!} pattern suffix to specify our
    own base assumption names: running
    \ecinput{examps/parts/tactics/introduction/3-1.ec}{}{}{}{}
    produces
    \ecinput{examps/parts/tactics/introduction/3-1.1.ec}{}{}{}{}

    To see how the \ec{->} rewriting pattern works, suppose
    the current goal is
    \ecinput{examps/parts/tactics/introduction/4-1.0.ec}{}{}{}{}
    Then running
    \ecinput{examps/parts/tactics/introduction/4-1.ec}{}{}{}{}
    produces
    \ecinput{examps/parts/tactics/introduction/4-1.1.ec}{}{}{}{}
    Alternatively, one can introduce the assumption \ec{x = y},
    and then use the \ec{->>} substitution pattern:
    if the current goal is
    \ecinput{examps/parts/tactics/introduction/8-1.0.ec}{}{}{}{}
    then running
    \ecinput{examps/parts/tactics/introduction/8-1.ec}{}{}{}{}
    produces
    \ecinput{examps/parts/tactics/introduction/8-1.1.ec}{}{}{}{}

    To see how a view may be applied to a not-yet-introduced formula
    assumption, suppose the current goal is
    \ecinput{examps/parts/tactics/introduction/5-1.0.ec}{}{}{}{}
    Then running
    \ecinput{examps/parts/tactics/introduction/5-1.ec}{}{}{}{}
    produces
    \ecinput{examps/parts/tactics/introduction/5-1.1.ec}{}{}{}{}
    And then running
    \ecinput{examps/parts/tactics/introduction/5-2.ec}{}{}{}{}
    on this goal produces
    \ecinput{examps/parts/tactics/introduction/5-2.1.ec}{}{}{}{}

    Finally, let's see examples of how a disjunction assumption
    may be destructed, either using the \ec{case} tactic followed
    by a case introduction pattern, or by making the
    case introduction pattern do the destruction.
    For the first case, if the current goal is
    \ecinput{examps/parts/tactics/introduction/6-1.0.ec}{}{}{}{}
    then running
    \ecinput{examps/parts/tactics/introduction/6-1.ec}{}{}{}{}
    produces the two goals
    \ecinput{examps/parts/tactics/introduction/6-1.1.ec}{}{}{}{}
    and
    \ecinput{examps/parts/tactics/introduction/6-1.2.ec}{}{}{}{}
    And for the second case, if the current goal is
    \ecinput{examps/parts/tactics/introduction/7-1.0.ec}{}{}{}{}
    then running
    \ecinput{examps/parts/tactics/introduction/7-1.ec}{}{}{}{}
    produces the two goals
    \ecinput{examps/parts/tactics/introduction/7-1.1.ec}{}{}{}{}
    and
    \ecinput{examps/parts/tactics/introduction/7-1.2.ec}{}{}{}{}
    Note how we used the clear pattern to discard the assumption
    \ec{X}.
  \end{tsyntax}
\end{tactic}

\paragraph{Generalization.}

The generalization tactical moves assumptions from the context into
the conclusion and generalizes subterms or formulas of the conclusion.

\begin{tactic}[$\tau$: $\;\pi_1 \cdots \pi_n$]{generalization}
  \begin{tsyntax}[empty]{$\tau$: $\;\pi_1 \cdots \pi_n$}
    Generalize the patterns $\pi_1, \ldots, \pi_n$, starting from
    $\pi_n$ and going back, and then run tactic $\tau$.  \emph{This
      tactical is only applicable to certain tactics: \rtactic{move},
      \rtactic{case} (just the version that destructs the top
      assumption of a goal's conclusion) and \rtactic{elim}.}

    \begin{itemize}
    \item When $\pi$ is an assumption from the context, it's moved
      back into the conclusion, using universal quantification or an
      implication, as appropriate. If one assumption depends on
      another, one can't generalize the later without also
      generalizing the former.

      For example, if the current goal is
      \ecinput{examps/parts/tactics/generalize/2-1.0.ec}{}{}{}{} then
      running \ecinput{examps/parts/tactics/generalize/2-1.ec}{}{}{}{}
      produces
      \ecinput{examps/parts/tactics/generalize/2-1.1.ec}{}{}{}{} In
      this example, one can't generalize \ec{x} without also
      generalizing \ec{eq_xy}.

    \item $\pi$ may also be a subformula or subterm of the goal, or
      \ec{_}, which stands for the whole goal, possibly prefixed by an
      occurrence selector. This replaces the formula or subterm with a
      universally quantified identifier of the approprate type.

      For example, if the current goal is
      \ecinput{examps/parts/tactics/generalize/1-1.0.ec}{}{}{}{} then
      running \ecinput{examps/parts/tactics/generalize/1-1.ec}{}{}{}{}
      produces
      \ecinput{examps/parts/tactics/generalize/1-1.1.ec}{}{}{}{}
      Alternatively, running
      \ecinput{examps/parts/tactics/generalize/3-1.ec}{}{}{}{}
      produces
      \ecinput{examps/parts/tactics/generalize/3-1.1.ec}{}{}{}{}
    \end{itemize}
  \end{tsyntax}
\end{tactic}

\subsection{Tactics}

% --------------------------------------------------------------------
\begin{tactic}{idtac}
\end{tactic}

% --------------------------------------------------------------------
\begin{tactic}[move | move: $\;\pi_1 \cdots \pi_n$]{move}
  \begin{tsyntax}{move}
     Does nothing, equivalent to \rtactic{idtac}. This form is mainly
     used in conjonction with an introduction pattern (see
     Section~\ref{s:intro-pattern}), e.g. \ls!move=> $\iota_1 \cdots \iota_n$!.
  \end{tsyntax}

  \begin{tsyntax}{move: $\;\pi_1 \cdots \pi_n$}
    Generalize the patterns $\pi_1, \cdots, \pi_n$, starting from
    $\pi_n$ and going back.
    %See Section~\ref{s:gen-pattern} for more
    %information on the generalization mechanism.
  \end{tsyntax}
\end{tactic}

% --------------------------------------------------------------------
\begin{tactic}{clear}
\end{tactic}

% --------------------------------------------------------------------
\begin{tactic}{assumption}
  \begin{tsyntax}[empty]{assumption}
    Search in the context for a hypothesis that is convertible to the
    goal's conclusion, solving the goal if one is found. Fail if none
    can be found.

    For example, if the current goal is
    \ecinput{../examps/parts/tactics/assumption/1-1.0.ec}{}{}{}{} then
    running
    \ecinput{../examps/parts/tactics/assumption/1-1.ec}{}{}{}{} solves
    the goal.
  \end{tsyntax}
\end{tactic}

% --------------------------------------------------------------------
\begin{tactic}{reflexivity}
  \begin{tsyntax}[empty]{reflexivity}
  Solve goals of the form \ec{b = b} (up to computation).

  For example, if the current goal is
  \ecinput{../examps/parts/tactics/reflexivity/1-1.0.ec}{}{}{}{} then
  running \ecinput{../examps/parts/tactics/reflexivity/1-1.ec}{}{}{}{}
  solves the goal.
  \end{tsyntax}
\end{tactic}

% --------------------------------------------------------------------
\begin{tactic}{left}
  \begin{tsyntax}[empty]{left}
    Reduce a goal whose conclusion is a disjunction to one whose
    conclusion is its left member.

    For example, if the current goal is
    \ecinput{examps/parts/tactics/left/1-1.0.ec} then
    running \ecinput{examps/parts/tactics/left/1-1.ec}
    produces the goal
    \ecinput{examps/parts/tactics/left/1-1.1.ec}
  \end{tsyntax}
\end{tactic}

% --------------------------------------------------------------------
\begin{tactic}{right}
  \begin{tsyntax}[empty]{right}
  Reduce a disjunctive goal to its right member.

  For example, if the current goal is
  \ecinput{../examps/parts/tactics/right/1-1.0.ec}{}{}{}{} then
  running \ecinput{../examps/parts/tactics/right/1-1.ec}{}{}{}{}
  produces the goal
  \ecinput{../examps/parts/tactics/right/1-1.1.ec}{}{}{}{}
  \end{tsyntax}
\end{tactic}

% --------------------------------------------------------------------
\begin{tactic}[exists $\;e$]{exists}
  \begin{tsyntax}[empty]{exists $\;e$}
    Reduces proving an existential to proving the witness $e$ satisfies
    the existential's body.

    For example, if the current goal is
    \ecinput{../examps/parts/tactics/exists/1-1.0.ec}{}{}{}{} then
    running \ecinput{../examps/parts/tactics/exists/1-1.ec}{}{}{}{}
    produces the goal
    \ecinput{../examps/parts/tactics/exists/1-1.1.ec}{}{}{}{}
  \end{tsyntax}
\end{tactic}

% --------------------------------------------------------------------
\begin{tactic}{split}
  \begin{tsyntax}[empty]{split}
  Break an intrinsically conjunctive goal into its component subgoals.
  For instance, it can:
  \begin{itemize}
  \item close any goal that is convertible to \ec{true} or provable by
    \ec{reflexivity},
    \item replace a logical equivalence by the direct and indirect implications,
    \item replace a goal of the form \ec{$\phi_1\!$ /\\ $\;\,\phi_2$} by the two
      subgoals for $\phi_1$ and $\phi_2$. The same applies for a goal of
      the form \ec{$\phi_1\!$ && $\;\,\phi_2$},
    \item replace an equality between $n$-tuples by $n$ equalities
          on their components.
  \end{itemize}

  For example, if the current goal is
  \ecinput{../examps/parts/tactics/split/1-1.0.ec}{}{}{}{} then
  running \ecinput{../examps/parts/tactics/split/1-1.ec}{}{}{}{}
  produces the goals
  \ecinput{../examps/parts/tactics/split/1-1.1.ec}{}{}{}{}
  and
  \ecinput{../examps/parts/tactics/split/1-1.2.ec}{}{}{}{}
  And if the current goal is
  \ecinput{../examps/parts/tactics/split/2-1.0.ec}{}{}{}{} then
  running \ecinput{../examps/parts/tactics/split/2-1.ec}{}{}{}{}
  produces the goals
  \ecinput{../examps/parts/tactics/split/2-1.1.ec}{}{}{}{}
  and
  \ecinput{../examps/parts/tactics/split/2-1.2.ec}{}{}{}{}
  Repeating the last example with \ec{&&} rather than \ec{/\\},
  if the current goal is
  \ecinput{../examps/parts/tactics/split/2a-1.0.ec}{}{}{}{} then
  running \ecinput{../examps/parts/tactics/split/2a-1.ec}{}{}{}{}
  produces the goals
  \ecinput{../examps/parts/tactics/split/2a-1.1.ec}{}{}{}{}
  and
  \ecinput{../examps/parts/tactics/split/2a-1.2.ec}{}{}{}{}
  This illustrates the difference between \ec{/\\} and \ec{&&}.
  And if the current goal is
  \ecinput{../examps/parts/tactics/split/3-1.0.ec}{}{}{}{} then
  running \ecinput{../examps/parts/tactics/split/3-1.ec}{}{}{}{}
  produces the goals
  \ecinput{../examps/parts/tactics/split/3-1.1.ec}{}{}{}{}
  and
  \ecinput{../examps/parts/tactics/split/3-1.2.ec}{}{}{}{}
  \end{tsyntax}
\end{tactic}

% --------------------------------------------------------------------
\begin{tactic}{congr}
  \begin{tsyntax}[empty]{congr}
  Replace a goal of the form \ec{f t$_1$ ... t$_n$ = f u$_1$ ... u$_n$}
  with the subgoals \ec{t$_i$ = u$_i$} for all \ec{$i$}. Subgoals solvable
  by \ec{reflexivity} are automatically closed.
  \end{tsyntax}
\end{tactic}

% --------------------------------------------------------------------
\begin{tactic}[subst | subst x]{subst}
  \begin{tsyntax}[empty]{subst}
  Search for the first equation of the form \ec{x = f} or \ec{f = x} in the context
  and replace all the occurrences of \ec{x} by \ec{f} everywhere in the context and the
  goal before clearing it. If no identifier is given, repeatedly apply the tactic to
  all identifiers for which such an equation exists.
  \end{tsyntax}
\end{tactic}


% --------------------------------------------------------------------
\begin{tactic}{trivial}
  \begin{tsyntax}[empty]{trivial}
  Try to solve the goal by using a mixture of low-level tactics.
  This tactic is called by the intro-pattern \ec{//}.
  \end{tsyntax}
\end{tactic}

% --------------------------------------------------------------------
\begin{tactic}{done}
  \begin{tsyntax}[empty]{done}
  \fix{Missing description of done}.
  \end{tsyntax}
\end{tactic}

% --------------------------------------------------------------------
\begin{tactic}{simplify}
  \begin{tsyntax}[empty]{simplify}
  \fix{Missing description of simplify}.
  \end{tsyntax}
\end{tactic}

% --------------------------------------------------------------------
\begin{tactic}[progress | progress $\;\tau$]{progress}
  \begin{tsyntax}[empty]{progress}
  Break the goal into multiple \emph{simpler} ones by repeatedly applying
  \ec{split}, \ec{subst} and \ec{move=>}. The tactic $\tau$ given to
  \ec{progress} is tentatively applied after each step.

  For example, if the current goal is
  \ecinput{examps/parts/tactics/progress/1-1.0.ec}{}{}{}{} then
  running \ecinput{examps/parts/tactics/progress/1-1.ec}{}{}{}{}
  solves the goal.
  \end{tsyntax}

  \fix{Describe \ec{progress} options.}
\end{tactic}

% --------------------------------------------------------------------
\begin{tactic}{smt}
  \begin{tsyntax}[empty]{smt}
  \fix{Missing description of smt}.
  \end{tsyntax}
\end{tactic}

% --------------------------------------------------------------------
\begin{tactic}{admit}
  \begin{tsyntax}[empty]{admit}
  Close the current goal by admitting it.

  For example, if the current goal is
  \ecinput{examps/parts/tactics/admit/1-1.0.ec} then
  running \ecinput{examps/parts/tactics/admit/1-1.ec}
  solves the goal.
  \end{tsyntax}
\end{tactic}


% --------------------------------------------------------------------
\begin{tactic}[change $\;\phi$]{change}
  \begin{tsyntax}[empty]{change $\;\phi$}
  Change the current goal for $\phi$ --- $\phi$ must be \emph{convertible}
  to the current goal.

  For example, if the current goal is
  \ecinput{../examps/parts/tactics/change/1-1.0.ec}{}{}{}{} then
  running \ecinput{../examps/parts/tactics/change/1-1.ec}{}{}{}{}
  produces the goal
  \ecinput{../examps/parts/tactics/change/1-1.1.ec}{}{}{}{}
  \end{tsyntax}
\end{tactic}

% --------------------------------------------------------------------
\begin{tactic}[pose x := $\;\pi$]{pose}
  \begin{tsyntax}[empty]{pose}
  Search for the first subterm \ec{p} of the goal matching $\pi$ and
  leading to the full instantiation of the pattern. Then introduce,
  after instantiation, the local definition \ec{x := p} and abstract
  all occurrences of \ec{p} in the goal as \ec{x}. An occurence
  selector can be used (see \rtactic{rewrite}).
  \end{tsyntax}
\end{tactic}

\begin{tactic}[have $\;\iota$: $\;\phi$]{have}
  \begin{tsyntax}[empty]{have}
    Logical cut. Generate two subgoals: one for the cut formula
    $\phi$, and one for \ec{$\phi$ => $\;\psi$} where $\psi$ is the
    current goal's conclusion. Moreover, the introduction pattern
    \ec{$\iota$} is applied to the second subgoal.

  For example, if the current goal is
  \ecinput{../examps/parts/tactics/have/1-1.0.ec}{}{}{}{} then
  running \ecinput{../examps/parts/tactics/have/1-1.ec}{}{}{}{}
  produces the goals
  \ecinput{../examps/parts/tactics/have/1-1.1.ec}{}{}{}{}
  and
  \ecinput{../examps/parts/tactics/have/1-1.2.ec}{}{}{}{}
  \end{tsyntax}

  \begin{tsyntax}{have $\;\iota$: $\;\phi$ by $\;\tau$}
  Attempts to use tactic $\tau$ to close the first subgoal (corresponding to
  the cut formula $\phi$), and fails if impossible.
  \end{tsyntax}
\end{tactic}


% --------------------------------------------------------------------
\begin{tactic}[cut $\;\iota$: $\;\phi$]{cut}
  \begin{tsyntax}[empty]{cut}
  Logical cut. Generate two subgoals: one for the cut formula $\phi$,
  and one for $\phi \Rightarrow G$ where $G$ is the current goal. Moreover,
  the intro-pattern \tct{$\iota$} is applied to the second subgoal.
  \end{tsyntax}
\end{tactic}


% --------------------------------------------------------------------
\begin{tactic}[apply (p : proof-term)]{apply}
  \begin{tsyntax}[empty]{apply (p : proof-term)}
   If \tct{p} is a proof-term for the pattern (formula)
  \begin{center}
    \tct{forall (x1 : t1) ... (xn : tn), A1 -> ... -> An -> B}
  \end{center}
  \noindent then \tct{apply} tries to match B with the current G. If the
  match succeeds and leads to the full instantiation of the pattern,
  then the goal is replaced, after instantiation, with the $n$ subgoals
  \tct{A1, ..., An}.
  \end{tsyntax}
\end{tactic}

% --------------------------------------------------------------------
\begin{tactic}{exact (p : proofterm)}
  \begin{tsyntax}[empty]{exact}
  Equivalent to \ec{by apply (p : proofterm)}, i.e. apply the given
  proof-term and the try to close the goals with \ec{trivial} - failing
  if not all goals can be closed.
  \end{tsyntax}
\end{tactic}

% --------------------------------------------------------------------
\begin{tactic}[rewrite $\;\pi_1 \cdots \pi_n$ | rewrite $\;\pi_1 \cdots \pi_n$ in $\;H$]{rewrite}
  \begin{tsyntax}{rewrite $\;\pi_1 \cdots \pi_n$}
  Rewrite the rewrite-pattern $\pi_1 \cdots \pi_n$ from left to right,
  where the $\pi_i$ can be of the following form:
  \begin{itemize}
  \item one of \ec{//}, \ec{/=}, \ec{//=},
  \item a proof-term $p$, or
  \item a pattern prefixed by \ec{/} (slash).
  \end{itemize}
  The two last forms can be prefixed by a direction indicator (the
  sign \ec{-}, see Subsection~\ref{subsec:occsels}), followed by an
  occurrence selector (see Subsection~\ref{subsec:occsels}), followed
  (for proof-terms only) by a repetition marker (\ec{!}, \ec{?},
  \ec{n!} or \ec{n?}). All these prefixes are optional.

  \smallskip
  Depending on the form of $\pi$, \ec{rewrite $\;\pi$} does the following:
    \begin{itemize}
    \item For \ec{//}, \ec{/=}, and \ec{//=}, see
      Subsection~\ref{subsec:introgen}.

    \item If $\pi$ is a proof-term with conclusion $f_1=f_2$, then
      \ec{rewrite} searches for the first subterm of the goal's
      conclusion matching $f_1$ and resulting in the full
      instantiation of the pattern.  It then replaces, after
      instantiation of the pattern, all the occurrences of $f_1$ by
      $f_2$ in the goal's conclusion, and creates new subgoals for the
      instantiations of the assuptions of $p$.  If no subterms of the
      goal's conclusion match $f_1$ or if the pattern cannot be fully
      instantiated by matching, the tactic fails.  The tactic works
      the same if the pattern ends by \ec{$f_1\!$ <=> $\;f_2$}. If the
      direction indicator \ec{-} is given, \ec{rewrite} works in the
      reverse direction, searching for a match of $f_2$ and then
      replacing all occurrences of $f_2$ by $f_1$.

    \item If $\pi$ is a \ec{/}-prefixed pattern of the form
      $o\,p_1\,\cdots\,p_n$, with $o$ a defined symbol, then
      \ec{rewrite} searches for the first subterm of the goal's
      conclusion matching $o\,p_1\,\cdots\,p_n$ and resulting in the
      full instantiation of the pattern. It then replaces, after
      instantiation of the pattern, all the occurrences of
      $o\,p_1\,\cdots\,p_n$ by the $\beta\delta$ head-normal form of
      $o\,p_1\,\cdots\,p_n$, where the $\delta$-reduction is
      restricted to subterms headed by the symbol $o$. If no subterms
      of the goal's conclusion match $o\,p_1\,\cdots\,p_n$ or if the
      pattern cannot be fully instantiated by matching, the tactic
      fails. If the direction indicator \ec{-} is given, \ec{rewrite}
      works in the reverse direction, searching for a match of the
      $\beta\delta_o$ head-normal of $o\,p_1\,\cdots\,p_n$ and then
      replacing all occurrences of this head-normal form with
      $o\,p_1\,\cdots\,p_n$.
    \end{itemize}
    
    \smallskip
    The occurrence selector restricts which occurrences of the
    matching pattern are replaced in the goal's conclusion---see
    Subsection~\ref{subsec:occsels}.

    Repetition markers allow the repetition of the same rewriting. For
    instance, \ec{rewrite $\;\pi$} leads to \ec{do! rewrite
      $\;\pi$}. See the tactical \ec{do} for more information.
    
    Lastly, \ec{rewrite $\;\pi_1 \cdots \pi_n$} is equivalent to
    \ec{rewrite} $\;\pi_1$; ...; \ec{rewrite} $\;\pi_n$.

    \smallskip
    For example, if the current goal is
    \ecinput{examps/parts/tactics/rewrite/1-1.0.ec}{}{}{}{} then
    running \ecinput{examps/parts/tactics/rewrite/1-1.ec}{}{}{}{}
    produces \ecinput{examps/parts/tactics/rewrite/1-1.1.ec}{}{}{}{}
    from which
    running \ecinput{examps/parts/tactics/rewrite/1-2.ec}{}{}{}{}
    produces \ecinput{examps/parts/tactics/rewrite/1-2.1.ec}{}{}{}{}
    from which
    running \ecinput{examps/parts/tactics/rewrite/1-3.ec}{}{}{}{}
    produces \ecinput{examps/parts/tactics/rewrite/1-3.1.ec}{}{}{}{}
  \end{tsyntax}

  \begin{tsyntax}{rewrite $\;\pi_1 \cdots \pi_n$ in $\;H$} Like the
    preceding case, except rewriting is done in the hypothesis
    $H$ instead of in the goal's conclusion.  Rewriting using a proof
    term is only allowed when the proof term was defined globally
    or before the assumption $H$.

    For example, if the current goal is
    \ecinput{examps/parts/tactics/rewrite/2-1.0.ec}{}{}{}{} then
    running \ecinput{examps/parts/tactics/rewrite/2-1.ec}{}{}{}{}
    produces \ecinput{examps/parts/tactics/rewrite/2-1.1.ec}{}{}{}{}
  \end{tsyntax}
\end{tactic}


% --------------------------------------------------------------------
\begin{tactic}[case $\;\phi$ | case]{case}
  \begin{tsyntax}{case $\;\phi$}
    Do an excluded-middle case analysis on $\phi$, substituting $\phi$
    in the goal's conclusion.

    For example, if the current goal is
    \ecinput{examps/parts/tactics/case/1-1.0.ec}{}{}{}{} then
    running \ecinput{examps/parts/tactics/case/1-1.ec}{}{}{}{}
    produces the goals
    \ecinput{examps/parts/tactics/case/1-1.1.ec}{}{}{}{}
    and
    \ecinput{examps/parts/tactics/case/1-1.2.ec}{}{}{}{}
  \end{tsyntax}

  \begin{tsyntax}{case}
    Destruct the top assumption of the goal's conclusion, generating
    subgoals that are dependent upon the kind of assumption
    destructed. \emph{This form of the tactic can be followed by
    the generalization tactical---see Subsection~\ref{subsec:introgen}.}

    \begin{itemize}
    \item (\textbf{conjunction})
    For example, if the current goal is
    \ecinput{examps/parts/tactics/case/2-1.0.ec}{}{}{}{} then
    running \ecinput{examps/parts/tactics/case/2-1.ec}{}{}{}{}
    produces the goal
    \ecinput{examps/parts/tactics/case/2-1.1.ec}{}{}{}{}
    \ec{&&} works identically.

    \item (\textbf{disjunction})
    For example, if the current goal is
    \ecinput{examps/parts/tactics/case/3-1.0.ec}{}{}{}{} then
    running \ecinput{examps/parts/tactics/case/3-1.ec}{}{}{}{}
    produces the goals
    \ecinput{examps/parts/tactics/case/3-1.1.ec}{}{}{}{}
    and
    \ecinput{examps/parts/tactics/case/3-1.2.ec}{}{}{}{}
    \ec{||} works identically.

    \item (\textbf{existential})
    For example, if the current goal is
    \ecinput{examps/parts/tactics/case/4-1.0.ec}{}{}{}{} then
    running \ecinput{examps/parts/tactics/case/4-1.ec}{}{}{}{}
    produces the goal
    \ecinput{examps/parts/tactics/case/4-1.1.ec}{}{}{}{}

    \item (\textbf{unit}) Substitutes \ec{tt} for the assumption in
      the remainder of the conclusion.

    \item (\textbf{bool})
    For example, if the current goal is
    \ecinput{examps/parts/tactics/case/5-1.0.ec}{}{}{}{} then
    running \ecinput{examps/parts/tactics/case/5-1.ec}{}{}{}{}
    produces the goals
    \ecinput{examps/parts/tactics/case/5-1.1.ec}{}{}{}{}
    and
    \ecinput{examps/parts/tactics/case/5-1.2.ec}{}{}{}{}

    \item (\textbf{product type})
    For example, if the current goal is
    \ecinput{examps/parts/tactics/case/6-1.0.ec}{}{}{}{} then
    running \ecinput{examps/parts/tactics/case/6-1.ec}{}{}{}{}
    produces the goal
    \ecinput{examps/parts/tactics/case/6-1.1.ec}{}{}{}{}

    \item (\textbf{inductive datatype})
    Consider the inductive datatype declaration:
\begin{easycrypt}{}{}
type tree = [Leaf | Node of bool & tree & tree].
\end{easycrypt}
    Then, if the current goal is
    \ecinput{examps/parts/tactics/case/7-1.0.ec}{}{}{}{} then
    running \ecinput{examps/parts/tactics/case/7-1.ec}{}{}{}{}
    produces the goals
    \ecinput{examps/parts/tactics/case/7-1.1.ec}{}{}{}{}
    and
    \ecinput{examps/parts/tactics/case/7-1.2.ec}{}{}{}{}
    \end{itemize}
  \end{tsyntax}
\end{tactic}

% --------------------------------------------------------------------
\begin{tactic}[elim | elim /$L$]{elim}
  \begin{tsyntax}{elim}
    Eliminates the top assumption of the goal's conclusion, generating
    subgoals that are dependent upon the kind of assumption
    eliminated. \emph{This tactic can be followed by
    the generalization tactical---see Subsection~\ref{subsec:introgen}.}

    \ec{elim} mostly works identically to \rtactic{case}, the exception
    being inductive datatype and the integers (for which a built-in
    induction principle is applied---see the other form).

    Consider the inductive datatype declaration:
\begin{easycrypt}{}{}
type tree = [Leaf | Node of bool & tree & tree].
\end{easycrypt}
    Then, if the current goal is
    \ecinput{examps/parts/tactics/elim/1-1.0.ec}
    running \ecinput{examps/parts/tactics/elim/1-1.ec}
    produces the goals
    \ecinput{examps/parts/tactics/elim/1-1.1.ec}
    and
    \ecinput{examps/parts/tactics/elim/1-1.2.ec}
  \end{tsyntax}

  \begin{tsyntax}{elim /$L$}
    Eliminates the top assumption of the goal's conclusion using the
    supplied induction principle lemma. \emph{This tactic can be
      followed by the generalization tactical---see
      Subsection~\ref{subsec:introgen}.}
    For example, consider the declarations
\begin{easycrypt}{}{}
type tree = [Leaf | Node of bool & tree & tree].
op rev (tr : tree) : tree =
  with tr = Leaf => Leaf
  with tr = Node b tr1 tr2 => Node b (rev tr1) (rev tr2).
\end{easycrypt}
and suppose we've already proved
\begin{easycrypt}{}{}
lemma IndPrin :
  forall (p : tree -> bool) (tr : tree),
  p Leaf =>
  (forall (b : bool) (tr1 tr2 : tree),
   p tr1 => p tr2 => p(Node b tr1 tr2)) =>
  p tr.
\end{easycrypt}
    Then, if the current goal is
    \ecinput{examps/parts/tactics/elim/2-1.0.ec}
    running \ecinput{examps/parts/tactics/elim/2-1.ec}
    produces the goals
    \ecinput{examps/parts/tactics/elim/2-1.1.ec}
    and
    \ecinput{examps/parts/tactics/elim/2-1.2.ec}
  \end{tsyntax}

\smallskip
When we consider the \ec{Int} theory in Chapter~\ref{chap:library},
we'll discuss the induction principle on the integers.
\end{tactic}


% --------------------------------------------------------------------
\begin{tactic}{algebra}
\end{tactic}


\subsection{Tacticals}
\label{subsec:tacticals}

Tactics can be combined together, composed and modified by
\emph{tacticals}. We've already seen the introduction and generalization
tacticals, which turn a tactic $\tau$ and a list of patterns into
a composite tactic, which may then combined with other tactics.

\begin{tactic}[$\tau_1$; $\;\tau_2$]{sequence}
  \begin{tsyntax}[empty]{t1; t2}
  Apply $\tau_2$ to all the subgoals generated by $\tau_1$.
  Sequencing groups to the left, so that
  \ec{$\tau_1$; $\;\tau_2$; $\;\tau_3$} means
  \ec{($\tau_1$; $\;\tau_2$); $\;\tau_3$}.
  
  For example, if the current goal is
  \ecinput{../examps/parts/tactics/sequence-tactical/1-1.0.ec}{}{}{}{} then
  running \ecinput{../examps/parts/tactics/sequence-tactical/1-1.ec}{}{}{}{}
  produces the goals
  \ecinput{../examps/parts/tactics/sequence-tactical/1-1.1.ec}{}{}{}{}
\end{tsyntax}
\end{tactic}

\begin{tactic}[$\tau_1$; \[$\;\tau'_1$ | $\;\cdots$ | $\;\tau'_n$\]]{sequence with branching}
  \begin{tsyntax}[empty]{}
    Run $\tau_1$, which must generate exactly $n$ subgoals, $G_1,\ldots,G_n$.
    Then apply $\tau'_i$ to $G_i$, for all $i$.

  For example, if the current goal is
  \ecinput{../examps/parts/tactics/sequence-branching-tactical/1-1.0.ec}{}{}{}{} then
  running \ecinput{../examps/parts/tactics/sequence-branching-tactical/1-1.ec}{}{}{}{}
  solves the goal.
  \end{tsyntax}
\end{tactic}

\begin{tactic}[try $\;\tau$]{failure recovery}\label{tactic-try}
  \begin{tsyntax}[empty]{try t}
  Execute the tactic $\tau$ if it succeeds; do nothing (leave the
  goal unchanged) if it fails.

  \paragraph{Remark.}
  By default, \EasyCrypt proofs are run in \ec{strict} mode. In this
  mode, \ec{smt} failures cannot be caught using \ec{try}. This allows
  \EasyCrypt to always build the proof tree correctly, even in weak
  check mode, where \ec{smt} calls are assumed to succeed. Inside a
  strict proof, weak check mode can be turned on and off at will,
  allowing for the fast replay of proof sections during
  development. In any event, we recommend \emph{never} using \ec{try
    smt}: a little thought is much more cost-effective than failing
  \ec{smt} calls.
  \end{tsyntax}
\end{tactic}

\begin{tactic}[do! $\;\tau$]{tactic repetition}
  \begin{tsyntax}[empty]{do! t}
    Apply $\tau$ to the current goal, then repeatedly apply it to all
    subgoals, stopping on a branch only when it fails. An error is
    produced it $\tau$ does not apply to the current goal.
  \end{tsyntax}

  For example, if the current goal is
  \ecinput{../examps/parts/tactics/do-tactical/1-1.0.ec}{}{}{}{} then
  running \ecinput{../examps/parts/tactics/do-tactical/1-1.ec}{}{}{}{}
  produces the goals
  \ecinput{../examps/parts/tactics/do-tactical/1-1.1.ec}{}{}{}{}
  and
  \ecinput{../examps/parts/tactics/do-tactical/1-1.2.ec}{}{}{}{}

  \paragraph{Variants.}\strut

  \begin{tabularx}{\textwidth}{@{}ll@{}}
  {\ec{do? $\;\tau$}} & apply $\tau$ 0 or more times, until it fails\\
  {\ec{do $\;n$! $\;\tau$}} & apply $\tau$ with depth exactly $n$\\
  {\ec{do $\;n$? $\;\tau$}} & apply $\tau$ with depth at most $n$
  \end{tabularx}
\end{tactic}

\begin{tactic}[$\tau$; first $\;\tau_2$]{goal selection}
  \begin{tsyntax}[empty]{t1; first t2}
    Apply the tactic $\tau_1$, then apply $\tau_2$ on the first
    subgoal generated by \ec{t1}, leaving the other goals unchanged.
    An error is produced if no subgoals are generated by $\tau_1$.

  \paragraph{Variants.}\strut

  \noindent\begin{tabularx}{\textwidth}{@{}ll@{}}
    {\ec{$\tau_1$; first $\;n$ $\;\tau_2$}} & apply $\tau_2$ on the
    first $n$ subgoals generated by $\tau_1$\\[.4cm]
    {\ec{$\tau_1$; last $\;\tau_2$}} & apply $\tau_2$ on the last subgoal
    generated by $\tau_1$\\[.4cm]
    {\ec{$\tau_1$; last $\;n$ $\;\tau_2$}} & apply $\tau_2$ on the last $n$
    subgoals generated by $\tau_1$\\[.4cm]
    {\ec{$\tau$; first $\;n\!$ last}} & \parbox{250pt}{reorder the subgoals
      generated by $\tau$, moving the first $n$ to the end of the
      list} \\[.4cm]
    {\ec{$\tau$; last $\;n\!$ first}} & \parbox{250pt}{reorder the subgoals
      generated by $\tau$, moving the last $n$ to the beginning of the
      list} \\[.4cm]
    {\ec{$\tau$; last first}} & \parbox{250pt}{reorder the subgoals generated
    by $\tau$, moving the last one to the beginning of the list}\\[.4cm]
    {\ec{$\tau$; first last}} & \parbox{250pt}{reorder the subgoals
     generated by $\tau$, moving the first one to the end of the list}
  \end{tabularx}

  \bigskip
  For example, if the current goal is
  \ecinput{../examps/parts/tactics/sequence-reordering-tactical/1-1.0.ec}{}{}{}{} then
  running \ecinput{../examps/parts/tactics/sequence-reordering-tactical/1-1.ec}{}{}{}{}
  produces the goals
  \ecinput{../examps/parts/tactics/sequence-reordering-tactical/1-1.1.ec}{}{}{}{}
  and
  \ecinput{../examps/parts/tactics/sequence-reordering-tactical/1-1.2.ec}{}{}{}{}
  \end{tsyntax}
\end{tactic}

\begin{tactic}[by $\;\tau$]{closing goals}
  \begin{tsyntax}[empty]{by t}
  Apply the tactic $\tau$ and try to close all the generated subgoals using
  \rtactic{trivial}. Fail if not all subgoals can be closed.
  \end{tsyntax}

  \paragraph{Remark.} Inside the a lemma's proof, \ec{by []} is
  equivalent to \ec{by trivial}.  But the form
\begin{easycrypt}{}{}
lemma #$\;\cdots$# by [].
\end{easycrypt}
  means
\begin{easycrypt}{}{}
lemma #$\;\cdots$# by (trivial; smt).
\end{easycrypt}
\end{tactic}
