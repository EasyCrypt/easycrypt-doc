\chapter{Specifications}
\label{Specifications}

In this chapter, we present \EasyCrypt's language for writing
cryptographic specifications.  We start by presenting its typed
expression language, go on to consider its module language for
expressing cryptographic games, and conclude by presenting its
ambient logic---which includes judgments of the \hl, \phl and \prhl
logics.

\EasyCrypt\ has a typed expression language based on the polymorphic
typed lambda calculus. Expressions are guaranteed to terminate,
although their values may be under-specified.  Its type system has:
\begin{itemize}
\item several pre-defined base types;

\item product (tuple) and record types;

\item user-defined abbreviations for types and parameterized types; and

\item user-defined concrete datatypes (like lists and trees).
\end{itemize}
In its expression language:
\begin{itemize}
\item one may use operators imported from the \EasyCrypt\ library, e.g.,
  for the pre-defined base types;

\item user-defined operators may be defined, including by
  structural recursion on concrete datatypes.
\end{itemize}
For each type, there is a type of probability distributions over that
type.

\EasyCrypt's modules consist of typed global variables and procedures.
The body of a procedure consists of local variable declarations followed
by a sequence of statements:
\begin{itemize}
\item ordinary assignments;

\item random assignments, assigning values chosen
  from distributions to variables;

\item procedure calls, whose results are assigned to variables;

\item conditional (if-then-else) statements;

\item while loops; and

\item return statements (which may only appear at the end of
  procedures).
\end{itemize}
A module's procedures modules may refer to the global variables of
previously declared modules. Modules may be parameterized by abstract
modules, which may be used to model adversaries; and modules
types---or interfaces---may be formalized, describing modules with at
least certain specified typed procedures.

\EasyCrypt\ has four logics: a probabilistic, relational Hoare logic
(\prhl), relating pairs of procedures; a probabilistic Hoare logic
(\phl) allowing one to carry out proofs about the probability of a
procedure's execution resulting in a postcondition holding; an
ordinary (possibilistic) Hoare logic (\hl); and an ambient
higher-order logic for proving general mathematical facts, as well as
for connecting judgments from the other logics

Proofs are carried out using tactics, which is the focus of
Chapter~\ref{Tactics}.  \EasyCrypt\ also has ways (theories and
sections) of structuring specifications and proofs, which will be
described in Chapter~\ref{Structuring}. In Chapter~\ref{Library},
we'll survey the \EasyCrypt\ Library, which consists of numerous
theories, defining mathematical structures (like groups, rings
and fields), data structures (like finite sets and maps), and
cryptographic constructions (like random oracles and different
forms of encryption).

\section{Lexical Categories}
\label{Lexical}

\EasyCrypt's language has a number of lexical categories:
\begin{itemize}
\item \textbf{Keywords}. \EasyCrypt\ has the following
  \emph{keywords}: \ecn{admit}, \ecn{algebra},
  \ecn{alias}, \ecn{apply}, \ecn{as},
  \ecn{assert}, \ecn{assumption}, \ecn{auto},
  \ecn{axiom}, \ecn{axiomatized}, \ecn{beta},
  \ecn{by}, \ecn{byequiv}, \ecn{byphoare},
  \ecn{bypr}, \ecn{call}, \ecn{case},
  \ecn{cfold}, \ecn{change}, \ecn{class},
  \ecn{clear}, \ecn{clone}, \ecn{congr},
  \ecn{conseq}, \ecn{const}, \ecn{cut},
  \ecn{declare}, \ecn{delta}, \ecn{do},
  \ecn{done}, \ecn{eager}, \ecn{elif},
  \ecn{elim}, \ecn{else}, \ecn{end},
  \ecn{equiv}, \ecn{exact}, \ecn{exfalso},
  \ecn{exists}, \ecn{expect}, \ecn{export},
  \ecn{fel}, \ecn{fieldeq}, \ecn{first},
  \ecn{fission}, \ecn{forall}, \ecn{fun},
  \ecn{fusion}, \ecn{generalize}, \ecn{glob},
  \ecn{goal}, \ecn{have}, \ecn{hint},
  \ecn{hoare}, \ecn{hypothesis}, \ecn{idtac},
  \ecn{if}, \ecn{import}, \ecn{in},
  \ecn{inline}, \ecn{instance}, \ecn{intros},
  \ecn{iota}, \ecn{islossless}, \ecn{kill},
  \ecn{last}, \ecn{left}, \ecn{lemma},
  \ecn{let}, \ecn{local}, \ecn{logic},
  \ecn{modpath}, \ecn{module}, \ecn{move},
  \ecn{nolocals}, \ecn{nosmt}, \ecn{of},
  \ecn{op}, \ecn{phoare}, \ecn{pose},
  \ecn{Pr}, \ecn{pr_bounded}, \ecn{pred},
  \ecn{print}, \ecn{proc}, \ecn{progress},
  \ecn{proof}, \ecn{prover}, \ecn{qed},
  \ecn{rcondf}, \ecn{rcondt}, \ecn{realize},
  \ecn{reflexivity}, \ecn{require}, \ecn{res},
  \ecn{return}, \ecn{rewrite}, \ecn{right},
  \ecn{ringeq}, \ecn{rnd}, \ecn{rwnormal},
  \ecn{search}, \ecn{section}, \ecn{seq},
  \ecn{sim}, \ecn{simplify}, \ecn{skip},
  \ecn{smt}, \ecn{sp}, \ecn{split},
  \ecn{splitwhile}, \ecn{strict}, \ecn{subst},
  \ecn{swap}, \ecn{symmetry}, \ecn{then},
  \ecn{theory}, \ecn{timeout}, \ecn{Top},
  \ecn{transitivity}, \ecn{trivial}, \ecn{try},
  \ecn{type}, \ecn{unroll}, \ecn{var},
  \ecn{while}, \ecn{why3}, \ecn{with},
  \ecn{wp} and \ecn{zeta}.

\item \textbf{Identifiers}. An \emph{identifier} is a sequence of
  letters, digits, underscores (\ecn{_}) and apostrophes
  (\ecn{'}) that begins with a letter or underscore, and isn't
  equal to an underscore or a keyword other than \ecn{expect},
  \ecn{first}, \ecn{last}, \ecn{left},
  \ecn{right} or \ecn{strict}.

\item \textbf{Operator names}. An \emph{operator name} is an
  identifier, a binary operator name, a unary operator name, or
  a mixfix operator name.

\item \textbf{Binary operator names}. A \emph{binary operator name}
  is:
  \begin{itemize}
  \item a nonempty sequence of equal signs (\ec{=}), less
  than signs (\ec{<}), greater than signs (\ec{>}), forward slashes
  (\ec{/}), backward slashes (\ec{\\}), plus signs (\ec{+}), minus
  signs (\ec{-}), times signs (\ec{*}), vertical bars (\ec{|}), colons
  (\ec{:}), ampersands (\ec{&}), up arrows (\ec{^}) and percent signs
  (\ec{\%}); or

  \item a backtick mark (\ec{`}), followed by a nonempty sequence of one
    of these characters, followed by a backtick mark; or

  \item a backward slash followed by a nonempty sequence of letters,
    digits, underscores and apostrophes.
  \end{itemize}

  A binary operator name is an \emph{infix operator name} iff it is
  surrounded by backticks, or begins with a backslash, or:
  \begin{itemize}
  \item it is neither \ecn{<<} nor \ecn{>>}; and
  \item it doesn't contain a colon, unless it is a sequence of colons
    of length at least two; and
  \item it doesn't contain \ecn{=>}, except if it is \ecn{=>}; and
  \item it doesn't contain \ecn{|}, except if it is \ecn{||}; and
  \item it doesn't contain \ecn{/}, except if it is \ecn{/},
    \ecn{/\\}, or a sequence of slashes of length at least 3.
  \end{itemize}

  The precedence hierarchy for infix operators is (from lowest to highest):
  \begin{itemize}
  \item \ecn{=>} (right-associative);

  \item \ecn{<=>} (non-associative);

  \item \ecn{||} and \ecn{\\/} (right-associative);

  \item \ecn{&&} and \ecn{/\\} (right-associative);

  \item \ecn{=} and \ecn{<>} (non-associative);

  \item \ecn{<}, \ecn{>}, \ecn{<=} and \ecn{>=}
    (left-associative);

  \item \ecn{-} and \ecn{+} (left-associative);

  \item \ecn{*}, and any nonempty combination of \ecn{/} and
    \ecn{\%} (other than \ecn{//}, which is illegal)
    (left-associative);

  \item all other infix operators except sequences of colons
    (left-associative);

  \item sequences of colons of length at least two (right-associative).
  \end{itemize}

\item \textbf{Unary operator names}. A \emph{unary operator name} is a
  negation sign (\ec{!}), a nonempty sequence of plus signs (\ec{+}),
  a nonempty sequence of minus signs (\ec{-}), or a backward slash
  followed by a nonempty sequence of letters, digits, underscores and
  apostrophes.  A \emph{prefix operator name} is any unary operator
  name not consisting of either two more plus signs or two or more
  minus signs.

\item \textbf{Mixfix operator names}. A \emph{mixfix operator name} is
  of the following sequences of characters: \ec{`|_|}, \ec{[]},
  \ec{_.[_]} or \ec{_.[_<-_]}.  (We'll see below how they may be used
  in mixfix form.)

\item \textbf{Record projections}. A \emph{record projection} is an
  identifier.

\item \textbf{Constructor names}. A \emph{constructor name} is an identifier
  or a symbolic operator name.

\item \textbf{Type variables}. A \emph{type variable} consists of an
  apostrophe followed by a sequence of letters, digits, underscores
  and apostrophes that begins with a lowercase letter or underscore,
  and isn't equal to an underscore.

\item \textbf{Type or type operator names}. A \emph{type or type
  operator name} is an identifier.

\item \textbf{Variable names}. A \emph{variable} name is an identifier
  that doesn't begin with an uppercase letter.

\item \textbf{Procedure names}. A \emph{procedure} name is an identifier
  that doesn't begin with an uppercase letter.

\item \textbf{Module names}. A \emph{module name} is an identifier that
  begins with an uppercase letter.

\item \textbf{Module type names}. A \emph{module type name} is an
  identifier that begins with an uppercase letter.

\item \textbf{Memory identifiers}. A \emph{memory identifier} consists
  an ampersand followed by either a nonempty sequence of digits or an
  identifier whose initial character isn't an upper case letter.
\end{itemize}

\section{Script Structure, Printing and Searching}

An \EasyCrypt\ script consists of a sequence of \emph{steps},
terminated by dots (\ec{.}). Steps may:
\begin{itemize}
\item declare types and type operators;

\item declare operators and predicates;

\item declare modules or module types;

\item state axioms or lemmas;

\item apply tactics;

\item require (make available) theories;

\item print types, operators, predicates, modules, module types,
  axioms and lemmas;

\item search for lemmas involving operators.
\end{itemize}

To print an entity, one may say:
\begin{easycrypt}{}{}
print type t.
print op f.
print pred p.
print module Foo.
print module type FOO.
print axiom foo.
print lemma goo.
\end{easycrypt}
The entity kind may be omitted, in which case all entities with the
given name are printed. \ec{print op} and \ec{print pred} may be used
interchangeably, and may be applied to record field projections and
datatype constructors, as well as to operators and predicates---all of which
share the same name space.
\ec{print axiom} and \ec{print lemma} are also interchangeable---axioms
and lemmas share the same name space.

To search for axioms and lemmas involving all of a list of operators,
one can say
\begin{easycrypt}{}{}
search f.
search (+).
search (+) (-).  (* search for axioms/lemmas involving both operators *)
\end{easycrypt}
(Infix operators must be parenthesized.)

Declared/stated entities may refer to previously declared/stated entities,
but not to themselves or later ones (with the exception of recursively
declared operators on datatypes, and to references to a module's own
global variables).

\section{Typed Expressions}

\subsection{Types}

\EasyCrypt's \emph{types} are formed from its built-in types and
type operator using function types, product (tuple) types, and
user-defined types and type constructors, which include record types
and datatypes. \EasyCrypt's types are required to be inhabited---i.e.,
nonempty.

There are four built-in types:
\begin{itemize}
\item the type \ec{unit}, whose single element is \ec{tt} (which may
  also be written \ec{()});

\item the type \ec{bool} of booleans, \ec{true} and \ec{false};

\item the type \ec{int} of integers; and

\item the type \ec{real} of reals.
\end{itemize}

In addition, for every type $t$ there is a type \ec{$t\;$distr}, which
should be axiomatized so as to be a \emph{sub-distribution} of type
$t$, i.e., so that each element of $t$ is assigned a probability in
such a way that the sum of the probabilities of all of $t$'s element
is no more than $1$. When the sum is equal to $1$, we have a
\emph{distribution}. \ec{distr} is a \emph{type operator} taking a
single type argument:
\begin{easycrypt}{}{}
type 'a distr
\end{easycrypt}
Here, \ec{'a} is a \emph{type variable}, marking the type parameter of
\ec{distr}; if we substitute an actual type for \ec{'a} in \ec{'a distr},
we get a type.

Given types $t_1$ and $t_2$, \ec{$t_1\;$->$\;t_2$} is a function type,
the type of total function transforming inputs of type $t_1$ to
outputs of type $t_2$. \ec{->} associates to the right, so that
\ec{int -> bool -> real} means \ec{int -> (bool -> real)}.

Given types $t_1$ and $t_2$, \ec{$t_1\;$*$\;t_2$} is a product (tuple)
type, whose elements are pairs $(x_1, x_2)$ where $x_1$ and $x_2$ are
elements of $t_1$ and $t_2$, respectively. Tuples may have more than
two components: e.g., \ec{int * int * real} is the type of triples
$(x_1, x_2, x_3)$ where $x_1$ and $x_2$ are integers and $x_3$ is a real.
Note that \ec{int * (int * real)}, \ec{(int * int) * real} and
\ec{int * int * real} are distinct types; the first two are pair types.
\ec{*} has higher precedence than \ec{->}, so that, e.g.,
\ec{int * int -> bool * real} means \ec{(int * int) -> (bool * real)}.

Record types may be declared like this:
\begin{easycrypt}{}{}
type t = { x : int; y : bool; }.
type u = { y : real; yy : int; yyy : real; }.
\end{easycrypt}
Here \ec{t} is the type of records with field projections \ec{x} of
type \ec{int}, and \ec{y} of type \ec{bool}. The order of projections
is irrelevant.  Different record types can't use overlapping
projections, and record projections must be disjoint from operators
(see below). Records may have any non-zero number of fields; values of
type \ec{u} are record with three fields. We may also define record
type operators, as in:
\begin{easycrypt}{}{}
type 'a t = { x : 'a; f : 'a -> 'a; }.
type ('a, 'b) u = { f : 'a -> 'b; x : 'a; }.
\end{easycrypt}
Then, a value $v$ of type \ec{int t} would have fields \ec{x} and
\ec{f} of types \ec{int} and \ec{int -> int}, respectively; and
a value $v$ of type \ec{(int, bool) u} would have fields \ec{x} and
\ec{f} with types \ec{int} and \ec{int -> bool}, respectively.

Datatypes and datatype operators may be declared like this:
\begin{easycrypt}{}{}
type enum = [ First | Second | Third ].
type either_int_bool = [ First of int | Second of bool ].
type ('a, 'b) either = [ First of 'a | Second of 'b ].
type intlist = [
  | Nil
  | Cons of (int * intlist) ].
type 'a list = [
  | Nil
  | Cons of 'a & 'a list ].
\end{easycrypt}
Here, \ec{First}, \ec{Second}, \ec{Third}, \ec{Nil} and \ec{Cons} are
constructors, and must be distinct from all operators, record
projections and other constructors.  \ec{enum} is an enumerated type
with the three elements \ec{First}, \ec{Second} and \ec{Third}. The
elements of \ec{either_int_bool} consist of \ec{First} applied to an
integer, or \ec{Second} applied to a boolean, and the datatype
operator \ec{either} is simply its generalization to arbitrary types
\ec{'a} and \ec{'b}.  \ec{intlist} is an inductive datatype: its
elements are \ec{Nil} and the results of applying \ec{Cons} to a pairs
of the form $(x, \mathit{ys})$, where $x$ is an integer and
$\mathit{ys}$ is a previously constructed \ec{intlist}.  Note that a
vertical bar (\ec{|}) is permitted before the first constructor of a
datatype. Finally, \ec{list} is the generalization of \ec{intlist} to
lists over an arbitrary type \ec{'a}, but with a twist. The use of
\ec{&} means that \ec{Cons} is ``curried'': instead of applying
\ec{Cons} to a pair $(x, \mathit{ys})$, one gives it \ec{$x$ : 'a} and
\ec{$\mathit{ys}$ : 'a list} one at a time, as in \ec{Cons$\,x\,$
  $\,\mathit{ys}$}.  Unsurprisingly, more than one occurrence of
\ec{&} is allowed in a constructor's definition. E.g., here is the
datatype for binary trees whose leaves and internal nodes are labeled
by integers:
\begin{easycrypt}{}{}
type tree = [
  | Leaf of int
  | Cons of tree & int & tree
].
\end{easycrypt}
\ec{Cons $\,\mathit{tr}_1\,$ $x\,$ $\mathit{tr}_2$} will be the tree
constructed from an integer $x$ and trees $\mathit{tr}_1$ and
$\mathit{tr}_2$.
\EasyCrypt\ must be able to convince itself that a datatype is
nonempty, most commonly because it has at least one constructor
taking no arguments, or only arguments not involving the datatype.

Types and type operators that are simply abbreviations for pre-existing
types may be declared, as in:
\begin{easycrypt}{}{}
type t = int * bool.
type ('a, b) arr = 'a -> 'b.
\end{easycrypt}
Then, e.g., \ec{(int, bool) arr} is the same type as \ec{int -> bool}.

Finally, abstract types and type operators may be declared, as in:
\begin{easycrypt}{}{}
type t.
type ('a, b) u.
type t, ('a, b) u.
\end{easycrypt}
We'll see later how such types and type operators may be used.

\subsection{Expressions}

We'll now survey \EasyCrypt's typed expressions. Anonymous functions
are written \ec{fun ($x$ : $\,t_1$) => $\;e$}, where $x$ is an identifier,
$t_1$ is a type, and $e$ is an expression---probably involving $x$.
If $e$ has type $t_2$ under the assumption that $x$ has type $t_1$,
then the anonymous function will have type \ec{$t_1$ -> $\;\,t_2$}.
Function application is written using juxtapositioning, so that if
$e_1$ has type \ec{$t_1$ -> $\;\,t_2$}, and $e_2$ has type $t_1$, then
$e_1\,e_2$ has type $t_2$. Function application associates to the
left, and anonymous functions extend as far to the right as possible.
\EasyCrypt\ infers the types of the bound variables of anonymous function
when it can. Nested anonymous functions may be abbreviated by
collecting all their bound variables together. E.g., consider the
expression
\begin{easycrypt}{}{}
(fun (x : int) => fun (y : int) => fun (z : bool) => y) 0 1 false
\end{easycrypt}
which evaluates to \ec{1}. It may be abbreviated to
\begin{easycrypt}{}{}
(fun (x y : int, z : bool) => y) 0 1 false
\end{easycrypt}
or
\begin{easycrypt}{}{}
(fun (x : int) (y : int) (z : bool) => y) 0 1 false
\end{easycrypt}
or (letting \EasyCrypt\ carry out type inference)
\begin{easycrypt}{}{}
(fun x y z => y) 0 1 false
\end{easycrypt}
In the type inference, only the type of \ec{y} is determined, but
that's acceptable.

An operator may be declared by specifying its type and giving the
expression to be evaluated. E.g.,
\begin{easycrypt}{}{}
op x : int = 3.
op f : int -> bool -> int = fun (x : int) (y : bool) => x.
op g : bool -> int = f 1.
op y : int = g true.
op z = f 1 true.
\end{easycrypt}
Here \ec{f} is a curried function---it takes its arguments one
at a time. Hence \ec{y} and \ec{z} have the same value: \ec{1}.
As illustrated by the declaration of \ec{z}, one may omit the
operator's type when it can be inferred from its expression.
The declaration of \ec{f} may be abbreviated to
\begin{easycrypt}{}{}
op f (x : int) (y : bool) = x.
\end{easycrypt}
or
\begin{easycrypt}{}{}
op f (x : int, y : bool) = x.
\end{easycrypt}

\emph{Polymorphic} operators may be declared, as in
\begin{easycrypt}{}{}
op g ['a, 'b] : 'a -> 'b -> 'a = fun (x : 'a, y : 'b) => x.
\end{easycrypt}
or
\begin{easycrypt}{}{}
op g ['a, 'b] (x : 'a, y : 'b) = x.
\end{easycrypt}
or
\begin{easycrypt}{}{}
op g (x : 'a, y : 'b) = x.
\end{easycrypt}
Here \ec{g} has all the types formed
by substituting types for the types variable \ec{'a} and \ec{'b}
in \ec{'a -> 'b -> 'a}. This allows us to use \ec{g} at different
types
\begin{easycrypt}{}{}
op a = g true 0.
op b = g 0 false.
\end{easycrypt}
making \ec{a} and \ec{b} evaluate to \ec{true} and \ec{0}, respectively.

\emph{Abstract} operators may be declared, i.e., ones whose values
are unspecified. E.g., we can declare
\begin{easycrypt}{}{}
op x : int.
op f : int -> int.
op g ['a, 'b] : 'a -> 'b -> 'a.
\end{easycrypt}
Equivalently, \ec{f} and \ec{g} may be declared like this:
\begin{easycrypt}{}{}
op f (x : int) : int.
op g ['a, 'b] (x : 'a, y : 'b) : 'a.
\end{easycrypt}
One may declare multiple abstract operators of the same type:
\begin{easycrypt}{}{}
op f, g : int -> int.
op g, h ['a, 'b] : 'a -> 'b -> 'a.
\end{easycrypt}
We'll see later how abstract operators may be used.

Binary operators may be declared and used with infix notation (as long
as they are infix operators). One parenthesizes a binary operator when
declaring it and using it in non-infix form (i.e., as a value).  If
$\mathit{io}$ is an infix operator and $e_1,e_2$ are expressions, then
$e_1\mathbin{\mathit{io}}e_2$ is translated to \ec{($\mathit{io}$)
  $\,e_1$ $\,e_2$}, whenever the latter expression is
well-typed. E.g., if we declare
\begin{easycrypt}{}{}
op (--) ['a, 'b] (x : 'a) (y : 'b) = x.
op x : int = (--) 0 true.
op x' : int = 0 -- true.
op y : bool = (--) true 0.
op y' : bool = true -- 0.
\end{easycrypt}
then \ec{x} and \ec{x'} evaluate to \ec{0}, and
\ec{y} and \ec{y'} evaluate to \ec{true}.

Unary operators may be declared and used with prefix notation
(as long as they are prefix operators).
One (square) brackets a unary operator when
declaring it and using it in non-prefix form (i.e., as a value).
If $\mathit{po}$ is
a prefix operator and $e$ is an expression, then
$\mathit{po}\,e$ is translated to
\ec{[$\mathit{po}$] $\,e$}, whenever the latter
expression is well-typed. E.g., if we declare
\begin{easycrypt}{}{}
op x : int.
op f : int -> int.
op [!] : int -> int.
op y : int = ! f x.
op y' : int = [!](f x).
\end{easycrypt}
then \ec{y} and \ec{y'} both evaluate to the result of applying the
abstract operator \ec{!} of type \ec{int -> int} to the result of
applying the abstract operator \ec{f} of type \ec{int -> int} to the
abstract value \ec{x} of type \ec{int}.  Function application has
higher precedence than prefix operators, which have higher precedence
than infix operators, prefix operators group to the right, and infix
operators have the associativities and relative precedences that were
detailed in Section~\ref{Lexical}.

The four mixfix operators may be declared and used as follows. They
are (double) quoted when being declared or used in non-mixfix form
(i.e., as values).
\begin{itemize}
\item (\ec{[]})\quad \ec{[]} is translated to \ec{\"[]\"}. E.g.,
  if we declare
\begin{easycrypt}{}{}
op "[]" : int = 3.
op x : int = [].
\end{easycrypt}
then \ec{x} will evaluate to \ec{3}.

\item (\ec{`|_|})\quad If $e$ is an expression, then \ec{`|$e$|} is
  translated to \ec{\"`|_|\" $\,e$}, as long as the latter expression
  is well-typed.  E.g., if we declare
\begin{easycrypt}{}{}
op "`|_|" : int -> bool.
op x : bool = "`|_|" 3.
op y : bool = `|3|.
\end{easycrypt}
then \ec{y} will evaluate to the same value as \ec{x}.

\item (\ec{_.[_]})\quad If $e_1,e_2$ are expressions, then
    \ec{$e_1$.[$e_2$]} is translated to \ec{\"_.[_]\" $\,e_1$
      $\,e_2$}, whenever the latter expression is well-typed. E.g., if
    we declare
\begin{easycrypt}{}{}
op "_.[_]" : int -> int -> bool.
op x : bool = "_.[_]" 3 4.
op y : bool = 3.[4].
\end{easycrypt}
then \ec{y} will evaluate to the same value as \ec{x}.

\item (\ec{_.[_<-_]})\quad If $e_1,e_2,e_3$ are expressions, then
    \ec{$e_1$.[$e_2$ <- $\,e_3$]} is translated to \ec{\"_.[_<-_]\"
      $\,e_1$ $\,e_2$ $\,e_3$}, whenever the latter expression is
    well-typed. E.g., if we declare
\begin{easycrypt}{}{}
op "_.[_<-_]" : int -> int -> int -> bool.
op x : bool = "_.[_<-_]" 3 4 5.
op y : bool = 3.[4 <- 5].
\end{easycrypt}
then \ec{y} will evaluate to the same value as \ec{x}.
\end{itemize}
In addition, if $e_1,\ldots,e_n$ are expressions then
\begin{center}
\ec{[$e_1$; $\;\ldots\,$; $\,e_n$]}  
\qquad is translated to \qquad
\ec{$e_1$ :: $\;\ldots$ :: $\;e_n$ :: []}
\end{center}
whenever the latter expression is well-typed.
The initial argument of \ec{\"_.[_]\"} and \ec{\"_.[_<-_]\"} have
higher precedence than even function application. E.g., one
can't omit the parentheses in
\begin{easycrypt}{}{}
op f : int -> int.
op y : bool = (f 3).[4].
op z : bool = (f 3).[4 <- 5].
\end{easycrypt}

Some operators are built-in to \EasyCrypt, automatically understood
by its ambient logic:
\begin{easycrypt}{}{}
op (=) ['a]: 'a -> 'a -> bool.

op [!] : bool -> bool.
op (||) : bool -> bool -> bool.
op (\/) : bool -> bool -> bool.
op (&&) : bool -> bool -> bool.
op (/\) : bool -> bool -> bool.
op (=>) : bool -> bool -> bool.
op (<=>) : bool -> bool -> bool.

op mu : 'a distr -> ('a -> bool) -> real.
\end{easycrypt}
The operator \ec{=} is equality. On the booleans, we have negation
\ec{!}, two forms of disjunction (\ec{\\/} and \ec{||}) and conjunction
(\ec{/\\} and \ec{&&}), implication (\ec{=>}) and if-and-only-if
(\ec{<=>}).  The two disjunctions (respectively, conjunctions) are
semantically equivalent, but are treated differently by \EasyCrypt
proof engine. The associativities and precedences of the infix
operators were given in Section~\ref{Lexical}, and (as a prefix
operator) \ec{!} has higher precedence than all of them. The
expression \ec{$e_1$ <> $\;e_2$} is treated as \ec{!($e_1$ =
  $\;e_2$)}. \ec{<>} is not an operator, but it has the precedence and
non-associative status of Section~\ref{Lexical}.
The intended meaning of \ec{mu $\,d$ $\,p$} is the probability that
randomly choosing a value of the given type from the sub-distribution
$d$ will satisfy the function $p$ (in the sense of causing it to return
\ec{true}).

If $e$ is an expression of type \ec{int}, then
\ec{$e$\%r} is the corresponding real.

If $e_1$ is an expression of type \ec{bool} and $e_2,e_3$ are expressions
of some type $t$, then the \emph{conditional expression}
\ec{$e_1$ ? $\,e_2$ : $\,e_3$} evaluates to $e_2$, if $e_1$ evaluates to
\ec{true}, and evaluates to $e_3$, if $e_1$ evaluates to \ec{false}.
E.g., if we write
\begin{easycrypt}{}{}
op x : int = (3 < 4) ? 4 + 7 : (9 - 1).
\end{easycrypt}
then \ec{x} evaluates to \ec{11}. The conditional expression's
precedence at its first argument is lower than function
application, but higher than the prefix operators; its second argument
needn't be parenthesized; and the precedence at its third argument is
lower than the prefix operators, but higher than the infix operators.

For the built-in types \ec{bool}, \ec{int} and \ec{real}, and the type
operator \ec{distr}, the \EasyCrypt\ Library (see
Chapter~\ref{Library}) provides corresponding theories, \ec{Bool},
\ec{Int}, \ec{Real} and \ec{Distr}. These theories provide various
operations, axioms, etc.  To make use of a theory, one must
``require'' it.  E.g.,
\begin{easycrypt}{}{}
require Bool Int Real Distr.
\end{easycrypt}
will make the theories just mentioned available. This
would allow us to write, e.g.,
\begin{easycrypt}{}{}
op x = Int.(+) 3 4.
\end{easycrypt}
making \ec{x} evaluate to \ec{7}. But to be able to use \ec{+}
and the other operators provided by \ec{Int}
in infix form and without qualification (specifying which theory to
find them in), we need to import \ec{Int}. If we do
\begin{easycrypt}{}{}
import Bool Int Real.
op x : int = 3 + 4 - 7 * 2.
op y : real = 5%r * 3%r / 2%r.
op z : bool = x%r >= y.
\end{easycrypt}
we'll end up with \ec{z} evaluating to \ec{false}.
One may combine requiring and importing in one step:
\begin{easycrypt}{}{}
require import Bool Int Real Distr.
\end{easycrypt}
We'll cover theories and their usage in detail in
Chapter~\ref{Structuring}.

Requiring the theory \ec{Bool} makes available the value \ec{\{0,1\}}
of type \ec{bool distr}, which is the uniform distribution on
the booleans. (No whitespace is allowed in the name for this distribution,
and the \ec{0} must come before the \ec{1}.)
Requiring the theory \ec{Distr} make available syntax for the uniform
distribution of integers from a finite range. If $e_1$ and $e_2$
are expressions of type \ec{int} denoting $n_1$ and $n_2$, respectively,
then \ec{[$e_1$..$e_2$]} is the value of type \ec{int distr}
that is the uniform distribution on the set of
all integers that are greater-than-or-equal to $n_1$ and less-than-or-equal-to
$n_2$---unless $n_1>n_2$, in which case it is the sub-distribution assigning
probability $0$ to all integers.

Values of product (tuple) and record types are constructed and
destructed as follows:
\begin{easycrypt}{}{}
op x : int * int * bool = (3, 4, true).
op b : bool = x .` 3.
type t = { u : int; v : bool; }.
op y : t = {| v = false; u = 10; |}.
op a : bool = y .` v.
\end{easycrypt}
Then, \ec{b} evaluates to \ec{true}, and \ec{a} evaluates to \ec{false}.
Note the field order in the declaration of \ec{y} was allowed to be
a permutation of that of the record type \ec{t}.

When we declare a datatype, its constructors are available to
us as values. E.g, if we declare
\begin{easycrypt}{}{}
type ('a, 'b) either = [Fst of 'a | Snd of 'b].
op x : (int, bool) either = Fst 10.
op y : int -> (int, bool) either = Fst.
op z : (int, bool) either = y 10.
\end{easycrypt}
then \ec{z} evaluates to the same result as \ec{x}.

We can declare operators using pattern matching on the constructors
of datatypes. E.g., continuing the previous example, we can
declare and use an operator \ec{fst} by:
\begin{easycrypt}{}{}
op fst ['a, 'b] (def : 'a) (ei : ('a, 'b) either) : 'a =
  with ei = Fst a => a
  with ei = Snd b => def.
op l1 : (int, bool) either = Fst 10.
op l2 : (int, bool) either = Snd true.
op m1 : int = fst (-1) l1.
op m2 : int = fst (-1) l2.
\end{easycrypt}
Here, \ec{m1} will evaluate to \ec{10}, whereas \ec{m2} will
evaluate to \ec{-1}.
Such operator declarations may be recursive, as long as \EasyCrypt\
can determine that the recursion is well-founded. E.g., here is one
way of declaring an operator \ec{length} that computes the length
of a list:
\begin{easycrypt}{}{}
type 'a list = [Nil | Cons of 'a & 'a list].
op len ['a] (acc : int, xs : 'a list) : int =
  with xs = Nil => acc
  with xs = Cons y ys => len (acc + 1) ys.
op length ['a] (xs : 'a list) = len 0 xs.
op xs = Cons 0 (Cons 1 (Cons 2 Nil)).
op n : int = length xs.
\end{easycrypt}
Then \ec{n} will evaluate to \ec{3}.

\section{Module System}

\subsection{Modules}

\EasyCrypt's modules consist of typed global variables and procedures,
which have different name spaces.
Listing~\ref{SimpMod} contains the definition of a simple module,
\ec{M}, which exemplifies much of the module language.
\ecinput{examps/specifications-examp1.ec}{Simple
  Module}{}{SimpMod}
\ec{M} has one \emph{global variable}---\ec{x}---which is used by the
\emph{procedures} of \ec{M}---\ec{init}, \ec{incr}, \ec{get} and
\ec{main}. Global variables must be declared before the procedures
that use them.

The procedure \ec{init} (``initialize'') has a \emph{parameter} (or
\emph{argument}) \ec{bnd} (``bound'') of type \ec{int}.  \ec{init}
uses a \emph{random assignment} to assign to \ec{x} an integer chosen
uniformly from the integers whose absolute values are at most
\ec{bnd}.  The return type of \ec{init} is \ec{unit}, whose only
element is \ec{tt}; this is implicitly returned by \ec{init} upon
exit.

The procedure \ec{incr} (``increment''), increments the value of
\ec{x} by its parameter \ec{n}. The procedure \ec{get} takes no
parameters, but simply returns the value of \ec{x}, using
a \emph{return statement}---which is only allowed as the
final statement of a procedure.

And the \ec{main} procedures takes no parameters, and returns a boolean
that's computed as follows:
\begin{itemize}
\item It declares a local variable, \ec{n}, of type \ec{int}---local
  in the sense that other procedures can't access or affect it.

\item It uses a \emph{procedure call} to call the procedure \ec{init}
  with a bound of \ec{100}, causing \ec{x} to be initialized to an
  integer between \ec{-100} and \ec{100}.

\item It calls \ec{incr} twice, with \ec{10} and then \ec{-50}.

\item It uses a \emph{procedure call assignment} to call the procedure
  \ec{get} with no arguments, and assign \ec{get}'s return value to \ec{n}.

\item It evaluates the boolean expression \ec{n < 0}, and returns
  the value of this expression as its boolean result.
\end{itemize}

As we've seen, each declaration or statement of a procedure is
terminated with a semicolon.  \EasyCrypt tries to infer the types of
local variables. So, in the above example, one could write
\ec{var n;} for the local declaration of \ec{n}.
One may combine multiple local variable declarations, as in:
\begin{easycrypt}{}{}
var x, y, z : int;
var u, v;
var x, y, z : int = 10;
var x, y, z = 10;
\end{easycrypt}
Procedure parameters are variables; they may be modified during the
execution of their procedures.  A procedure's parameters and local
variables must be distinct variable names.  The three kinds of
assignment statements differ according to their allowed right-hand
sides (rhs):
\begin{itemize}
\item The rhs of a random assignment must be a single
  (sub-)distribution. When choosing from a proper sub-distribution,
  the random assignment may fail, causing the procedure call that
  invoked it to fail to terminate.

\item The rhs of an ordinary assignment may be an arbitrary expression
  (which doesn't include use of procedures).

\item the rhs of a procedure call assignment must be a single procedure
  call.  
\end{itemize}
If the rhs of an assignment produces a tuple value, its left-hand side
may use pattern matching, as in
\begin{easycrypt}{}{}
(x, y, z) <- #\ldots#;
\end{easycrypt}
in the case where $\ldots$ produces a triple.

The two remaining kinds of statements are illustrated in
Listing~\ref{CondWhileExamp}: \emph{conditionals} and \emph{while
  loops}.
\ecinput{examps/specifications-examp2.ec}{Conditionals
  and While Loops} {}{CondWhileExamp}
\ec{N} has a single procedure,
\ec{loop}, which begins by initializing a local variable \ec{y} to
\ec{0}. It then enters a while loop, which continues executing until
(which may never happen) \ec{y} becomes \ec{10} or more. At each
iteration of the while loop, an integer between \ec{1} and \ec{10} is
randomly chosen and assigned to the local variable \ec{z}. The
conditional is used to behave differently depending upon whether the
value of \ec{z} is less-than-or-equal-to \ec{5} or not.
\begin{itemize}
\item When the answer is ``yes'', \ec{y} is decremented by \ec{z}.

\item When the answer is ``no'', \ec{y} is incremented by \ec{z - 5}.
\end{itemize}
Once (if) the while loop is exited---which means \ec{y} is now \ec{10}
or more---the procedure returns \ec{y}'s value as its return value.

When the body of a while loop, or the then or else part of a conditional,
has a single statement, the curly braces may be omitted. E.g., the
conditional of the preceding example could be written:
\begin{easycrypt}{}{}
if (z <= 5) y <- y - z;
else y <- y + (z - 5);
\end{easycrypt}
And when the else part of a conditional is empty (consists of
\ec{\{\}}), it may be omitted, as in:
\begin{easycrypt}{}{}
if (z <= 5) y <- y - z;
\end{easycrypt}

As illustrated in Listing~\ref{UsingOtherModuleExamp}, modules may
access the global variables, and call the procedures, of previously
declared modules.
\ecinput{examps/specifications-examp3.ec}{One
  Module Using Another Module} {}{UsingOtherModuleExamp} Procedure
\ec{g} of \ec{N} both accesses the global variable \ec{x} of module
\ec{M} (\ec{M.x}), and calls \ec{M}'s procedure, \ec{f} (\ec{M.f}).
The parameter list of \ec{g} could equivalently be written:
\begin{easycrypt}{}{}
n : int, m : int, b : bool  
\end{easycrypt}
A module may refer to its own global variables using its own module
name, allowing us to write
\begin{easycrypt}{}{}
proc f() : unit = {
  M.x <- M.x + 1;
}
\end{easycrypt}
for the definition of procedure \ec{M.f}.  The procedure \ec{h} of
\ec{N} is an alias for procedure \ec{M.f}: calling it is equivalent to
directly calling \ec{M.f}.
One declare a module name to be an alias for a module, as in
\begin{easycrypt}{}{}
module L = N.
\end{easycrypt}

\subsection{Procedure Execution}

A procedure call has a context---a \emph{memory} recording the global
variables of all declared modules.  Even if a global variable of a
module isn't explicitly initialized, it will still have a well-defined
value of the variable's type in that memory.

When the procedure is called, the memory is extended with initial
values for the procedure's parameters (the actual arguments to the
function) and local variables. The intial values for the local
variables are thus arbitrary values of the variable's types. This is
modeled in \EasyCrypt's program logics by our not knowing anything
about them; in the operational semantics, it can be modeled by
using symbolic values and doing symbolic evaluation.

When a procedure returns, its return value is stored in a special
component \ec{res} of its memory, but the procedure's parameters
and local variables are removed from the memory.

When one procedure calls another, the calling procedure's parameters
and local variables are stored on an auxiliary stack before the call.
Upon return from the procedure call, those saved variable are loaded
back into the memory, and the \ec{res} component of the memory is
removed and stored (in the case of a procedure call assignment) in the
variable on the left hand side of the procedure call.

For example, given this declaration
\begin{easycrypt}{}{}
module X = {
  proc f() : int = {
    var x : int;
    return x - x;
  }
}.
\end{easycrypt}
the procedure \ec{X.f} always returns \ec{0}, even though nothing is
known about which integer is used as the intial value for the local
variable \ec{x}.

\subsection{Module Types}

\EasyCrypt's \emph{module types} specify the types of a set of
procedures.  E.g., consider the module type \ec{OR} :
\ecinput{examps/specifications-examp4.ec}{} {3-7}{} \ec{OR} describes
minimum expectations for a ``guessing oracle''---that it provide at
least procedures with the specified types.  The order of the
procedures in a module type is irrelevant. In a procedure's type, one
may combine multiple parameters of the same type, as in:
\begin{easycrypt}{}{}
proc init(secret tries : int) : unit
\end{easycrypt}
The names of procedure parameters used in module types
are purely for documentation purposes; one may elide them instead
using underscores, writing, e.g.,
\begin{easycrypt}{}{}
proc init(_ : int, _ : int) : unit
\end{easycrypt}
Note that module types say nothing about the global variables a module
should have. Modules types have a different name space than modules.

Listing~\ref{GuessOrModule} contains an example guessing oracle
implementation.  \ecinput{examps/specifications-examp4.ec}{Guessing
  Oracle Module} {9-30}{GuessOrModule} Its \ec{init} procedure stores
the supplied secret in the global variable \ec{sec}, initializes the
allowed number of guesses in the global variable \ec{tris}, and
initializes the \ec{guessed} global variable to record that the secret
hasn't yet been guessed.  If more allowed tries remain, the \ec{guess}
procedure updates \ec{guessed} to take into account the supplied
guess, and decrements the allowed number of tries; otherwise, it does
nothing.  And its \ec{guessed} procedure returns the value of
\ec{guessed}, indicating whether the secret has been successfully
guessed, so far.
\ec{Or} \emph{satisfies} the specification of the
module type \ec{OR}, and we can ask \EasyCrypt\ to check this by
supplying that module type when declaring \ec{Or}, as in
Listing~\ref{GuessOrModuleCheck}.
\ecinput{examps/specifications-examp5.ec}{Guessing Oracle Module with
  Module Type Check} {9-30}{GuessOrModuleCheck}

Supplying a module type \emph{doesn't} change the result of a module
declaration. E.g., if we had omitted \ec{guessed} from the module type
\ec{OR}, the module \ec{Or} would still have had the procedure
\ec{guessed}. Furthermore, when declaring a module, we can ask
\EasyCrypt\ to check whether it satisfies multiple module types, as
in:
\begin{easycrypt}{}{}
module type A = { proc f() : unit }.
module type B = { proc g() : unit }.
module X : A, B = {
  var x, y : int
  proc f() : unit = { x <- x + 1; }
  proc g() : unit = { y <- y + 1; }
}.
\end{easycrypt}

Suppose we want to declare a cryptographic game using our guessing
oracle, parameterized by an adversary with access to the \ec{guess}
procedure of the oracle, and which provides two procedures---one
for choosing the range in which the guessing game will operate, and
one for doing the guessing.  We'd like to write something like:
\begin{easycrypt}{}{}
module type GAME = {
  proc main() : bool
}.

module Game(Adv : ADV) : GAME = {
  module A = Adv(Or)
  proc main() : bool = { #\ldots# }
}.
\end{easycrypt}
Thus, the module type \ec{ADV} for adversaries must be parameterized
by an implementation of \ec{OR}. Given the adversary procedures we have
in mind, the syntax for this is
\begin{easycrypt}{}{}
module type ADV(O : OR) = {
  proc chooseRange() : int * int
  proc doGuessing() : unit
}.
\end{easycrypt}
But this declaration would give the adversary access to all of \ec{O}'s
procedures, which isn't what we want. Instead, we can write
\begin{easycrypt}{}{}
module type ADV(O : OR) = {
  proc chooseRange() : int * int {}
  proc doGuessing() : unit {O.guess}
}.
\end{easycrypt}
meaning that \ec{chooseRange} has no access to the oracle, and
\ec{doGuessing} may only call its \ec{guess} procedure. Using
this notation, our original attempt would have to be written
\begin{easycrypt}{}{}
module type ADV(O : OR) = {
  proc chooseRange() : int * int {O.init O.guess O.guessed}
  proc doGuessing() : unit {O.init O.guess O.guessed}
}
\end{easycrypt}
Finally, we can specify that \ec{chooseRange} must initialize
all of the adversary's global variables (if any) by using a
star annotation:
\begin{easycrypt}{}{}
module type ADV(O : OR) = {
  proc * chooseRange() : int * int {}
  proc doGuessing() : unit {O.guess}
}.
\end{easycrypt}

The full Guessing Game example is contained in Listing~\ref{FullGuessing}.
\ecinput{examps/specifications-examp5.ec}{Full
  Guessing Game Example} {}{FullGuessing}
%\ecinput{examps/specifications-examp5.ec}{Full
%  Guessing Game Example (II)} {57-82}{FullGuessingII}
%\ec{SimpAdv} is a simple implementation of an adversary.
The inclusion of the constraint \ec{ADV(O)} in \ec{SimpAdv}'s declaration
\begin{easycrypt}{}{}
module SimpAdv(O : OR) : ADV(O) = #\ldots#
\end{easycrypt}
makes \EasyCrypt\ check that: its implementation of \ec{chooseRange}
doesn't use \ec{O} at all; its implementation of \ec{doGuessing}
doesn't use any of \ec{O}'s procedures other than \ec{guess}; and
that \ec{chooseRange} initializes \ec{SimpAdv}'s global variables.
Its \ec{chooseRange} procedure chooses the range of \ec{1} to \ec{100},
and initializes global variables recording this range and the
number of guesses it will make (see the code of \ec{Game} to see
why \ec{10} is a sensible choice). The \ec{doGuessing} procedure
makes \ec{10} random guesses.

Despite \ec{SimpAdv} being a parameterized module, to refer to one
of its global variables from another module one ignores the parameter,
saying, e.g.,
\begin{easycrypt}{}{}
module X = {
  proc f() : int = {
    return SimpAdv.tries;
  }
}.
\end{easycrypt}
On the other hand, to call one of \ec{SimpAdv}'s procedures, one needs
to specify which oracle parameter it will use, as in:
\begin{easycrypt}{}{}
module X = {
  proc f() : unit = {
    SimpAdv(Or).doGuessing();
  }
}.
\end{easycrypt}

The module \ec{Game} gives its adversary parameter, \ec{Adv}, the
concrete guessing oracle \ec{Or}, calling the resulting module \ec{A}. Its
main function then uses \ec{Or} and \ec{A} to run the game.
\begin{itemize}
\item It calls \ec{A}'s \ec{chooseRange} procedure to get the adversary's
  choice of guessing range. If the range doesn't have at least ten elements,
  it returns \ec{false} without doing anything else---the adversary
  has supplied a range that's too small.

\item Otherwise, it uses \ec{Or.init} to initialize the guessing
  oracle with a secret that's randomly chosen from the range, plus a
  number of allowed guesses that's one tenth of the range's size.

\item It then calls \ec{A.doGuessing}, allowing the adversary to
  attempt the guess the secret.

\item Finally, it calls \ec{Or.guessed} to learn whether the
  adversary has guessed the secret, returning this boolean
  value as its result.
\end{itemize}

Finally, the declaration
\begin{easycrypt}{}{}
module SimpGame = Game(SimpAdv).
\end{easycrypt}
declares \ec{SimpGame} to be the specialization of \ec{Game} to
our simple adversary, \ec{SimpAdv}. When processing this declaration,
\EasyCrypt's type checker verifies that \ec{SimpAdv} satisfies
the specification \ec{ADV}.
The reader might be wondering what---if anything---prevents us writing
a version of \ec{SimpAdv} that directly accesses/calls the global
variables and procedures of \ec{Or} (or of \ec{Game}, were \ec{SimpAdv}
declared after it), violating our understanding of the adversary's
power. The answer is that \EasyCrypt's type checker isn't in a position
to do this. Instead, we'll see in the next section how such
constraints are modeled using \EasyCrypt's logic.

\subsection{Global Variables}
\label{GlobalVariables}

The set of all \emph{global variables of} a module $M$ is the
union of
\begin{itemize}
\item the set of global variables that are declared in $M$; and

\item the set of all global variables declared in modules that
  \emph{could} be read or written by a series of procedure calls
  beginning with a call of one of $M$'s procedures. By ``could'' we
  mean the read/write analysis assumes the execution of both branches
  of conditionals, the execution of while loop bodies, and the
  terminal of while loops.
\end{itemize}
To print the global variables of a module $M$, one runs:
\begin{easycrypt}{}{}
print glob #$M$#.
\end{easycrypt}

For example, suppose we make these declarations:
\begin{easycrypt}{}{}
module Y1 = {
  var y, z : int
  proc f() : unit = { y <- 0; }
  proc g() : unit = { }
}.
module Y2 = {
  var y : int
  proc f() : unit = { Y1.f(); }
}.
module Y3 = {
  var y : int
  proc f() : unit = { Y1.g(); }
}.
module type X = {
  proc f() : unit
}.
module Z(X : X) = {
  var y : int
  proc f() : unit = { X.f(); }
}.
\end{easycrypt}
Then: the set of global variables of \ec{Y1} consists of \ec{Y1.y} and
\ec{Y1.z}; the set of global variables of \ec{Y2} consists of
\ec{Y1.y} and \ec{Y2.y}; the set of global variables of \ec{Y3}
consists of \ec{Y3.y}; the set of global variables of \ec{Z} consists
of \ec{Z.y}; and the set of global variables of \ec{Z(Y1)} consists of
\ec{Z.y} and \ec{Y1.y}. In the case of \ec{Z}, because its parameter
\ec{X} is abstract, no global variables are obtained from \ec{X}.

For every module $M$, there is a corresponding type, \ec{glob $\,M$},
where a value of type \ec{glob $\,M$} is a tuple consisting of
a value for each of the global variables of $M$. Nothing can be done with
values of such types other than compare them for equality.

\section{Logics}

\subsection{Formulas}

The \emph{formulas} of \EasyCrypt's ambient logic are formed by
adding to \EasyCrypt's expressions
\begin{itemize}
\item universal and existential quantification,

\item application of built-in and user-defined predicates,

\item probability expressions and lossless assertions, and

\item \hl, \phl and \prhl judgments,
\end{itemize}
and identifying the formulas with the extended expressions of type
\ec{bool}.
This means we automatically have all boolean operators as operators on
formulas, with their normal precedences and associativities, including
negation
\begin{easycrypt}{}{}
op [!] : bool -> bool.
\end{easycrypt}
the two semantically equivalent disjunctions
\begin{easycrypt}{}{}
op (||) : bool -> bool -> bool.
op (\/) : bool -> bool -> bool.
\end{easycrypt}
the two semantically equivalent conjunctions
\begin{easycrypt}{}{}
op (&&) : bool -> bool -> bool.
op (/\) : bool -> bool -> bool.
\end{easycrypt}
implication
\begin{easycrypt}{}{}
op (=>) : bool -> bool -> bool.
\end{easycrypt}
and if-and-only-if
\begin{easycrypt}{}{}
op (<=>) : bool -> bool -> bool.
\end{easycrypt}

The quantifiers' bound identifiers are typed, although \EasyCrypt\
will attempt to infer their types if they are omitted. Universal and
existential quantification are written as
\begin{easycrypt}{}{}
forall (#$x$# : #$t$#), #$\,\phi$#
\end{easycrypt}
and
\begin{easycrypt}{}{}
exists (#$x$# : #$t$#), #$\,\phi$#
\end{easycrypt}
respectively, where the formula $\phi$ typically involves the
identifier $x$ of type $t$. We can abbreviate nested universal or
existential quantification in the style of nested anonymous functions,
writing, e.g.,
\begin{easycrypt}{}{}
forall (x : int, y : int, z : bool), #\ldots#
forall (x y : int, z : bool), #\ldots#
forall (x y : int) (z : bool), #\ldots#
exists (x : int, y : int, z : bool), #\ldots#
exists (x y : int, z : bool), #\ldots#
exists (x y : int) (z : bool), #\ldots#
\end{easycrypt}
Quantification extends as far to the right as possible, i.e.,
has lower precedence than the binary operations on formulas.

Abstract \emph{predicates} may be defined as in:
\begin{easycrypt}{}{}
pred P0 : ().
pred P1 : (int).
pred P2 : (int, int * bool).
pred P3 : (int, int * bool, real -> int).
\end{easycrypt}
Then, \ec{P0}, \ec{P1}, \ec{P2} and \ec{P3} are extended expressions
of types \ec{bool}, \ec{int -> bool}, \ec{int -> int * bool -> bool}
and \ec{int -> int * bool -> (real -> int) -> bool}, respectively.
Thus, if $e_1$, $e_2$ and $e_3$ are extended expressions of types
\ec{int}, \ec{int * bool} and \ec{real -> int}, respectively, then
\ec{P0}, \ecn{P1 $\,e_1$}, \ecn{P2 $\,e_1$ $\,e_2$} and \ecn{P3
  $\,e_1$ $\,e_2$ $\,e_3$} are formulas.
One may also declare \ec{P0} by simply
\begin{easycrypt}{}{}
pred P0.
\end{easycrypt}

Concrete predicates are defined in a way that is similar to how
operators are declared.  E.g., if we declare
\begin{easycrypt}{}{}
pred Q (x y : int, z : bool) = x = y /\ z.
\end{easycrypt}
or
\begin{easycrypt}{}{}
pred Q (x : int) (y : int) (z : bool) = x = y /\ z.
\end{easycrypt}
then \ec{Q} is an extended expression of type
\begin{easycrypt}{}{}
int -> int -> bool -> bool
\end{easycrypt}
meaning that, e.g.,
\begin{easycrypt}{}{}
(fun (b : bool -> bool) => b true) (Q 3 4)
\end{easycrypt}
is a formula.
And here is how polymorphic predicates may be defined:
\begin{easycrypt}{}{}
pred R ['a, 'b] : ('a, 'a * 'b).
pred R' ['a, 'b] (x : 'a, y : 'a * 'b) = (y.`1 = x).
\end{easycrypt}

Extended expressions also include program memories, although there
isn't a type of memories, and anonymous functions and operators can't
take memories as inputs.  If \ec{&$m$} is a memory and $x$ is a
program variable that's in \ec{&$m$}'s domain, then \ec{x\{$m$\}} is
the extended expression for the value of $x$ in \ec{&$m$}.
Quantification over memories is allowed:
\begin{easycrypt}{}{}
forall &#$m$#, #$\phi$#
\end{easycrypt}
Here, \ec{&$m$} ranges over all memories with domains equal to the
set of all global variables of all currently declared modules.
E.g., suppose we have declared:
\begin{easycrypt}{}{}
module X = { var x : int }.
module Y = { var y : int }.
\end{easycrypt}
Then, this is a (true) formula:
\begin{easycrypt}{}{}
forall &m, X.x{m} < Y.y{m} => X.x{m} + 1 <= Y.y{m}
\end{easycrypt}

\EasyCrypt's logics can introduce memories whose domains include not
just the global variables of modules but also:
\begin{itemize}
\item the local variables and parameters of
  procedures; and

\item \ec{res}, whose value in a memory resulting from running a
  procedure will record the result (return value) of the procedure.
\end{itemize}
There is no way for the user to introduce such memories directly.  We
can't do anything with memories other than look up the values of
variables in them. In particular, formulas can't test or assert the
equality of memories.

If $M$ is a module and \ec{&$m$} is a memory, then \ec{(glob
  $\,M$)\{$m$\}} is the value of type \ec{glob $\,M$} consisting of the
tuple whose components are the values of all the global variables of
$M$ in \ec{&$m$}.  (See Subsection~\ref{GlobalVariables} for the
definition of the set of all global variables of a module.)

For convenience, we have the following derived syntax for formulas: If
$\phi$ is a formula and \ec{&$m$} is a memory, then \ec{$\phi$\{$m$\}}
means the formula in which every subterm $u$ of $\phi$ consisting of a
variable or \ec{res} or \ec{glob $\,M$}, for a module
$M$, is replaced by \ec{$u$\{$m$\}}.  For example,
\begin{easycrypt}{}{}
(Y1.y = Y1.z => Y1.z = Y1.y){m}
\end{easycrypt}
expands to
\begin{easycrypt}{}{}
Y1.y{m} = Y1.z{m} => Y1.z{m} = Y1.y{m}
\end{easycrypt}
The parentheses are necessary, because \ec{$\_$\{$m$\}} has
higher precedence than even function application.
We say that \ec{&$m$} satisfies $\phi$ iff \ec{$\phi$\{$m$\}} holds.

Extended expressions also include modules, although there isn't a type
of modules, and anonymous functions and operators can't take modules
as inputs.  Quantification over modules is allowed. If $T$ is a module
type, and $M$ is a module name, then
\begin{easycrypt}{}{}
forall (#$M$# <: #$T$#), #$\phi$#
\end{easycrypt}
means
\begin{center}
 for all modules $M$ satisfying $T$, $\phi$ holds.
\end{center}
Formulas can't talk about module equality.

There is also a variant form of module quantification of the
the form
\begin{easycrypt}{}{}
forall (#$M$# <: #$T$#{#$N_1$#,#\ldots#,#$N_l$#}), #$\phi$#
\end{easycrypt}
where $N_1,\ldots,N_l$ are modules, for $l\geq 1$. Its meaning is
\begin{center}
  for all modules $M$ satisfying $T$ whose sets of global
  variables\\are disjoint from the sets of global variables of the
  $N_i$, $\phi$ holds.
\end{center}

Finally, \EasyCrypt's ambient logic has probability expressions, \hl,
\phl and \prhl judgments, and lossless assertions:
\begin{itemize}
\item (\textbf{Probability Expressions})\quad A \emph{probability expression}
  has the form
  \begin{center}
    \ec{Pr[$M$.$p$($e_1$,$\,\ldots$,$\,e_n$) @ &$m$ : $\,\phi$]}
  \end{center}
where:
\begin{itemize}
\item $p$ is a procedure of module $M$ that takes $n$ arguments, whose
types agree with the types of the $e_i$;

\item \ec{&$m$} is a memory whose domain is the global variables of
  all declared modules;

\item the formula $\phi$ may involve the term \ec{res}, whose
  type is \ec{$M$.$p$}'s return type, as well as global variables
  of modules.
\end{itemize}
Occurrences in $\phi$ of bound identifiers (bound outside
the probability expression) whose names conflict with parameters and local
variables of \ec{$M$.$p$} will refer to the bound identifiers, not the
parameters/local variables.

The informal meaning of the probability expression is
the probability that running \ec{$M$.$p$} with arguments
$e_1$, \ldots, $e_n$, and initial memory \ec{&$m$} will terminate
in a final memory satisfying $\phi$. To run \ec{$M$.$p$}:
\begin{itemize}
\item \ec{&$m$} is extended to map \ec{$M$.$p$}'s parameters to the
 $e_i$, and to map the procedure's local variables to \emph{arbitrary}
 initial values;

\item the body of the procedure is run in this extended memory.
\end{itemize}
If initializing the procedure's local variables to different values
affects the result of running the procedure's body, the probability
may be poorly defined.

\item (\hl Judgments)\quad A \hl \emph{judgment} has the form
  \begin{center}
    \ec{hoare[$M$.$p$ : $\,\phi$ ==> $\,\psi$]}
  \end{center}
where:
\begin{itemize}
\item $p$ is a procedure of module $M$;

\item the formula $\phi$ may involve the global variables of declared
  modules, as well as the parameters of \ec{$M$.$p$};

\item the formula $\psi$ may involve the term \ec{res}, whose type is
  \ec{$M$.$p$}'s return type, as well as the global variables of
  declared modules.
\end{itemize}
Occurrences in $\phi$ and $\psi$ of bound identifiers (bound outside
the judgment) whose names conflict with parameters and local
variables of \ec{$M$.$p$} will refer to the bound identifiers, not the
parameters/local variables.

The informal meaning of the \hl judgment is that, for all initial
memories \ec{&$m$} satisfying $\phi$ and whose domains consist of the
global variables of declared modules plus the parameters and local
variables of \ec{$M$.$p$}, if running the body of \ec{$M$.$p$} in
\ec{&$m$} results in termination with a memory, that memory
satisfies $\psi$.

\item (\phl Judgments)\quad A \phl \emph{judgment} has one of the forms
  \begin{center}
    \ec{phoare[$M$.$p$ : $\,\phi$ ==> $\,\psi$] < $\,e$} \\
    \ec{phoare[$M$.$p$ : $\,\phi$ ==> $\,\psi$] = $\,e$} \\
    \ec{phoare[$M$.$p$ : $\,\phi$ ==> $\,\psi$] > $\,e$}
  \end{center}
where:
\begin{itemize}
\item $p$ is a procedure of module $M$;

\item the formula $\phi$ may involve the global variables of declared
  modules, as well as the parameters of \ec{$M$.$p$};

\item the formula $\psi$ may involve the term \ec{res}, whose type is
  \ec{$M$.$p$}'s return type, as well as the global variables of
  declared modules;

\item $e$ is an expression of type \ec{real}.
\end{itemize}
Occurrences in $\phi$ and $\psi$ and of bound identifiers (bound
outside the judgment) whose names conflict with parameters or local
variables of \ec{$M$.$p$} will refer to the bound identifiers, not the
parameters/local variables. \ec{e} will have to be parenthesized
unless it is a constant or nullary operator.

The informal meaning of the \phl judgment is that, for all initial
memories \ec{&$m$} satisfying $\phi$ and whose domains consist of the
global variables of declared modules plus the parameters and local
variables of \ec{$M$.$p$}, that the probability that
\begin{center}
  running the body of \ec{$M$.$p$} in \ec{&$m$} results in\\
  termination with a memory satisfying $\psi$
\end{center}
has the indicated relation to the value of $e$.

\item (\prhl Judgments)\quad A \prhl \emph{judgment} has the form
  \begin{center}
     \ec{equiv[$M$.$p$ ~ $\,N$.$q$ : $\,\phi$ ==> $\,\psi$]}
  \end{center}
where:
\begin{itemize}
\item $p$ is a procedure of module $M$, and $q$ is a procedure of module
 $N$;

\item the formula $\phi$ may involve the global variables of declared
  modules, the parameters of \ec{$M$.$p$}, which must be interpreted
  in memory \ec{&1} (e.g., \ec{x\{1\}}), and the parameters of
  \ec{$N$.$q$}, which must be interpreted in memory \ec{&2};

\item the formula $\psi$ may involve the global variables of declared
  modules, \ec{res\{1\}}, which has the type of \ec{$M$.$p$}'s return type,
  and \ec{res\{2\}}, which has the type of \ec{$N$.$q$}'s return type.
\end{itemize}
Occurrences in $\psi$ of bound identifiers (bound outside the
judgment) whose names conflict with parameters and local variables of
\ec{$M$.$p$} and \ec{$N$.$q$} will refer to the bound identifiers, not
the parameters and local variables, even if they are enclosed in
memory references (e.g., \ec{x\{1\}}). If \ec{&1} (resp., \ec{&2}) is
a bound memory (outside the judgment), then all references to \ec{&1}
(resp., \ec{&2}) in $\phi$ and $\psi$ are renamed to use a fresh
memory.

The informal meaning of the \prhl judgment is that, for all initial
memories \ec{&1} whose domains consist of the global variables of
declared modules plus the parameters and local variables of
\ec{$M$.$p$}, for all initial memories \ec{&2} whose domains consist
of the global variables of declared modules plus the parameters and
local variables of \ec{$N$.$q$}, if $\phi$ holds, then
\fix{missing definition}

\item (Lossless Assertions)\quad A \emph{lossless assertion} has the form
  \begin{center}
    \ec{islossless $\,M$.$p$}
  \end{center}
  and is simply an abbreviation for
  \begin{center}
    \ec{phoare[$M$.$p$ : true ==> true] = 1\%r}
  \end{center}
\end{itemize}

Consider these declarations:
\begin{easycrypt}{}{}
module G1 = {
  proc f() : bool = {
    var x : bool;
    x <$ {0,1};
    return x;
  }
}.

module G2 = {
  proc f() : bool = {
    var x, y : bool;
    x <$ {0,1}; y <$ {0,1};
    return x ^^ y;  (* ^^ is exclusive or *)
  }
}.
\end{easycrypt}
Then:
\begin{itemize}
\item If \ec{&$m$} is a memory,
\begin{easycrypt}{}{}
Pr[G1.f() @ &m : res]
\end{easycrypt}
is the probability that \ec{G1.f()} returns \ec{true} when run
in this memory. (It's value is \ec{1\%r / 2\%r}.)

\item The \hl judgment
\begin{easycrypt}{}{}
hoare[G2.f : true ==> !res]
\end{easycrypt}
says that, if \ec{G2.f()} halts (which we know it will), then
its return value will be \ec{false}. (This judgment is \ec{false}.)

\item The \phl judgment
\begin{easycrypt}{}{}
phoare[G2.f : true ==> res] = (1%r / 2%r)
\end{easycrypt}
says that the probability of \ec{G2.f()} returning \ec{true} is
\ec{1\%r / 2\%r}. (This judgment is \ec{true}.)

\item The \prhl judgment
\begin{easycrypt}{}{}
equiv[G1.f ~ G2.f : true ==> res]
\end{easycrypt}
says that \ec{G1.f()} and \ec{G2.f()} are equally likely to return
\ec{true} as well as equally likely to return \ec{false}.
(This judgment is \ec{true}.)

\item The lossless assertion
\begin{easycrypt}{}{}
lossless G2.f    
\end{easycrypt}
says that \ec{G2.f()} always terminates, no matter what memory it's
run in. (This judgment is \ec{true}.)
\end{itemize}

\subsection{Axioms and Lemmas}

One states an \emph{axiom} or \emph{lemma} by giving a well-typed formula with
no free identifiers, as in:
\begin{easycrypt}{}{}
axiom Sym : forall (x y : int), x = y => y = x.
lemma Sym : forall (x y : int), x = y => y = x.
\end{easycrypt}
The difference between axioms and lemmas is that axioms are trusted
by \EasyCrypt, whereas lemmas must be proved, in the steps that
follow. The \emph{proof} of a lemma has the form
\begin{easycrypt}{}{}
proof.
#$\mathit{step}_1$#, #\ldots# #$\mathit{step}_1$#.
qed.
\end{easycrypt}
Actually the \ec{proof.} step is optional, but it's good style to include
it. The steps of the proof consist of tactic applications as well
as \ec{print} and \ec{search} commands.
The \ec{qed.} step saves the lemma, making it available for reuse;
it's only allowed when the proof is complete.
When the proof for a lemma has a very simple form, the proof may be
included as part of the lemma's statement:
\begin{easycrypt}{}{}
lemma #$\mathit{name}$# : #$\phi$# by [#$\mathit{tactic}$].
\end{easycrypt}
or
\begin{easycrypt}{}{}
lemma #$\mathit{name}$# : #$\phi$# by [].
\end{easycrypt}
In the first case, the proof consists of a single tactic; the meaning of
\ec{by []} will be described in Chapter~\ref{Tactics}.

One may also parameterize an axiom of lemma by the free identifiers of
its formula, as in:
\begin{easycrypt}{}{}
lemma Sym (x : int) (y : int) : x = y => y = x.
\end{easycrypt}
or
\begin{easycrypt}{}{}
lemma Sym (x y : int) : x = y => y = x.
\end{easycrypt}
Parameterizing an axiom or lemma by the free identifiers of its formula
has the same logical meaning as universally quantifying those identifiers
in the formula. But we'll see in Chapter~\ref{Tactics} why the
parameterized form makes an axiom or lemma easier to apply.

Polymorphic axioms and lemmas may be stated using a syntax reminiscent
of the one for operators:
\begin{easycrypt}{}{}
lemma Sym ['a] (x y : 'a) : x = y => y = x.
lemma PairEq ['a, 'b] (x x' : 'a) (y y' : 'b) :
  x = x' => y = y' => (x, y) = (x', y').
\end{easycrypt}
or
\begin{easycrypt}{}{}
lemma Sym (x y : 'a) : x = y => y = x.
lemma PairEq (x x' : 'a) (y y' : 'b) :
  x = x' => y = y' => (x, y) = (x', y').
\end{easycrypt}

We can axiomatize the meaning of abstract types, operators and relations.
E.g., an abstract type of monoids may be axiomatized by:
\begin{easycrypt}{}{}
type monoid.
op id : monoid.
op (+) : monoid -> monoid -> monoid.
axiom LeftIdentity (x : monoid) :
  id + x = x.
axiom RightIdentity (x : monoid) :
  x + id = x.
axiom Associative (x y z : monoid) :
  x + (y + z) = (x + y) + z.
\end{easycrypt}
Here, we know nothing about \ec{monoid} other than these axioms.

One must be careful with axioms, however, because it's easy to introduce
inconsistencies, allowing one to prove false formulas. E.g., because
all types must be nonempty in \EasyCrypt, writing
\begin{easycrypt}{}{}
type t.
axiom Empty : !(exists (x : t), true).
\end{easycrypt}
will allow us to prove \ec{false}.

Axioms and lemmas may be parameterized by
memories and modules. Consider the declarations:
\begin{easycrypt}{}{}
module type T = {
  proc f() : unit
}.
module G(X : T) = {
  var x : int
  proc g() : unit = {
    X.f();
  }
}.
\end{easycrypt}
Then lemma \ec{Lossless}
\begin{easycrypt}{}{}
lemma Lossless (X <: T) : islossless X.f => islossless G(X).g.
\end{easycrypt}
which is parameterized by an abstract module \ec{X} of module type
\ec{T}, says that \ec{G(X).g} always terminates, no matter the memory
it's run in, as long as this is true of \ec{X.f}.
Lemma \ec{Invar}
\begin{easycrypt}{}{}
lemma Invar (X <: T{G}) (n : int) :
  islossless X.f =>
  phoare[G(X).g : G.x = n ==> G.x = n] = 1%r.
\end{easycrypt}
which is parameterized by an abstract module \ec{X} of module type
\ec{T} that is guaranteed not to access or modify \ec{G.x}, and an
integer \ec{n}, says that, assuming \ec{X.f} is lossless, if
\ec{G(X).g()} is run in a memory giving \ec{G.x} the value \ec{n},
then \ec{G(X).g()} is guaranteed to terminate in a memory in which
\ec{G.x}'s value is still \ec{n}.  Finally lemma \ec{Invar'}
\begin{easycrypt}{}{}
lemma Invar' (X <: T{G}) (n : int) &m :
  islossless X.f => G.x{m} = n =>
  Pr[G(X).g() @ &m : G.x = n] = 1%r.
\end{easycrypt}
which has the parameters of \ec{Invar} plus a memory \ec{&m},
says the same thing as \ec{Invar}, but using a probability
expression rather than a \phl judgment.
