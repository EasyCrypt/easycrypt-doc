\chapter{Specifications}
\label{Specifications}

In this chapter, we see how cryptographic specifications may be written
in EasyCrypt. We start by considering its typed expression language,
go on to consider its module language for expressing cryptographic games,
and conclude by considering its four logics.

\EasyCrypt\ has a typed expression language based on the polymorphic
typed lambda calculus. Expressions are guaranteed to terminate,
although their values may be under-specified.  Its type system has:
\begin{itemize}
\item several pre-defined base types;

\item product (tuple) types;

\item user-defined abbreviations for types and parameterized types; and

\item user-defined concrete datatypes (like lists and trees).
\end{itemize}
In its expression language:
\begin{itemize}
\item operators for the pre-defined base types may be imported from
  the standard library;

\item user-defined operations may be defined, including by
  structural recursion on concrete datatypes.
\end{itemize}
For each type, there is a type of probability distributions over that
type.

\EasyCrypt's modules consist of typed global variables and procedures.
The body of a procedure consists of local variable declartions followed
by a sequence of statements:
\begin{itemize}
\item ordinary assignments;

\item random assignments, assigning values chosen
  from distributions to variables;

\item procedure calls, whose results are assigned to variables;

\item conditional (if-then-else) statements;

\item while loops; and

\item return statements (which may only appear at the end of
  procedures).
\end{itemize}
A procedure's modules may refer to the global variables of previously
declared modules.

\EasyCrypt\ has four logics: a probabilistic, relational Hoare logic
(\prhl), relating pairs of procedures; a probabilistic Hoare logic
(\phl) allowing one to carry out proofs about the probability of a
procedure's execution resulting in a postcondition holding; an
ordinary (possibilistic) Hoare logic (\hl); and an ambient higher
order logic for proving general mathematical facts, as well as for
connecting judgments from the other logics

Proofs are carried out using tactics, which is the focus of
Chapter~\ref{Tactics}.  \EasyCrypt\ also has ways (theories and
sections) of structuring specifications and proofs, which will be
described in Chapter~\ref{Structuring}. In Chapter~\ref{Library},
we'll survey the \EasyCrypt\ Library, which consists of numerous
theories, definining mathematical structures (like groups, rings
and fields), data structures (like finite sets and maps), and
cryptographic constructions (like random oracles and different
forms of encryption).

\section{Typed Expressions}

\section{Modules}

\section{Logics}
