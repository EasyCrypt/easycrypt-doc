\chapter{Tactics}
\label{chap:tactics}

Proofs in \EasyCrypt are carried out using \emph{tactics}, logical
rules embodying general reasoning principles, which transform the
current lemma (or \emph{goal}) into zero or more
\emph{subgoals}---sufficient conditions for the lemma/goal to
hold. Simple ambient logic goals may be automatically proved using SMT
solvers.

In this chapter, we introduce \EasyCrypt's proof engine, before
describing the tactics for \EasyCrypt's four logics: ambient, \prhl,
\phl and \hl.

\section{Proof Engine}

\EasyCrypt's proof engine works with goal lists, where a \emph{goal}
has two parts:
\begin{itemize}
\item A \emph{context} consisting of a
  \begin{itemize}
  \item a set of type variables, and

  \item an \emph{ordered} set of \emph{assumptions}, consisting of
    identifiers with their types, memories, module names with their
    module types and restrictions, local definitions, and
    \emph{hypotheses}, i.e., formulas. An identifier's type may
    involve the type variables, the local definitions and formulas may
    involve the type variables, identifiers, memories and module
    names.
  \end{itemize}

\item A \emph{conclusion}, consisting of a single formula, with
  the same constraints as the assumption formulas.
\end{itemize}
Informally, to prove a goal, one must show the conclusion to be true,
given the truth of the hypotheses, for all valid instantiations
of the assumption identifiers, memories and module names.

For example,
\ecinput{examps/parts/tactics-examp0-2.1.ec}{}{}{}{}
is a goal.
And, in the context of the declarations
\ecinput{examps/tactics-examp2.ec}{}{3-11}{}{}
this is a goal:
\ecinput{examps/parts/tactics-examp2-2.0.ec}{}{}{}{}
The conclusion of this goal is just a nonlinear rendering of the formula
\begin{easycrypt}{}{}
phoare [G(X).g : G.x = n ==> G.x = n] = 1%r.
\end{easycrypt}
\EasyCrypt's pretty printer renders \prhl, \phl and \hl judgements
in such a nonlinear style when the judgements appear as
(as opposed to in) the conclusions of goals.

Internally, \EasyCrypt's proof engine also works with \prhl, \phl and
\hl judgments involving lists of statements rather than procedure
names, which we'll call \emph{statement judgements}, below. For
example, given this declaration
\ecinput{examps/tactics-examp1.ec}{}{3-9}{}{} this is an \phl
statement judgement:
\ecinput{examps/parts/tactics-examp1-3.0.ec}{}{}{}{} The pre- and
post-conditions of a statement judgement may refer to the parameters
and local variables of the \emph{procedure context} of the
conclusion---\ec{M.f} in the preceding example. They may also refer
to the memories \ec{&$1$} and \ec{&$2$} in the case of \prhl statement
judgements.  When a statement judgement appears anywhere other than as
the conclusion of a goal, the pretty printer renders it in abbreviated
linear syntax.  E.g., the preceding goal is rendered as
\begin{easycrypt}{}{}
hoare[if (x %% 3 = 1) {...} : x %% 3 = n ==> x %% 3 = n %% 2 + 1]
\end{easycrypt}
Statement judgements can't be directly input by the user.

We use the term \emph{program} to refer to \emph{either} a procedure
appearing in a \prhl, \phl or \hl judgement, \emph{or} a statement
list appearing in a \prhl, \phl or \hl statement judgement. In the
case of \prhl (statement) judgements, we speak of the \emph{left} and
\emph{right} programs, also using \emph{program 1} for the left
program, and \emph{program 2} for the right one. We will only speak of
a program's \emph{length} when it's a statement list we are referring
to. By the \emph{empty} program, we mean the statement list with no
statements.

When the proof of a lemma is begun, the proof engine starts out with
a single goal, consisting of the lemma's statement. E.g.,
the lemma
\ecinput{examps/tactics-examp0.ec}{}{1-3}{}{}
gives rise to the goal
\ecinput{examps/parts/tactics-examp0-1.0.ec}{}{}{}{}
For parameterized lemmas, the goal includes the lemma's parameters
as assumptions. E.g.,
\begin{easycrypt}{}{}
lemma PairEq (x x' : 'a) (y y' : 'b) :
  x = x' => y = y' => (x, y) = (x', y').
\end{easycrypt}
gives rise to
\ecinput{examps/parts/tactics-examp0-2.0.ec}{}{}{}{}

\EasyCrypt's tactics, when applicable, reduce the first goal to zero
or more subgoals.  E.g., if the first goal is
\ecinput{examps/parts/tactics-examp1-3.0.ec}{}{}{}{}
then applying the \ec{if} tactic (handle a conditional) reduces
(replaces) this goal with the two goals
\ecinput{examps/parts/tactics-examp1-3.1.ec}{}{}{}{} and
\ecinput{examps/parts/tactics-examp1-3.2.ec}{}{}{}{}
(leaving the remaining goals, if any, unchanged).
If the first goal is
\ecinput{examps/parts/tactics-examp1-6.0.ec}{}{}{}{}
then applying the \ec{smt} tactic (try to solve the goal using
SMT provers) solves the goal, i.e., replaces it with no subgoals.
Applying a tactic may fail; in this case an error message is issued
and the list of goals is left unchanged.

A lemma's proof may be saved, using the step \ec{qed}, when the list
of goals becomes empty. And this must be done before anything else may
be done.

\paragraph{Remark.}
In the descriptions of \EasyCrypt's tactics given in the following two
sections, unless otherwise specified, you should assume that the
subgoals to which a tactic reduces a goal have the same contexts as
that original goal.

\section{Matching}

Statement patterns are an extension of statements.  \fix{Add
  explanation of new \ec{replace} tactic, which uses statement
  patterns.}

\begin{tabular}{| l | l |}
  \hline
  Blocks & \\
  \hline
  \texttt{\{\}}  & empty statement \\
  \texttt{\{s\}} & start and end anchors around \texttt{s} \\
  \texttt{\{s]}  & start anchor at the beginning of \texttt{s} \\
  \texttt{[s\}}  & end anchor at the end of \texttt{s} \\
  \texttt{[s]}   & find \texttt{s} at any position in the statements, 
                  without looking into possible branches \\
  \texttt{<s>}   & find \texttt{s} at any position in the statements, 
                  and looking into possible branches \\
  \hline
\end{tabular}

\begin{tabular}{| l | l |}
  \hline
  Statements&\\
  \hline
  \texttt{\_} & any sequence of statements \\
  \texttt{n!\_} & a sequence of  \texttt{n} statements \\
  \texttt{s as X} &  \texttt{s} that is associated with the name  \texttt{X}\\
  \texttt{X} & interpreted as : \texttt{\_ as X} \\
  \hline
  \texttt{!s} & repeat  \texttt{s}, can be zero time \\
  \texttt{?s} &  \texttt{s} is to appear once, or zero time \\
  \texttt{n!s} & repeat  \texttt{s}  \texttt{n} times \\
  \texttt{[n..m]!s} & repeat  \texttt{s} at least  \texttt{n} times, up to  \texttt{m} times \\
  \texttt{[n..]!s} & repeat  \texttt{s} at least  \texttt{n} times \\
  \texttt{[..m]!s} & repeat  \texttt{s} up to  \texttt{m} times \\
  \texttt{$\sim$s} & apply not greedy carateristics to  \texttt{s} if that makes sense\\
  \hline
  \texttt{s1 ; s2} &  \texttt{s1} followed directly by  \texttt{s2} \\
  \texttt{s1 s2} &  \texttt{s1} followed directly by  \texttt{s2} \\
  \texttt{s1 | s2} &  \texttt{s1} or  \texttt{s2} \\
%  \texttt{s1 ;; s2} & set the separation between s1 and s2 for the seq tactic,
%                      it's the same behaviour as ``;''\\
  \hline
  \texttt{\_ <- \_ ;} & any affectation \\
  \texttt{\_ <\$ \_ ;} & any sample \\
  \texttt{\_ <@ \_ ;} & any procedure call \\
  \texttt{if ;} & any  \texttt{if} statement \\
  \texttt{while ;} & any  \texttt{while} statement \\
  \texttt{if \_ bt else bf} & an \texttt{if} statement where  \texttt{bt} is the
  block matched in the true branch's body,\\
  & and \texttt{bf} the block matched in the false branch's body\\
  \texttt{while \_ b} & a \texttt{while} branch where the block  \texttt{b}
  is matched in the body of the loop\\
  \hline
\end{tabular}

% Ambient Logic Section
\section{Ambient logic}
\label{sec:ambientlogic}

In this section, we describe the proof terms, tactics and tacticals of
\EasyCrypt's ambient logic.

\subsection{Proof Terms}
\label{subsec:proofterms}

Formulas introduce identifier and formula assumptions using universal
quantifiers and implications. For example, the formula
\begin{easycrypt}{}{}
forall (x y : bool), x = y => forall (z : bool), y = z => x = z.
\end{easycrypt}
introduces the assumptions
\begin{easycrypt}{}{}
x     : bool
y     : bool
eq_xy : x = y
z     : bool
eq_yz : y = z
\end{easycrypt}
(where the names of the two formulas were chosen to be meaningful),
and has \ec{x = z} as its conclusion. We refer to the first assumption
of a formula as the formula's \emph{top assumption}. E.g., the top
assumption of the preceding formula is \ec{x : bool}.

\EasyCrypt has \emph{proof terms}, which partially describe how
to prove a formula.  Their syntax is described in Figure~\ref{fig:proofterms},
where $X$ ranges over lemma (or formula assumption) names.
\begin{figure}
  \begin{center}
  \begin{tabular}{rcl>{\bf}l}
    $p$ & ::=
      & {\ec{_}} & proof hole \\
     && {\ec{($X$, $\;q_1$, $\;\ldots$, $\;q_n$)}} & lemma application \\[.2cm]
    $q$ & ::=
      & {$e$} & expression \\
      && {$p$} & proof term \\
  \end{tabular}
  \end{center}
  \caption{\label{fig:proofterms}Proof Terms}
\end{figure}
A proof term for a lemma (or formula assumption) $X$ has components
corresponding to the assumptions introduced by $X$.  A component
corresponding to a variable consists of an expression of the
variable's type. The proof term is explaining how the instantiation of
the lemma's conclusion with these expressions may be proved.  A
formula component consists of a proof term explaining how the
instantiation of the formula may be proved.  Proof holes will get
turned into subgoals when a proof term is used in backward reasoning,
e.g., by the \rtactic{apply} tactic. \fix{Need explanation of how a
  proof term may be used in forward reasoning.}

Consider, e.g., the following declarations and axioms
\begin{easycrypt}{}{}
pred P : int.
pred Q : int.
pred R : int.
axiom P (x : int) : P x.
axiom Q (x : int) : P x => Q x.
axiom R (x : int) : P(x + 1) => Q x => R x.
\end{easycrypt}
Then, given that \ec{x : int} is an assumption,
\begin{easycrypt}{}{}
(R x (P(x + 1)) (Q x (P x)))
\end{easycrypt}
is a proof term proving the conclusion \ec{R x}. And
\begin{easycrypt}{}{}
(R x _ (Q x _))
\end{easycrypt}
is a proof term that turns proofs of \ec{P(x + 1)} and \ec{P x}
into proofs of \ec{R x}. When used in backward reasoning, it
will reduce a goal with conclusion \ec{R x} to subgoals with
conclusions \ec{P(x + 1)} and \ec{P x}.
\fix{Can it be used in forward reasoning?}

Some of a proof term's expressions may be replaced by \ec{_}, asking
\EasyCrypt to infer them from the context.  Going even further, one
may abbreviate a one-level proof term all of whose components are
\ec{_} to just its lemma name. For example, we can write \ec{R} for
\ec{(R _ _ _)}.  When used in backward reasoning, it will reduce a
goal with conclusion \ec{R x} to subgoals with conclusions \ec{P(x +
  1)} and \ec{Q x}. \fix{In forward reasoning they aren't equivalent---why?}

\subsection{Occurrence Selectors and Rewriting Directions}
\label{subsec:occsels}

Some ambient logic tactics use \emph{occurrence selectors} to restrict
their operation to certain occurrences of a term or formula in a
goal's conclusion or formula assumption. The syntax is \ec{\{$i_1$,
  $\;\ldots$, $\;i_n$\}}, specifying that only occurrences $i_1$
throught $i_n$ of the term/formula in a depth-first, left-to-right
traversal of the goal's conclusion or formula assumption should be
operated on. Specifying \ec{\{- $i_1$, $\;\ldots$, $\;i_n$\}}
restricts attention to all occurrences \emph{not} in the following
list. They may also be empty, meaning that all applicable occurrences
should be operated on.

Some ambient logic tactics use \emph{rewriting directions}, $\mathit{dir}$,
which may either be empty (meaning rewriting from left to right), or \ec{-},
meaning rewriting from right to left.

\subsection{Introduction and Generalization}
\label{subsec:introgen}

\paragraph{Introduction.}

One moves the assumptions of a goal's conclusion into the goal's
context using the introduction tactical.  This tactical uses
introduction patterns, which are defined in Figure~\ref{fig:intropat}.
\begin{figure}
  \begin{center}
  \begin{tabular}{rcl>{\bf}l}
    $\iota$ & ::=
      & {$b$} & name \\
      && {\ec{_}} & no name \\
      && {\ec{+}} & auto revert \\
      && {\ec{?}} & find name \\
      && {$\mathit{occ}\;$\ec{->}} & rewrite using assumption \\
      && {$\mathit{occ}\;$\ec{<-}} & rewrite in reverse using assumption \\
      && {\ec{->>}} & substitute using assumption \\
      && {\ec{<<-}} & substitute in reverse using assumption \\
      && {\ec{/$p$}} & replace assumption by applying proof term \\
      && {\ec{\{$a_1\cdots a_n$\}}} & clear introduced assumptions \\
      && {\ec{/=}} & simplify \\
      && {\ec{//}} & trivial \\
      && {\ec{//=}} & simplify then trivial \\
      && {\ec{$\mathit{dir}\;\mathit{occ}\;$@/$\mathit{op}$}} & unfold definition of operator \\
      && {\ec{[$\iota_{11}\cdots\iota_{1{m_1}}$ | $\;\cdots$ | $\;\iota_{r1}\cdots\iota_{r{m_r}}$]}} & case pattern \\[.2cm]
    $\mathit{b}$ & ::=
      & {$x$} & identifier \\
      && {$M$} & module name \\
      && {\ec{&$m$}} & memory name \\
  \end{tabular}
  \end{center}
  \caption{\label{fig:intropat} Introduction Patterns}
\end{figure}
In this definition, $\mathit{occ}$ ranges over occurrence selectors,
and $\mathit{dir}$ ranges over directions---see
Subsection~\ref{subsec:occsels}).

If a list $\iota_1,\ldots,\iota_n$ of introduction patterns consists
entirely of \ec{//} (apply the \rtactic{trivial} tactic), \ec{/=}
(apply the \rtactic{simplify} tactic) and \ec{//=} (apply the
\ec{simplify} and then \ec{trivial}), then \emph{applying}
$\iota_1,\ldots,\iota_n$ \emph{to} a list of goals $G_1,\ldots,G_m$ is
done by applying the tactics corresponding to the $\iota_i$ in order
to each $G_j$, causing some of the goals to be solved and thus
disappear and some of the goals to be simplified.

\begin{tactic}[$\tau$=>$\;\iota_1 \cdots \iota_n$]{introduction}
  \begin{tsyntax}[empty]{}
    Runs the tactic $\tau$, matching the resulting goals, $G_1,\ldots,G_l$,
    with the introduction patterns $\iota_1,\ldots,\iota_n$:
    \begin{itemize}
    \item Suppose $k$ is such that all of $\iota_1,\ldots,\iota_{k-1}$
      are \ec{//}, \ec{/=} and \ec{//=}, and either $k>n$ or $\iota_k$
      is not \ec{//}, \ec{/=} or \ec{//=}.

    \item Let $G'_1,\ldots,G'_{l'}$ be the goals resulting from
      applying $\iota_1,\ldots,\iota_{k-1}$ to $G_1,\ldots,G_l$.

    \item If $l'=0$, the tactical produces no subgoals.

    \item Otherwise, if $k>n$, the tactical's result is
      $G'_1,\ldots,G'_{l'}$.

    \item Otherwise, if $\iota_k$ is not a case pattern, each subgoal
      $G'_i$ is matched against $\iota_k,\ldots,\iota_n$ by the
      procedure described below, with the resulting subgoals being
      collected into a list of goals (maintaining order viz a viz the
      indices $i$) as the tactical's result.

    \item Otherwise, $\iota_k$ is a case pattern
          \ec{[$\iota_{11}\cdots\iota_{1{m_1}}$ | $\;\cdots$ | $\;\iota_{r1}\cdots\iota_{r{m_r}}$]}.

     \item If $\tau$ is not equivalent to \rtactic{idtac}, the tactic fails
       unless $r = l'$, in which case each $G'_i$ is matched against 
        \begin{displaymath}
          \iota_{i1}\cdots\iota_{i{m_i}}\iota_{k+1}\cdots \iota_n
        \end{displaymath}
       by the procedure described below, with the resulting subgoals
       being collected into the tactical's result.

     \item Otherwise, $\tau$ is equivalent to \rtactic{idtac} (and so
       $l'=1$). In this case $G'_1$ is matched against
       $\iota_k,\ldots,\iota_n$ by the procedure described below, with
       the resulting subgoals being collected into a list of goals as
       the tactical's result.
    \end{itemize}

    \paragraph{Matching a single goal against a list of patterns:}

    To match a goal $G$ against a list of introduction patterns
    $\iota_1,\ldots,\iota_n$, the introduction patterns are processed
    from left-to-right, as follows:
    \begin{itemize}
    \item ($b$)\quad The top assumption (universally quantified
      identifier, module name or memory; or left side of implication)
      is consumed, and introduced with this name. Fails if the top
      assumption has neither of these forms.

    \item (\ec{$b\,$!})\quad Same as the preceding case, except that
      $b$ is used as the base of the introduced name, extending the
      base to avoid naming conflicts.

    \item (\ec{_})\quad Same as the preceding case, except the
      assumption is introduced with an anonymous name (which can't be
      uttered by the user).

    \item (\ec{+})\quad Same as the preceding case, except that after
      a branch of the procedure completes, yielding a goal, the
      assumption will be reverted, i.e., un-introduced (using a
      universal quantifier or implication as appropriate).

    \item (\ec{?})\quad Same as the preceding case, except \EasyCrypt
      chooses the name by which the assumption is introduced (using
      universally quantified names as assumption bases).

    \item ($\mathit{occ}\;$\ec{->})\quad Consume the top assumption,
      which must be an equality, and use it as a left-to-right rewriting
      rule in the remainder of the goal's conclusion, restricting rewriting
      to the specified occurrences of the equality's left side.

    \item ($\mathit{occ}\;$\ec{<-})\quad The same as the preceding case,
      except the rewriting is from right-to-left.

    \item (\ec{->>})\quad The same as \ec{->}, except the consumed
      equality assuption is used to perform a left-to-right substitution
      in the entire goal, i.e., in its assumptions, as well as its
      conclusion.

    \item (\ec{<<-})\quad The same as the preceding case, except
      the substitution is from right-to-left.

    \item (\ec{/$p$})\quad Replace the top assumption by the result
    of applying the proof term $p$ to it using forward reasoning.

    \item (\ec{\{$a_1\cdots a_n$\}})\quad Doesn't affect the goal's
      conclusion, but clears the specified assumptions, i.e., removes
      them. Fails if one or more of the assumptions can't be cleared,
      because a remaining assumption depends upon it.

    \item (\ec{/=})\quad Apply \rtactic{simplify} to goal's conclusion.

    \item (\ec{//})\quad Apply \rtactic{trivial} to goal's conclusion;
      this may solve the goal, i.e., so that the procedure's current
      branch yields no goals.

    \item (\ec{/=})\quad Apply \rtactic{simplify} and then \rtactic{trivial}
      to goal's conclusion; this may solve the goal, so that the
      procedure's current branch yields no goals.

    \item ({\ec{$\mathit{dir}\;\mathit{occ}\;$@/$\mathit{op}$}})\quad
      Unfold (fold, if the direction is \ec{-}) the definition of
      operator $\mathit{op}$ at the specified occurrences of the
      goal's conclusion. See the \rtactic{rewrite} tactic for the
      details.

    \item (\ec{[$\iota_{11}\cdots\iota_{1{m_1}}$ | $\;\cdots$ | $\;\iota_{r1}\cdots\iota_{r{m_r}}$]})\quad
      \begin{itemize}
      \item If $r=0$, then the top assumption of the goal is destructed
        using the \rtactic{case} tactic, the resulting goals are
        matched against $\iota_2,\ldots,\iota_n$, and their subgoals
        are assembled into a list of goals.

      \item Otherwise $r>0$. The goal's top assumption is destructed
        using the \rtactic{case} tactic, yielding subgoals
        $H_1,\ldots H_p$.  If $p\neq r$, the procedure fails. Otherwise
        each subgoal $H_i$ is matched against
        \begin{displaymath}
          \iota_{i1}\cdots\iota_{i{m_i}}\iota_2\cdots \iota_n
        \end{displaymath}
        with the resulting goals being collected into a list as
        the procedure's result.
      \end{itemize}
    \end{itemize}

    The following examples use the tactic \rtactic{move}, which is
    equivalent to \rtactic{idtac}.
    In its simplest form, the introduction tactical simply gives names
    to assumptions.  For example, if the current goal is
    \ecinput{examps/parts/tactics/introduction/1-1.0.ec}{}{}{}{}
    then running
    \ecinput{examps/parts/tactics/introduction/1-1.ec}{}{}{}{}
    produces
    \ecinput{examps/parts/tactics/introduction/1-1.1.ec}{}{}{}{}
    Alternatively, we can use the introduction pattern \ec{?}
    to let \EasyCrypt choose the assumption names, using
    \ec{H} as a base for formula assumptions and starting
    from the identifier names given in universal quantifiers:
    \ecinput{examps/parts/tactics/introduction/2-1.ec}{}{}{}{}
    produces
    \ecinput{examps/parts/tactics/introduction/2-1.1.ec}{}{}{}{}

    To see how the \ec{->} rewriting pattern works, suppose
    the current goal is
    \ecinput{examps/parts/tactics/introduction/4-1.0.ec}{}{}{}{}
    Then running
    \ecinput{examps/parts/tactics/introduction/4-1.ec}{}{}{}{}
    produces
    \ecinput{examps/parts/tactics/introduction/4-1.1.ec}{}{}{}{}
    Alternatively, one can introduce the assumption \ec{x = y},
    and then use the \ec{->>} substitution pattern:
    if the current goal is
    \ecinput{examps/parts/tactics/introduction/8-1.0.ec}{}{}{}{}
    then running
    \ecinput{examps/parts/tactics/introduction/8-1.ec}{}{}{}{}
    produces
    \ecinput{examps/parts/tactics/introduction/8-1.1.ec}{}{}{}{}

    To see how a view may be applied to a not-yet-introduced formula
    assumption, suppose the current goal is
    \ecinput{examps/parts/tactics/introduction/5-1.0.ec}{}{}{}{}
    Then running
    \ecinput{examps/parts/tactics/introduction/5-1.ec}{}{}{}{}
    produces
    \ecinput{examps/parts/tactics/introduction/5-1.1.ec}{}{}{}{}
    And then running
    \ecinput{examps/parts/tactics/introduction/5-2.ec}{}{}{}{}
    on this goal produces
    \ecinput{examps/parts/tactics/introduction/5-2.1.ec}{}{}{}{}

    Finally, let's see examples of how a disjunction assumption
    may be destructed, either using the \ec{case} tactic followed
    by a case introduction pattern, or by making the
    case introduction pattern do the destruction.
    For the first case, if the current goal is
    \ecinput{examps/parts/tactics/introduction/6-1.0.ec}{}{}{}{}
    then running
    \ecinput{examps/parts/tactics/introduction/6-1.ec}{}{}{}{}
    produces the two goals
    \ecinput{examps/parts/tactics/introduction/6-1.1.ec}{}{}{}{}
    and
    \ecinput{examps/parts/tactics/introduction/6-1.2.ec}{}{}{}{}
    And for the second case, if the current goal is
    \ecinput{examps/parts/tactics/introduction/7-1.0.ec}{}{}{}{}
    then running
    \ecinput{examps/parts/tactics/introduction/7-1.ec}{}{}{}{}
    produces the two goals
    \ecinput{examps/parts/tactics/introduction/7-1.1.ec}{}{}{}{}
    and
    \ecinput{examps/parts/tactics/introduction/7-1.2.ec}{}{}{}{}
    Note how we used the clear pattern to discard the assumption
    \ec{X}.
  \end{tsyntax}
\end{tactic}

\paragraph{Generalization.}

The generalization tactical moves assumptions from the context into
the conclusion and generalizes subterms or formulas of the conclusion.

\begin{tactic}[$\tau$: $\;\pi_1 \cdots \pi_n$]{generalization}
  \begin{tsyntax}[empty]{$\tau$: $\;\pi_1 \cdots \pi_n$}
    Generalize the patterns $\pi_1, \ldots, \pi_n$, starting from
    $\pi_n$ and going back, and then run tactic $\tau$.  \emph{This
      tactical is only applicable to certain tactics: \rtactic{move},
      \rtactic{case} (just the version that destructs the top
      assumption of a goal's conclusion) and \rtactic{elim}.}

    \begin{itemize}
    \item When $\pi$ is an assumption from the context, it's moved
      back into the conclusion, using universal quantification or an
      implication, as appropriate. If one assumption depends on
      another, one can't generalize the later without also
      generalizing the former.

      For example, if the current goal is
      \ecinput{examps/parts/tactics/generalize/2-1.0.ec}{}{}{}{} then
      running \ecinput{examps/parts/tactics/generalize/2-1.ec}{}{}{}{}
      produces
      \ecinput{examps/parts/tactics/generalize/2-1.1.ec}{}{}{}{} In
      this example, one can't generalize \ec{x} without also
      generalizing \ec{eq_xy}.

    \item $\pi$ may also be a subformula or subterm of the goal, or
      \ec{_}, which stands for the whole goal, possibly prefixed by an
      occurrence selector. This replaces the formula or subterm with a
      universally quantified identifier of the approprate type.

      For example, if the current goal is
      \ecinput{examps/parts/tactics/generalize/1-1.0.ec}{}{}{}{} then
      running \ecinput{examps/parts/tactics/generalize/1-1.ec}{}{}{}{}
      produces
      \ecinput{examps/parts/tactics/generalize/1-1.1.ec}{}{}{}{}
      Alternatively, running
      \ecinput{examps/parts/tactics/generalize/3-1.ec}{}{}{}{}
      produces
      \ecinput{examps/parts/tactics/generalize/3-1.1.ec}{}{}{}{}
    \end{itemize}
  \end{tsyntax}
\end{tactic}

\subsection{Tactics}

% --------------------------------------------------------------------
\begin{tactic}{idtac}
\end{tactic}

% --------------------------------------------------------------------
\begin{tactic}[move | move: $\;\pi_1 \cdots \pi_n$]{move}
  \begin{tsyntax}{move}
     Does nothing, equivalent to \rtactic{idtac}. This form is mainly
     used in conjonction with an introduction pattern (see
     Section~\ref{s:intro-pattern}), e.g. \ls!move=> $\iota_1 \cdots \iota_n$!.
  \end{tsyntax}

  \begin{tsyntax}{move: $\;\pi_1 \cdots \pi_n$}
    Generalize the patterns $\pi_1, \cdots, \pi_n$, starting from
    $\pi_n$ and going back.
    %See Section~\ref{s:gen-pattern} for more
    %information on the generalization mechanism.
  \end{tsyntax}
\end{tactic}

% --------------------------------------------------------------------
\begin{tactic}{clear}
\end{tactic}

% --------------------------------------------------------------------
\begin{tactic}{assumption}
  \begin{tsyntax}[empty]{assumption}
    Search in the context for a hypothesis that is convertible to the
    goal's conclusion, solving the goal if one is found. Fail if none
    can be found.

    For example, if the current goal is
    \ecinput{../examps/parts/tactics/assumption/1-1.0.ec}{}{}{}{} then
    running
    \ecinput{../examps/parts/tactics/assumption/1-1.ec}{}{}{}{} solves
    the goal.
  \end{tsyntax}
\end{tactic}

% --------------------------------------------------------------------
\begin{tactic}{reflexivity}
  \begin{tsyntax}[empty]{reflexivity}
  Solve goals of the form \ec{b = b} (up to computation).

  For example, if the current goal is
  \ecinput{../examps/parts/tactics/reflexivity/1-1.0.ec}{}{}{}{} then
  running \ecinput{../examps/parts/tactics/reflexivity/1-1.ec}{}{}{}{}
  solves the goal.
  \end{tsyntax}
\end{tactic}

% --------------------------------------------------------------------
\begin{tactic}{left}
  \begin{tsyntax}[empty]{left}
    Reduce a goal whose conclusion is a disjunction to one whose
    conclusion is its left member.

    For example, if the current goal is
    \ecinput{examps/parts/tactics/left/1-1.0.ec} then
    running \ecinput{examps/parts/tactics/left/1-1.ec}
    produces the goal
    \ecinput{examps/parts/tactics/left/1-1.1.ec}
  \end{tsyntax}
\end{tactic}

% --------------------------------------------------------------------
\begin{tactic}{right}
  \begin{tsyntax}[empty]{right}
  Reduce a disjunctive goal to its right member.

  For example, if the current goal is
  \ecinput{../examps/parts/tactics/right/1-1.0.ec}{}{}{}{} then
  running \ecinput{../examps/parts/tactics/right/1-1.ec}{}{}{}{}
  produces the goal
  \ecinput{../examps/parts/tactics/right/1-1.1.ec}{}{}{}{}
  \end{tsyntax}
\end{tactic}

% --------------------------------------------------------------------
\begin{tactic}[exists $\;e$]{exists}
  \begin{tsyntax}[empty]{exists $\;e$}
    Reduces proving an existential to proving the witness $e$ satisfies
    the existential's body.

    For example, if the current goal is
    \ecinput{../examps/parts/tactics/exists/1-1.0.ec}{}{}{}{} then
    running \ecinput{../examps/parts/tactics/exists/1-1.ec}{}{}{}{}
    produces the goal
    \ecinput{../examps/parts/tactics/exists/1-1.1.ec}{}{}{}{}
  \end{tsyntax}
\end{tactic}

% --------------------------------------------------------------------
\begin{tactic}{split}
  \begin{tsyntax}[empty]{split}
  Break an intrinsically conjunctive goal into its component subgoals.
  For instance, it can:
  \begin{itemize}
  \item close any goal that is convertible to \ec{true} or provable by
    \ec{reflexivity},
    \item replace a logical equivalence by the direct and indirect implications,
    \item replace a goal of the form \ec{$\phi_1\!$ /\\ $\;\,\phi_2$} by the two
      subgoals for $\phi_1$ and $\phi_2$. The same applies for a goal of
      the form \ec{$\phi_1\!$ && $\;\,\phi_2$},
    \item replace an equality between $n$-tuples by $n$ equalities
          on their components.
  \end{itemize}

  For example, if the current goal is
  \ecinput{../examps/parts/tactics/split/1-1.0.ec}{}{}{}{} then
  running \ecinput{../examps/parts/tactics/split/1-1.ec}{}{}{}{}
  produces the goals
  \ecinput{../examps/parts/tactics/split/1-1.1.ec}{}{}{}{}
  and
  \ecinput{../examps/parts/tactics/split/1-1.2.ec}{}{}{}{}
  And if the current goal is
  \ecinput{../examps/parts/tactics/split/2-1.0.ec}{}{}{}{} then
  running \ecinput{../examps/parts/tactics/split/2-1.ec}{}{}{}{}
  produces the goals
  \ecinput{../examps/parts/tactics/split/2-1.1.ec}{}{}{}{}
  and
  \ecinput{../examps/parts/tactics/split/2-1.2.ec}{}{}{}{}
  Repeating the last example with \ec{&&} rather than \ec{/\\},
  if the current goal is
  \ecinput{../examps/parts/tactics/split/2a-1.0.ec}{}{}{}{} then
  running \ecinput{../examps/parts/tactics/split/2a-1.ec}{}{}{}{}
  produces the goals
  \ecinput{../examps/parts/tactics/split/2a-1.1.ec}{}{}{}{}
  and
  \ecinput{../examps/parts/tactics/split/2a-1.2.ec}{}{}{}{}
  This illustrates the difference between \ec{/\\} and \ec{&&}.
  And if the current goal is
  \ecinput{../examps/parts/tactics/split/3-1.0.ec}{}{}{}{} then
  running \ecinput{../examps/parts/tactics/split/3-1.ec}{}{}{}{}
  produces the goals
  \ecinput{../examps/parts/tactics/split/3-1.1.ec}{}{}{}{}
  and
  \ecinput{../examps/parts/tactics/split/3-1.2.ec}{}{}{}{}
  \end{tsyntax}
\end{tactic}

% --------------------------------------------------------------------
\begin{tactic}{congr}
  \begin{tsyntax}[empty]{congr}
  Replace a goal of the form \ec{f t$_1$ ... t$_n$ = f u$_1$ ... u$_n$}
  with the subgoals \ec{t$_i$ = u$_i$} for all \ec{$i$}. Subgoals solvable
  by \ec{reflexivity} are automatically closed.
  \end{tsyntax}
\end{tactic}

% --------------------------------------------------------------------
\begin{tactic}[subst | subst x]{subst}
  \begin{tsyntax}[empty]{subst}
  Search for the first equation of the form \ec{x = f} or \ec{f = x} in the context
  and replace all the occurrences of \ec{x} by \ec{f} everywhere in the context and the
  goal before clearing it. If no identifier is given, repeatedly apply the tactic to
  all identifiers for which such an equation exists.
  \end{tsyntax}
\end{tactic}


% --------------------------------------------------------------------
\begin{tactic}{trivial}
  \begin{tsyntax}[empty]{trivial}
  Try to solve the goal by using a mixture of low-level tactics.
  This tactic is called by the intro-pattern \ec{//}.
  \end{tsyntax}
\end{tactic}

% --------------------------------------------------------------------
\begin{tactic}{done}
  \begin{tsyntax}[empty]{done}
  \fix{Missing description of done}.
  \end{tsyntax}
\end{tactic}

% --------------------------------------------------------------------
\begin{tactic}{simplify}
  \begin{tsyntax}[empty]{simplify}
  \fix{Missing description of simplify}.
  \end{tsyntax}
\end{tactic}

% --------------------------------------------------------------------
\begin{tactic}[progress | progress $\;\tau$]{progress}
  \begin{tsyntax}[empty]{progress}
  Break the goal into multiple \emph{simpler} ones by repeatedly applying
  \ec{split}, \ec{subst} and \ec{move=>}. The tactic $\tau$ given to
  \ec{progress} is tentatively applied after each step.

  For example, if the current goal is
  \ecinput{examps/parts/tactics/progress/1-1.0.ec}{}{}{}{} then
  running \ecinput{examps/parts/tactics/progress/1-1.ec}{}{}{}{}
  solves the goal.
  \end{tsyntax}

  \fix{Describe \ec{progress} options.}
\end{tactic}

% --------------------------------------------------------------------
\begin{tactic}{smt}
  \begin{tsyntax}[empty]{smt}
  \fix{Missing description of smt}.
  \end{tsyntax}
\end{tactic}

% --------------------------------------------------------------------
\begin{tactic}{admit}
  \begin{tsyntax}[empty]{admit}
  Close the current goal by admitting it.

  For example, if the current goal is
  \ecinput{examps/parts/tactics/admit/1-1.0.ec} then
  running \ecinput{examps/parts/tactics/admit/1-1.ec}
  solves the goal.
  \end{tsyntax}
\end{tactic}


% --------------------------------------------------------------------
\begin{tactic}[change $\;\phi$]{change}
  \begin{tsyntax}[empty]{change $\;\phi$}
  Change the current goal for $\phi$ --- $\phi$ must be \emph{convertible}
  to the current goal.

  For example, if the current goal is
  \ecinput{../examps/parts/tactics/change/1-1.0.ec}{}{}{}{} then
  running \ecinput{../examps/parts/tactics/change/1-1.ec}{}{}{}{}
  produces the goal
  \ecinput{../examps/parts/tactics/change/1-1.1.ec}{}{}{}{}
  \end{tsyntax}
\end{tactic}

% --------------------------------------------------------------------
\begin{tactic}[pose x := $\;\pi$]{pose}
  \begin{tsyntax}[empty]{pose}
  Search for the first subterm \ec{p} of the goal matching $\pi$ and
  leading to the full instantiation of the pattern. Then introduce,
  after instantiation, the local definition \ec{x := p} and abstract
  all occurrences of \ec{p} in the goal as \ec{x}. An occurence
  selector can be used (see \rtactic{rewrite}).
  \end{tsyntax}
\end{tactic}

\begin{tactic}[have $\;\iota$: $\;\phi$]{have}
  \begin{tsyntax}[empty]{have}
    Logical cut. Generate two subgoals: one for the cut formula
    $\phi$, and one for \ec{$\phi$ => $\;\psi$} where $\psi$ is the
    current goal's conclusion. Moreover, the introduction pattern
    \ec{$\iota$} is applied to the second subgoal.

  For example, if the current goal is
  \ecinput{../examps/parts/tactics/have/1-1.0.ec}{}{}{}{} then
  running \ecinput{../examps/parts/tactics/have/1-1.ec}{}{}{}{}
  produces the goals
  \ecinput{../examps/parts/tactics/have/1-1.1.ec}{}{}{}{}
  and
  \ecinput{../examps/parts/tactics/have/1-1.2.ec}{}{}{}{}
  \end{tsyntax}

  \begin{tsyntax}{have $\;\iota$: $\;\phi$ by $\;\tau$}
  Attempts to use tactic $\tau$ to close the first subgoal (corresponding to
  the cut formula $\phi$), and fails if impossible.
  \end{tsyntax}
\end{tactic}


% --------------------------------------------------------------------
\begin{tactic}[cut $\;\iota$: $\;\phi$]{cut}
  \begin{tsyntax}[empty]{cut}
  Logical cut. Generate two subgoals: one for the cut formula $\phi$,
  and one for $\phi \Rightarrow G$ where $G$ is the current goal. Moreover,
  the intro-pattern \tct{$\iota$} is applied to the second subgoal.
  \end{tsyntax}
\end{tactic}


% --------------------------------------------------------------------
\begin{tactic}[apply (p : proof-term)]{apply}
  \begin{tsyntax}[empty]{apply (p : proof-term)}
   If \tct{p} is a proof-term for the pattern (formula)
  \begin{center}
    \tct{forall (x1 : t1) ... (xn : tn), A1 -> ... -> An -> B}
  \end{center}
  \noindent then \tct{apply} tries to match B with the current G. If the
  match succeeds and leads to the full instantiation of the pattern,
  then the goal is replaced, after instantiation, with the $n$ subgoals
  \tct{A1, ..., An}.
  \end{tsyntax}
\end{tactic}

% --------------------------------------------------------------------
\begin{tactic}{exact (p : proofterm)}
  \begin{tsyntax}[empty]{exact}
  Equivalent to \ec{by apply (p : proofterm)}, i.e. apply the given
  proof-term and the try to close the goals with \ec{trivial} - failing
  if not all goals can be closed.
  \end{tsyntax}
\end{tactic}

% --------------------------------------------------------------------
\begin{tactic}[rewrite $\;\pi_1 \cdots \pi_n$ | rewrite $\;\pi_1 \cdots \pi_n$ in $\;H$]{rewrite}
  \begin{tsyntax}{rewrite $\;\pi_1 \cdots \pi_n$}
  Rewrite the rewrite-pattern $\pi_1 \cdots \pi_n$ from left to right,
  where the $\pi_i$ can be of the following form:
  \begin{itemize}
  \item one of \ec{//}, \ec{/=}, \ec{//=},
  \item a proof-term $p$, or
  \item a pattern prefixed by \ec{/} (slash).
  \end{itemize}
  The two last forms can be prefixed by a direction indicator (the
  sign \ec{-}, see Subsection~\ref{subsec:occsels}), followed by an
  occurrence selector (see Subsection~\ref{subsec:occsels}), followed
  (for proof-terms only) by a repetition marker (\ec{!}, \ec{?},
  \ec{n!} or \ec{n?}). All these prefixes are optional.

  \smallskip
  Depending on the form of $\pi$, \ec{rewrite $\;\pi$} does the following:
    \begin{itemize}
    \item For \ec{//}, \ec{/=}, and \ec{//=}, see
      Subsection~\ref{subsec:introgen}.

    \item If $\pi$ is a proof-term with conclusion $f_1=f_2$, then
      \ec{rewrite} searches for the first subterm of the goal's
      conclusion matching $f_1$ and resulting in the full
      instantiation of the pattern.  It then replaces, after
      instantiation of the pattern, all the occurrences of $f_1$ by
      $f_2$ in the goal's conclusion, and creates new subgoals for the
      instantiations of the assuptions of $p$.  If no subterms of the
      goal's conclusion match $f_1$ or if the pattern cannot be fully
      instantiated by matching, the tactic fails.  The tactic works
      the same if the pattern ends by \ec{$f_1\!$ <=> $\;f_2$}. If the
      direction indicator \ec{-} is given, \ec{rewrite} works in the
      reverse direction, searching for a match of $f_2$ and then
      replacing all occurrences of $f_2$ by $f_1$.

    \item If $\pi$ is a \ec{/}-prefixed pattern of the form
      $o\,p_1\,\cdots\,p_n$, with $o$ a defined symbol, then
      \ec{rewrite} searches for the first subterm of the goal's
      conclusion matching $o\,p_1\,\cdots\,p_n$ and resulting in the
      full instantiation of the pattern. It then replaces, after
      instantiation of the pattern, all the occurrences of
      $o\,p_1\,\cdots\,p_n$ by the $\beta\delta$ head-normal form of
      $o\,p_1\,\cdots\,p_n$, where the $\delta$-reduction is
      restricted to subterms headed by the symbol $o$. If no subterms
      of the goal's conclusion match $o\,p_1\,\cdots\,p_n$ or if the
      pattern cannot be fully instantiated by matching, the tactic
      fails. If the direction indicator \ec{-} is given, \ec{rewrite}
      works in the reverse direction, searching for a match of the
      $\beta\delta_o$ head-normal of $o\,p_1\,\cdots\,p_n$ and then
      replacing all occurrences of this head-normal form with
      $o\,p_1\,\cdots\,p_n$.
    \end{itemize}
    
    \smallskip
    The occurrence selector restricts which occurrences of the
    matching pattern are replaced in the goal's conclusion---see
    Subsection~\ref{subsec:occsels}.

    Repetition markers allow the repetition of the same rewriting. For
    instance, \ec{rewrite $\;\pi$} leads to \ec{do! rewrite
      $\;\pi$}. See the tactical \ec{do} for more information.
    
    Lastly, \ec{rewrite $\;\pi_1 \cdots \pi_n$} is equivalent to
    \ec{rewrite} $\;\pi_1$; ...; \ec{rewrite} $\;\pi_n$.

    \smallskip
    For example, if the current goal is
    \ecinput{examps/parts/tactics/rewrite/1-1.0.ec}{}{}{}{} then
    running \ecinput{examps/parts/tactics/rewrite/1-1.ec}{}{}{}{}
    produces \ecinput{examps/parts/tactics/rewrite/1-1.1.ec}{}{}{}{}
    from which
    running \ecinput{examps/parts/tactics/rewrite/1-2.ec}{}{}{}{}
    produces \ecinput{examps/parts/tactics/rewrite/1-2.1.ec}{}{}{}{}
    from which
    running \ecinput{examps/parts/tactics/rewrite/1-3.ec}{}{}{}{}
    produces \ecinput{examps/parts/tactics/rewrite/1-3.1.ec}{}{}{}{}
  \end{tsyntax}

  \begin{tsyntax}{rewrite $\;\pi_1 \cdots \pi_n$ in $\;H$} Like the
    preceding case, except rewriting is done in the hypothesis
    $H$ instead of in the goal's conclusion.  Rewriting using a proof
    term is only allowed when the proof term was defined globally
    or before the assumption $H$.

    For example, if the current goal is
    \ecinput{examps/parts/tactics/rewrite/2-1.0.ec}{}{}{}{} then
    running \ecinput{examps/parts/tactics/rewrite/2-1.ec}{}{}{}{}
    produces \ecinput{examps/parts/tactics/rewrite/2-1.1.ec}{}{}{}{}
  \end{tsyntax}
\end{tactic}


% --------------------------------------------------------------------
\begin{tactic}[case $\;\phi$ | case]{case}
  \begin{tsyntax}{case $\;\phi$}
    Do an excluded-middle case analysis on $\phi$, substituting $\phi$
    in the goal's conclusion.

    For example, if the current goal is
    \ecinput{examps/parts/tactics/case/1-1.0.ec}{}{}{}{} then
    running \ecinput{examps/parts/tactics/case/1-1.ec}{}{}{}{}
    produces the goals
    \ecinput{examps/parts/tactics/case/1-1.1.ec}{}{}{}{}
    and
    \ecinput{examps/parts/tactics/case/1-1.2.ec}{}{}{}{}
  \end{tsyntax}

  \begin{tsyntax}{case}
    Destruct the top assumption of the goal's conclusion, generating
    subgoals that are dependent upon the kind of assumption
    destructed. \emph{This form of the tactic can be followed by
    the generalization tactical---see Subsection~\ref{subsec:introgen}.}

    \begin{itemize}
    \item (\textbf{conjunction})
    For example, if the current goal is
    \ecinput{examps/parts/tactics/case/2-1.0.ec}{}{}{}{} then
    running \ecinput{examps/parts/tactics/case/2-1.ec}{}{}{}{}
    produces the goal
    \ecinput{examps/parts/tactics/case/2-1.1.ec}{}{}{}{}
    \ec{&&} works identically.

    \item (\textbf{disjunction})
    For example, if the current goal is
    \ecinput{examps/parts/tactics/case/3-1.0.ec}{}{}{}{} then
    running \ecinput{examps/parts/tactics/case/3-1.ec}{}{}{}{}
    produces the goals
    \ecinput{examps/parts/tactics/case/3-1.1.ec}{}{}{}{}
    and
    \ecinput{examps/parts/tactics/case/3-1.2.ec}{}{}{}{}
    \ec{||} works identically.

    \item (\textbf{existential})
    For example, if the current goal is
    \ecinput{examps/parts/tactics/case/4-1.0.ec}{}{}{}{} then
    running \ecinput{examps/parts/tactics/case/4-1.ec}{}{}{}{}
    produces the goal
    \ecinput{examps/parts/tactics/case/4-1.1.ec}{}{}{}{}

    \item (\textbf{unit}) Substitutes \ec{tt} for the assumption in
      the remainder of the conclusion.

    \item (\textbf{bool})
    For example, if the current goal is
    \ecinput{examps/parts/tactics/case/5-1.0.ec}{}{}{}{} then
    running \ecinput{examps/parts/tactics/case/5-1.ec}{}{}{}{}
    produces the goals
    \ecinput{examps/parts/tactics/case/5-1.1.ec}{}{}{}{}
    and
    \ecinput{examps/parts/tactics/case/5-1.2.ec}{}{}{}{}

    \item (\textbf{product type})
    For example, if the current goal is
    \ecinput{examps/parts/tactics/case/6-1.0.ec}{}{}{}{} then
    running \ecinput{examps/parts/tactics/case/6-1.ec}{}{}{}{}
    produces the goal
    \ecinput{examps/parts/tactics/case/6-1.1.ec}{}{}{}{}

    \item (\textbf{inductive datatype})
    Consider the inductive datatype declaration:
\begin{easycrypt}{}{}
type tree = [Leaf | Node of bool & tree & tree].
\end{easycrypt}
    Then, if the current goal is
    \ecinput{examps/parts/tactics/case/7-1.0.ec}{}{}{}{} then
    running \ecinput{examps/parts/tactics/case/7-1.ec}{}{}{}{}
    produces the goals
    \ecinput{examps/parts/tactics/case/7-1.1.ec}{}{}{}{}
    and
    \ecinput{examps/parts/tactics/case/7-1.2.ec}{}{}{}{}
    \end{itemize}
  \end{tsyntax}
\end{tactic}

% --------------------------------------------------------------------
\begin{tactic}[elim | elim /$L$]{elim}
  \begin{tsyntax}{elim}
    Eliminates the top assumption of the goal's conclusion, generating
    subgoals that are dependent upon the kind of assumption
    eliminated. \emph{This tactic can be followed by
    the generalization tactical---see Subsection~\ref{subsec:introgen}.}

    \ec{elim} mostly works identically to \rtactic{case}, the exception
    being inductive datatype and the integers (for which a built-in
    induction principle is applied---see the other form).

    Consider the inductive datatype declaration:
\begin{easycrypt}{}{}
type tree = [Leaf | Node of bool & tree & tree].
\end{easycrypt}
    Then, if the current goal is
    \ecinput{examps/parts/tactics/elim/1-1.0.ec}
    running \ecinput{examps/parts/tactics/elim/1-1.ec}
    produces the goals
    \ecinput{examps/parts/tactics/elim/1-1.1.ec}
    and
    \ecinput{examps/parts/tactics/elim/1-1.2.ec}
  \end{tsyntax}

  \begin{tsyntax}{elim /$L$}
    Eliminates the top assumption of the goal's conclusion using the
    supplied induction principle lemma. \emph{This tactic can be
      followed by the generalization tactical---see
      Subsection~\ref{subsec:introgen}.}
    For example, consider the declarations
\begin{easycrypt}{}{}
type tree = [Leaf | Node of bool & tree & tree].
op rev (tr : tree) : tree =
  with tr = Leaf => Leaf
  with tr = Node b tr1 tr2 => Node b (rev tr1) (rev tr2).
\end{easycrypt}
and suppose we've already proved
\begin{easycrypt}{}{}
lemma IndPrin :
  forall (p : tree -> bool) (tr : tree),
  p Leaf =>
  (forall (b : bool) (tr1 tr2 : tree),
   p tr1 => p tr2 => p(Node b tr1 tr2)) =>
  p tr.
\end{easycrypt}
    Then, if the current goal is
    \ecinput{examps/parts/tactics/elim/2-1.0.ec}
    running \ecinput{examps/parts/tactics/elim/2-1.ec}
    produces the goals
    \ecinput{examps/parts/tactics/elim/2-1.1.ec}
    and
    \ecinput{examps/parts/tactics/elim/2-1.2.ec}
  \end{tsyntax}

\smallskip
When we consider the \ec{Int} theory in Chapter~\ref{chap:library},
we'll discuss the induction principle on the integers.
\end{tactic}


% --------------------------------------------------------------------
\begin{tactic}{algebra}
\end{tactic}


\subsection{Tacticals}
\label{subsec:tacticals}

Tactics can be combined together, composed and modified by
\emph{tacticals}. We've already seen the introduction and generalization
tacticals, which turn a tactic $\tau$ and a list of patterns into
a composite tactic, which may then combined with other tactics.

\begin{tactic}[$\tau_1$; $\;\tau_2$]{sequence}
  \begin{tsyntax}[empty]{t1; t2}
  Apply $\tau_2$ to all the subgoals generated by $\tau_1$.
  Sequencing groups to the left, so that
  \ec{$\tau_1$; $\;\tau_2$; $\;\tau_3$} means
  \ec{($\tau_1$; $\;\tau_2$); $\;\tau_3$}.
  
  For example, if the current goal is
  \ecinput{examps/parts/tactics/sequence-tactical/1-1.0.ec}{}{}{}{} then
  running \ecinput{examps/parts/tactics/sequence-tactical/1-1.ec}{}{}{}{}
  produces the goals
  \ecinput{examps/parts/tactics/sequence-tactical/1-1.1.ec}{}{}{}{}
\end{tsyntax}
\end{tactic}

\begin{tactic}[$\tau_1$; \[$\;\tau'_1$ | $\;\cdots$ | $\;\tau'_n$\]]{sequence with branching}
  \begin{tsyntax}[empty]{}
    Run $\tau_1$, which must generate exactly $n$ subgoals, $G_1,\ldots,G_n$.
    Then apply $\tau'_i$ to $G_i$, for all $i$.

  For example, if the current goal is
  \ecinput{examps/parts/tactics/sequence-branching-tactical/1-1.0.ec}{}{}{}{} then
  running \ecinput{examps/parts/tactics/sequence-branching-tactical/1-1.ec}{}{}{}{}
  solves the goal.
  \end{tsyntax}
\end{tactic}

\begin{tactic}[try $\;\tau$]{failure recovery}\label{tactic-try}
  \begin{tsyntax}[empty]{try t}
  Execute the tactic $\tau$ if it succeeds; do nothing (leave the
  goal unchanged) if it fails.

  \paragraph{Remark.}
  By default, \EasyCrypt proofs are run in \ec{strict} mode. In this
  mode, \ec{smt} failures cannot be caught using \ec{try}. This allows
  \EasyCrypt to always build the proof tree correctly, even in weak
  check mode, where \ec{smt} calls are assumed to succeed. Inside a
  strict proof, weak check mode can be turned on and off at will,
  allowing for the fast replay of proof sections during
  development. In any event, we recommend \emph{never} using \ec{try
    smt}: a little thought is much more cost-effective than failing
  \ec{smt} calls.
  \end{tsyntax}
\end{tactic}

\begin{tactic}[do! $\;\tau$]{tactic repetition}
  \begin{tsyntax}[empty]{do! t}
    Apply $\tau$ to the current goal, then repeatedly apply it to all
    subgoals, stopping on a branch only when it fails. An error is
    produced it $\tau$ does not apply to the current goal.
  \end{tsyntax}

  For example, if the current goal is
  \ecinput{examps/parts/tactics/do-tactical/1-1.0.ec}{}{}{}{} then
  running \ecinput{examps/parts/tactics/do-tactical/1-1.ec}{}{}{}{}
  produces the goals
  \ecinput{examps/parts/tactics/do-tactical/1-1.1.ec}{}{}{}{}
  and
  \ecinput{examps/parts/tactics/do-tactical/1-1.2.ec}{}{}{}{}

  \paragraph{Variants.}\strut

  \begin{tabularx}{\textwidth}{@{}ll@{}}
  {\ec{do? $\;\tau$}} & apply $\tau$ 0 or more times, until it fails\\
  {\ec{do $\;n$! $\;\tau$}} & apply $\tau$ with depth exactly $n$\\
  {\ec{do $\;n$? $\;\tau$}} & apply $\tau$ with depth at most $n$
  \end{tabularx}
\end{tactic}

\begin{tactic}[$\tau$; first $\;\tau_2$]{goal selection}
  \begin{tsyntax}[empty]{t1; first t2}
    Apply the tactic $\tau_1$, then apply $\tau_2$ on the first
    subgoal generated by \ec{t1}, leaving the other goals unchanged.
    An error is produced if no subgoals are generated by $\tau_1$.

  \paragraph{Variants.}\strut

  \noindent\begin{tabularx}{\textwidth}{@{}ll@{}}
    {\ec{$\tau_1$; first $\;n$ $\;\tau_2$}} & apply $\tau_2$ on the
    first $n$ subgoals generated by $\tau_1$\\[.4cm]
    {\ec{$\tau_1$; last $\;\tau_2$}} & apply $\tau_2$ on the last subgoal
    generated by $\tau_1$\\[.4cm]
    {\ec{$\tau_1$; last $\;n$ $\;\tau_2$}} & apply $\tau_2$ on the last $n$
    subgoals generated by $\tau_1$\\[.4cm]
    {\ec{$\tau$; first $\;n\!$ last}} & \parbox{250pt}{reorder the subgoals
      generated by $\tau$, moving the first $n$ to the end of the
      list} \\[.4cm]
    {\ec{$\tau$; last $\;n\!$ first}} & \parbox{250pt}{reorder the subgoals
      generated by $\tau$, moving the last $n$ to the beginning of the
      list} \\[.4cm]
    {\ec{$\tau$; last first}} & \parbox{250pt}{reorder the subgoals generated
    by $\tau$, moving the last one to the beginning of the list}\\[.4cm]
    {\ec{$\tau$; first last}} & \parbox{250pt}{reorder the subgoals
     generated by $\tau$, moving the first one to the end of the list}
  \end{tabularx}

  \bigskip
  For example, if the current goal is
  \ecinput{examps/parts/tactics/sequence-reordering-tactical/1-1.0.ec}{}{}{}{} then
  running \ecinput{examps/parts/tactics/sequence-reordering-tactical/1-1.ec}{}{}{}{}
  produces the goals
  \ecinput{examps/parts/tactics/sequence-reordering-tactical/1-1.1.ec}{}{}{}{}
  and
  \ecinput{examps/parts/tactics/sequence-reordering-tactical/1-1.2.ec}{}{}{}{}
  \end{tsyntax}
\end{tactic}

\begin{tactic}[by $\;\tau$]{closing goals}
  \begin{tsyntax}[empty]{by t}
  Apply the tactic $\tau$ and try to close all the generated subgoals using
  \rtactic{trivial}. Fail if not all subgoals can be closed.
  \end{tsyntax}

  \paragraph{Remark.} Inside the a lemma's proof, \ec{by []} is
  equivalent to \ec{by trivial}.  But the form
\begin{easycrypt}{}{}
lemma #$\;\cdots$# by [].
\end{easycrypt}
  means
\begin{easycrypt}{}{}
lemma #$\;\cdots$# by (trivial; smt).
\end{easycrypt}
\end{tactic}


% Program Logic Section
\section{Program Logics}
\label{sec:programlogics}

In this section, we describe the tactics of \EasyCrypt's three program
logics: \prhl, \phl and \hl.  There are five rough classes of program
logic tactics:
\begin{enumerate}
\item those that actually reason about the program in Hoare logic
  style;

\item those that correspond to semantics-preserving program
  transformations or compiler optimizations;

\item those that operate at the level of specifications,
  strenghtening, combining or splitting goals without modifying the
  program;

\item tactics that automate the application of other tactics;

\item advanced tactics for handling eager/lazy sampling and bounding
  the probability of failure.
\end{enumerate}
We discuss these five classes in turn.

\subsection{Tactics for Reasoning about Programs}
\label{subsec:reasoningprograms}

Some of the program reasoning tactics discussed in this subsection
have two modes when used on goals whose conclusions are \prhl
statement judgements.  Their default mode is to operate on both
programs at once. When a side is specified (using \ec{$\tau$\{1\}} or
\ec{$\tau$\{2\}}), a one-sided variant is used, with 1 referring to the
left program, and 2 to the right one.

\medskip

% --------------------------------------------------------------------
\begin{tactic}{proc}
  \begin{tsyntax}{proc}
    Turn a goal whose conclusion is a \prhl (resp., \phl) judgement
    involving concrete procedures (resp., a concrete procedure)
    into one whose conclusion is a \prhl (resp., \phl) statement
    judgement by replacing the \emph{concrete} procedures (resp.,
    procedure) by their (resp., its) bodies (resp., body).

  \bigskip
  For example, if the current goal is
  \ecinput{../examps/parts/tactics/proc/1-1.0.ec}{}{}{}{} then
  running \ecinput{../examps/parts/tactics/proc/1-1.ec}{}{}{}{}
  produces the goal
  \ecinput{../examps/parts/tactics/proc/1-1.1.ec}{}{}{}{}
  \end{tsyntax}

  \begin{tsyntax}{proc $\;I$}
    Reduce a goal whose conclusion is a \prhl (resp., \phl) judgement
    involving an abstract procedure to \prhl (resp., \phl) judgements
    on the oracles the procedure may query.

  \bigskip
  For example, given the declarations
  \ecinput{examps/tactics/proc/2.ec}{}{3-20}{}
  if the current goal is
  \ecinput{../examps/parts/tactics/proc/2-1.0.ec}{}{}{}{} then
  running \ecinput{../examps/parts/tactics/proc/2-1.ec}{}{}{}{}
  produces the goals
  \ecinput{../examps/parts/tactics/proc/2-1.1.ec}{}{}{}{}
  and
  \ecinput{../examps/parts/tactics/proc/2-1.2.ec}{}{}{}{}
  and
  \ecinput{../examps/parts/tactics/proc/2-1.3.ec}{}{}{}{}
  and
  \ecinput{../examps/parts/tactics/proc/2-1.4.ec}{}{}{}{}
  The tactic would fail without the module restriction \ec{T\{Or\}} on
  \ec{M}, as then \ec{M} could directly manipulate \ec{Or.x}.
  \end{tsyntax}

  \begin{tsyntax}{proc B I}
  Derive a specification for an \emph{abstract} procedure from an
  ``upto-failure'' invariant on the oracles it may query. The failure
  event \ec{B} is evaluated in the right memory. The left oracles
  must be lossless once the bad event occurs. The right oracles must
  guarantee the stability of the failure event with probability 1.
  \end{tsyntax}

  \begin{tsyntax}{proc B I I'}
  Similar to \ec{proc B I}, with an additional invariant once the bad
  event occurs. This is particularly useful when additional facts
  about the state need to be known to prove the losslessness and
  stability conditions.
  \end{tsyntax}

  \begin{tsyntax}{proc*}
  Derive a specification on procedures from a specification on a
  program whose code consists in a call to that procedure. This tactic
  is particularly useful in combination with \rtactic{inline} when
  faced with a \prhl judgment where one of the procedures is concrete
  and the other is abstract.
  \end{tsyntax}
\end{tactic}

% --------------------------------------------------------------------
\begin{tactic}{skip}
\end{tactic}

% --------------------------------------------------------------------
\begin{tactic}{seq}
  \begin{tsyntax}{seq $\;n_1$ $\;n_2$ : $\;R$}
    \textbf{\prhl sequence rule.} If $n_1$ and $n_2$ are natural
    numbers and the goal's conclusion is a \prhl statement judgement
    with precondition $P$, postcondition $Q$ and such that the lengths
    of the first and second programs are at least $n_1$ and $n_2$,
    respectively, then reduce the goal to two subgoals:
    \begin{itemize}
    \item A first goal whose conclusion has precondition $P$,
      postcondition $R$, first program consisting of the first $n_1$
      statements of the original goal's first program, and second
      program consisting of the first $n_2$ statements of the original
      goal's second program.

    \item A second goal whose conclusion has precondition $R$,
      postcondition $Q$, first program consisting of all but the first
      $n_1$ statements of the original goal's first program, and
      second program consisting of all but the first $n_2$ statements
      of the original goal's second program.
    \end{itemize}

  \bigskip
  For example, if the current goal is
  \ecinput{../examps/parts/tactics/seq/1-1.0.ec}{}{}{}{} then
  running \ecinput{../examps/parts/tactics/seq/1-1.ec}{}{}{}{}
  produces the goals
  \ecinput{../examps/parts/tactics/seq/1-1.1.ec}{}{}{}{}
  and
  \ecinput{../examps/parts/tactics/seq/1-1.2.ec}{}{}{}{}
  \end{tsyntax}

  \begin{tsyntax}{seq $\;n$ : $\;R$}
  \textbf{\hl sequence rule.} If $n$ is a natural
    number and the goal's conclusion is an \hl statement judgement
    with precondition $P$, postcondition $Q$ and such that the length
    of the program is at least $n$, then reduce the goal to two subgoals:
    \begin{itemize}
    \item A first goal whose conclusion has precondition $P$,
      postcondition $R$, and program consisting of the first $n$
      statements of the original goal's program.

    \item A second goal whose conclusion has precondition $R$,
      postcondition $Q$, and program consisting of all but the first
      $n$ statements of the original goal's program.
    \end{itemize}

  \bigskip
  For example, if the current goal is
  \ecinput{../examps/parts/tactics/seq/2-1.0.ec}{}{}{}{} then
  running \ecinput{../examps/parts/tactics/seq/2-1.ec}{}{}{}{}
  produces the goals
  \ecinput{../examps/parts/tactics/seq/2-1.1.ec}{}{}{}{}
  and
  \ecinput{../examps/parts/tactics/seq/2-1.2.ec}{}{}{}{}
  \end{tsyntax}

%  \begin{tsyntax}{seq $\ n$: R $\ \delta_1\ \delta_2\ \delta_3\ \delta_4$ I}
%  Non-relational probabilistic sequence rule. Argument \ec{I} is
%  optional (and defaults to $\mathsf{true}$). When one of
%  $(\delta_1,\delta_2)$ (resp. $(\delta_3,\delta_4)$) is 0, the other
%  can be replaced with a wildcard \ec{_}, and the corresponding goal
%  is not generated, as it is not relevant to the proof. When none of
%  the $\delta$s are given, the following default values are used:
%  $\delta_1 = 1$, $\delta_2 = \delta$, $\delta_3 = 0$.
%
%  \paragraph{Examples:}\strut
%  
%  \begin{cmathpar}
%  \texample[\phl{}]
%    {\ec{seq $\ \left|c\right|$: R $\ \delta_1$ $\ \delta_2$ $\ \delta_3$ $\ \delta_4$ I}}
%    {\HL{P}{c}{I} \\
%     \pHL{P}{c}{R}{\diamond}{\delta_1}  \\
%     \pHL{R \wedge I}{c'}{Q}{\diamond}{\delta_2} \\
%     \pHL{P}{c}{\neg R}{\diamond}{\delta_3} \\
%     \pHL{\neg R \wedge I}{c'}{Q}{\diamond}{\delta_4} \\
%     \delta_1 \delta_2 + \delta_3 \delta_4 \diamond \delta}
%    {\pHL{P}{c;c'}{Q}{\diamond}{\delta}}
%  \end{cmathpar}
%
%  \begin{cmathpar}
%  \texample[\phl{}]
%    {\ec{seq $\ \left|c\right|$: R $\ \delta_1$ $\ \delta_2\ \_\ 0$}}
%    {\HL{P}{c}{\mathsf{true}} \\
%     \pHL{P}{c}{R}{\diamond}{\delta_1} \\\\
%     \pHL{R \wedge I}{c'}{Q}{\diamond}{\delta_2} \\
%     \pHL{\neg R \wedge I}{c'}{Q}{\diamond}{0} \\
%     \delta_1 \delta_2 \diamond \delta}
%    {\pHL{P}{c;c'}{Q}{\diamond}{\delta}}
%  \end{cmathpar}
%
%  \textbf{Note:} Since most tactics implicitly apply the \rtactic{seq}
%  rule, most \phl tactics take optional final arguments corresponding
%  to the $\delta$s and \ec{I}.
%  \end{tsyntax}
\end{tactic}

% --------------------------------------------------------------------
\begin{tactic}{sp}
\end{tactic}

% --------------------------------------------------------------------
\begin{tactic}{wp}
  \begin{tsyntax}{wp}
    If the goal's conclusion is a \prhl, \phl or \hl statement
    judgement, consume the longest suffix(es) of the conclusion's
    program(s) consisting entirely of statements built-up from
    ordinary assignments (not random assignments or procedure call
    assignments) and \ec{if} statements, replacing the conclusion's
    postcondition by the weakest precondition $R$ such that the
    statement judgement consisting of $R$, the consumed suffix(es)
    and the conclusion's original postcondition holds.

    \bigskip For example, if the current goal is
    \ecinput{examps/parts/tactics/wp/1-1.0.ec} then
    running \ecinput{examps/parts/tactics/wp/1-1.ec}
    produces the goal
    \ecinput{examps/parts/tactics/wp/1-1.1.ec}
  \end{tsyntax}

  \begin{tsyntax}{wp $\;n_1$ $\;n_2$}
    In \prhl, let \ec{wp} consume \emph{exactly} $n_1$ statements of
    the first program and $n_2$ statements of the second
    program. Fails if this isn't possible.
  \end{tsyntax}

  \begin{tsyntax}{wp $\;n$}
    In \phl and \hl, let \ec{wp} consume \emph{exactly} $n$ statements
    of the program. Fails if this isn't possible.
  \end{tsyntax}
\end{tactic}

% --------------------------------------------------------------------
\begin{tactic}{rnd}
  When describing the variants of this tactic, we ony consider random
  assignments whose left-hand sides consist of single identifiers.
  The generalization to multiple assignment, when distributions over
  tuple types are sampled, is straightforward.

  \bigskip
  \begin{tsyntax}{rnd | rnd $\;f$ | rnd $\;f$ $\;g$} If the conclusion
    is a \prhl statement judgement whose programs end with random
    assignments \ec{$x_1\!$ <$\$$ $\;d_1$} and \ec{$x_2\!$ <$\$$
      $\;d_2$}, and $f$ and $g$ are functions between the types of
    $x_1$ and $x_2$, then consume those random assignments, replacing
    the conclusion's postcondition by the probabilistic weakest
    precondition of the random assignments wrt.\ $f$ and $g$.  The new
    postcondition checks that:
    \begin{itemize}
    \item $f$ and $g$ are an isomorphism between the distributions
      $d_1$ and $d_2$;

    \item for all elements $u$ in the support of $d_1$, the result
      of substituting $u$ and $f\,u$ for \ec{$x_1$\{1\}} and
      \ec{$x_2$\{2\}} in the conclusion's original postcondition
      holds.
    \end{itemize}
    When $g$ is $f$, it can be omitted. When $f$ is the identity, it
    can be omitted.

    \bigskip For example, if the current goal is
    \ecinput{examps/parts/tactics/rnd/2-1.0.ec}{}{}{}{} then
    running \ecinput{examps/parts/tactics/rnd/2-1.ec}{}{}{}{}
    produces the goal
    \ecinput{examps/parts/tactics/rnd/2-1.1.ec}{}{}{}{}
    Note that if one uses the other isomorphism between \ec{\{0,1\}} and
    \ec{[2..3]} the generated subgoal will be false.
  \end{tsyntax}

  \begin{tsyntax}{rnd\{1\} | rnd\{2\}}
    If the conclusion is a \prhl statement judgement whose designated
    program (1 or 2) ends with a random assignment 
    \ec{$x\!$ <$\$$ $\;d$}, then consume that random assignment,
    replacing the conclusion's postcondition with a check that:
    \begin{itemize}
    \item the weight of $d$ is $1$ (so the random assignment can't fail);

    \item for all elements $u$ in the support of $d$, the result
      of substituting $u$ for \ec{$x$\{$i$\}}---where $i$ is the
      selected side---in the conclusion's original
      postcondition holds.
    \end{itemize}

    \bigskip For example, if the current goal is
    \ecinput{examps/parts/tactics/rnd/3-1.0.ec}{}{}{}{} then
    running \ecinput{examps/parts/tactics/rnd/3-1.ec}{}{}{}{}
    produces the (false!) goal
    \ecinput{examps/parts/tactics/rnd/3-1.1.ec}{}{}{}{}
  \end{tsyntax}

  \begin{tsyntax}{rnd}
    If the conclusion is an \hl statement judgement whose program ends
    with a random assignment, then consume that random assignment,
    replacing the conclusion's postcondition by the possibilistic
    weakest precondition of the random assignment.

    \bigskip For example, if the current goal is
    \ecinput{examps/parts/tactics/rnd/1-1.0.ec}{}{}{}{} then
    running \ecinput{examps/parts/tactics/rnd/1-1.ec}{}{}{}{}
    produces the goal
    \ecinput{examps/parts/tactics/rnd/1-1.1.ec}{}{}{}{}
  \end{tsyntax}

  \begin{tsyntax}{rnd | rnd $\;E$}
    In \phl, compute the probabilistic weakest precondition of a
    random sampling with respect to event $E$. When $E$ is not
    specified, it is inferred from the current postcondition.
  \end{tsyntax}
\end{tactic}

% --------------------------------------------------------------------
\begin{tactic}{if}
  \begin{tsyntax}[empty]{if}
  \fix{Missing description of if}.
  \end{tsyntax}
\end{tactic}

% --------------------------------------------------------------------
\begin{tactic}{while}
  \begin{tsyntax}{while $\;I$}
    Here $I$ is an \emph{invariant} (formula), which may reference
    variables of the two programs, interpreted in their memories.  If
    the goal's conclusion is a \prhl statement judgement whose
    programs both end with \ec{while} statements, reduce the goal to
    two subgoals whose conclusions are \prhl statement judgements:
    \begin{itemize}
    \item One whose first and second programs are the bodies of the
      first and second while statements, whose precondition is the
      conjunction of $I$ and the while statements' boolean expressions (the
      first of which is interpreted in memory \ec{&1}, and the second
      of which is interpreted in \ec{&2}) and whose postcondition is
      the conjunction of $I$ and the assertion that the while statements'
      boolean expressions (interpreted in the appropriate memories)
      are equivalent.

    \item One whose precondition is the original goal's precondition,
      whose first and second programs are all the results of removing
      the while statements from the two programs, and whose postcondition is
      the conjunction of:
      \begin{itemize}
      \item the conjunction of $I$ and the assertion that the while statements'
        boolean expressions are equivalent; and

      \item the assertion that, for all values of the variables
        \emph{modified} by the while statements, if the while statements'
        boolean expressions don't hold, but $I$ holds, then the
        original goal's postcondition holds (in $I$, the while statements'
        boolean expressions, and the postcondition, variables modified
        by the while statements are replaced by universally quantified
        identifiers; otherwise, the boolean expressions are
        interpreted in the program's respective memories, and the
        memory references of $I$ and the postcondition are
        maintained).
      \end{itemize}
    \end{itemize}

  \medskip
  For example, if the current goal is
  \ecinput{examps/parts/tactics/while/1-1.0.ec}{}{}{}{} then
  running \ecinput{examps/parts/tactics/while/1-1.ec}{}{}{}{}
  produces the goals
  \ecinput{examps/parts/tactics/while/1-1.1.ec}{}{}{}{}
  and
  \ecinput{examps/parts/tactics/while/1-1.2.ec}{}{}{}{}
  \end{tsyntax}

  \begin{tsyntax}{while\{1\} $\;I$ $\;v$ | while\{2\} $\;I$ $\;v$}
    Here $I$ is an \emph{invariant} (formula) and $v$ is a
    \emph{termination variant} integer expression, both of which may reference
    variables of the two programs, interpreted in their memories.  If
    the goal's conclusion is a \prhl statement judgement whose
    designated program (1 or 2) ends with a \ec{while} statement,
    reduce the goal to two subgoals;
    \begin{itemize}
    \item One whose conclusion is a \phl statement judgement, saying that
      running the body of the while statement in a memory in which
      $I$ holds and the while statement's boolean expression is true
      is guaranteed to result in termination in a memory in which
      $I$ holds and in which the value of the variant expression $v$
      has decreased by at least $1$. (More precisely, the \phl statement
      judgment is universally quantified by the memory of the non-designated
      program and the initial value of $v$. References to the variables
      of the nondesignated program in $I$ and $v$ are interpreted in this
      memory; reference to the variables of the designed program have
      their memory references removed.)

    \item One whose conclusion is a \prhl statement judgement whose
      precondition is the original goal's precondition, whose
      designated program is the result of removing the while statement
      from the original designated program, whose other program is
      unchanged, and whose postcondition is the conjunction of $I$ and
      the assertion that, for all values of the variables modified by
      the while statement, that the conjunction of the following
      formulas holds:
      \begin{itemize}
      \item the assertion that, if $I$ holds, but the variant
        expression $v$ is not positive, then the while statement's
        boolean expression is false;

      \item the assertion that, if the while statement's boolean expression
        doesn't hold, but $I$ holds, then the original goal's postcondition
        holds.
      \end{itemize}
    \end{itemize}

  \bigskip
  For example, if the current goal is
  \ecinput{examps/parts/tactics/while/3-1.0.ec}{}{}{}{} then
  running \ecinput{examps/parts/tactics/while/3-1.ec}{}{}{}{}
  produces the goals
  \ecinput{examps/parts/tactics/while/3-1.1.ec}{}{}{}{}
  and
  \ecinput{examps/parts/tactics/while/3-1.2.ec}{}{}{}{}
  \end{tsyntax}

  \begin{tsyntax}{while $\;I$}
    Here $I$ is an \emph{invariant} (formula), which may reference
    variables of the program.  If the goal's conclusion is an \hl
    statement judgement ending with a \ec{while} statement, reduce the
    goal to two subgoals whose conclusions are \hl statement
    judgements:
    \begin{itemize}
    \item One whose program is the body of the while statement, whose
      precondition is the conjunction of $I$ and the while statement's
      boolean expression, and whose postcondition is $I$.

    \item One whose precondition is the original goal's precondition,
      whose program is the result of removing the while statement from
      the original program, and whose postcondition is the conjunction
      of:
      \begin{itemize}
      \item $I$; and

      \item the assertion that, for all values of the variables
        \emph{modified} by the while statement, if the while statement's boolean
        expression doesn't hold, but $I$ holds, then the original
        goal's postcondition holds (in $I$, the while statement's boolean
        expression, and the postcondition, variables modified by the
        while statement are replaced by universally quantified
        identifiers).
      \end{itemize}
    \end{itemize}

  \bigskip
  For example, if the current goal is
  \ecinput{examps/parts/tactics/while/2-1.0.ec}{}{}{}{} then
  running \ecinput{examps/parts/tactics/while/2-1.ec}{}{}{}{}
  produces the goals
  \ecinput{examps/parts/tactics/while/2-1.1.ec}{}{}{}{}
  and
  \ecinput{examps/parts/tactics/while/2-1.2.ec}{}{}{}{}
  \end{tsyntax}

  \begin{tsyntax}{while $\;I$ $\;v$}
  \phl version...

%  Where \ec{v} is an integer-valued expression. In \phl, performs a
%  weakest precondition computation over a loop, using \ec{I} as
%  invariant and \ec{v} as a decreasing variant to prove
%  termination. In addition to the two invariant-related subgoals (see
%  above), two subgoals regarding the variant are generated; one
%  requiring that the variant be less than 0 exactly when the loop
%  condition is false, and the other requiring that the variant
%  decreases strictly.
  \end{tsyntax}
\end{tactic}

% --------------------------------------------------------------------
\begin{tactic}{call}
  \begin{tsyntax}{call (_ : $\;P$ ==> $\;Q$)}
    If the goal's conclusion is a \prhl or \hl statement judgement
    whose program(s) end(s) with (a) procedure call(s) or (a) procedure
    call assignment(s), then generate two subgoals:
  \begin{itemize}
  \item One whose conclusion is a judgement of the same kind whose
    precondition is $P$, whose procedure(s) are/is the procedure(s)
    being called, and whose postcondition is $Q$.

  \item One whose conclusion is a statement judgement of the same
    kind whose precondition is the original goal's precondition,
    whose program(s) are/is the result of removing the procedure
    call(s) from the program(s), and whose postcondition is the
    conjunction of
    \begin{itemize}
    \item the result of replacing the procedure's/procedures'
      parameter(s) by their actual argument(s) in $P$; and

    \item the assertion that, for all values of the global variable(s)
      modified by the procedure(s) and the result(s) of the procedure
      call(s), if $Q$ holds (where these quantified identifiers have
      been substituted for the modified variables and procedure
      results), then the original goal's postcondition holds (where
      the modified global variables and occurrences of the variable(s)
      (if any) to which the result(s) of the call(s) of the
      procedure(s) are/is assigned have been replaced by the
      appropriate quantified identifiers).
    \end{itemize}
  \end{itemize}

  \medskip
  For example, if the current goal is
  \ecinput{../examps/parts/tactics/call/1-1.0.ec}{}{}{}{}
  and the procedures \ec{M.f} and \ec{N.f} have a single parameter,
  \ec{y}, then running
  \ecinput{../examps/parts/tactics/call/1-1.ec}{}{}{}{} produces the
  goals \ecinput{../examps/parts/tactics/call/1-1.1.ec}{}{}{}{} and
  \ecinput{../examps/parts/tactics/call/1-1.2.ec}{}{}{}{}

  \bigskip
  Alternatively, a proof term whose conclusion is a \prhl or
  \hl judgement involving the procedure(s) called at the
  end(s) of the program(s) may be supplied as the argument to
  \ec{call}, in which case only the second subgoal need be
  generated.

  \medskip
  For example, in the start-goal of the preceding example,
  if the lemma \ec{M_N_f} is
  \ecinput{examps/tactics/call/1.ec}{}{31-33}{}
  then running
  \ecinput{../examps/parts/tactics/call/1-2.ec}{}{}{}{} produces the
  goal \ecinput{../examps/parts/tactics/call/1-2.1.ec}{}{}{}{}
  \end{tsyntax}

  \begin{tsyntax}{call\{1\} (_ : $\;P$ ==> $\;Q$) | call\{2\} (_ : $\;P$ ==> $\;Q$)}
    If the goal's conclusion is a \prhl statement judgement whose
    designated program ends with a procedure call, then generate two
    subgoals:
  \begin{itemize}
  \item One whose conclusion is a \phl judgement whose precondition is
    $P$, whose procedure is the procedure being called, whose
    postcondition is $Q$, and whose bound part specifies equality with
    probability 1.
    (Consequently, $P$ and $Q$ may not mention \ec{&1} and
    \ec{&2}.)

  \item One whose conclusion is a \prhl statement judgement whose
    precondition is the original goal's precondition, whose programs
    are the result of removing the procedure call from the designated
    program, and leaving the other program unchanged, and whose
    postcondition is the conjunction of
    \begin{itemize}
    \item the result of replacing the procedure's
      parameter(s) by their actual argument(s) in $P$; and

    \item the assertion that, for all values of the global variable(s)
      modified by the procedure and the result of the procedure call,
      if $Q$ holds (where these quantified identifiers have been
      substituted for the modified variables and procedure result),
      then the original goal's postcondition holds (where the modified
      global variables and occurrences the variable (if any) to which
      the result of the procedure call is assigned have been replaced
      by the appropriate quantified identifiers).
    \end{itemize}
  \end{itemize}

  For example, if the current goal is
  \ecinput{../examps/parts/tactics/call/2-1.0.ec}{}{}{}{}
  then running
  \ecinput{../examps/parts/tactics/call/2-1.ec}{}{}{}{} produces the
  goals \ecinput{../examps/parts/tactics/call/2-1.1.ec}{}{}{}{} and
  \ecinput{../examps/parts/tactics/call/2-1.2.ec}{}{}{}{}

  \bigskip Alternatively, a proof term whose conclusion is a \phl
  judgement specifying equality with probability 1 and involving the
  procedure called at the end of the designated program may be
  supplied as the argument to \ec{call}, in which case only the second
  subgoal need be generated.

  \medskip
  For example, in the start-goal of the preceding example,
  if the lemma \ec{M_f} is
  \ecinput{examps/tactics/call/2.ec}{}{23-24}{}
  then running
  \ecinput{../examps/parts/tactics/call/2-2.ec}{}{}{}{} produces the
  goal \ecinput{../examps/parts/tactics/call/2-2.1.ec}{}{}{}{}
  \end{tsyntax}

  \begin{tsyntax}{call (_ : $\;I$)}
    If the conclusion is a \prhl or \hl statement judgement whose
    program(s) end(s) with (a) call(s) of (a) \emph{concrete}
    procedure(s), then use the specification argument to \ec{call}
    generated from the \emph{invariant} $I$, and automatically apply
    \ec{proc} to its first subgoal.  In the \prhl case, its
    precondition will assume equality of the procedures' parameters,
    and its postcondition will assert equality of the results of the
    procedure calls.

    \medskip
    For example, if the current goal is
    \ecinput{../examps/parts/tactics/call/1-3.0.ec}{}{}{}{}
    and modules \ec{M} and \ec{N} contain
    \ecinput{examps/tactics/call/1.ec}{}{4-8}{} and
    \ecinput{examps/tactics/call/1.ec}{}{18-22}{}
    respectively, then
    running \ecinput{../examps/parts/tactics/call/1-3.ec}{}{}{}{}
    produces the goals
    \ecinput{../examps/parts/tactics/call/1-3.1.ec}{}{}{}{} and
    \ecinput{../examps/parts/tactics/call/1-3.2.ec}{}{}{}{}
  \end{tsyntax}

  \begin{tsyntax}{call (_ : $\;I$)}
    If the conclusion is a \prhl or \hl statement judgement whose
    program(s) end(s) with (a) call(s) of the same \emph{abstract}
    procedure, then use the specification argument to \ec{call}
    generated from the \emph{invariant} $I$, and automatically apply
    \ec{proc $\;I$} to its first subgoal, pruning the first two
    subgoals the application generates, because their conclusions
    consist of ambient logic formulas that are true by construction,
    and pruning the next goal (showing the losslessness of the abstract
    procedure given the losslessness of the abstract oracles it uses), if
    trivial suffices to solve it.
    In the \prhl case, its precondition will assume equality of the
    procedure's parameters and of the global variables of the module
    containing the procedure, and its postcondition will assume
    equality of the results of the procedure calls and of the global
    variables of the containing module.

    \medskip
    For example, given the declarations
    \ecinput{examps/tactics/call/3.ec}{}{3-24}{}
    if the current goal is
    \ecinput{../examps/parts/tactics/call/3-1.0.ec}{}{}{}{}
    then running
    \ecinput{../examps/parts/tactics/call/3-1.ec}{}{}{}{}
    produces the goals
    \ecinput{../examps/parts/tactics/call/3-1.1.ec}{}{}{}{} and
    \ecinput{../examps/parts/tactics/call/3-1.2.ec}{}{}{}{}
  \end{tsyntax}

  \begin{tsyntax}{call (_ : $\;B$, $\;I$)}
    If the conclusion is a \prhl statement judgement whose programs
    end with calls of the same \emph{abstract} procedure, then use the
    specification argument to \ec{call} generated from the \emph{bad
      event} $B$ and \emph{invariant} $I$, and automatically apply
    \ec{proc $\;B$ $\;I$} to its first subgoal, pruning the first two
    subgoals the application generates, because their conclusions
    consist of ambient logic formulas that are true by construction,
    and pruning the next goal (showing the losslessness of the abstract
    procedure given the losslessness of the abstract oracles it uses), if
    trivial suffices to solve it.
    The specification's precondition will assume equality of the
    procedure's parameters and of the global variables of the module
    containing the procedure as well as $I$, and its postcondition
    will assert $I$ and the equality of the results of the procedure
    calls and of the global variables of the containing module---but
    only when $B$ does not hold.

    \medskip
    For example, given the declarations
    \ecinput{examps/tactics/call/4.ec}{}{3-51}{}
    if the current goal is
    \ecinput{../examps/parts/tactics/call/4-1.0.ec}{}{}{}{}
    then running
    \ecinput{../examps/parts/tactics/call/4-1.ec}{}{}{}{}
    produces the goals
    \ecinput{../examps/parts/tactics/call/4-1.1.ec}{}{}{}{} and
    \ecinput{../examps/parts/tactics/call/4-1.2.ec}{}{}{}{} and
    \ecinput{../examps/parts/tactics/call/4-1.3.ec}{}{}{}{} and
    \ecinput{../examps/parts/tactics/call/4-1.4.ec}{}{}{}{}
  \end{tsyntax}

  \begin{tsyntax}{call (_ : $\;B$, $\;I$, $\;J$)}
    If the conclusion is a \prhl statement judgement whose programs
    end with calls of the same \emph{abstract} procedure, then use the
    specification argument to \ec{call} generated from the \emph{bad
      event} $B$ and \emph{invariants} $I$ and $J$, and automatically
    apply \ec{proc $\;B$ $\;I$ $\;J$} to its first subgoal, pruning the
    first two subgoals the application generates, because their
    conclusions consist of ambient logic formulas that are true by
    construction.  The specification's precondition will assume
    equality of the procedure's parameters and of the global variables
    of the module containing the procedure as well as $I$, and its
    postcondition will assert
    \begin{itemize}
    \item $I$ and the equality of the results of the procedure calls
      and of the global variables of the containing module---if $B$
      does not hold; and

    \item $J$---if $B$ does hold.
    \end{itemize}

    \medskip
    For example, given the declarations of the preceding example
    if the current goal is
    \ecinput{../examps/parts/tactics/call/4-2.0.ec}{}{}{}{}
    then running
    \ecinput{../examps/parts/tactics/call/4-2.ec}{}{}{}{}
    produces the goals
    \ecinput{../examps/parts/tactics/call/4-2.1.ec}{}{}{}{} and
    \ecinput{../examps/parts/tactics/call/4-2.2.ec}{}{}{}{} and
    \ecinput{../examps/parts/tactics/call/4-2.3.ec}{}{}{}{} and
    \ecinput{../examps/parts/tactics/call/4-2.4.ec}{}{}{}{}
  \end{tsyntax}
\end{tactic}


\subsection{Tactics for Transforming Programs}
\label{subsec:transformingprograms}

% --------------------------------------------------------------------
\begin{tactic}{swap}
All versions of the tactic work for \prhl (an optional side can be given),
\phl and \hl statement judgements. We'll describe their operation
in terms of a single program (list of statements).

\medskip
\begin{tsyntax}{swap $\;n$ $\;m$ $\;l$}
  Fails unless $1\leq n < m \leq l$ and the program has at least $l$
  statements. Swaps the statement block from positions $n$ through
  $m-1$ with the statement block from $m$ through $l$, failing if these
  blocks of statements aren't independent.
\end{tsyntax}

\begin{tsyntax}{swap [$n$..$m$] $\;k$}
  Fails unless $1\leq n \leq m$ and the program has at
  least $m$ statements.
  \begin{itemize}
  \item If $k$ is non-negative, move the statement block from $n$
    through $m$ forward $k$ positions, failing if the program doesn't
    have at least $m + k$ statements or if the swapped statements
    blocks aren't independent.

  \item If $k$ is negative, move the statement block from $n$ through
    $m$ backward $-k$ positions, failing if $n + k < 1$ or if the
    swapped statement blocks aren't independent.
  \end{itemize}
\end{tsyntax}

\begin{tsyntax}{swap $\;n$\ $k$}
  Equivalent to \ec{swap [$n$..$n$] $\;k$}.
\end{tsyntax}

\begin{tsyntax}{swap $\;k$}
  If $k$ is non-negative, equivalent to \ec{swap 1 $\;k$}.
  If $k$ is negative, equivalent to \ec{swap $\;n$ $\;k$},
  where $n$ is the length of the program.
\end{tsyntax}

\medskip For example, suppose the current goal is
  \ecinput{examps/parts/tactics/swap/1-1.0.ec}{}{}{}{}
  Then running
  \ecinput{examps/parts/tactics/swap/1-1.ec}{}{}{}{}
  produces goal
  \ecinput{examps/parts/tactics/swap/1-1.1.ec}{}{}{}{}
  From which running
  \ecinput{examps/parts/tactics/swap/1-2.ec}{}{}{}{}
  produces goal
  \ecinput{examps/parts/tactics/swap/1-2.1.ec}{}{}{}{}
  From which running
  \ecinput{examps/parts/tactics/swap/1-3.ec}{}{}{}{}
  produces goal
  \ecinput{examps/parts/tactics/swap/1-3.1.ec}{}{}{}{}
  From which running
  \ecinput{examps/parts/tactics/swap/1-4.ec}{}{}{}{}
  produces goal
  \ecinput{examps/parts/tactics/swap/1-4.1.ec}{}{}{}{}
  From which running
  \ecinput{examps/parts/tactics/swap/1-5.ec}{}{}{}{}
  produces goal
  \ecinput{examps/parts/tactics/swap/1-5.1.ec}{}{}{}{}
\end{tactic}

% --------------------------------------------------------------------
\begin{tactic}{inline}
  \begin{tsyntax}{inline $\;M_1$.$p_1$ $\;\cdots$ $\;M_n$.$p_n$}
    Inline the selected \emph{concrete} procedures in both programs,
    with \prhl, and in the program, with \hl and \phl, until no more
    inlining of these procedures is possible.

    To inline a procedure call, the procedure's parameters are
    assigned the values of their arguments (fresh parameter
    identifiers are used, as necessary, to avoid naming
    conflicts). This is followed by the body of the procedure. Finally,
    the procedure's return value is assigned to the identifiers (if
    any) to which the procedure call's result is assigned.
  \end{tsyntax}

  \begin{tsyntax}{inline\{1\} $\;M_1$.$p_1$ $\;\cdots$ $\;M_n$.$p_n$ | inline\{2\} $\;M_1$.$p_1$ $\;\cdots$ $\;M_n$.$p_n$}
    Do the inlining in just the first or second program, in the \prhl case.
  \end{tsyntax}

  \begin{tsyntax}{inline* | inline\{1\}* | inline\{2\}*}
    Inline all concrete procedures, continuing until no more inlining
    is possible.
  \end{tsyntax}

  \begin{tsyntax}{inline $\;\mathit{occs}$ $\;M$.$p$ | inline\{1\} $\;\mathit{occs}$ $\;M$.$p$ | inline\{2\} $\;\mathit{occs}$ $\;M$.$p$}
    Inline just the specified occurrences of $M$.$p$, where
    $\mathit{occs}$ is a parenthesized nonempty sequence of positive
    numbers \ec{($n_1$ $\;\cdots$ $\;n_l$)}. E.g., \ec{(1 3)} means the
    first and third occurrences of the procedure.  In the \prhl case,
    a side \ec{\{1\}} or \ec{\{2\}} must be specified.
  \end{tsyntax}

  \bigskip
  For example, given the declarations
  \ecinput{examps/tactics/inline/1.ec}{}{3-18}{}
  if the current goal is
  \ecinput{examps/parts/tactics/inline/1-1.0.ec}{}{}{}{} then
  running \ecinput{examps/parts/tactics/inline/1-1.ec}{}{}{}{}
  produces the goal
  \ecinput{examps/parts/tactics/inline/1-1.1.ec}{}{}{}{}
  From which running
  \ecinput{examps/parts/tactics/inline/1-2.ec}{}{}{}{}
  produces the goal
  \ecinput{examps/parts/tactics/inline/1-2.1.ec}{}{}{}{}
  From which running
  \ecinput{examps/parts/tactics/inline/1-3.ec}{}{}{}{}
  produces the goal
  \ecinput{examps/parts/tactics/inline/1-3.1.ec}{}{}{}{}
  And, if the current goal is
  \ecinput{examps/parts/tactics/inline/2-1.0.ec}{}{}{}{} then
  running \ecinput{examps/parts/tactics/inline/2-1.ec}{}{}{}{}
  produces the goal
  \ecinput{examps/parts/tactics/inline/2-1.1.ec}{}{}{}{}
\end{tactic}


% --------------------------------------------------------------------
\begin{tactic}{rcondf}
\end{tactic}

% --------------------------------------------------------------------
\begin{tactic}{rcondt}
  \begin{tsyntax}{rcondt $\;n$}
    If the goal's conclusion is an \hl statement judgement whose $n$th
    statement is an \ec{if} statement, reduce the goal to two
    subgoals.
    \begin{itemize}
    \item One whose concludion is an \hl statement judgement whose
      precondition is the original goal's precondition, program is the
      first $n-1$ statements of the original goal's program, and
      postcondition is the boolean expression of the \ec{if}
      statement.
   
    \item One whose conclusion is an \hl statement judgement that's
      the same as that of the original goal except that the \ec{if}
      statement has been replaced by its \ec{then} part.
    \end{itemize}

    \medskip For example, if the current goal is
    \ecinput{examps/parts/tactics/rcondt/1-1.0.ec}{}{}{}{} then
    running \ecinput{examps/parts/tactics/rcondt/1-1.ec}{}{}{}{}
    produces the goals
    \ecinput{examps/parts/tactics/rcondt/1-1.1.ec}{}{}{}{} and
    \ecinput{examps/parts/tactics/rcondt/1-1.2.ec}{}{}{}{}
  \end{tsyntax}

  \begin{tsyntax}{rcondt\{1\} $\;n$ | rcondt\{2\} $\;n$}
    If the goal's conclusion is a \prhl statement judgement where the
    $n$th statement of the designated program is an \ec{if} statement,
    reduce the goal to two subgoals.
    \begin{itemize}
    \item One whose conclusion is an \hl statement judgement whose
      precondition is the original goal's precondition, program is the
      first $n-1$ statements of the original goal's designated
      program, and postcondition is the boolean expression of the
      \ec{if} statement. Actually, the \hl statement judgement is
      universally quantified by a memory of the non-designated
      program, and references in the precondition to variables of the
      non-designated program are interpreted in that memory.
   
    \item One whose conclusion is a \prhl statement judgement that's
      the same as that of the original goal except that the \ec{if}
      statement has been replaced by its \ec{then} part.
    \end{itemize}

  \medskip
  For example, if the current goal is
  \ecinput{examps/parts/tactics/rcondt/1-2.0.ec}{}{}{}{} then
  running \ecinput{examps/parts/tactics/rcondt/1-2.ec}{}{}{}{}
  produces the goals
  \ecinput{examps/parts/tactics/rcondt/1-2.1.ec}{}{}{}{}
  and
  \ecinput{examps/parts/tactics/rcondt/1-2.2.ec}{}{}{}{}
  \end{tsyntax}
\end{tactic}


% --------------------------------------------------------------------
\begin{tactic}{splitwhile}
  \begin{tsyntax}[empty]{splitwhile}
  \fix{Missing description of splitwhile}.
  \end{tsyntax}
\end{tactic}

% --------------------------------------------------------------------
\begin{tactic}{unroll}
  \begin{tsyntax}{unroll $\;n$}
    If the goal's conclusion is an \hl statement judgement whose $n$th
    statement is a \ec{while} statement, then insert before that
    statement an \ec{if} statement whose boolean expression is the
    \ec{while} statement's boolean expression, whose \ec{then} part is
    the \ec{while} statements's body, and whose \ec{else} part is
    empty.

    \medskip For example, if the current goal is
    \ecinput{../examps/parts/tactics/unroll/1-1.0.ec}{}{}{}{} then
    running \ecinput{../examps/parts/tactics/unroll/1-1.ec}{}{}{}{}
    produces the goal
    \ecinput{../examps/parts/tactics/unroll/1-1.1.ec}{}{}{}{}
  \end{tsyntax}

  \begin{tsyntax}{unroll\{1\} $\;n$ | unroll\{2\} $\;n$}
    If the goal's conclusion is an \prhl statement judgement where the
    $n$th statement of the designated program is a \ec{while}
    statement, then insert before that statement an \ec{if} statement
    whose boolean expression is the \ec{while} statement's boolean
    expression, whose \ec{then} part is the \ec{while} statements's
    body, and whose \ec{else} part is empty.

    \medskip For example, if the current goal is
    \ecinput{../examps/parts/tactics/unroll/1-2.0.ec}{}{}{}{} then
    running \ecinput{../examps/parts/tactics/unroll/1-2.ec}{}{}{}{}
    produces the goal
    \ecinput{../examps/parts/tactics/unroll/1-2.1.ec}{}{}{}{}
    from which running
    running \ecinput{../examps/parts/tactics/unroll/1-3.ec}{}{}{}{}
    produces the goal
    \ecinput{../examps/parts/tactics/unroll/1-3.1.ec}{}{}{}{}
  \end{tsyntax}
\end{tactic}

% --------------------------------------------------------------------
\begin{tactic}{fission}
\end{tactic}

% --------------------------------------------------------------------
\begin{tactic}{fusion}
  \begin{tsyntax}{fusion $\;c$!$l$ @ $\;m$, $\;n$}
    \hl statement judgement version.  Fails unless $0\leq l$ and
    $0\leq m$ and $0\leq n$ and the $c$th statement of the program
    is a \ec{while} statement, and there are at least $l$ statements right
    before the \ec{while} statement, at its level,
    and the part of the program beginning from the $l$ statements
    before the while loop may be uniquely matched against
\begin{easycrypt}{}{}
#$s_1$# while (#$e$#) { #$s_2$# #$s_4$# }
#$s_1$# while (#$e$#) { #$s_3$# #$s_4$# }
\end{easycrypt}
    where:
    \begin{itemize}
    \item $s_1$ has length $l$;

    \item $s_2$ has length $m$;

    \item $s_3$ has length $n$;

    \item $e$ doesn't reference the variables written by $s_2$ and $s_3$;

    \item $s_1$ and $s_4$ don't read or write the variables written by
      $s_2$ and $s_3$;

    \item $s_2$ and $s_3$ don't write the variables written by $s_1$
      and $s_4$;

    \item $s_2$ and $s_3$ don't read or write the variables written by
      the other.
    \end{itemize}
    The tactic replaces
\begin{easycrypt}{}{}
#$s_1$# while (#$e$#) { #$s_2$# #$s_4$# }
#$s_1$# while (#$e$#) { #$s_3$# #$s_4$# }
\end{easycrypt}
by
\begin{easycrypt}{}{}
#$s_1$# while (#$e$#) { #$s_2$# #$s_3$# #$s_4$# }
\end{easycrypt}

    \medskip For example, if the current goal is
    \ecinput{examps/parts/tactics/fusion/1-1.0.ec}{}{}{}{} then
    running \ecinput{examps/parts/tactics/fusion/1-1.ec}{}{}{}{}
    produces the goal
    \ecinput{examps/parts/tactics/fusion/1-1.1.ec}{}{}{}{}
  \end{tsyntax}

  \begin{tsyntax}{fusion $\;c$ @ $\;m$, $\;n$}
    Equivalent to \ec{fusion $\;c$!1 @ $\;m$, $\;n$}.
  \end{tsyntax}

  \begin{tsyntax}{fusion\{1\} $\;\cdots$ | fusion\{2\} $\;\cdots$}
    The \prhl versions of the above variants, working on the
    designated program.
  \end{tsyntax}
\end{tactic}


% --------------------------------------------------------------------
\begin{tactic}{alias}
  \begin{tsyntax}{alias $\;c$ with $\;x$}
    If the goal's conclusion is an \hl statement judgement whose
    program's $c$th statement is an assignment statement, and $x$ is
    an identifier, then replace the assignment statement by the
    following two statements:
    \begin{itemize}
    \item an assignment statement of the same kind as the original
      assignment statement (ordinary, random, procedure call) whose
      left-hand side is $x$, and whose right-hand side is the
      right-hand side of the original assignment statement;

    \item an ordinary assignment statement whose left-hand side is
      the left-hand side of the original assignment statement,
      and whose right-hand side is $x$.
    \end{itemize}
    If $x$ is a local variable of the program, a fresh name is
    generated by adding digits to the end of $x$.
  \end{tsyntax}

  \begin{tsyntax}{alias $\;c$}
    Equivalent to \ec{alias $\;c$ with x}.
  \end{tsyntax}

  \begin{tsyntax}{alias $\;c$ $\;x$ = $\;e$}
    If the program has an $c$th statement, and the expression $e$ is
    well-typed in the context of the program, insert before the $c$th
    statement an ordinary assignment statement whose left-hand side is
    $x$ and whose right-hand side is $e$.
    If $x$ is a local variable of the program, a fresh name is
    generated by adding digits to the end of $x$.
  \end{tsyntax}

  \begin{tsyntax}{alias\{1\} $\;\cdots$ | alias\{2\} $\;\cdots$}
    The \prhl versions of the preceding forms, where the aliasing
    is done in the designated program.
  \end{tsyntax}

  \bigskip For example, if the current goal is
  \ecinput{../examps/parts/tactics/alias/1-1.0.ec}{}{}{}{} then running
  \ecinput{../examps/parts/tactics/alias/1-1.ec}{}{}{}{} produces the
  goal \ecinput{../examps/parts/tactics/alias/1-1.1.ec}{}{}{}{}
  from which running
  \ecinput{../examps/parts/tactics/alias/1-2.ec}{}{}{}{} produces the
  goal \ecinput{../examps/parts/tactics/alias/1-2.1.ec}{}{}{}{}
  from which running
  \ecinput{../examps/parts/tactics/alias/1-3.ec}{}{}{}{} produces the
  goal \ecinput{../examps/parts/tactics/alias/1-3.1.ec}{}{}{}{}
\end{tactic}

% --------------------------------------------------------------------
\begin{tactic}{cfold}
  \begin{tsyntax}{cfold $\;n$ ! $\;m$}
    Fails unless $n\geq 1$ and $m\geq 0$.  If the goal's conclusion is
    an \hl statement judgement in which statement $n$ of the
    judgement's program is an ordinary assignment statement in which
    constant values are assigned to local identifiers, and the
    following statement block of length $m$ does not write any of
    those identifiers, then replace all occurrences of the assigned
    identifiers in that statement block by the constants assigned to
    them, and move the assignment statement to after the modified
    statement block.

    \medskip For example, if the current goal is
    \ecinput{../examps/parts/tactics/cfold/1-1.0.ec}{}{}{}{} then
    running \ecinput{../examps/parts/tactics/cfold/1-1.ec}{}{}{}{}
    produces the goal
    \ecinput{../examps/parts/tactics/cfold/1-1.1.ec}{}{}{}{}
    from which
    running \ecinput{../examps/parts/tactics/cfold/1-2.ec}{}{}{}{}
    produces the goal
    \ecinput{../examps/parts/tactics/cfold/1-2.1.ec}{}{}{}{}
    from which
    running \ecinput{../examps/parts/tactics/cfold/1-3.ec}{}{}{}{}
    produces the goal
    \ecinput{../examps/parts/tactics/cfold/1-3.1.ec}{}{}{}{}
    from which
    running \ecinput{../examps/parts/tactics/cfold/1-4.ec}{}{}{}{}
    produces the goal
    \ecinput{../examps/parts/tactics/cfold/1-4.1.ec}{}{}{}{}
  \end{tsyntax}

  \begin{tsyntax}{cfold\{1\} $\;n$ ! $\;m$ | cfold\{2\} $\;n$ ! $\;m$}
    Like the \hl version, but operating on the designed program of
    a \prhl judgement's conclusion.
  \end{tsyntax}

  \begin{tsyntax}{cfold\{1\} $\;n$ | cfold\{2\} $\;n$}
    Like the general cases, but where $m$ is set so as to be the
    number of statements after the assignment statement.
  \end{tsyntax}
\end{tactic}

% --------------------------------------------------------------------
\begin{tactic}{kill}
  \begin{tsyntax}{kill $\;n$ ! $\;m$}
    Fails unless $n\geq 1$ and $m\geq 0$.  If the goal's conclusion is
    an \hl statement judgement in which the statement block from
    position $n$ to $n + m - 1$ is well-defined (when $m = 0$, this
    block is empty), and the variables written by this statement
    block aren't used in the judgement's postcondition or read by the
    rest of the program, then reduce the goal to two subgoals.
    \begin{itemize}
    \item One whose conclusion is a \phl statement judgement whose pre-
       and postconditions are \ec{true}, whose program is the
       statement block, and whose bound part is \ec{= 1\%r}.

    \item One that's identical to the original goal except that the
      statement block has been removed.
    \end{itemize}

    \medskip For example, if the current goal is
    \ecinput{../examps/parts/tactics/kill/1-1.0.ec}{}{}{}{} then
    running \ecinput{../examps/parts/tactics/kill/1-1.ec}{}{}{}{}
    produces the goals
    \ecinput{../examps/parts/tactics/kill/1-1.1.ec}{}{}{}{}
    and
    \ecinput{../examps/parts/tactics/kill/1-1.2.ec}{}{}{}{}
  \end{tsyntax}

  \begin{tsyntax}{kill\{1\} $\;n$ ! $\;m$ | kill\{2\} $\;n$ ! $\;m$}
    Like the \hl case but for \prhl judgements, where the statement
    block to be killed is in the designated program.

    \medskip For example, if the current goal is
    \ecinput{../examps/parts/tactics/kill/1-2.0.ec}{}{}{}{} then
    running \ecinput{../examps/parts/tactics/kill/1-2.ec}{}{}{}{}
    produces the goals
    \ecinput{../examps/parts/tactics/kill/1-2.1.ec}{}{}{}{}
    and
    \ecinput{../examps/parts/tactics/kill/1-2.2.ec}{}{}{}{}
  \end{tsyntax}

  \begin{tsyntax}{kill $\;n$ | kill\{1\} $\;n$ | kill\{2\} $\;n$}
    Like the general cases, but with $m = 1$.
  \end{tsyntax}

  \begin{tsyntax}{kill $\;n$ ! * | kill\{1\} $\;n$ ! * | kill\{2\} $\;n$ ! *}
    Like the general cases, but with $m$ set so that the statement
    block to be killed is the rest of the (designated) program.
  \end{tsyntax}
\end{tactic}

% --------------------------------------------------------------------
\begin{tactic}{modpath}
\end{tactic}


\subsection{Tactics for Reasoning about Specifications}

% --------------------------------------------------------------------
\begin{tactic}{symmetry}
  \begin{tsyntax}{symmetry}
  In \prhl, swaps the two programs, transforming the pre and
  postconditions by swapping the memories they refer to.

  \textbf{Examples:} In the following, $\symrel{\cdot}$ inverses its
  argument relation. (That is, for any relation $R$ and any
  $(m_1,m_2)\in{R}$, we have $(m_2,m_1)\in\symrel{R}$.)
  $$
  \inferrule*[left=(\prhl)]%%
    {\pRHL{\symrel{P}}{c_2}{c_1}{\symrel{Q}}}%%
    {\pRHL{P}{c_1}{c_2}{Q}}%%
    \quad\raisebox{.7em}{\tct{symmetry}}
  $$
  \end{tsyntax}
\end{tactic}

% --------------------------------------------------------------------
\begin{tactic}{transitivity}
  \begin{tsyntax}{transitivity c ($P_1$ ==> $\ Q_1$) ($P_2$ ==> $\ Q_2$)}
  In \prhl, applies the transitivity of program equivalence using the
  specified program and specifications. When the goal is a judgment on
  procedures, \ec{c} should be a procedure. When the goal is a
  judgment on statements, \ec{c} should be a statement, and the
  tactic then takes a side argument, used to decide the procedure
  context under which local variables from \ec{c} are evaluated.

  \textbf{Examples:}
  \begin{mathpar}
  \inferrule%%
    {\forall \mem{m_1}\ \mem{m_2}.\, \Rel{P}{\mem{m_1}}{\mem{m_2}} \Rightarrow
        \exists \mem{m}.\, \Rel{P_1}{\mem{m_1}}{\mem{m}}
                           \wedge \Rel{P_2}{\mem{m}}{\mem{m_2}} \\%
     \forall \mem{m_1}\ \mem{m}\ \mem{m_2}.\,
        \Rel{Q_1}{\mem{m_1}}{\mem{m}} \Rightarrow
        \Rel{Q_2}{\mem{m}}{\mem{m_2}} \Rightarrow
        \Rel{Q}{\mem{m_1}}{\mem{m_2}} \\%
     \pRHL{P_1}{f_1}{f}{Q_1} \\%
     \pRHL{P_2}{f}{f_2}{Q_2}}%%
    {\pRHL{P}{f_1}{f_2}{Q}}%%
    \quad\mbox{(\prhl)\quad\parbox{200pt}{\ec{transitivity f ($P_1$ ==> $\ Q_1$) ($P_2$ ==> $\ Q_2$)}}} \\
  \inferrule%%
    {\forall \mem{m_1}\ \mem{m_2}.\, \Rel{P}{\mem{m_1}}{\mem{m_2}} \Rightarrow
        \exists \mem{m}.\, \Rel{P_1}{\mem{m_1}}{\mem{m}}
                           \wedge \Rel{P_2}{\mem{m}}{\mem{m_2}} \\%
     \forall \mem{m_1}\ \mem{m}\ \mem{m_2}.\,
        \Rel{Q_1}{\mem{m_1}}{\mem{m}} \Rightarrow
        \Rel{Q_2}{\mem{m}}{\mem{m_2}} \Rightarrow
        \Rel{Q}{\mem{m_1}}{\mem{m_2}} \\%
     \pRHL{P_1}{s_1}{s}{Q_1} \\%
     \pRHL{P_2}{s}{s_2}{Q_2}}%%
    {\pRHL{P}{s_1}{s_2}{Q}}%%
    \quad\mbox{(\prhl)\quad\parbox{200pt}{\ec{transitivity$\{$1$\}$ $\ \{$ s $\ \}$ ($P_1$ ==> $\ Q_1$) ($P_2$ ==> $\ Q_2$)}}} \\
  \end{mathpar}

  \textbf{Note:} In practice, the existential quantification over
  memory $\mem{m}$ in the first generated subgoal is replaced with an
  existential quantification over the program variables appearing in $P$,
  $P_1$, ot $P_2$.
  \end{tsyntax}
\end{tactic}

% --------------------------------------------------------------------
\begin{tactic}{conseq}
  \begin{tsyntax}{conseq <specification>}
  Rule of consequence. Proves a specification by weakening of a
  stronger result. Any one of the specification places can be filled
  with a wildcard \tct{_} to keep the value it contains in the current
  goal and trivially discharge the corresponding subgoal.

  \textbf{Examples:} In the following, $\leq^\uparrow$ (resp. $=^\uparrow$,
  $\geq^\uparrow$) is $\Leftarrow$ (resp. $\Leftrightarrow$ and
  $\Rightarrow$).
  \begin{mathpar}
  \inferrule*[left=(\prhl),rightskip=10em]%%
    {P' \Rightarrow P \\%
     Q \Rightarrow Q' \\%
     \pRHL{P}{c}{c'}{Q}}%%
    {\pRHL{P'}{c}{c'}{Q'}}%%
    \quad\raisebox{.7em}{\tct{conseq (_: P ==> Q)}} \\
  \inferrule*[left=(\prhl),rightskip=10em]%%
    {Q \Rightarrow Q' \\%
     \pRHL{P'}{c}{c'}{Q}}%%
    {\pRHL{P'}{c}{c'}{Q'}}%%
    \quad\raisebox{.7em}{\tct{conseq (_: _ ==> Q)}} \\
  \inferrule*[left=(\phl),rightskip=10em]%%
    {P' \Rightarrow \delta \mathrel{\diamond} \delta' \\%
     P' \Rightarrow P \\%
     Q \mathrel{\diamond^\uparrow} Q' \\%
     \pHL{P}{c}{Q}{\diamond}{\delta}}%%
    {\pHL{P'}{c}{Q'}{\diamond}{\delta'}}%%
    \quad\raisebox{.7em}{\tct{conseq (_: P ==> Q: $\delta$)}} \\
  \inferrule*[left=(\hl),rightskip=10em]%%
    {P' \Rightarrow P \\%
     Q \Rightarrow Q' \\%
     \HL{P}{c}{Q}}%%
    {\HL{P'}{c}{Q'}}%%
    \quad\raisebox{.7em}{\tct{conseq (_: P ==> Q)}} \\
  \end{mathpar}
  \end{tsyntax}

  \begin{tsyntax}{conseq <lemma>}
  Only works on procedures. Same as \tct{conseq <specification>}, but
  the specification to use is inferred from the lemma provided. Raises
  an error if the lemma does not refer to the expected procedure(s).
  \end{tsyntax}

  \begin{tsyntax}{conseq* <specification>}
  Same as \tct{conseq <specification>}, but the subgoal corresponding
  to the postcondition is refined by a ``may modify'' analysis.
  \end{tsyntax}

  \fix{Missing descriptions of combining variants of conseq}.
\end{tactic}

% --------------------------------------------------------------------
\begin{tactic}[case]{case-pl}
  \begin{tsyntax}{case $\;e$}
    If the goal's conclusion is a \prhl, \hl or \phl \emph{statement}
    judgement and $e$ is well-typed in the goal's context, split the
    goal into two goals:
    \begin{itemize}
    \item a first goal in which $e$ is added as a conjunct to the
      conclusion's precondition; and

    \item a second goal in which $!e$ is added as a conjunct to the
      conclusion's precondition.
    \end{itemize}

    \medskip For example, if the current goal is
    \ecinput{examps/parts/tactics/case_pl/1-1.0.ec} then
    running \ecinput{examps/parts/tactics/case_pl/1-1.ec}
    produces the goals
    \ecinput{examps/parts/tactics/case_pl/1-1.1.ec}
    and
    \ecinput{examps/parts/tactics/case_pl/1-1.2.ec}
    And if the current goal is
    \ecinput{examps/parts/tactics/case_pl/2-1.0.ec} then
    running \ecinput{examps/parts/tactics/case_pl/2-1.ec}
    produces the goals
    \ecinput{examps/parts/tactics/case_pl/2-1.1.ec}
    and
    \ecinput{examps/parts/tactics/case_pl/2-1.2.ec}
  \end{tsyntax}
\end{tactic}

% --------------------------------------------------------------------
\begin{tactic}{phoare split}
  \fxfatal{Update waiting for overhaul of \phl.}

  \begin{tsyntax}{phoare split $\ \delta_{A}$ $\ \delta_{B}$ $\ \delta_{AB}$}
  Splits a \phl judgment whose postcondition is a conjunction or
  disjunction into three \phl judgments following the definition of
  the probability of a disjunction of events.

  \paragraph{Examples:}\strut

  \begin{cmathpar}
  \texample[\phl{}]
    {\ec{phoare split $\ \delta_{A}$ $\ \delta_{B}$ $\ \delta_{AB}$}}
    {\delta_{A} + \delta_{B} - \delta_{AB} \diamond \delta \\
     \pHL{P}{c}{A}{\diamond}{\delta_{A}} \\
     \pHL{P}{c}{B}{\diamond}{\delta_{B}} \\
     \pHL{P}{c}{A \wedge B}{\invrel{\diamond}}{\delta_{AB}}}
    {\pHL{P}{c}{A \vee B}{\diamond}{\delta}}

  \texample[\phl{}]
    {\ec{phoare split $\ \delta_{A}$ $\ \delta_{B}$ $\ \delta_{AB}$}}
    {\delta_{A} + \delta_{B} - \delta_{AB} \diamond \delta \\
     \pHL{P}{c}{A}{\diamond}{\delta_{A}} \\
     \pHL{P}{c}{B}{\diamond}{\delta_{B}} \\
     \pHL{P}{c}{A \vee B}{\invrel{\diamond}}{\delta_{AB}}}
    {\pHL{P}{c}{A \wedge B}{\diamond}{\delta}}
  \end{cmathpar}
  \end{tsyntax}

  \begin{tsyntax}{phoare split $\ {!}$ $\ \delta_{\top}$ $\ \delta_{!}$}
  Splits a \phl judgment into two judgments whose postcondition are
  true and the negation of the original postcondition, respectively.

  \paragraph{Examples:}\strut

  \begin{cmathpar}
  \texample[\phl{}]
    {\ec{phoare split ! $\ \delta_{\top}$ $\ \delta_{!}$}}
    {\delta_{\top} - \delta_{!} \diamond \delta \\
     \pHL{P}{c}{\mathsf{true}}{\diamond}{\delta_{\top}} \\
     \pHL{P}{c}{!Q}{\invrel{\diamond}}{\delta_{!}}}
    {\pHL{P}{c}{Q}{\diamond}{\delta}}
  \end{cmathpar}
  \end{tsyntax}

  \begin{tsyntax}{phoare split $\ \delta_{A}$ $\ \delta_{!A}$: A}
  Splits a \phl judgment following an event $A$.

  \paragraph{Examples:}\strut

  \begin{cmathpar}
  \texample[\phl{}]
    {\ec{phoare split $\ \delta_{A}$ $\ \delta_{!A}$: A}}
    {\delta_{A} + \delta_{!A} \diamond \delta \\
     \pHL{P}{c}{Q \wedge A}{\diamond}{\delta_{A}} \\
     \pHL{P}{c}{Q \wedge \neg A}{\diamond}{\delta_{!A}}}
    {\pHL{P}{c}{Q}{\diamond}{\delta}}
  \end{cmathpar}
  \end{tsyntax}  
\end{tactic}

% --------------------------------------------------------------------
\begin{tactic}{byequiv}
  \begin{tsyntax}{byequiv (_ : $\;P$ ==> $\;Q$)}
    If the goal's conclusion has the form
    \begin{center}
      \ec{Pr[$M_1$.$p_1$($a_{1,1}$, $\ldots$, $a_{1,n_1}$) @ &$m_1$ : $\;E_1$] =
          Pr[$M_2$.$p_2$($a_{2,1}$, $\ldots$, $a_{2,n_2}$) @ &$m_2$ : $\;E_2$]},
    \end{center}
    reduce the goal to three subgoals:
    \begin{itemize}
    \item One with conclusion
          \ec{equiv[$M_1$.$p_1$ ~ $\;M_2$.$p_2$ : $\;P$ ==> $\;Q$]};

    \item One whose conclusion says that $P$ holds, where
      references to memories \ec{&1} and \ec{&2} have been replaced
      by \ec{&$m_1$} and \ec{&$m_2$}, respectively, and references
      to the formal parameters of \ec{$M_1$.$p_1$} and
      \ec{$M_2$.$p_2$} have been replaced by their arguments;

    \item One whose conclusion says that $Q$ implies that
      \ec{$E_1$\{1\} <=> $\;E_2$\{2\}}.
    \end{itemize}

    The argument to \ec{byequiv} may be replaced by a proof term for
    \ec{equiv[$M_1$.$p_1$ ~ $\;M_2$.$p_2$ : $\;P$ ==> $\;Q$]}, in which
    case the first subgoal isn't generated.
    Furthermore, either or both of $P$ and $Q$ may be replaced by
    \ec{_}, asking that the pre- or postcondition be inferred.
    Supplying no argument to \ec{equiv} is the same as replacing
    both $P$ and $Q$ by \ec{_}. By default, inference of $Q$ attempts
    to infer a conjuction of equalities implying   
    \ec{$E_1$\{1\} <=> $\;E_2$\{2\}}. Passing the \ec{[-eq]} option to
    \ec{conseq} takes $Q$ to \emph{be} \ec{$E_1$\{1\} <=> $\;E_2$\{2\}}.

    \medskip
    \emph{The other variants of the tactic behave similarly with
    regards to the use of proof terms and specification inference.}

    \medskip For example, consider the module
    \ecinput{examps/tactics/byequiv/1.ec}{}{3-10}{}
    If the current goal is
    \ecinput{examps/parts/tactics/byequiv/1-1.0.ec}{}{}{}{} then
    running \ecinput{examps/parts/tactics/byequiv/1-1.ec}{}{}{}{}
    produces the goals
    \ecinput{examps/parts/tactics/byequiv/1-1.1.ec}{}{}{}{}
    and
    \ecinput{examps/parts/tactics/byequiv/1-1.2.ec}{}{}{}{}
    and
    \ecinput{examps/parts/tactics/byequiv/1-1.3.ec}{}{}{}{}
    Given the lemma
    \ecinput{examps/tactics/byequiv/1.ec}{}{24-24}{}
    if the current goal is
    \ecinput{examps/parts/tactics/byequiv/1-2.0.ec}{}{}{}{} then
    running \ecinput{examps/parts/tactics/byequiv/1-2.ec}{}{}{}{}
    produces the goals
    \ecinput{examps/parts/tactics/byequiv/1-2.1.ec}{}{}{}{}
    and
    \ecinput{examps/parts/tactics/byequiv/1-2.2.ec}{}{}{}{}
    And, if the current goal is
    \ecinput{examps/parts/tactics/byequiv/1-3.0.ec}{}{}{}{} then
    running \ecinput{examps/parts/tactics/byequiv/1-3.ec}{}{}{}{}
    produces the goals
    \ecinput{examps/parts/tactics/byequiv/1-3.1.ec}{}{}{}{}
    and
    \ecinput{examps/parts/tactics/byequiv/1-3.2.ec}{}{}{}{}
    and
    \ecinput{examps/parts/tactics/byequiv/1-3.3.ec}{}{}{}{}
  \end{tsyntax}

  \begin{tsyntax}{byequiv (_ : $\;P$ ==> $\;Q$)}
    If the goal's conclusion has the form
    \begin{center}
      \ec{Pr[$M_1$.$p_1$($a_{1,1}$, $\ldots$, $a_{1,n_1}$) @ &$m_1$ : $\;E_1$] <=
          Pr[$M_2$.$p_2$($a_{2,1}$, $\ldots$, $a_{2,n_2}$) @ &$m_2$ : $\;E_2$]},
    \end{center}
    then \ec{conseq} behaves the same as in the first variant
    except that the conclusion of the third subgoal says that $Q$
    implies \ec{$E_1$\{1\} => $\;E_2$\{2\}}.

    \medskip For example, if the current goal is
    \ecinput{examps/parts/tactics/byequiv/2-1.0.ec}{}{}{}{} then
    running \ecinput{examps/parts/tactics/byequiv/2-1.ec}{}{}{}{}
    produces the goals
    \ecinput{examps/parts/tactics/byequiv/2-1.1.ec}{}{}{}{}
    and
    \ecinput{examps/parts/tactics/byequiv/2-1.2.ec}{}{}{}{}
    and
    \ecinput{examps/parts/tactics/byequiv/2-1.3.ec}{}{}{}{}
    And, if the current goal is
    \ecinput{examps/parts/tactics/byequiv/2-2.0.ec}{}{}{}{} then
    running \ecinput{examps/parts/tactics/byequiv/2-2.ec}{}{}{}{}
    produces the goals
    \ecinput{examps/parts/tactics/byequiv/2-2.1.ec}{}{}{}{}
    and
    \ecinput{examps/parts/tactics/byequiv/2-2.2.ec}{}{}{}{}
    and
    \ecinput{examps/parts/tactics/byequiv/2-2.3.ec}{}{}{}{}
  \end{tsyntax}

  \begin{tsyntax}{byequiv (_ : $\;P$ ==> $\;Q$)}
    If the goal's conclusion has the form
    \begin{center}
      \ec{Pr[$M_1$.$p_1$($a_{1,1}$, $\ldots$, $a_{1,n_1}$) @ &$m_1$ : $\;E_1$] <=} \\
      \ec{Pr[$M_2$.$p_2$($a_{2,1}$, $\ldots$, $a_{2,n_2}$) @ &$m_2$ : $\;E_2$] +
          Pr[$M_2$.$p_2$($a_{2,1}$, $\ldots$, $a_{2,n_2}$) @ &$m_2$ : $\;B_2$]},
    \end{center}
    then \ec{conseq} behaves the same as in the first variant
    except that the conclusion of the third subgoal says that $Q$
    implies \ec{!$B_2$\{2\} => $\;E_1$\{1\} => $\;E_2$\{2\}}.

    \medskip For example, if the current goal is
    \ecinput{examps/parts/tactics/byequiv/3-1.0.ec}{}{}{}{} then
    running \ecinput{examps/parts/tactics/byequiv/3-1.ec}{}{}{}{}
    produces the goals
    \ecinput{examps/parts/tactics/byequiv/3-1.1.ec}{}{}{}{}
    and
    \ecinput{examps/parts/tactics/byequiv/3-1.2.ec}{}{}{}{}
    \fixme{Why is the second subgoal pruned? (Compare with first and
    second variants, where the corresponding subgoal isn't pruned.)}
    And, if the current goal is
    \ecinput{examps/parts/tactics/byequiv/3-2.0.ec}{}{}{}{} then
    running \ecinput{examps/parts/tactics/byequiv/3-2.ec}{}{}{}{}
    produces the goals
    \ecinput{examps/parts/tactics/byequiv/3-2.1.ec}{}{}{}{}
    and
    \ecinput{examps/parts/tactics/byequiv/3-2.2.ec}{}{}{}{}
    \fixme{Why is the second subgoal pruned?}
  \end{tsyntax}

  \begin{tsyntax}{byequiv (_ : $\;P$ ==> $\;Q$) : $\;B_1$}
    If the goal's conclusion has the form
    \begin{center}
      \ec{`| Pr[$M_1$.$p_1$($a_{1,1}$, $\ldots$, $a_{1,n_1}$) @ &$m_1$ : $\;E_1$] -
          Pr[$M_2$.$p_2$($a_{2,1}$, $\ldots$, $a_{2,n_2}$) @ &$m_2$ : $\;E_2$] | <=} \\
      \ec{Pr[$M_2$.$p_2$($a_{2,1}$, $\ldots$, $a_{2,n_2}$) @ &$m_2$ : $\;B_2$]},
    \end{center}
    then \ec{conseq} behaves the same as in the first variant
    except that the conclusion of the third subgoal says that $Q$
    implies
    \begin{center}
      \ec{($B_1$\{1\} <=> $\;B_2$\{2\}) /\\ ! $\;B_2$\{2\} => ($\;E_1$\{1\} <=> $\;E_2$\{2\})}
    \end{center}

    \medskip For example, if the current goal is
    \ecinput{examps/parts/tactics/byequiv/4-1.0.ec}{}{}{}{} then
    running \ecinput{examps/parts/tactics/byequiv/4-1.ec}{}{}{}{}
    produces the goals
    \ecinput{examps/parts/tactics/byequiv/4-1.1.ec}{}{}{}{}
    and
    \ecinput{examps/parts/tactics/byequiv/4-1.2.ec}{}{}{}{}
    and
    \ecinput{examps/parts/tactics/byequiv/4-1.3.ec}{}{}{}{}
    Given the lemma
    \ecinput{examps/tactics/byequiv/4.ec}{}{37-38}{}
    if the current goal is
    \ecinput{examps/parts/tactics/byequiv/4-2.0.ec}{}{}{}{} then
    running \ecinput{examps/parts/tactics/byequiv/4-2.ec}{}{}{}{}
    produces the goals
    \ecinput{examps/parts/tactics/byequiv/4-2.1.ec}{}{}{}{}
    and
    \ecinput{examps/parts/tactics/byequiv/4-2.2.ec}{}{}{}{}
  \end{tsyntax}
\end{tactic}

%%  \begin{tsyntax}{byequiv [option]? <spec>}
%%  Derives a probability relation from a \prhl judgement on the
%%  procedures involved. \ec{<spec>} can include wildcards when the
%%  tactic should infer the pre or postcondition. In addition,
%%  \ec{<spec>} can be extended with a failure event to infer precise
%%  applications of the Fundamental Lemma.
%%
%%  \textbf{Options:} By default, (\ec{eq} option) specification
%%  inference attempts to infer a conjunction of equalities sufficient
%%  to imply the desired relation. Passing the \ec{-eq} option
%%  overrides this behaviour, instead using the trivial relation on
%%  events.
%%
%%  \paragraph{Examples:}\strut
%%
%%  \begin{cmathpar}
%%    \texample
%%      {\ec{byequiv (_: P ==> Q)}}
%%      {\pRHL{P}{f_1}{f_2}{Q} \\\\
%%       {m_1[\Arg\mapsto\vec{a}_1]} \rel{P} {m_2[\Arg\mapsto\vec{a}_2]} \\
%%       Q \Rightarrow E_1\{1\}  \Leftrightarrow E_2\{2\}}
%%      {\PR{f_1}{\vec{a}_1}{\mem{m_1}}{E_1} = \PR{f_2}{\vec{a}_2}{m_2}{E_2}}
%%  \end{cmathpar}
%%
%%  \begin{cmathpar}
%%    \texample
%%      {\ec{byequiv (_: P ==> Q)}}
%%      {\pRHL{P}{f_1}{f_2}{Q} \\\\
%%       {m_1[\Arg\mapsto\vec{a}_1]} \rel{P} {m_2[\Arg\mapsto\vec{a}_2]} \\
%%       Q \Rightarrow E_1\{1\}  \Rightarrow E_2\{2\}}
%%      {\PR{f_1}{\vec{a}_1}{\mem{m_1}}{E_1} \leq \PR{f_2}{\vec{a}_2}{m_2}{E_2}}
%%  \end{cmathpar}
%%
%%  \begin{cmathpar}
%%    \texample
%%      {\ec{byequiv (_: P ==> Q)}}
%%      {\pRHL{P}{f_1}{f_2}{Q} \\\\
%%       {m_1[\Arg\mapsto\vec{a}_1]} \rel{P} {m_2[\Arg\mapsto\vec{a}_2]} \\
%%       Q \Rightarrow E_2\{2\}  \Rightarrow E_1\{1\}}
%%      {\PR{f_1}{\vec{a}_1}{\mem{m_1}}{E_1} \geq \PR{f_2}{\vec{a}_2}{m_2}{E_2}}
%%  \end{cmathpar}
%%
%%  \begin{cmathpar}
%%    \texample
%%      {\ec{byequiv (_: P ==> Q)}}
%%      {\pRHL{P}{f_1}{f_2}{Q} \\
%%       {m_1[\Arg\mapsto\vec{a}_1]} \rel{P} {m_2[\Arg\mapsto\vec{a}_2]} \\
%%       Q \Rightarrow \neg B_2\{2\} \Rightarrow E_1\{1\}  \Rightarrow E_2\{2\}}
%%      {\PR{f_1}{\vec{a}_1}{\mem{m_1}}{E_1}
%%       \leq \PR{f_2}{\vec{a}_2}{m_2}{E_2}
%%          + \PR{f_2}{\vec{a}_2}{m_2}{B_2}}
%%  \end{cmathpar}
%%
%%  \begin{cmathpar}
%%    \texample
%%      {\ec{byequiv (_: P ==> Q) : B$_1$}}
%%      {\pRHL{P}{f_1}{f_2}{Q} \\
%%       {m_1[\Arg\mapsto\vec{a}_1]} \rel{P} {m_2[\Arg\mapsto\vec{a}_2]} \\
%%       Q \Rightarrow
%%         (B_1\{1\} \Leftrightarrow B_2\{2\})
%%         \wedge (\neg B_2\{2\} \Rightarrow E_1\{1\} \Leftrightarrow E_2\{2\})}
%%      {| \PR{f_1}{\vec{a}_1}{\mem{m_1}}{E_1} - \PR{f_2}{\vec{a}_2}{m_2}{E_2} |
%%       \leq \PR{f_2}{\vec{a}_2}{m_2}{B_2}}
%%  \end{cmathpar}
%%
%%  \begin{cmathpar}
%%    \texample
%%      {\ec{byequiv [-eq] (_: P ==> _)}}
%%      {\pRHL{P}{f_1}{f_2}{E_1\{1\} \Leftrightarrow E_2\{2\}} \\
%%       {m_1[\Arg\mapsto\vec{a}_1]} \rel{P} {m_2[\Arg\mapsto\vec{a}_2]}}
%%      {\PR{f_1}{\vec{a}_1}{\mem{m_1}}{E_1} = \PR{f_2}{\vec{a}_2}{m_2}{E_2}}
%%  \end{cmathpar}
%% \end{tsyntax}
%%
%%  \begin{tsyntax}{byequiv <lemma>}
%%  Same as \ec{byequiv <spec>}, but the specification to use
%%  is inferred from the lemma provided. Raises an error if the lemma
%%  does not refer to the expected procedures. Inference options have no
%%  effect in this setting.
%%  \end{tsyntax}

% --------------------------------------------------------------------
\begin{tactic}{byphoare}
\end{tactic}

% --------------------------------------------------------------------
\begin{tactic}{hoare}
  \begin{tsyntax}[empty]{hoare}
  \fix{Missing description of hoare}.
  \end{tsyntax}
\end{tactic}

% --------------------------------------------------------------------
\begin{tactic}{bypr}
  \begin{tsyntax}{bypr $\;e_1$ $\;e_2$}
    If the goal's conclusion has the form
    \begin{center}
      \ec{equiv[$M$.$p$ ~ $\;N$.$q$ : $\;P$ ==> $\;Q$]},
    \end{center}
    and the $e_i$ are expressions of the same type possibily
    involving memories \ec{&1} and \ec{&2} for \ec{$M$.$p$} and
    \ec{$N$.$q$}, respectively, then reduce the goal to two subgoals:
    \begin{itemize}
    \item One whose conclusion says that for all memories \ec{&1}
      and \ec{&2} for \ec{$M$.$p$} and \ec{$N$.$q$}, if $e_1 = e_2$,
      then $Q$ holds; and

    \item One whose conclusion says that, for all memories \ec{&1} and
      \ec{&2} for \ec{$M$.$p$} and \ec{$N$.$q$} and values $a$ of the
      common type of the $e_i$, if $P$ holds, then the probability of
      running \ec{$M$.$p$} in memory \ec{&1} and with arguments
      consisting of the values of its formal parameters in \ec{&1} and
      terminating in a memory in which the value of $e_1$ (replacing
      references to \ec{&1} with reference to this memory) is $a$ is
      the same as the probability of running \ec{$N$.$q$} in memory
      \ec{&2} and with arguments consisting of the values of its
      formal parameters in \ec{&2} and terminating in a memory in
      which the value of $e_2$ (replacing references to \ec{&2} with
      reference to this memory) is $a$.
    \end{itemize}

    \medskip For example, consider the modules
    \ecinput[linerange=3-20]{examps/tactics/bypr/1.ec}
    If the current goal is
    \ecinput{examps/parts/tactics/bypr/1-1.0.ec} then
    running \ecinput{examps/parts/tactics/bypr/1-1.ec}
    produces the goals
    \ecinput{examps/parts/tactics/bypr/1-1.1.ec}
    and
    \ecinput{examps/parts/tactics/bypr/1-1.2.ec}
  \end{tsyntax}

  \begin{tsyntax}{bypr}
    If the goal's conclusion has the form
    \begin{center}
      \ec{hoare[$M$.$p$ : $\;P$ ==> $\;Q$]},
    \end{center}
    then reduce the goal to one whose conclusion says that, for all
    memories \ec{&m} for \ec{$M$.$p$} such that \ec{$P$\{$m$\}} holds,
    the probability of running \ec{$M$.$p$} in memory \ec{&$m$} and
    with arguments consisting of the values of its formal parameters
    in \ec{&$m$} and terminating in a memory satisfying \ec{!$Q$} is $0$.

    \medskip For example, consider the module
    \ecinput[linerange=3-10]{examps/tactics/bypr/2.ec}
    If the current goal is
    \ecinput{examps/parts/tactics/bypr/2-1.0.ec} then
    running \ecinput{examps/parts/tactics/bypr/2-1.ec}
    produces the goal
    \ecinput{examps/parts/tactics/bypr/2-1.1.ec}
  \end{tsyntax}
\end{tactic}

%%  Derives a program judgment from a probability relation or an exact
%%  probability. Only applies to judgments on procedures.
%%
%%  \paragraph{Examples:}\strut
%%  
%%  \begin{cmathpar}
%%    \texample[\prhl{}]
%%      {\ec{bypr (r$_1$) (r$_2$)}}
%%      {\forall \mem{m_1}, \mem{m_2}, a.\,
%%          r_1 = a \Rightarrow
%%          r_2 = a \Rightarrow
%%          {\mem{m_1}} \rel{Q} {\mem{m_2}} \\
%%       \forall \vec{a}_1, \vec{a}_2, \mem{m_1}, \mem{m_2}, a.\,
%%         {\mem{m_1}[\Arg\mapsto\vec{a}_1]} \rel{P} {\mem{m_2}[\Arg\mapsto\vec{a}_2]} \Rightarrow \\
%%         \PR{f_1}{\vec{a}_1}{\mem{m_1}}{a = r_1} = \PR{f_2}{\vec{a}_2}{\mem{m_2}}{a = r_2}}
%%      {\pRHL{P}{f_1}{f_2}{Q}}
%%  \end{cmathpar}
%%
%%  \begin{cmathpar}
%%    \texample[\phl{}]
%%      {\ec{bypr}}
%%      {\forall \mem{m}, \vec{a}.\, P\ m[\Arg\mapsto\vec{a}] \Rightarrow
%%          \PR{f}{\vec{a}}{m}{E} \mathrel{\diamond} \delta}
%%      {\pHL{P}{f}{E}{\diamond}{\delta}}
%%  \end{cmathpar}
%%
%%  \begin{cmathpar}
%%    \texample[\hl{}]
%%      {\ec{bypr}}
%%      {\forall \mem{m}, \vec{a}.\, P\ m[\Arg\mapsto\vec{a}] \Rightarrow
%%          \PR{f}{\vec{a}}{m}{\neg E} \mathop{=}0\%r}
%%      {\HL{P}{f}{E}}
%%  \end{cmathpar}
%%  \end{tsyntax}
%%\end{tactic}

% --------------------------------------------------------------------
\begin{tactic}{exists*}
  \begin{tsyntax}[empty]{exists*}

  \end{tsyntax}
\end{tactic}

% --------------------------------------------------------------------
\begin{tactic}{elim*}
  \begin{tsyntax}[empty]{elim*}
  Destruct existential quantifications at the head of a
  precondition. Such existential quantifications may be introduced by
  \rtactic{sp} or \rtactic{exists*}.

  \paragraph{Examples:}\strut

  \begin{cmathpar}
  \texample[\prhl{}]{\ec{elim*}}%%
    {\forall x.\, \pRHL{x = \inmem{M.x}{1} \wedge P}{c_1}{c_2}{Q}}%%
    {\pRHL{\exists x, x = \inmem{M.x}{1} \wedge P}{c_1}{c_2}{Q}}

  \texample[\phl{}]{\ec{elim*}}%%
    {\forall x.\, \pHL{x = M.x \wedge P}{c}{Q}{\diamond}{\delta}}%%
    {\pHL{\exists x, x = M.x \wedge P}{c}{Q}{\diamond}{\delta}}

  \texample[\hl{}]{\ec{elim*}}%%
    {\forall x.\, \HL{x = M.x \wedge P}{c}{Q}}%%
    {\HL{\exists x, x = M.x \wedge P}{c}{Q}}  
  \end{cmathpar}

  \end{tsyntax}
\end{tactic}

% --------------------------------------------------------------------
\begin{tactic}{exfalso}
  \begin{tsyntax}{exfalso}
  Combines \rtactic{conseq}, \rtactic{byequiv}, \rtactic{byphoare},
  \rtactic{hoare} and \rtactic{bypr} to strengthen the precondition
  into $\mathsf{false}$ and to discharge the resulting trivial goal.

  \textbf{Examples:}
  \begin{mathpar}
  \inferrule%%
    {P \Rightarrow \mathsf{false}}%%
    {\pRHL{P}{c}{c'}{Q}}%%
    \quad\mbox{(\prhl)\quad\parbox{200pt}{\tct{exfalso}}} \\
  \inferrule%%
    {P \Rightarrow \mathsf{false}}%%
    {\pHL{P}{c}{Q}{\diamond}{\delta}}%%
    \quad\mbox{(\phl)\quad\parbox{200pt}{\tct{exfalso}}} \\
  \inferrule%%
    {P \Rightarrow \mathsf{false}}%%
    {\HL{P}{c}{Q}}%%
    \quad\mbox{(\hl)\quad\parbox{200pt}{\tct{exfalso}}} \\
  \end{mathpar}
  \fix{Move \tct{exfalso} to automatic tactics}?
  \end{tsyntax}
\end{tactic}


\subsection{Automated Tactics}
\label{subsec:automatedtactics}

% --------------------------------------------------------------------
\begin{tactic}{auto}
\end{tactic}

% --------------------------------------------------------------------
\begin{tactic}{sim}
  \ec{sim} attempts to solve a goal whose conclusion is a \prhl
  judgement or statement judgement by working backwards, propagating
  and extending a conjunction of equalties between variables of the
  two programs, verifying that the conclusion's precondition implies
  the final conjuction of equalities.  It's capable of working
  backwards through \ec{if} and \ec{while} statements and handing
  random assignments, but only when the programs are sufficiently
  similar (thus its name). Sometimes this process only partly
  succeeds, leaving a statement judgement whose programs are prefixes
  of the original programs.

  \begin{tsyntax}{sim}
    Without any arguments, \ec{sim} attemps to infer the conjuction of
    program variable equalities from the conclusion's postcondition.

    \medskip For example, if the current goal is
    \ecinput{examps/parts/tactics/sim/1-1.0.ec}{}{}{}{} then
    running \ecinput{examps/parts/tactics/sim/1-1.ec}{}{}{}{}
    produces the goal
    \ecinput{examps/parts/tactics/sim/1-1.1.ec}{}{}{}{}
    which \ec{auto} is able to solve.
  \end{tsyntax}

  \begin{tsyntax}{sim / $\;\phi$ : $\;\mathit{eqs}$}
    One may give the starting conjuction, $\mathit{eqs}$, of equalities
    explicitly, and may also specifiy an invariant $\phi$ on the
    global variables of the programs.

    \medskip For example, if the current goal is
    \ecinput{examps/parts/tactics/sim/1-2.0.ec}{}{}{}{} then
    running \ecinput{examps/parts/tactics/sim/1-2.ec}{}{}{}{}
    produces the goals
    \ecinput{examps/parts/tactics/sim/1-2.1.ec}{}{}{}{}
    and
    \ecinput{examps/parts/tactics/sim/1-2.2.ec}{}{}{}{}
    which \ec{smt} and \ec{auto;smt}, respectively, are
    able to solve.
  \end{tsyntax}

  \begin{tsyntax}{sim $\;\mathit{proceq}_1$ $\;\ldots$ $\;\mathit{proceq}_1$ / $\;\phi$ : $\;\mathit{eqs}$}
    In its most general form, one may also supply a sequence of
    procedure global equality specifications of the form
    \begin{center}
      \ec{($M$.$p$ ~ $\;N$.$q$ : $\;\mathit{eqs}$)},
    \end{center}
    where $\mathit{eqs}$ is a conjuction of global variable
    equalities. When \ec{sim} encounters a pair of procedure calls
    consisting of a call to \ec{$M$.$p$} in the first program and
    \ec{$N$.$q$} in the second program, it will generate a subgoal
    whose conclusion is a \prhl judgment between \ec{$M$.$p$} and
    \ec{$N$.$q$}, whose precondition assumes equality of its
    arguments, $\mathit{eqs}$ and $\phi$, and whose postcondition
    requires equality of the calls' results, $\mathit{eqs}$ and
    $\phi$.

    One may also replace \ec{$M$.$p$ ~ $\;N$.$q$} by \ec{_},
    meaning that the same conjunction of global variable equalities
    is used for all procedure calls.

    \medskip For example, if the current goal is
    \ecinput{examps/parts/tactics/sim/2-1.0.ec}{}{}{}{} then
    running \ecinput{examps/parts/tactics/sim/2-1.ec}{}{}{}{}
    produces the goals
    \ecinput{examps/parts/tactics/sim/2-1.1.ec}{}{}{}{}
    and
    \ecinput{examps/parts/tactics/sim/2-1.2.ec}{}{}{}{}
    and
    \ecinput{examps/parts/tactics/sim/2-1.3.ec}{}{}{}{}
    and
    \ecinput{examps/parts/tactics/sim/2-1.4.ec}{}{}{}{}
    which \ec{smt}, \ec{proc;auto;smt}, \ec{proc;auto;smt}
    and \ec{auto}, respectively, are able to solve.

    \medskip And, if the current goal is
    \ecinput{examps/parts/tactics/sim/2-2.0.ec}{}{}{}{} then
    running \ecinput{examps/parts/tactics/sim/2-2.ec}{}{}{}{}
    produces the goals
    \ecinput{examps/parts/tactics/sim/2-2.1.ec}{}{}{}{}
    and
    \ecinput{examps/parts/tactics/sim/2-2.2.ec}{}{}{}{}
    and
    \ecinput{examps/parts/tactics/sim/2-2.3.ec}{}{}{}{}
    and
    \ecinput{examps/parts/tactics/sim/2-2.4.ec}{}{}{}{}
    which \ec{smt}, \ec{proc;auto;smt}, \ec{proc;auto;smt}
    and \ec{auto}, respectively, are able to solve.
  \end{tsyntax}
\end{tactic}

%%  \begin{tsyntax}{sim <pos>? <hintgeqs>* <hintinv>? <eqs>?}\\
%%    where \begin{tabular}{lrl}
%%       <pos>      & = & <uint> <uint> \\
%%       <hintgeqs> & = & (<procname>? $\sim$ <procname>? : <formula>) \\
%%                  & | & ($\_$? : <formula>  \\
%%       <eqs>      & = & : <formula> \\
%%    \end{tabular}
%%  
%%  \fix{Missing description of sim}.
%%  \end{tsyntax}
%%\end{tactic}


\subsection{Advanced Tactics}
\label{subsec:advancedtactics}

% --------------------------------------------------------------------
\begin{tactic}{eager}
  \begin{tsyntax}[empty]{eager}
  \fix{Missing description of eager}.
  \fix{Missing descriptions for all eager <tactic> variants}.
  \end{tsyntax}
\end{tactic}

% --------------------------------------------------------------------
\begin{tactic}{fel}
  \begin{tsyntax}{fel $\;\mathit{init}$ $\;\mathit{ctr}$ $\;\mathit{stepub}$
                      $\;\mathit{bound}$ $\;\mathit{bad}$ $\;\mathit{conds}$
                      $\;\mathit{inv}$}
    ``fel'' stands for ``failure event lemma''. To use this tactic,
    one must load the theory \ec{FelTactic}. To be applicable, the
    current goal's conclusion must have the form
    \begin{center}
      \ec{Pr[$M$.$p$($a_1$, $\;\ldots$, $\;a_r$) @ &$m$ : $\;\phi$] <= $\;\mathit{ub}$}.
    \end{center}
    Here:
    \begin{itemize}
    \item $\mathit{ub}$ (``upper bound'') is an expression of type \ec{real}.

    \item $\mathit{ctr}$ is the \emph{counter}, an expression of
      type \ec{int} involving program variables.

    \item $\mathit{bad}$ is an expression of type \ec{bool} involving
      program variables. It is the ``bad'' or ``failure'' event.

    \item $\mathit{inv}$ is an optional invariant on program
      variables; if it's omitted, \ec{true} is used.

    \item $\mathit{init}$ is a natural number no bigger than the
      number of statements in $M$.$p$. It is the length of the initial
      part of the procedure that ``initializes'' the failure event
      lemma---causing $\mathit{ctr}$ to become $0$ and $\mathit{bad}$
      to become \ec{false} and establishing $\mathit{inv}$.  The
      non-initialization part of the procedure may not \emph{directly}
      use the program variables on which $\mathit{ctr}$,
      $\mathit{bad}$ and $\mathit{inv}$ depend. These variables may
      only be modified by concrete procedures \ec{$M$.$p$} may
      directly or indirectly call---such procedures are called
      \emph{oracle procedures}.  If \ec{$M$.$p$} directly or
      indirectly calls an abstract procedure, there must be a module
      constraint saying that the abstract procedure may not modify the
      program variables determining the values of $\mathit{ctr}$,
      $\mathit{bad}$ and $\mathit{inv}$ or that are used by the oracle
      procedures.

    \item $\mathit{bound}$ is an expression of type \ec{int}. It must
      be the case that
      \begin{center}
        \ec{$\phi$ /\\ $\;\mathit{inv}$ => $\;\mathit{bad}$ /\\
            $\;\mathit{ctr}$ <= $\;\mathit{bound}$}.
      \end{center}

    \item $\mathit{conds}$ is a list of \emph{procedure preconditions}
      \begin{center}
        \ec{[$N_1$.$p_1$ : $\;\phi_1$; $\;\ldots$; $\;N_l$.$p_l$ : $\;\phi_l$]},
      \end{center}
      where the $N_i$.$p_i$ are procedures, and the $\phi_i$ are
      expressions of type \ec{bool} involving program variables and
      procedure parameters.  When a procedure's precondition is true,
      it must increase the counter's value; when it isn't true, it
      must not decrease the counter's value, and must preserve the
      value of $\mathit{bad}$. Whether a procedure's precondition
      holds or not, the invariant $\mathit{inv}$ must be preserved.

    \item $\mathit{stepub}$ is a function of type \ec{int -> real},
      providing an upper bound as a function of the
      counter's current value. When a procedure's precondition, the
      invariant $\mathit{inv}$, \ec{!$\mathit{bad}$} and \ec{0 <=
        $\;\mathit{ctr}$ < $\;\mathit{bound}$} hold, the probability
      that $\mathit{bad}$ becomes set during that call must be
      upper-bounded by the application of $\mathit{stepub}$ to the
      counter's value.  In addition, it must be the case that the
      summation of \ec{$\mathit{stepub}$ $\;i$}, as $i$ ranges from
      $0$ to $\mathit{bound} - 1$, is upper-bounded by $\mathit{ub}$.

    \end{itemize}
    The subgoals generated by \ec{fel} enforce the above rules. The
    best way to understand the details is via an example.

    \medskip For example, consider the declarations
    \ecinput[linerange=30-76]{examps/tactics/fel/1.ec} Here, the oracle has
    a boolean variable \ec{won}, which is the bad event. It also has a
    list of integers \ec{gens}, all of which are within the range $1$
    to \ec{upp}, inclusive---the integers ``generated'' so far. The
    counter is the size of \ec{gens}. The procedure \ec{gen} randomly
    generates such an integer, setting \ec{won} to \ec{true} if the
    integer was previously generated. And the procedure \ec{add} adds
    a new integer to the list of generated integers, without possibily
    setting \ec{bad}. Both \ec{gen} and \ec{add} do nothing when the
    counter reaches the bound \ec{n}.  The adversary has access to both
    \ec{gen} and \ec{bad}.

    If the current goal is
    \ecinput{examps/parts/tactics/fel/1-1.0.ec} then
    running \ecinput{examps/parts/tactics/fel/1-1.ec}
    produces the goals
    \ecinput{examps/parts/tactics/fel/1-1.1.ec}
    and
    \ecinput{examps/parts/tactics/fel/1-1.2.ec}
    and
    \ecinput{examps/parts/tactics/fel/1-1.3.ec}
    and
    \ecinput{examps/parts/tactics/fel/1-1.4.ec}
    and
    \ecinput{examps/parts/tactics/fel/1-1.5.ec}
    and
    \ecinput{examps/parts/tactics/fel/1-1.6.ec}
    and
    \ecinput{examps/parts/tactics/fel/1-1.7.ec}
    and
    \ecinput{examps/parts/tactics/fel/1-1.8.ec}
    and
    \ecinput{examps/parts/tactics/fel/1-1.9.ec}
  \end{tsyntax}
\end{tactic}


